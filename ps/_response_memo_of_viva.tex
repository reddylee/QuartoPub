% Options for packages loaded elsewhere
% Options for packages loaded elsewhere
\PassOptionsToPackage{unicode}{hyperref}
\PassOptionsToPackage{hyphens}{url}
\PassOptionsToPackage{dvipsnames,svgnames,x11names}{xcolor}
%
\documentclass[
  12pt,
  letterpaper,
  DIV=11,
  numbers=noendperiod]{scrartcl}
\usepackage{xcolor}
\usepackage{amsmath,amssymb}
\setcounter{secnumdepth}{-\maxdimen} % remove section numbering
\usepackage{iftex}
\ifPDFTeX
  \usepackage[T1]{fontenc}
  \usepackage[utf8]{inputenc}
  \usepackage{textcomp} % provide euro and other symbols
\else % if luatex or xetex
  \usepackage{unicode-math} % this also loads fontspec
  \defaultfontfeatures{Scale=MatchLowercase}
  \defaultfontfeatures[\rmfamily]{Ligatures=TeX,Scale=1}
\fi
\usepackage{lmodern}
\ifPDFTeX\else
  % xetex/luatex font selection
  \setmainfont[]{Times New Roman}
  \setsansfont[]{Arial}
  \setmonofont[]{Courier New}
\fi
% Use upquote if available, for straight quotes in verbatim environments
\IfFileExists{upquote.sty}{\usepackage{upquote}}{}
\IfFileExists{microtype.sty}{% use microtype if available
  \usepackage[]{microtype}
  \UseMicrotypeSet[protrusion]{basicmath} % disable protrusion for tt fonts
}{}
\usepackage{setspace}
\makeatletter
\@ifundefined{KOMAClassName}{% if non-KOMA class
  \IfFileExists{parskip.sty}{%
    \usepackage{parskip}
  }{% else
    \setlength{\parindent}{0pt}
    \setlength{\parskip}{6pt plus 2pt minus 1pt}}
}{% if KOMA class
  \KOMAoptions{parskip=half}}
\makeatother
% Make \paragraph and \subparagraph free-standing
\makeatletter
\ifx\paragraph\undefined\else
  \let\oldparagraph\paragraph
  \renewcommand{\paragraph}{
    \@ifstar
      \xxxParagraphStar
      \xxxParagraphNoStar
  }
  \newcommand{\xxxParagraphStar}[1]{\oldparagraph*{#1}\mbox{}}
  \newcommand{\xxxParagraphNoStar}[1]{\oldparagraph{#1}\mbox{}}
\fi
\ifx\subparagraph\undefined\else
  \let\oldsubparagraph\subparagraph
  \renewcommand{\subparagraph}{
    \@ifstar
      \xxxSubParagraphStar
      \xxxSubParagraphNoStar
  }
  \newcommand{\xxxSubParagraphStar}[1]{\oldsubparagraph*{#1}\mbox{}}
  \newcommand{\xxxSubParagraphNoStar}[1]{\oldsubparagraph{#1}\mbox{}}
\fi
\makeatother


\usepackage{longtable,booktabs,array}
\usepackage{calc} % for calculating minipage widths
% Correct order of tables after \paragraph or \subparagraph
\usepackage{etoolbox}
\makeatletter
\patchcmd\longtable{\par}{\if@noskipsec\mbox{}\fi\par}{}{}
\makeatother
% Allow footnotes in longtable head/foot
\IfFileExists{footnotehyper.sty}{\usepackage{footnotehyper}}{\usepackage{footnote}}
\makesavenoteenv{longtable}
\usepackage{graphicx}
\makeatletter
\newsavebox\pandoc@box
\newcommand*\pandocbounded[1]{% scales image to fit in text height/width
  \sbox\pandoc@box{#1}%
  \Gscale@div\@tempa{\textheight}{\dimexpr\ht\pandoc@box+\dp\pandoc@box\relax}%
  \Gscale@div\@tempb{\linewidth}{\wd\pandoc@box}%
  \ifdim\@tempb\p@<\@tempa\p@\let\@tempa\@tempb\fi% select the smaller of both
  \ifdim\@tempa\p@<\p@\scalebox{\@tempa}{\usebox\pandoc@box}%
  \else\usebox{\pandoc@box}%
  \fi%
}
% Set default figure placement to htbp
\def\fps@figure{htbp}
\makeatother





\setlength{\emergencystretch}{3em} % prevent overfull lines

\providecommand{\tightlist}{%
  \setlength{\itemsep}{0pt}\setlength{\parskip}{0pt}}



 


\KOMAoption{captions}{tableheading}
\makeatletter
\@ifpackageloaded{caption}{}{\usepackage{caption}}
\AtBeginDocument{%
\ifdefined\contentsname
  \renewcommand*\contentsname{Table of contents}
\else
  \newcommand\contentsname{Table of contents}
\fi
\ifdefined\listfigurename
  \renewcommand*\listfigurename{List of Figures}
\else
  \newcommand\listfigurename{List of Figures}
\fi
\ifdefined\listtablename
  \renewcommand*\listtablename{List of Tables}
\else
  \newcommand\listtablename{List of Tables}
\fi
\ifdefined\figurename
  \renewcommand*\figurename{Figure}
\else
  \newcommand\figurename{Figure}
\fi
\ifdefined\tablename
  \renewcommand*\tablename{Table}
\else
  \newcommand\tablename{Table}
\fi
}
\@ifpackageloaded{float}{}{\usepackage{float}}
\floatstyle{ruled}
\@ifundefined{c@chapter}{\newfloat{codelisting}{h}{lop}}{\newfloat{codelisting}{h}{lop}[chapter]}
\floatname{codelisting}{Listing}
\newcommand*\listoflistings{\listof{codelisting}{List of Listings}}
\makeatother
\makeatletter
\makeatother
\makeatletter
\@ifpackageloaded{caption}{}{\usepackage{caption}}
\@ifpackageloaded{subcaption}{}{\usepackage{subcaption}}
\makeatother
\usepackage{bookmark}
\IfFileExists{xurl.sty}{\usepackage{xurl}}{} % add URL line breaks if available
\urlstyle{same}
\hypersetup{
  pdftitle={Response Memo: Thesis Corrections Based on Viva Feedback},
  colorlinks=true,
  linkcolor={blue},
  filecolor={Maroon},
  citecolor={Blue},
  urlcolor={blue},
  pdfcreator={LaTeX via pandoc}}


\title{Response Memo: Thesis Corrections Based on Viva Feedback}
\author{}
\date{}
\begin{document}
\maketitle


\setstretch{1.618}
\textbf{Dear Professor Böhmelt and Professor Powell,}

Thank you for your insightful feedback and constructive critiques during
my viva examination. I have carefully reviewed your examiner reports and
our discussions, incorporating substantial revisions to address your
concerns. This memo outlines the changes made, directly referencing your
feedback and explaining how the revised thesis enhances focus, clarity,
and analytical rigor.

The revisions include a significant restructuring of the thesis,
refining its scope and strengthening its arguments. Below, I detail the
new structure and specific changes made in response to your comments.

\subsubsection{Revised Thesis Structure}\label{revised-thesis-structure}

To address your feedback, I have removed the chapter on classic coup
determinants (original Chapter 2) to sharpen the focus on autocoups and
their comparative dynamics with coups. The revised structure is as
follows:

Chapter 1: Introduction

Chapter 2: Conceptualizing and Analysing Autocoups (Expanded from the
original Chapter 3)

Chapter 3: Determinants of Autocoup Attempts (Expanded from a section
within the original Chapter 3)

Chapter 4: Power Acquisition and Leadership Survival (Revised from the
original Chapter 4)

Chapter 5: Coups, Autocoups, and Democracy (New Substantive Chapter)

Chapter 6: Conclusion and Future Research Directions

\subsubsection{Specific Revisions Addressing Examiner
Feedback}\label{specific-revisions-addressing-examiner-feedback}

\paragraph{Chapter 2: Conceptualizing autocoup and introducing
dataset}\label{chapter-2-conceptualizing-autocoup-and-introducing-dataset}

\textbf{Conceptual Clarity:} I have refined the definition of autocoups,
clearly distinguishing them from broader power expansion efforts and
emphasizing tenure extension through unconstitutional means. Chapter 2
now includes a detailed rationale for this conceptualization to enhance
theoretical precision.

\textbf{Dataset Refinement (Addressing Professor Böhmelt's Concern):}
Professor Böhmelt, you highlighted potential overlap between
coup-installed leaders and autocoup leaders, which could introduce
post-treatment bias in the survival analysis (Chapter 4). To address
this, I conducted a thorough review of the autocoup dataset, excluding
cases where leaders initially assumed power via a coup. This ensures
that the ``autocoup leader'' category only includes those who entered
power through constitutional means. As a result, the dataset for the
survival analysis was reduced from 110 to 83 events. While this reduces
the sample size, it significantly improves analytical rigor and clarity
in distinguishing coup-installed and autocoup leaders, directly
addressing your concern.

\paragraph{Chapter 3: Determinants of autocoup
attempts}\label{chapter-3-determinants-of-autocoup-attempts}

\textbf{Rationale for Autocoups by Personalist Leaders (Addressing
Professor Powell's query):} Professor Powell, you questioned why leaders
classified as ``personalist'' would stage autocoups to extend tenure. In
the revised Chapter 3, I clarify that the GWF regime type coding
reflects a leader's tenure, which often evolves over time. Many
personalist leaders initially assume power within party, military, or
democratic structures before consolidating a personalist regime.
Autocoups often mark a critical step in this consolidation, removing
constitutional barriers to extend rule and solidify personal control. I
have included case studies, such as leaders in Russia and Belarus, to
illustrate how autocoups facilitate the transition to personalist rule.

\paragraph{Chapter 4: Power acquisition and leadership
survival}\label{chapter-4-power-acquisition-and-leadership-survival}

\textbf{Regime Type as Covariate (Addressing Professor Powell's
suggestion):} Following your recommendation, Professor Powell, I
incorporated regime type as a key covariate in the survival analysis
models. The revised results indicate that, when controlling for regime
type, the effect of power acquisition method (coup vs.~autocoup) on
leadership survival is no longer statistically significant. This finding
aligns with literature emphasizing the role of institutional context in
shaping political outcomes and underscores regime type as a primary
determinant of leader longevity, beyond irregular mechanisms of power
acquisition.

\textbf{Impact of Dataset Refinement:} The exclusion of overlapping
cases from the autocoup dataset (noted in Chapter 2) enhances the
clarity of comparisons between coup and autocoup leaders in this
analysis, ensuring robust results.

\paragraph{Chapter 5: Coups, autocoups, and
democracy}\label{chapter-5-coups-autocoups-and-democracy}

\textbf{New Substantive Chapter:} This new chapter examines the
comparative impact of coups and autocoups on democratic institutions and
trajectories. Building on prior analyses, it provides a comprehensive
assessment of how these irregular transitions affect democratic quality
and stability over time, enriching the thesis's contribution to the
literature.

\paragraph{Additional Revisions}\label{additional-revisions}

\textbf{Literature Engagement:} I have strengthened engagement with
existing studies, particularly in the introductory sections, ensuring
comprehensive referencing to address your feedback on literature
integration.

\textbf{Formatting and Style:} Bullet points in the main text have been
converted to narrative paragraphs to improve academic flow and
coherence.

\textbf{Clarity and Concision:} The entire manuscript has been reviewed
to enhance readability, eliminate redundancy, and maintain a
professional tone.

These revisions aim to address your feedback comprehensively, enhancing
the thesis's coherence, analytical depth, and scholarly rigor. I am
deeply grateful for your expertise and rigorous review, which have
significantly strengthened this work. Please let me know if further
clarifications or adjustments are needed.

Sincerely,

Zhu Qi




\end{document}

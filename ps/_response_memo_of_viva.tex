% Options for packages loaded elsewhere
% Options for packages loaded elsewhere
\PassOptionsToPackage{unicode}{hyperref}
\PassOptionsToPackage{hyphens}{url}
\PassOptionsToPackage{dvipsnames,svgnames,x11names}{xcolor}
%
\documentclass[
  12pt,
  letterpaper,
  DIV=11,
  numbers=noendperiod]{scrartcl}
\usepackage{xcolor}
\usepackage{amsmath,amssymb}
\setcounter{secnumdepth}{-\maxdimen} % remove section numbering
\usepackage{iftex}
\ifPDFTeX
  \usepackage[T1]{fontenc}
  \usepackage[utf8]{inputenc}
  \usepackage{textcomp} % provide euro and other symbols
\else % if luatex or xetex
  \usepackage{unicode-math} % this also loads fontspec
  \defaultfontfeatures{Scale=MatchLowercase}
  \defaultfontfeatures[\rmfamily]{Ligatures=TeX,Scale=1}
\fi
\usepackage{lmodern}
\ifPDFTeX\else
  % xetex/luatex font selection
  \setmainfont[]{Times New Roman}
  \setsansfont[]{Arial}
  \setmonofont[]{Courier New}
\fi
% Use upquote if available, for straight quotes in verbatim environments
\IfFileExists{upquote.sty}{\usepackage{upquote}}{}
\IfFileExists{microtype.sty}{% use microtype if available
  \usepackage[]{microtype}
  \UseMicrotypeSet[protrusion]{basicmath} % disable protrusion for tt fonts
}{}
\usepackage{setspace}
\makeatletter
\@ifundefined{KOMAClassName}{% if non-KOMA class
  \IfFileExists{parskip.sty}{%
    \usepackage{parskip}
  }{% else
    \setlength{\parindent}{0pt}
    \setlength{\parskip}{6pt plus 2pt minus 1pt}}
}{% if KOMA class
  \KOMAoptions{parskip=half}}
\makeatother
% Make \paragraph and \subparagraph free-standing
\makeatletter
\ifx\paragraph\undefined\else
  \let\oldparagraph\paragraph
  \renewcommand{\paragraph}{
    \@ifstar
      \xxxParagraphStar
      \xxxParagraphNoStar
  }
  \newcommand{\xxxParagraphStar}[1]{\oldparagraph*{#1}\mbox{}}
  \newcommand{\xxxParagraphNoStar}[1]{\oldparagraph{#1}\mbox{}}
\fi
\ifx\subparagraph\undefined\else
  \let\oldsubparagraph\subparagraph
  \renewcommand{\subparagraph}{
    \@ifstar
      \xxxSubParagraphStar
      \xxxSubParagraphNoStar
  }
  \newcommand{\xxxSubParagraphStar}[1]{\oldsubparagraph*{#1}\mbox{}}
  \newcommand{\xxxSubParagraphNoStar}[1]{\oldsubparagraph{#1}\mbox{}}
\fi
\makeatother


\usepackage{longtable,booktabs,array}
\usepackage{calc} % for calculating minipage widths
% Correct order of tables after \paragraph or \subparagraph
\usepackage{etoolbox}
\makeatletter
\patchcmd\longtable{\par}{\if@noskipsec\mbox{}\fi\par}{}{}
\makeatother
% Allow footnotes in longtable head/foot
\IfFileExists{footnotehyper.sty}{\usepackage{footnotehyper}}{\usepackage{footnote}}
\makesavenoteenv{longtable}
\usepackage{graphicx}
\makeatletter
\newsavebox\pandoc@box
\newcommand*\pandocbounded[1]{% scales image to fit in text height/width
  \sbox\pandoc@box{#1}%
  \Gscale@div\@tempa{\textheight}{\dimexpr\ht\pandoc@box+\dp\pandoc@box\relax}%
  \Gscale@div\@tempb{\linewidth}{\wd\pandoc@box}%
  \ifdim\@tempb\p@<\@tempa\p@\let\@tempa\@tempb\fi% select the smaller of both
  \ifdim\@tempa\p@<\p@\scalebox{\@tempa}{\usebox\pandoc@box}%
  \else\usebox{\pandoc@box}%
  \fi%
}
% Set default figure placement to htbp
\def\fps@figure{htbp}
\makeatother





\setlength{\emergencystretch}{3em} % prevent overfull lines

\providecommand{\tightlist}{%
  \setlength{\itemsep}{0pt}\setlength{\parskip}{0pt}}



 


\KOMAoption{captions}{tableheading}
\makeatletter
\@ifpackageloaded{caption}{}{\usepackage{caption}}
\AtBeginDocument{%
\ifdefined\contentsname
  \renewcommand*\contentsname{Table of contents}
\else
  \newcommand\contentsname{Table of contents}
\fi
\ifdefined\listfigurename
  \renewcommand*\listfigurename{List of Figures}
\else
  \newcommand\listfigurename{List of Figures}
\fi
\ifdefined\listtablename
  \renewcommand*\listtablename{List of Tables}
\else
  \newcommand\listtablename{List of Tables}
\fi
\ifdefined\figurename
  \renewcommand*\figurename{Figure}
\else
  \newcommand\figurename{Figure}
\fi
\ifdefined\tablename
  \renewcommand*\tablename{Table}
\else
  \newcommand\tablename{Table}
\fi
}
\@ifpackageloaded{float}{}{\usepackage{float}}
\floatstyle{ruled}
\@ifundefined{c@chapter}{\newfloat{codelisting}{h}{lop}}{\newfloat{codelisting}{h}{lop}[chapter]}
\floatname{codelisting}{Listing}
\newcommand*\listoflistings{\listof{codelisting}{List of Listings}}
\makeatother
\makeatletter
\makeatother
\makeatletter
\@ifpackageloaded{caption}{}{\usepackage{caption}}
\@ifpackageloaded{subcaption}{}{\usepackage{subcaption}}
\makeatother
\usepackage{bookmark}
\IfFileExists{xurl.sty}{\usepackage{xurl}}{} % add URL line breaks if available
\urlstyle{same}
\hypersetup{
  pdftitle={Response Memo: Thesis Corrections Based on Viva Feedback},
  colorlinks=true,
  linkcolor={blue},
  filecolor={Maroon},
  citecolor={Blue},
  urlcolor={blue},
  pdfcreator={LaTeX via pandoc}}


\title{Response Memo: Thesis Corrections Based on Viva Feedback}
\author{}
\date{}
\begin{document}
\maketitle


\setstretch{1.618}
I wish to convey my profound gratitude to Professor Böhmelt and
Professor Powell for their insightful and constructive feedback during
my viva examination. Having thoroughly reviewed the examiners reports
and the discussions held during the viva, I have implemented substantial
revisions to address the concerns raised. This memorandum delineates the
modifications made, directly referencing the feedback provided, and
elucidates how these changes have enhanced the thesis in terms of focus,
clarity, and analytical rigour.

\subsubsection{1. Overview of Revisions}\label{overview-of-revisions}

The thesis has undergone significant structural reorganisation, with a
refined scope and strengthened core arguments. The following sections
outline the revised structure and detail the specific amendments made in
response to the examiners comments.

\subsubsection{2. Revised Thesis
Structure}\label{revised-thesis-structure}

To enhance the analytical focus, as suggested, I have removed the
chapter on classical coup determinants (formerly Chapter 2). This
adjustment enables the thesis to concentrate more directly on autocoups
and their comparative dynamics with coups. The revised structure is as
follows:

\textbf{Chapter 1:} Introduction

\textbf{Chapter 2:} Conceptualising and Analysing Autocoups (expanded
from the original Chapter 3)

\textbf{Chapter 3:} Determinants of Autocoup Attempts (developed from a
section within the original Chapter 3)

\textbf{Chapter 4:} Power Acquisition and Leadership Survival (revised
from the original Chapter 4)

\textbf{Chapter 5:} Coups, Autocoups, and Democracy (new substantive
chapter)

\textbf{Chapter 6:} Conclusion and Future Research Directions

\subsubsection{3. Specific Revisions in Response to Examiner
Feedback}\label{specific-revisions-in-response-to-examiner-feedback}

\paragraph{Chapter 2: Conceptualizing autocoup and introducing
dataset}\label{chapter-2-conceptualizing-autocoup-and-introducing-dataset}

\textbf{Conceptual Clarification:} The definition of autocoups has been
refined to distinguish them more clearly from broader efforts at
executive aggrandisement. The revised chapter underscores the extension
of tenure through extra-constitutional means and provides a detailed
rationale for this conceptualisation, thereby enhancing theoretical
precision.

\textbf{Dataset Refinement (Responding to Professor Böhmelt):}
Addressing concerns about potential post-treatment biasspecifically, the
inclusion of leaders who initially assumed power via a coup and later
conducted an autocoup---I have meticulously reviewed the dataset. Cases
where leaders entered power through a coup have been excluded from the
autocoup category. Consequently, the dataset used in the survival
analysis has been reduced from 110 to 83 events. While this reduces the
sample size, it significantly enhances the analytical clarity and
validity of the findings by maintaining a strict distinction between
coup-installed and autocoup leaders.

\paragraph{Chapter 3: Determinants of autocoup
attempts}\label{chapter-3-determinants-of-autocoup-attempts}

\textbf{Motivations of Personalist Leaders (Responding to Professor
Powell):} In response to the query regarding why personalist leaders
might undertake an autocoup, I have clarified that regime-type coding
(e.g., Geddes, Wright, and Frantz) reflects a leaders tenure and
evolution over time. Many personalist regimes emerge through a process
of consolidation following initial entry via party, military, or
democratic means. Autocoups often serve as a pivotal mechanism in this
transformation, enabling leaders to remove constitutional constraints
and entrench their authority. To elucidate this, I have incorporated
case studies, including examples from Russia and Belarus.

\paragraph{Chapter 4: Power acquisition and leadership
survival}\label{chapter-4-power-acquisition-and-leadership-survival}

\textbf{Inclusion of Non-Coup Leaders as Baseline (Responding to
Professor Böhmelt):} A new baseline category comprising non-coup
leadersthose who assumed office through neither coups nor autocoups,
typically via regular constitutional mechanismshas been introduced. This
refinement enhances the coherence of the survival models and provides a
more robust analytical framework for assessing the effects of different
modes of power acquisition.

\textbf{Regime Type as a Covariate (Responding to Professor Powell):}
Following Professor Powell's recommendation, regime type has been
incorporated as a covariate in the survival analysis. The revised
results indicate that, when regime type is controlled for, the method of
power acquisition (coup versus autocoup) no longer exerts a
statistically significant effect on leadership survival. This finding
aligns with the literature, which underscores the importance of
institutional context in shaping political outcomes, and reinforces the
conclusion that regime type is a critical determinant of leader
longevity.

\textbf{Impact of Dataset Refinement:} The exclusion of cases involving
prior coups (as detailed in Chapter 2) further clarifies the analysis,
enabling a more precise comparison between autocoup and coup leaders in
the context of survival modelling.

\paragraph{Chapter 5: Coups, autocoups, and
democracy}\label{chapter-5-coups-autocoups-and-democracy}

\textbf{New Substantive Chapter:} In response to recommendations made
during the viva, a new substantive chapter has been incorporated to
explore the broader implications of coups and autocoups for democratic
institutions and their developmental trajectories. Drawing on
longitudinal data, this chapter assesses their impact on democratic
quality and institutional stability, as measured by Polity V scores. By
situating these findings within the broader comparative politics
literature, the chapter enhances the thesis's contribution to scholarly
debates on regime transitions and democratic backsliding.

\subsubsection{4. Additional Revisions}\label{additional-revisions}

\textbf{Engagement with the Literature:} The literature review,
particularly in the introductory sections, has been expanded and
deepened to better contextualise the research and demonstrate engagement
with existing scholarship.

\textbf{Stylistic and Formatting Improvements:} Bullet points within the
main text have been reformulated into full narrative paragraphs to
enhance the academic tone and coherence.

\textbf{Clarity and Concision:} The entire manuscript has been revised
to improve readability, consistency, and professional presentation.

These revisions have been undertaken to address the feedback provided
during the viva and in the examiners reports comprehensively. They aim
to enhance the coherence, analytical depth, and scholarly quality of the
thesis. I am deeply grateful to Professor Böhmelt and Professor Powell
for their rigorous and detailed assessments, which have significantly
strengthened the final manuscript. Should further clarification or
additional revisions be required, I would be pleased to address them.




\end{document}

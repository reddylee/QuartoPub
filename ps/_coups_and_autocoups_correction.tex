% Options for packages loaded elsewhere
% Options for packages loaded elsewhere
\PassOptionsToPackage{unicode}{hyperref}
\PassOptionsToPackage{hyphens}{url}
\PassOptionsToPackage{dvipsnames,svgnames,x11names}{xcolor}
%
\documentclass[
  12pt,
]{report}
\usepackage{xcolor}
\usepackage[top = 3cm,bottom = 3cm,left = 3cm,right = 2.7cm]{geometry}
\usepackage{amsmath,amssymb}
\setcounter{secnumdepth}{5}
\usepackage{iftex}
\ifPDFTeX
  \usepackage[T1]{fontenc}
  \usepackage[utf8]{inputenc}
  \usepackage{textcomp} % provide euro and other symbols
\else % if luatex or xetex
  \usepackage{unicode-math} % this also loads fontspec
  \defaultfontfeatures{Scale=MatchLowercase}
  \defaultfontfeatures[\rmfamily]{Ligatures=TeX,Scale=1}
\fi
\usepackage{lmodern}
\ifPDFTeX\else
  % xetex/luatex font selection
  \setmainfont[]{Times New Roman}
  \setsansfont[]{Arial}
  \setmonofont[]{Courier New}
\fi
% Use upquote if available, for straight quotes in verbatim environments
\IfFileExists{upquote.sty}{\usepackage{upquote}}{}
\IfFileExists{microtype.sty}{% use microtype if available
  \usepackage[]{microtype}
  \UseMicrotypeSet[protrusion]{basicmath} % disable protrusion for tt fonts
}{}
\usepackage{setspace}
% Make \paragraph and \subparagraph free-standing
\makeatletter
\ifx\paragraph\undefined\else
  \let\oldparagraph\paragraph
  \renewcommand{\paragraph}{
    \@ifstar
      \xxxParagraphStar
      \xxxParagraphNoStar
  }
  \newcommand{\xxxParagraphStar}[1]{\oldparagraph*{#1}\mbox{}}
  \newcommand{\xxxParagraphNoStar}[1]{\oldparagraph{#1}\mbox{}}
\fi
\ifx\subparagraph\undefined\else
  \let\oldsubparagraph\subparagraph
  \renewcommand{\subparagraph}{
    \@ifstar
      \xxxSubParagraphStar
      \xxxSubParagraphNoStar
  }
  \newcommand{\xxxSubParagraphStar}[1]{\oldsubparagraph*{#1}\mbox{}}
  \newcommand{\xxxSubParagraphNoStar}[1]{\oldsubparagraph{#1}\mbox{}}
\fi
\makeatother


\usepackage{longtable,booktabs,array}
\usepackage{calc} % for calculating minipage widths
% Correct order of tables after \paragraph or \subparagraph
\usepackage{etoolbox}
\makeatletter
\patchcmd\longtable{\par}{\if@noskipsec\mbox{}\fi\par}{}{}
\makeatother
% Allow footnotes in longtable head/foot
\IfFileExists{footnotehyper.sty}{\usepackage{footnotehyper}}{\usepackage{footnote}}
\makesavenoteenv{longtable}
\usepackage{graphicx}
\makeatletter
\newsavebox\pandoc@box
\newcommand*\pandocbounded[1]{% scales image to fit in text height/width
  \sbox\pandoc@box{#1}%
  \Gscale@div\@tempa{\textheight}{\dimexpr\ht\pandoc@box+\dp\pandoc@box\relax}%
  \Gscale@div\@tempb{\linewidth}{\wd\pandoc@box}%
  \ifdim\@tempb\p@<\@tempa\p@\let\@tempa\@tempb\fi% select the smaller of both
  \ifdim\@tempa\p@<\p@\scalebox{\@tempa}{\usebox\pandoc@box}%
  \else\usebox{\pandoc@box}%
  \fi%
}
% Set default figure placement to htbp
\def\fps@figure{htbp}
\makeatother


% definitions for citeproc citations
\NewDocumentCommand\citeproctext{}{}
\NewDocumentCommand\citeproc{mm}{%
  \begingroup\def\citeproctext{#2}\cite{#1}\endgroup}
\makeatletter
 % allow citations to break across lines
 \let\@cite@ofmt\@firstofone
 % avoid brackets around text for \cite:
 \def\@biblabel#1{}
 \def\@cite#1#2{{#1\if@tempswa , #2\fi}}
\makeatother
\newlength{\cslhangindent}
\setlength{\cslhangindent}{1.5em}
\newlength{\csllabelwidth}
\setlength{\csllabelwidth}{3em}
\newenvironment{CSLReferences}[2] % #1 hanging-indent, #2 entry-spacing
 {\begin{list}{}{%
  \setlength{\itemindent}{0pt}
  \setlength{\leftmargin}{0pt}
  \setlength{\parsep}{0pt}
  % turn on hanging indent if param 1 is 1
  \ifodd #1
   \setlength{\leftmargin}{\cslhangindent}
   \setlength{\itemindent}{-1\cslhangindent}
  \fi
  % set entry spacing
  \setlength{\itemsep}{#2\baselineskip}}}
 {\end{list}}
\usepackage{calc}
\newcommand{\CSLBlock}[1]{\hfill\break\parbox[t]{\linewidth}{\strut\ignorespaces#1\strut}}
\newcommand{\CSLLeftMargin}[1]{\parbox[t]{\csllabelwidth}{\strut#1\strut}}
\newcommand{\CSLRightInline}[1]{\parbox[t]{\linewidth - \csllabelwidth}{\strut#1\strut}}
\newcommand{\CSLIndent}[1]{\hspace{\cslhangindent}#1}



\setlength{\emergencystretch}{3em} % prevent overfull lines

\providecommand{\tightlist}{%
  \setlength{\itemsep}{0pt}\setlength{\parskip}{0pt}}



 


\usepackage{booktabs}
\usepackage{caption}
\usepackage{longtable}
\usepackage{colortbl}
\usepackage{array}
\usepackage{anyfontsize}
\usepackage{multirow}
\usepackage{sectsty}
\chapterfont{\centering}
\usepackage{lscape}
\newcommand{\blandscape}{\begin{landscape}}
\newcommand{\elandscape}{\end{landscape}}
\makeatletter
\@ifpackageloaded{caption}{}{\usepackage{caption}}
\AtBeginDocument{%
\ifdefined\contentsname
  \renewcommand*\contentsname{Table of contents}
\else
  \newcommand\contentsname{Table of contents}
\fi
\ifdefined\listfigurename
  \renewcommand*\listfigurename{Figures}
\else
  \newcommand\listfigurename{Figures}
\fi
\ifdefined\listtablename
  \renewcommand*\listtablename{Tables}
\else
  \newcommand\listtablename{Tables}
\fi
\ifdefined\figurename
  \renewcommand*\figurename{Figure}
\else
  \newcommand\figurename{Figure}
\fi
\ifdefined\tablename
  \renewcommand*\tablename{Table}
\else
  \newcommand\tablename{Table}
\fi
}
\@ifpackageloaded{float}{}{\usepackage{float}}
\floatstyle{ruled}
\@ifundefined{c@chapter}{\newfloat{codelisting}{h}{lop}}{\newfloat{codelisting}{h}{lop}[chapter]}
\floatname{codelisting}{Listing}
\newcommand*\listoflistings{\listof{codelisting}{List of Listings}}
\makeatother
\makeatletter
\makeatother
\makeatletter
\@ifpackageloaded{caption}{}{\usepackage{caption}}
\@ifpackageloaded{subcaption}{}{\usepackage{subcaption}}
\makeatother
\usepackage{bookmark}
\IfFileExists{xurl.sty}{\usepackage{xurl}}{} % add URL line breaks if available
\urlstyle{same}
\hypersetup{
  colorlinks=true,
  linkcolor={blue},
  filecolor={Maroon},
  citecolor={Blue},
  urlcolor={blue},
  pdfcreator={LaTeX via pandoc}}


\author{}
\date{}
\begin{document}

\begin{titlepage}
  \begin{center}
    \vspace*{2cm}
    
    \Huge{\textbf{Leadership Transitions and Survival: Coups, Autocoups, and Power Dynamics}}
    
    \vspace{1.5cm}
    
    \Large{Zhu Qi}
    
    \vspace{5cm}
    
    \large{A thesis submitted for the degree of \\ Doctor of Philosophy in Political Science}
    
    \vspace{0.8cm}
    
    \large{Department of Government}
    \vspace{0.5cm}
    
    \large{University of Essex}
    
    \vspace{1.5cm}
    
    \large{September 2024}
    \vspace{2cm}
    
    
  \end{center}
\end{titlepage}

\renewcommand*\contentsname{Contents}
{
\hypersetup{linkcolor=}
\setcounter{tocdepth}{2}
\tableofcontents
}
\listoffigures
\listoftables

\setstretch{1.618}
\chapter*{Acknowledgements}\label{acknowledgements}
\addcontentsline{toc}{chapter}{Acknowledgements}

The completion of this thesis marks the culmination of a remarkable
journey, filled with dedication, perseverance, and moments of profound
joy. I am deeply grateful to the numerous individuals who have supported
and encouraged me throughout this endeavour.

I would like to express my sincerest appreciation to my supervisor,
Professor Kristian Skrede Gleditsch, whose guidance, expertise, and
unwavering support have been instrumental in shaping my research. His
constructive feedback and encouragement have been invaluable, and I am
profoundly grateful for his mentorship.

I am also grateful to Professor Han Dorussen, the chair of my board
panel, for his continuous support and thoughtful input. His insightful
comments and suggestions have significantly enhanced the quality and
depth of my research.

I would like to acknowledge the important contributions of my initial
co-supervisors, Dr.~Saurabh Pant and Professor David Siroky, who laid a
strong foundation for this work during the early stages of my research.
Although they are no longer at the University of Essex, their
instruction and guidance were instrumental in shaping the direction of
this project.

I have been fortunate to receive feedback and guidance from several
esteemed scholars in the field, including Dr.~Brian J Phillips,
Dr.~Prabin Khadka, and Dr. Winnie Xia. Their expertise and insights have
enriched this research, and I am grateful for their contributions.

I would also like to express my sincere gratitude to my examiners,
Professor Tobias Böhmelt and Professor Jonathan Powell. Their insightful
questions, constructive critiques, and engaging discussion during the
viva examination were invaluable in challenging my thinking and
highlighting areas for further refinement. Their expertise has
undoubtedly strengthened the final version of this thesis.

On a personal note, I would like to express my deepest gratitude to my
family, who have been a constant source of support and inspiration
throughout this journey. To my beloved wife, Ji Zhi, your patience,
love, and encouragement have been immeasurable. To my dear children,
Siyan and Sisheng, your joy and curiosity have motivated me to persevere
and strive for excellence.

I am also deeply grateful to my father for his enduring support and
belief in my abilities. To the cherished memory of my late mother, your
love, guidance, and values continue to shape my path and inspire my
endeavours. And to my three brothers, whose support enabled me to pursue
my PhD without worries, I am forever grateful.

While many individuals have contributed to the success of this work, I
take full responsibility for any errors or shortcomings that may remain.

\chapter*{Abstract}\label{abstract}
\addcontentsline{toc}{chapter}{Abstract}

This thesis addresses a notable lacuna in the study of irregular
leadership transitions by systematically incorporating
autocoups---instances wherein incumbent leaders extend their
constitutionally mandated terms through extra-constitutional means. It
refines the conceptual definition of autocoups by resolving prevailing
ambiguities, thereby aligning them more closely with conventional coup
frameworks. On the basis of this refined definition, the thesis
introduces a novel global dataset of autocoup events spanning the period
from 1945 to 2023, comprising 83 documented cases, 64 of which were
successful.

Utilising this dataset, the study conducts a large-N empirical analysis
to examine the structural determinants of autocoups. The findings
indicate that most power-centred regimes--presidential democracies and
personalist regimes--are significantly more susceptible to employing
autocoups as a strategy for power retention, in contrast to other regime
types. This pattern diverges from that of traditional coups, which have
historically been more prevalent in military regimes.

The analysis then shifts to the question of leadership survival,
employing survival analysis techniques to compare the political
longevity of leaders who assumed office via traditional coups with those
who retained power through autocoups. Contrary to the hypothesis that
autocoup leaders survive longer than their coup-installed counterparts,
the results reveal that---once very short-lived leaderships (less than
180 days) are excluded---the method of power acquisition does not exert
a statistically significant effect on leadership duration. Instead,
regime type emerges once again as the critical determinant: military
regimes exhibit a significantly higher hazard ratio for leadership
removal compared to the reference category of dominant-party regimes,
mirroring trends observed in classic coups.

The thesis also assesses the broader institutional ramifications of such
irregular power transitions, particularly with respect to
democratisation. Using Polity scores as a proxy for democratic quality
and applying a country-fixed effects model, the analysis demonstrates
that autocoups are associated with a sustained erosion of democratic
institutions both preceding and following their occurrence. In contrast,
while traditional coups precipitate an immediate and sharp decline in
democratic quality, they are often followed by democratic recovery or
transitions over time. These findings highlight the divergent political
trajectories engendered by coups and autocoups and call for greater
scholarly and policy attention to the consistently negative consequences
of autocoups for democratic governance.

Taken together, the findings underscore the distinct nature, drivers,
and consequences of coups and autocoups. This research makes several
substantive contributions: it clarifies the conceptual boundaries of
autocoups; provides a new empirical basis for their systematic study;
and offers robust comparative insights into how different modes of
irregular power transition influence both leadership survival and
institutional development. The implications are substantial for academic
scholarship and policy-making alike, particularly in the context of
global democratic backsliding and the resilience of political
institutions.

\emph{\textbf{Keywords:} Coups, Autocoups, Leadership transitions,
Leadership survival, Democratic resilience}

\chapter{Introduction}\label{introduction}

At the heart of contemporary political dynamics lies a fundamental
question: why are some political leaders prematurely removed from
office, while others succeed in extending their tenure beyond
constitutionally mandated limits? Furthermore, how does the mode of
their survival or removal shape political stability and democratic
institutions? This thesis seeks to address these critical questions by
analysing the structural and strategic foundations of irregular
leadership transitions.

\section{Motivations}\label{motivations}

The stability and resilience of political systems depend crucially on
the orderly transfer of power. When leadership transitions occur within
established institutional frameworks, they reinforce political
legitimacy and contribute to regime durability. In contrast, the
breakdown of conventional mechanisms for political succession often
precipitates instability, violence, and democratic backsliding. Among
the most disruptive of such breakdowns are irregular leadership
transitions, which leave lasting institutional legacies and
fundamentally alter the political trajectory of regimes. Understanding
the causes and consequences of these events remains central to the study
of political order and regime change.

The extant literature identifies a broad array of catalysts for
irregular leadership exits, including civil wars
(\citeproc{ref-kokkonen2019}{Kokkonen and Sundell 2019}), international
conflict (\citeproc{ref-demesquita1995}{Mesquita and Siverson 1995}),
ethnic divisions (\citeproc{ref-londregan1995}{Londregan, Bienen, and
Walle 1995}), economic crises (\citeproc{ref-miller2012}{Miller 2012};
\citeproc{ref-krishnarajan2019}{Krishnarajan 2019}), and natural
disasters (\citeproc{ref-quirozflores2012}{Quiroz Flores and Smith
2012}). Among these, coups d'état are particularly consequential due to
both their frequency and their direct displacement of incumbent leaders.
In autocratic regimes, coups account for nearly one-third of all
leadership exits---exceeding regular transitions, which constitute just
over one-fifth (\citeproc{ref-frantz2016}{Frantz and Stein 2016}).
Furthermore, over 63\% of non-constitutional removals in dictatorships
are attributable to coups (\citeproc{ref-svolik2009}{Svolik 2009}).

Consequently, coups have received extensive scholarly attention. A
substantial body of research explores their causes, outcomes, and
long-term implications for democracy and development
(\citeproc{ref-thyne2019}{Thyne and Powell 2019}). In particular, the
study of coup determinants has flourished, with scholars proposing
nearly one hundred explanatory variables. Yet, a widely accepted
baseline model remains elusive (\citeproc{ref-gassebner2016}{Gassebner,
Gutmann, and Voigt 2016}).

By contrast, another form of irregular power transition---the
autocoup---has received comparatively limited academic scrutiny. In an
autocoup, an incumbent extends their tenure by circumventing or
overriding constitutional term limits through extra-constitutional
means. While autocoups do not immediately result in leadership change,
they constitute a fundamental breach of institutional succession norms
and obstruct the expected regular transfer of power. As such, they
warrant classification as a critical, albeit understudied, variant of
irregular leadership transition.

This thesis argues that autocoups deserve systematic analysis alongside
traditional coups within a unified analytical framework. Despite
differences in execution, both coups and autocoups involve
extra-constitutional efforts to acquire or retain power, and both have
significant implications for leadership survival, regime stability, and
democratic integrity. Comparative analysis of these two forms of
irregular transition can reveal shared drivers, divergent outcomes, and
broader lessons for democratic resilience.

The urgency of this inquiry is underscored by the serious risks
associated with irregular transitions. Both coups and autocoups may
trigger immediate crises---ranging from institutional paralysis to civil
unrest---and leave deep institutional scars. More fundamentally, they
tend to dismantle constitutional checks and balances, undermine
electoral processes, and accelerate democratic decline or authoritarian
consolidation.

Historical cases illustrate these dangers vividly. Ghana's turbulent era
from 1979 to 1984 reflects the destabilising effects of classic coups.
Following Jerry Rawlings's 1979 coup, eight individuals, including three
former heads of state, were executed
(\citeproc{ref-pieterse1982}{Pieterse 1982}). Rawlings launched another
coup in 1981 and subsequently quashed three further coup attempts
(\citeproc{ref-haynes2022d}{Haynes 2022}). In contrast, the 1992
autocoup in Peru, orchestrated by President Alberto Fujimori,
exemplifies how an incumbent can dismantle democratic institutions
without a change in leadership. Fujimori dissolved Congress, suspended
the constitution, and ruled by decree
(\citeproc{ref-mauceri1995}{Mauceri 1995};
\citeproc{ref-cameron1998}{Maxwell A. Cameron 1998b}).

These patterns are increasingly salient in today's global political
landscape. According to Freedom House's Freedom in the World 2024
report, global political rights and civil liberties declined for the
eighteenth consecutive year in 2023, with setbacks recorded in 52
countries and improvements in only 21
(\citeproc{ref-freedomhouse2024freedom}{Freedom House 2024}). The
persistence of democratic erosion highlights the pressing need to
understand the mechanisms that facilitate it---including both coups and
autocoups.

This thesis seeks to advance both theoretical and empirical
understanding of irregular leadership transitions. It offers insights
with substantial implications for scholarly research and policy
formulation in fragile or democratising regimes.

\section{Research objectives and
contributions}\label{research-objectives-and-contributions}

In response to the pressing challenges posed by irregular leadership
transitions, this study undertakes a comprehensive comparative analysis
structured around four core research objectives. First, it seeks to
refine the conceptual definition of autocoups and introduce a novel
dataset amenable to large-N empirical analysis. Second, it aims to
identify the structural and institutional determinants of autocoups
through a systematic quantitative investigation. Third, it compares the
survival prospects of leaders who ascend to power via traditional coups
with those who extend their tenure through autocoups. Finally, it
assesses the divergent impacts of coups and autocoups on democratisation
trajectories and the resilience of political institutions.

By examining both coups and autocoups from 1950 to 2023, this thesis
addresses a significant gap in political science by developing and
applying a unified analytical framework that treats these events as
distinct yet interrelated forms of extra-constitutional power
transition. Through this lens, the study makes four principal
contributions to the literature on leadership dynamics, regime
stability, and institutional development.

\textbf{Conceptual clarification and empirical foundation for
autocoups:} This thesis advances conceptual clarity by situating
autocoups within the broader typology of irregular power transitions. It
offers a refined definition of autocoups---centred on the executive's
unilateral extension of tenure---and clearly differentiates them from
both executive aggrandisement and traditional military coups. Building
upon this conceptual framework, the study introduces an original dataset
of autocoups spanning the period from 1945 to 2023, documenting 83
incidents, of which 64 were successful. This dataset addresses a
long-standing empirical lacuna and provides a foundational basis for
systematic comparative analysis, thereby enabling future research into a
previously under-examined form of institutional disruption.

\textbf{First empirical analysis of determinants of autocoups:}
Utilising this newly compiled dataset, the thesis undertakes the first
empirical examination of the structural and institutional conditions
under which autocoups are likely to occur. The analysis finds that
leaders operating within power-concentrated systems---particularly
presidential democracies and personalist regimes---are significantly
more prone to extending their tenure through autocoups than those
operating in other regime types. These findings contribute to the
literature on the relationship between regime characteristics and
irregular power retention, highlighting the critical role of
institutional structures in shaping leaders' strategic decisions to
override term limits.

\textbf{Comparative analysis of leadership longevity between
coup-installed and autocoup leaders:} The research further contributes
to the study of leadership survival by comparing the tenure durations of
leaders who attain power through coups with those who extend it via
autocoups. Employing survival analysis on both the coup and autocoup
datasets, the study finds that, contrary to the assumption, the method
of power acquisition is not a significant predictor of leadership
longevity. However, it reaffirms that regime type plays a decisive role
in leader survival, irrespective of the mode of accession or retention.
The survival models indicate that leaders within military regimes
exhibit a higher hazard ratio of removal compared to those in
dominant-party regimes. These results underscore the influence of regime
structure on leadership durability and elite turnover.

\textbf{Comparative democratic implications of coups and autocoups:} The
thesis also investigates the differential impacts of coups and autocoups
on democratic development. Employing country-fixed effects regression
models and utilising Polity V scores as an indicator of democratic
quality, the analysis finds that autocoups are consistently associated
with gradual and sustained democratic erosion---both preceding and
following the event. In contrast, coups yield more heterogeneous
institutional outcomes. While they often result in immediate setbacks to
democratic norms, in certain instances they facilitate democratic
transitions, including shifts from autocracy to democracy. This
disaggregated analysis reveals the distinct trajectories and
institutional consequences engendered by different forms of irregular
power transitions.

\section{Policy implications}\label{policy-implications}

Although scholarly debate persists regarding the potential for coups to
inadvertently foster democratisation under certain conditions
(\citeproc{ref-thyne2014}{C. L. Thyne and Powell 2014};
\citeproc{ref-derpanopoulos2016}{Derpanopoulos et al. 2016};
\citeproc{ref-miller2016}{Miller 2016}), there exists a strong policy
consensus that coups constitute inherently illegitimate mechanisms of
political change. As violent disruptions of constitutional order, they
tend to inflict immediate institutional damage, precipitate instability,
and lead to unpredictable political trajectories. Accordingly, both
international and domestic policy responses have rightly prioritised
prevention---most notably through ``coup-proofing'' strategies designed
to insulate regimes from military intervention or elite defection
(\citeproc{ref-quinlivan1999}{Quinlivan 1999};
\citeproc{ref-pilster2012}{Pilster and Böhmelt 2012};
\citeproc{ref-powell}{Jonathan M. Powell, n.d.};
\citeproc{ref-albrecht2014}{Albrecht 2014a};
\citeproc{ref-carey2015}{Carey, Colaresi, and Mitchell 2015};
\citeproc{ref-brown2015}{C. S. Brown, Fariss, and McMahon 2015};
\citeproc{ref-sudduth2017}{Sudduth 2017}). However, as this thesis
demonstrates, such approaches have well-documented limitations
(\citeproc{ref-albrecht2014a}{Albrecht 2014b};
\citeproc{ref-reiter2020}{Reiter 2020}), and deeper structural power
dynamics within regimes are frequently more decisive in determining
vulnerability to both coups and autocoups.

The findings of this study yield several important policy implications,
particularly concerning institutional design, international responses,
and the monitoring of democratic backsliding. These implications will be
discussed in detail in the concluding chapter.

\section{Limitations and future
research}\label{limitations-and-future-research}

Whilst this study proposes a novel analytical framework for
understanding coups and autocoups, their effects on leadership survival,
and their broader institutional implications, several limitations
remain, indicating important directions for future research and
refinement.

A core challenge concerns the conceptual ambiguity that surrounds the
very definition of autocoups---particularly in borderline cases where
incumbents extend their authority through legal or quasi-legal
mechanisms. Future work might explore the normative and analytical
trade-offs involved in including such cases within the autocoup
category. Comparative analyses of `unconstitutional' versus
`extra-constitutional' extensions of executive tenure may assist in
clarifying whether these actions constitute variants of the same
phenomenon or represent analytically distinct processes. The case of
President Manuel Zelaya of Honduras in 2009, whose attempt to amend the
constitution to allow future re-election led to his removal by a
military coup (\citeproc{ref-muuxf1oz-portillo2019}{Muñoz-Portillo and
Treminio 2019}), illustrates the analytical complexity of identifying
autocoups and underscores the necessity of refining coding criteria and
interpretive clarity in future data collection efforts.

Given the long-term decline in traditional coups and the parallel rise
of autocoups, increased scholarly attention to the latter is imperative.
While this study focuses on tenure extension as the defining feature of
autocoups, broader forms of executive power expansion---whether within
or beyond the formal constitutional framework---merit systematic
investigation. To capture the full range of such practices, the
development of a dedicated dataset on executive power expansion
represents a crucial next step.

Moreover, the decreasing frequency of overt and dramatic regime
transitions since the early 2000s has coincided with a decline in
unambiguous shifts between democracy and autocracy. This trend
highlights the necessity of more sensitive instruments capable of
detecting incremental changes within regimes. Future empirical research
should focus on identifying and measuring such subtler transformations.

These limitations and promising avenues for future inquiry will be
explored in greater detail in the concluding chapter.

\section{Overview of the thesis}\label{overview-of-the-thesis}

This thesis examines the complex power dynamics underlying coups and
autocoups, with a particular focus on their consequences for leadership
survival and the democratisation or authoritarian transformation of
political regimes. It develops a unified analytical framework to study
these phenomena as distinct yet interconnected forms of irregular power
transition. Each chapter contributes to this overarching inquiry by
offering conceptual clarifications, empirical innovations, and
comparative insights.

\subsection*{Chapter 2: Autocoups: Conceptual Clarification and Dataset
Introduction}\label{chapter-2-autocoups-conceptual-clarification-and-dataset-introduction}
\addcontentsline{toc}{subsection}{Chapter 2: Autocoups: Conceptual
Clarification and Dataset Introduction}

Despite the increasing incidence of autocoups---particularly in the
post-Cold War era---their systematic study remains underdeveloped.
Existing scholarship suffers from conceptual fragmentation,
characterised by a proliferation of overlapping and inconsistently
defined terms (`self-coup', autogolpe, `executive aggrandisement', etc.)
(\citeproc{ref-marsteintredet2019}{Marsteintredet and Malamud 2019};
\citeproc{ref-baturo2022}{Baturo and Tolstrup 2022}). This conceptual
ambiguity complicates empirical analysis, as many datasets fail to
distinguish between tenure extension and other forms of executive power
consolidation---an essential distinction for this study. As a result,
methodological progress has been hindered, with most existing research
relying on qualitative case studies (\citeproc{ref-cameron1998}{Maxwell
A. Cameron 1998b}; \citeproc{ref-antonio2021}{Antonio 2021};
\citeproc{ref-pion-berlin2022}{Pion-Berlin, Bruneau, and Goetze 2022}),
rather than large-N analyses.

This chapter addresses these limitations by proposing a more precise and
theoretically grounded definition of the autocoup. It contends that
autocoups should be defined as attempts by incumbents to extend their
constitutionally mandated terms of office. By centring the definition on
tenure extension, the concept excludes broader forms of executive
aggrandisement that occur within existing constitutional time-frames,
and aligns autocoups conceptually with classic coups, both of which
involve disruptions to constitutionally prescribed leadership
succession. Accordingly, this study defines \textbf{an autocoup as the
extension of an incumbent leader's tenure beyond its original
constitutional limit, achieved through extra-constitutional means}.

Based on this refined definition, the chapter presents a significant
empirical contribution: a newly compiled global dataset of autocoups
spanning the period 1945 to 2023, identifying 83 distinct events, of
which 64 were successful. This dataset facilitates systematic
quantitative analysis and opens new avenues for comparative research on
irregular retention of power.

\subsection*{Chapter 3: Power Dynamics and Autocoup
Attempts}\label{chapter-3-power-dynamics-and-autocoup-attempts}
\addcontentsline{toc}{subsection}{Chapter 3: Power Dynamics and Autocoup
Attempts}

Due to long-standing conceptual and empirical constraints, existing
discussions of autocoups have relied largely on case-based approaches
(\citeproc{ref-baturo}{Baturo and Elgie, n.d.};
\citeproc{ref-marsteintredet2019}{Marsteintredet and Malamud 2019};
\citeproc{ref-baturo2022}{Baturo and Tolstrup 2022}). The dataset
introduced in Chapter 2 enables, for the first time, a large-N analysis
of the structural conditions underpinning autocoup attempts.

Drawing upon insights from the coup literature, this chapter
investigates a range of potential predictors---including economic
performance, succession rules, military influence, protest activity, and
media freedom. While these variables have been analysed in the context
of traditional coups, they often fail to account for persistent
cross-regime variation or the limited efficacy of so-called
`coup-proofing' strategies (\citeproc{ref-albrecht2014a}{Albrecht
2014b}; \citeproc{ref-reiter2020}{Reiter 2020}). Moreover, many studies
adopt overly simplistic regime typologies (e.g.~democracy versus
autocracy, or civilian versus military), thereby obscuring important
variation within regime types (\citeproc{ref-hiroi2013}{Hiroi and Omori
2013}; \citeproc{ref-schiel2019}{Schiel 2019}).

This chapter advances the argument that autocoup risk is shaped
fundamentally by the structural balance of power embedded in a regime's
founding configuration. Specifically, the likelihood of an autocoup is
determined by the equilibrium between incumbents and potential
institutional challengers---a balance largely set at regime inception
(\citeproc{ref-geddes2014}{Geddes, Wright, and Frantz 2014}). To test
this proposition, regime typologies are employed as proxies for internal
power structures.

Using both a standard logit model and a bias-reduced logit model
(Firth's penalised maximum likelihood estimation), the analysis reveals
that presidential democracies and personalist regimes are significantly
more prone to autocoup attempts than dominant-party regimes, when other
variables are held constant. Leaders in dominant-party and military
regimes, by contrast, do not significantly differ in their likelihood of
attempting an autocoup. These findings underscore the centrality of
regime type in shaping elite incentives for irregular tenure extension.

\subsection*{Chapter 4: Power Acquisition and Leadership Survival: A
Comparative Analysis of Coup-installed and Autocoup
Leaders}\label{chapter-4-power-acquisition-and-leadership-survival-a-comparative-analysis-of-coup-installed-and-autocoup-leaders}
\addcontentsline{toc}{subsection}{Chapter 4: Power Acquisition and
Leadership Survival: A Comparative Analysis of Coup-installed and
Autocoup Leaders}

Although a substantial body of literature has explored the tenure
survival of leaders who come to power through coups
(\citeproc{ref-gandhi2007}{Gandhi and Przeworski 2007};
\citeproc{ref-sudduth2017}{Sudduth 2017};
\citeproc{ref-easton2018}{Easton and Siverson 2018}), the absence of
comparable data on autocoups has long precluded systematic comparisons
between coup-installed leaders and those who prolong their rule via
autocoups. This chapter addresses that gap by offering the first
comparative survival analysis of these two categories within a unified
theoretical framework.

It argues that coup leaders typically face heightened legitimacy
deficits, political uncertainty, and institutional instability, whereas
autocoup leaders benefit from institutional continuity while
simultaneously removing key constraints. These distinct conditions shape
divergent pathways to political consolidation.

Surprisingly, the time-dependent Cox model shows no statistically
significant difference in survival risk between coup-installed and
autocoup leaders once relevant covariates---especially regime type---are
taken into account. Rather, regime characteristics exert a decisive
influence on leadership tenure: leaders in military and transitional
regimes face significantly higher risks of removal than those in
dominant-party regimes.

\subsection*{Chapter 5: Coups, autocoups, and
democracy}\label{chapter-5-coups-autocoups-and-democracy}
\addcontentsline{toc}{subsection}{Chapter 5: Coups, autocoups, and
democracy}

While the impact of coups on democratisation has received considerable
scholarly attention (\citeproc{ref-clayton2000}{Clayton and Onwumechili
2000}; \citeproc{ref-powell2014a}{Jonathan M. Powell 2014};
\citeproc{ref-thyne2020}{C. Thyne and Hitch 2020}), the consequences of
autocoups remain understudied due to the historical absence of relevant
data. This chapter addresses this lacuna through a quantitative analysis
of how coups and autocoups affect democratic institutions.

Whereas coups may produce leadership turnover or even regime change,
autocoups typically involve incumbents dismantling institutional
constraints without altering the core ruling coalition. Consequently,
their effects are best assessed not by regime-type transitions, but by
shifts in continuous measures of democratic quality, such as Polity V
scores.

Two key claims are advanced. First, leaders often begin to erode
institutional checks in anticipation of an autocoup, resulting in
declines in Polity scores prior to the event itself. Second, while coups
exhibit mixed outcomes---sometimes enabling democratisation---autocoups
almost invariably result in democratic erosion or deeper authoritarian
consolidation.

Empirical analysis using a country-fixed effects model confirms that
autocoups are associated with consistent declines in democratic quality
both before and after the event. By contrast, coups tend to cause an
immediate drop in Polity scores, although in some cases democratic
recovery follows. These findings highlight the uniquely insidious nature
of autocoups, which often proceed incrementally and under a legalistic
façade.

\subsection*{Chapter 6: Conclusion and future research
directions}\label{chapter-6-conclusion-and-future-research-directions}
\addcontentsline{toc}{subsection}{Chapter 6: Conclusion and future
research directions}

The concluding chapter synthesises the findings of the preceding
chapters, drawing attention to the structural, strategic, and
institutional dynamics that underpin irregular leadership transitions.
It contends that coups and autocoups are not simply disruptive events,
but strategic tools employed by elites to recalibrate or entrench
political authority. Their institutional legacies diverge: while coups
often destabilise regimes, autocoups typically consolidate autocratic
rule.

This chapter outlines the broader implications of these findings for
understanding the resilience of autocracy, the vulnerability of
democratic institutions, and the strategic calculus of political
leaders. It also proposes several directions for future research.

First, the broader phenomenon of executive power expansion---whether or
not it involves tenure extension---warrants systematic investigation.
Given its increasing frequency and corrosive institutional effects, the
development of a dedicated dataset on power expansions is strongly
recommended.

Second, as full regime transitions become less common, future research
should focus on detecting subtle, incremental changes in political
quality. Fine-grained indicators---such as modest shifts in Polity
scores---will be essential for monitoring democratic backsliding and
institutional recovery in hybrid or semi-authoritarian regimes.

\chapter{Autocoups: Conceptual Clarification and Dataset
Introduction}\label{sec-chapter3}

\section*{Abstract}\label{abstract-1}
\addcontentsline{toc}{section}{Abstract}

This chapter presents a refined conceptualisation of autocoups, defined
as instances in which incumbent leaders extend their constitutionally
mandated tenure through extra-constitutional means, typically by
circumventing or violating term limits. By critically reviewing and
synthesising overlapping terms---such as self-coup, autogolpe, and
executive takeover---the chapter delineates the conceptual boundaries of
the phenomenon, identifying tenure extension as its defining
characteristic. In distinguishing autocoups from broader and more
ambiguous forms of executive aggrandisement, it advances a more
analytically precise framework for studying irregular power extensions.
Building on this conceptual foundation, the chapter introduces a novel
global dataset of autocoup events from 1945 to 2023, identifying 83
distinct cases, of which 64 were successful. This empirical contribution
enables systematic, large-N analysis of an increasingly salient mode of
authoritarian consolidation.

\emph{\textbf{Keywords}: Autocoups, Coups, Irregular Power Transitions,
Leadership Tenure, Dataset}

\section{Introduction}\label{introduction-1}

The stability and resilience of political systems rest fundamentally on
the orderly transfer of power. When leadership succession takes place
within established constitutional frameworks, it reinforces the
legitimacy and durability of governing institutions. Conversely, the
breakdown of these norms and mechanisms often precipitates political
violence, institutional erosion, and prolonged instability.

While many leadership transitions occur without disruption, a
considerable proportion do not. In particular, authoritarian regimes and
fragile democracies frequently experience two principal forms of
irregular leadership outcomes: the premature removal of incumbents and
the extension of power beyond constitutional limits.

The former---forced removals of leaders prior to the completion of their
terms---has been extensively examined under the broader category of
irregular leadership transitions. These events have profound
consequences for regime stability, democratic legitimacy, and
institutional development. Accordingly, understanding their causes and
implications remains a core concern within political science.

The extant literature identifies a range of precipitating factors,
including civil war (\citeproc{ref-kokkonen2019}{Kokkonen and Sundell
2019}), international conflict (\citeproc{ref-demesquita1995}{Mesquita
and Siverson 1995}), ethnic cleavages
(\citeproc{ref-londregan1995}{Londregan, Bienen, and Walle 1995}), poor
economic performance (\citeproc{ref-miller2012}{Miller 2012};
\citeproc{ref-krishnarajan2019}{Krishnarajan 2019}), and natural
disasters (\citeproc{ref-quirozflores2012}{Quiroz Flores and Smith
2012}). Among these, however, coups d'état---typically defined as
illegal and overt attempts by the military or state elites to depose a
sitting executive (\citeproc{ref-powell2011}{Powell and Thyne
2011})---stand out as the most frequent and consequential source of
leadership change. In autocratic contexts, they account for
approximately one-third of all leader exits, surpassing even regular
transitions (\citeproc{ref-frantz2016}{Frantz and Stein 2016}), while
roughly two-thirds of non-constitutional removals in dictatorships are
attributable to coups (\citeproc{ref-svolik2009}{Svolik 2009}).
Unsurprisingly, coups have attracted considerable scholarly attention.
Researchers have explored their structural determinants, proximate
triggers, aftermath, and effects on democratic consolidation and
economic development (\citeproc{ref-thyne2019}{Thyne and Powell 2019}).

Yet this focus on traditional coups risks overlooking a distinct and
increasingly salient form of irregular transition: the autocoup. In this
chapter, an autocoup is defined as an instance in which an incumbent
leader extends their tenure by subverting or bypassing constitutional
term limits through extra-constitutional means. Despite their rising
incidence---particularly since the end of the Cold War---autocoups
remain under-theorised and under-examined. Conceptual fragmentation has
impeded progress, with a proliferation of overlapping and inconsistently
applied terms such as self-coup, autogolpe, and executive aggrandisement
(\citeproc{ref-marsteintredet2019}{Marsteintredet and Malamud 2019};
\citeproc{ref-baturo2022}{Baturo and Tolstrup 2022}). This lack of
definitional precision complicates data collection and comparative
analysis. Existing datasets often conflate tenure extensions with
broader forms of executive power consolidation, thereby failing to
isolate the specific mechanisms this study seeks to analyse
(\citeproc{ref-baturo2022}{Baturo and Tolstrup 2022}). Consequently,
scholarship has tended to rely on qualitative case studies
(\citeproc{ref-cameron1998}{Maxwell A. Cameron 1998b};
\citeproc{ref-antonio2021}{Antonio 2021};
\citeproc{ref-pion-berlin2022}{Pion-Berlin, Bruneau, and Goetze 2022}),
limiting the field's capacity for broader generalisation.

This chapter contends that these conceptual and empirical limitations
obscure a crucial dimension of contemporary politics. It thus proposes a
unified analytical framework for examining coups and autocoups as
distinct yet comparable strategies for undermining constitutional norms
governing leadership succession. This comparative approach is warranted
on three grounds.

First, both coups and autocoups constitute fundamental breaches of
constitutional order, with significant implications for democratic
resilience, political legitimacy, and institutional integrity. Analysing
them in tandem facilitates a systematic examination of how different
forms of irregular power transition shape trajectories of political
development and democratisation.

Second, while both disrupt established norms of succession, they operate
in opposite directions relative to the incumbent: coups terminate
leadership prematurely, whereas autocoups extend it beyond its
constitutionally mandated limit. This contrast offers a valuable lens
through which to explore the mechanisms of political survival and
authoritarian consolidation.

Third, a comparative framework helps to illuminate pressing contemporary
puzzles. For instance, how can the marked decline in coup frequency
since the 1990s (\citeproc{ref-bermeo2016}{Bermeo 2016}) be reconciled
with the sustained erosion of democratic governance---now in its
eighteenth consecutive year
(\citeproc{ref-freedomhouse2024freedom}{Freedom House 2024})? By
incorporating autocoups into the analytical schema, this study
highlights the growing significance of incremental, procedural
subversions of democracy, often orchestrated from within the existing
legal and institutional architecture.

To address these gaps, the chapter makes two principal contributions.
First, it offers conceptual clarification by redefining autocoups as a
subtype of irregular leadership transition centred specifically on
extra-constitutional tenure extension. This refined definition
distinguishes autocoups from broader, more diffuse forms of executive
aggrandisement. Second, it introduces a novel global dataset of autocoup
events from 1945 to 2023, compiled in accordance with this
reconceptualised framework, thereby enabling the first systematic
large-N analysis of the phenomenon.

The remainder of the chapter is structured as follows. Section 2 reviews
existing definitions related to power extension and executive
aggrandisement, culminating in a revised conceptualisation of autocoups.
Section 3 introduces the new dataset, detailing its scope, coding
criteria, and methodological foundations. Section 4 presents an initial
analysis through descriptive statistics and illustrative case studies.
The conclusion synthesises the chapter's key contributions and outlines
directions for future research.

\section{literature review and clarification of
definitions}\label{literature-review-and-clarification-of-definitions}

A significant limitation in the study of irregular leadership
transitions lies in the insufficient integration of research on
conventional coups and autocoups. While both constitute critical
mechanisms of extra-constitutional power transfer, they have typically
been examined in isolation, with limited attention paid to their
conceptual and empirical intersections.

This disjunction primarily arises from two factors: the historical
under-recognition of autocoups as a distinct subtype of irregular
transition, and the enduring conceptual ambiguity surrounding their
definition. Whereas classic coups are generally characterised by the
abrupt removal of incumbents, autocoups typically involve incumbent-led
efforts to retain or expand power by circumventing constitutional
constraints. However, the inconsistent usage of overlapping terms---such
as `self-coup', autogolpe, and `executive aggrandisement'---has further
muddied these distinctions.

Clarifying the definition of autocoups is thus a necessary step towards
constructing a comparative framework capable of capturing the full
spectrum of irregular power transitions. This section undertakes that
task by distinguishing autocoups from broader forms of executive power
consolidation, and by conceptually aligning them with traditional coups
through their shared transgression of constitutional norms.

\subsection*{Terminology}\label{terminology}
\addcontentsline{toc}{subsection}{Terminology}

Studies of autocoups employ a wide range of terms to describe the
extension of power or tenure by incumbent leaders. The most commonly
used is `self-coup', or autogolpe in Spanish
(\citeproc{ref-przeworski2000}{Przeworski et al. 2000};
\citeproc{ref-cameron1998a}{Maxwell A. Cameron 1998a};
\citeproc{ref-bermeo2016}{Bermeo 2016}; \citeproc{ref-helmke2017}{Helmke
2017}; \citeproc{ref-marsteintredet2019}{Marsteintredet and Malamud
2019}). This term gained scholarly prominence following the actions of
Peruvian President Alberto Fujimori in 1992, who dissolved Congress,
temporarily suspended the constitution, and ruled by decree
(\citeproc{ref-mauceri1995}{Mauceri 1995};
\citeproc{ref-cameron1998}{Maxwell A. Cameron 1998b}). However, as
Marsteintredet and Malamud (\citeproc{ref-marsteintredet2019}{2019})
rightly notes, `self-coup' is a potentially misleading label, as it
implies that the leader acts against themselves, whereas such moves are
typically aimed at other state institutions or constitutional
constraints.

A second category of terminology includes expressions such as
`presidential coup', `executive coup', `constitutional coup', `electoral
coup', `judicial coup', `slow-motion coup', `soft coup', and
`parliamentary coup' (\citeproc{ref-marsteintredet2019}{Marsteintredet
and Malamud 2019}). While these descriptors may offer insights into
specific mechanisms or contexts, their proliferation often leads to
conceptual confusion. Many centre on the method of power acquisition but
fail to consistently identify the perpetrator. Moreover, such
mechanisms---judicial rulings, legislative manoeuvres, or administrative
decrees---may be employed either by or against executive actors, further
complicating classification.

A third group of terms includes phrases such as `incumbent takeover',
`executive takeover', and `overstay'. For instance, incumbent takeover
refers to ``an event perpetuated by a ruling executive that
significantly reduces the formal and/or informal constraints on his/her
power'' (\citeproc{ref-baturo2022}{Baturo and Tolstrup 2022, 374}),
drawing on earlier work by Svolik (\citeproc{ref-svolik2014}{2014}).
Similarly, overstay denotes ``staying longer than the maximum term as it
stood when the candidate originally came into office''
(\citeproc{ref-ginsburg2011evasion}{Ginsburg, Melton, and Elkins 2011,
1844}). These terms help to clarify the identity of the actor (the
incumbent) and the nature of the action (power consolidation or term
extension), yet they often fail to convey the illegality or
unconstitutionality of such actions. In contrast to `coup', which
inherently implies an unlawful seizure of power, labels such as
`takeover' or `overstay' may inadvertently understate the normative
gravity of the events they describe.

Given that many existing terms tend to prioritise procedural mechanisms
over normative considerations, or even conflate legal and extra-legal
practices, this study posits `autocoup' as the most precise and
analytically coherent term. This term offers several key advantages.

\textbf{Definitional clarity and addressing the core essence:}
`Autocoup' precisely denotes an incumbent leader's extension of their
political tenure through extra-constitutional means. This definition
clearly distinguishes it from conventional coups, typically initiated by
external actors such as the military, and from other, more diffuse or
ill-defined instances of executive aggrandisement.

\textbf{Emphasis on severity and normative implications:} The suffix
`-coup' powerfully signifies the grave breach of constitutional order
inherent in such actions. In both academic research and policy
assessment contexts, the level of disruption and impact is no less than,
and may even exceed, that of conventional coups, thereby imbuing the
term with appropriate normative and critical weight.

\textbf{Accurate identification of the perpetrator:} The prefix `auto-'
directly identifies the incumbent leader as the instigator of such
events. This stands in stark contrast to conventional coups, which are
typically orchestrated by actors external to the incumbent's immediate
circle, such as the military or opposition factions, thereby
facilitating precise attribution.

\textbf{Promotion of conceptual coherence and comparative analysis:}
`Autocoup' shares an etymological root with `coup', ensuring an
intrinsic conceptual link and logical consistency. This enables
systematic comparative analysis of different yet related forms of
irregular leadership transition within a unified analytical framework,
thereby deepening the understanding of these phenomena.

In summary, the term `autocoup' not only pinpoints with precision both
the actor and the act itself but also clearly reveals the illegitimacy
and gravity of the behaviour, alongside its intrinsic theoretical
connections to conventional coups. It therefore stands as the most
accurate and analytically potent conceptual tool for capturing and
analysing such political phenomena, aligning perfectly with the unified
analytical framework this study seeks to establish.

\subsection*{Definition}\label{sec-definition}
\addcontentsline{toc}{subsection}{Definition}

While the use of precise terminology is undoubtedly important, a further
significant issue in previous definitions of autocoups lies in the
identification of their primary characteristic: is the central feature
the expansion of power, the extension of tenure, or a combination of
both? This question arises from the marked ambiguity surrounding
existing definitions of autocoups and related concepts.

Within political science, the notions of power expansion and tenure
extension frequently overlap or are applied ambiguously, thereby
contributing to conceptual confusion. To promote greater clarity, it is
essential to delineate these two frameworks more rigorously. Power
expansion refers to the process by which an incumbent accrues authority
beyond their original constitutional remit---typically through
centralisation, the weakening of institutional checks and balances, or
encroachments upon other branches of the state, such as the legislature
or judiciary. Tenure extension, by contrast, concerns efforts by a
leader to remain in office beyond the term originally prescribed, often
via constitutional amendments, the postponement or manipulation of
elections, or other mechanisms intended to bypass term limits.

Many existing definitions of autocoups conflate these dynamics or
disproportionately emphasise power expansion. For instance, Maxwell A.
Cameron (\citeproc{ref-cameron1998a}{1998a}) defines a self-coup as ``a
temporary suspension of the constitution and dissolution of congress by
the executive, who rules by decree until new legislative elections and a
referendum can be held to ratify a political system with broader
executive power'' (p.~220). Yet the concept of ``broader executive
power'' is inherently vague and open to contestation. Similarly, the
term incumbent takeover, defined as ``an event perpetuated by a ruling
executive that significantly reduces the formal and/or informal
constraints on his/her power'' (\citeproc{ref-baturo2022}{Baturo and
Tolstrup 2022, 374}), builds on earlier work by Svolik
(\citeproc{ref-svolik2014}{2014}) and centres on power expansion as
well. However, the dataset employing this definition encompasses both
power expansion and tenure extension. By contrast, the term overstay is
clearly defined as ``staying longer than the maximum term as it stood
when the candidate originally came into office''
(\citeproc{ref-ginsburg2011evasion}{Ginsburg, Melton, and Elkins 2011,
1844}), thereby focusing specifically on tenure extension.

This thesis contends that tenure extension ought to be regarded as the
primary and defining characteristic of an autocoup, for several reasons.
First, this focus aligns autocoups conceptually with traditional coups.
A classic coup is typically characterised by the forcible and premature
removal of a sitting executive; it does not necessarily entail a
reduction in the leader's powers, but rather a disruption of their
tenure. By the same logic, an autocoup should be defined by the
prolongation of tenure, not solely by the expansion of executive
authority. An incumbent may experience a diminution of power while
remaining in office---such an instance would not be coded as a coup.
Similarly, a leader who consolidates authority without exceeding term
limits would fall under the category of executive aggrandisement, but
not that of an autocoup.

Second, in practice, power expansion often functions as a strategic
means to enable tenure extension. The widely cited case of President
Alberto Fujimori in Peru exemplifies this dynamic. Although his 1992
actions involved the suspension of the constitution and the dissolution
of Congress, the ultimate objective was to ensure continued rule. The
1993 Constitution permitted him to seek a second term, which he won in
1995. Subsequently, a law of ``authentic interpretation'' passed by his
congressional allies enabled him to run again in 2000---a move steeped
in controversy. Although he secured re-election, his regime collapsed in
2000 amidst corruption and human rights scandals, prompting his flight
to Japan (\citeproc{ref-ezrow2019}{Ezrow 2019}). In this light, it is
illogical for incumbents to consolidate power unless they intend to
overstay in office; such actions merely strengthen their successors,
whose interests may diverge from their own.

Third, the measurement of power expansion presents greater
methodological challenges than the identification of tenure extension.
For example, Maxwell A. Cameron (\citeproc{ref-cameron1998a}{1998a})
defines a self-coup as involving both constitutional suspension and
congressional dissolution. Yet it remains unclear whether either act
alone constitutes an autocoup, whether both are required, or whether
they should be treated as distinct events. While the complexity of
measurement ought not to preclude the consideration of power expansion,
a clear point of departure is required. This study therefore designates
tenure extension as the definitional core of autocoups, leaving the
broader discussion of power expansion (and its inverse, power
contraction) to future research.

Based on these considerations, this study defines an autocoup as
\textbf{the extension of an incumbent leader's tenure in office beyond
the originally mandated limit, achieved through extra-constitutional
means.}

This definition places tenure extension at the centre of the concept,
while acknowledging that power expansion may coexist. First, the term
incumbent leader refers to the de facto national leader, irrespective of
their formal title. For instance, although Vladimir Putin formally
stepped down as President of Russia in 2008 and assumed the premiership,
effective political power remained in his hands. During this period, the
presidency---held by Dmitry Medvedev---functioned largely as a symbolic
office under Putin's continued control
(\citeproc{ref-chaisty2019}{Chaisty 2019}). To ensure consistency and
minimise arbitrariness, this study employs the Archigos dataset
(\citeproc{ref-goemans2009}{Goemans, Gleditsch, and Chiozza 2009}) to
determine whether an incumbent has effectively remained in power.

Second, although tenure extension is the definitional cornerstone, this
framework does not exclude simultaneous power expansion. Both may occur
in tandem, but the decisive criterion remains the act of exceeding one's
original time in office. In the Fujimori case, for example, the 1992
actions were not coded as an autocoup until the adoption of the 1993
constitutional amendment enabling his re-election.

Third, autocoups may be executed through both legal and illegal means.
For instance, Chadian President François Tombalbaye postponed general
elections until 1969 after coming to power in 1960. Similarly, Angolan
President José Eduardo dos Santos suspended elections during much of his
nearly four-decade rule (\citeproc{ref-baturo}{Baturo and Elgie, n.d.}).
These represent clear violations of constitutional norms. Other
instances---such as Putin's 2008 manoeuvre---may not be overtly illegal
but nonetheless undermine the constitutional spirit intended to limit
consecutive terms. Consequently, this definition emphasises the
functional illegitimacy of such actions, regardless of their formal
legality, particularly where the incumbent is the direct and principal
beneficiary.

Finally, an incumbent who seeks re-election in accordance with the
existing constitution is not engaging in an autocoup. However, should
they subsequently refuse to concede defeat and remain in power beyond
their lawful mandate, such conduct would indeed constitute an autocoup.

By clarifying these definitional boundaries, this chapter establishes
the conceptual foundation for the autocoup dataset introduced in the
following section.

\section{Introduction to the autocoup
dataset}\label{introduction-to-the-autocoup-dataset}

\subsection*{Defining the scope}\label{defining-the-scope}
\addcontentsline{toc}{subsection}{Defining the scope}

Classifying political events as autocoups often necessitates addressing
ambiguous or borderline cases. To ensure consistency and minimise
interpretive uncertainty, this study adopts a coding strategy grounded
in the definition articulated in the preceding section. Specifically,
only those instances in which incumbent leaders extend their originally
mandated term in office are coded as autocoups. Cases involving power
consolidation in the absence of tenure extension are excluded from the
dataset.

The temporal scope of the dataset spans the period from 1945 to the end
of 2023, reflecting the most recent data available at the time of
compilation. The geographical scope is global, encompassing leaders from
all countries and regions.

\subsection*{Classifying autocoups}\label{classifying-autocoups}
\addcontentsline{toc}{subsection}{Classifying autocoups}

In categorising autocoups, this study prioritises the methods employed
by incumbents, while outcomes constitute a secondary classificatory
dimension. Additional features are recorded where relevant information
is available.

\subsubsection*{Evasion of term limits}\label{evasion-of-term-limits}
\addcontentsline{toc}{subsubsection}{Evasion of term limits}

One of the most prevalent tactics in autocoups is the evasion of term
limits. Incumbents deploy ostensibly legal mechanisms to prolong their
hold on power, primarily through the manipulation of constitutional
provisions. Such manoeuvres may include pressuring legislatures or
courts to reinterpret term limits, amending the constitution to allow
extended terms, or replacing the constitution entirely. In some
instances, referendums are employed to confer a veneer of democratic
legitimacy. These extensions may range from the addition of a single
term to indefinite tenure.

\textbf{Changing the length of a term:} Incumbents may increase the
duration of a single term (e.g., from four to six years) without
altering the number of terms permitted. Examples include President David
Dacko (Central African Republic, 1962), President Grégoire Kayibanda
(Rwanda, 1973), and President Augusto Pinochet (Chile, 1988).

\textbf{Enabling re-election:} This involves modifying constitutional or
legal frameworks to permit re-election where it was previously barred.
For instance, President Carlos Menem of Argentina amended the
constitution in 1993 to allow himself to seek re-election, thereby
extending his tenure.

\textbf{Abolishing term limits:} President Paul Biya of Cameroon
successfully removed presidential term limits in 2008, thereby enabling
indefinite re-election.

\textbf{Declaring leadership for life:} This approach retains the
semblance of electoral competition, albeit often through manipulated or
uncontested elections. President Sukarno of Indonesia attempted to
declare himself president for life in 1963, although this effort
ultimately failed.

These strategies are frequently deployed in combination. For example,
President François Duvalier of Haiti first amended the constitution in
1961 to permit immediate re-election and subsequently declared himself
president for life in 1964.

\subsubsection*{Electoral manipulation and
rigging}\label{electoral-manipulation-and-rigging}
\addcontentsline{toc}{subsubsection}{Electoral manipulation and rigging}

The second most frequently observed strategy in autocoups involves the
manipulation of electoral processes to ensure the incumbent remains in
office.

\textbf{Delaying or cancelling elections:} The postponement of scheduled
elections without legitimate justification is a recurrent tactic.
President François Tombalbaye of Chad delayed general elections until
1969, having come to power in 1960. Similarly, President José Eduardo
dos Santos of Angola suspended elections throughout his tenure from 1979
to 2017.

\textbf{Rejecting unfavourable electoral outcomes:} Incumbents may
refuse to concede defeat and attempt to remain in office by
unconstitutional means. A prominent example is President Donald Trump of
the United States, who declined to accept the results of the 2020
presidential election and sought to overturn them.

\textbf{Electoral rigging:} Securing implausibly high vote shares is a
key indicator of electoral manipulation. This study codes elections in
which incumbents receive over \(90\%\) of the vote as indicative of an
autocoup. President Teodoro Obiang of Equatorial Guinea has consistently
achieved over \(95\%\) in multiparty elections since 1996.

\textbf{Exclusion of opposition:} Preventing opposition parties or
candidates from contesting elections---thereby converting them into de
facto uncontested contests---is considered a clear indicator of an
autocoup.

\subsubsection*{Installation of a
figurehead}\label{installation-of-a-figurehead}
\addcontentsline{toc}{subsubsection}{Installation of a figurehead}

Some incumbents seek to circumvent term limits by installing a trusted
proxy or figurehead, thereby retaining de facto control over state
affairs while relinquishing formal office.

A paradigmatic example is the 2008 presidential transition in Russia.
Confronted with constitutional term limits, President Vladimir Putin
endorsed Dmitry Medvedev as his successor, who was duly elected.
Medvedev then appointed Putin as Prime Minister. Despite the formal
shift in roles, most analysts agree that Putin retained substantial
influence, effectively rendering Medvedev a figurehead.

It is important that the identification of such cases be grounded in
objective criteria to avoid arbitrary classification. Accordingly, this
study relies on the Archigos dataset
(\citeproc{ref-goemans2009}{Goemans, Gleditsch, and Chiozza 2009}) to
determine leadership status. If a former officeholder is not recorded in
Archigos as the country's leader---despite wielding informal
power---they are not coded as engaging in a figurehead-style autocoup
within this dataset.

\subsubsection*{Reassignment of supreme
authority}\label{reassignment-of-supreme-authority}
\addcontentsline{toc}{subsubsection}{Reassignment of supreme authority}

This strategy entails restructuring the constitutional or legal
framework to create a new, more powerful office, which the incumbent
subsequently assumes after formally leaving their original post.

In 2017, Turkish Prime Minister Recep Tayyip Erdoğan orchestrated a
constitutional referendum that transformed Turkey from a parliamentary
to a presidential system. The newly empowered presidency carried
significantly enhanced executive authority. Erdoğan then ran for, and
won, the redefined presidency, thereby maintaining control under a
revised institutional arrangement.

\subsubsection*{One-time extension
arrangements}\label{one-time-extension-arrangements}
\addcontentsline{toc}{subsubsection}{One-time extension arrangements}

In certain cases, bespoke arrangements are enacted to extend an
incumbent's tenure without altering the broader constitutional
framework. These arrangements are explicitly tailored to the current
officeholder, with institutional rules on tenure or term limits intended
to resume their standard application for future leaders. For example, in
2004, Lebanon extended President Émile Lahoud's term by three years
through a one-off legal provision applying solely to his incumbency.

\subsection*{Data coding}\label{data-coding}
\addcontentsline{toc}{subsection}{Data coding}

The autocoup dataset is constructed on the basis of established datasets
and scholarly literature, thereby ensuring both reliability and
comprehensiveness. The principal sources employed for coding are listed
in Table Table~\ref{tbl-source}.

The Archigos dataset (\citeproc{ref-goemans2009}{Goemans, Gleditsch, and
Chiozza 2009}) and the Political Leaders' Affiliation Database (PLAD)
(\citeproc{ref-bomprezzi2024wedded}{Bomprezzi et al. 2024}) offer
detailed records of national leaders from 1875 to 2023. Although the
temporal focus of this study is limited to events occurring from 1945
onwards, these datasets are essential for identifying de facto leaders
and distinguishing them from nominal heads of state.

The Incumbent Takeover dataset (\citeproc{ref-baturo2022}{Baturo and
Tolstrup 2022}), which synthesises information from eleven separate
sources, provides a broad inventory of cases wherein executive actors
significantly curtailed institutional constraints on their authority. As
this dataset encompasses both power consolidation and tenure extension
cases, cross-referencing with Archigos
(\citeproc{ref-goemans2009}{Goemans, Gleditsch, and Chiozza 2009}) and
PLAD (\citeproc{ref-bomprezzi2024wedded}{Bomprezzi et al. 2024}) was
necessary to determine whether individual cases satisfied the
definitional criteria for an autocoup.

\begin{table}

\caption{\label{tbl-source}Main Data Sources for Coding the Autocoup
Dataset}

\centering{

\fontsize{12.0pt}{14.4pt}\selectfont
\begin{tabular*}{1\linewidth}{@{\extracolsep{\fill}}llrr}
\toprule
Dataset & Authors & Coverage & Obervations \\ 
\midrule\addlinespace[2.5pt]
Archigos & Goemans et al (2009) & 1875-2015 & 3409 \\ 
PLAD & Bomprezzi et al. (2024) & 1989-2023 & 1334 \\ 
Incumbent Takeover & Baturo and Tolstrup (2022) & 1913-2019 & 279 \\ 
\bottomrule
\end{tabular*}

}

\end{table}%

In total, 83 events were identified and coded as autocoups. Of these, 50
correspond to entries within the Incumbent Takeover dataset, while the
remaining 33 were newly identified and coded by the author through
cross-verification with supplementary materials, including Archigos
(\citeproc{ref-goemans2009}{Goemans, Gleditsch, and Chiozza 2009}), PLAD
(\citeproc{ref-bomprezzi2024wedded}{Bomprezzi et al. 2024}), and
contemporary news sources.

Although a majority of cases originate from the Incumbent Takeover
dataset, the present study does not constitute a replication of that
work. Of the 279 cases catalogued in Incumbent Takeover, 229 were
excluded from the current analysis on the grounds that they entailed
power consolidation without any accompanying attempt to extend the
leader's tenure. Such instances lie beyond the definitional scope of
autocoups as operationalised in this thesis. This conceptual refinement
constitutes the principal point of departure from the Incumbent Takeover
framework.

The final dataset comprises 14 structured variables, in addition to a
free-text field for supplementary notes. The variables are as follows:

\begin{itemize}
\item
  \textbf{Country identification:} Country code (ccode) and country name
  (country), following the standards of the Correlates of War project
  (\citeproc{ref-stinnett2002}{Stinnett et al. 2002}).
\item
  \textbf{Leader information:} Name of the de facto leader
  (leader\_name), coded in accordance with the conventions employed in
  the Archigos (\citeproc{ref-goemans2009}{Goemans, Gleditsch, and
  Chiozza 2009}) and PLAD (\citeproc{ref-bomprezzi2024wedded}{Bomprezzi
  et al. 2024}) datasets.
\item
  \textbf{Timeline variables:} Date the leader assumed office
  (entry\_date), date of departure (exit\_date), date of the
  autocoup-defining event (autocoup\_date), and commencement date of the
  extended term (extending\_date).
\item
  \textbf{Power transition methods:} Mode of accession (entry\_method),
  mode of departure (exit\_method), and binary indicators for regular or
  irregular entry (entry\_regular) and exit (exit\_regular).
\item
  \textbf{Autocoup characteristics:} Method of tenure extension
  (autocoup\_method) and outcome of the attempt (autocoup\_outcome),
  categorised as: ``failed and removed from office'', ``failed but
  completed original tenure'', or ``successful''. For successful cases,
  the duration of the additional term is calculated as the interval
  between extending\_date and exit\_date.
\item
  \textbf{Data source:} The principal dataset from which the case was
  coded (source).
\item
  \textbf{Additional notes:} Contextual commentary on exceptional or
  borderline cases (notes).
\end{itemize}

Several coding challenges and methodological decisions warrant further
elaboration. In instances where tenure extensions occurred
incrementally, the \texttt{autocoup\_date} corresponds to a pivotal
event---such as the passage of a constitutional amendment, a legislative
vote, or the outcome of a referendum. Where leaders attempted multiple
autocoups, details are consolidated in the notes field. Particular care
was taken to distinguish between mere power consolidation and explicit
efforts to prolong tenure, which necessitated triangulation across
multiple sources. Furthermore, assessing the success or failure of an
autocoup---particularly in under-reported contexts---frequently required
extensive background research and qualitative judgment.

\subsection*{Data descriptions}\label{data-descriptions}
\addcontentsline{toc}{subsection}{Data descriptions}

The primary coding process identified 83 instances of autocoups between
1945 and 2023, spanning 63 countries. This comprehensive dataset
provides a robust empirical foundation for analysing trends and patterns
in autocoup attempts across a wide array of political and institutional
contexts.

A breakdown of the methods employed by incumbents to extend their tenure
is presented in Table~\ref{tbl-autocoup_method}. The most prevalent
strategy is the legalisation or reintroduction of re-election,
accounting for 37 cases. This is followed by the removal of term limits
(10 cases) and the declaration of the leader as president for life (7
cases). Other tactics, such as the cancellation of scheduled elections
or the refusal to concede electoral defeat, appear less frequently.
Electoral rigging is recorded in only one case---primarily because it is
often difficult to verify with certainty, despite strong indications in
many instances.

\begin{table}

\caption{\label{tbl-autocoup_method}Autocoup methods and success rates
(1945-2023)}

\centering{

\fontsize{12.0pt}{14.4pt}\selectfont
\begin{tabular*}{0.99\linewidth}{@{\extracolsep{\fill}}lccr}
\toprule
Autocoup Method & Attempted & Succeeded & Success Rate \\ 
\midrule\addlinespace[2.5pt]
Enabling re-election & 37 & 26 & 70.3\% \\ 
Removing term limits & 10 & 10 & 100.0\% \\ 
Leader for life & 7 & 7 & 100.0\% \\ 
Delaying elections & 5 & 5 & 100.0\% \\ 
One-time arrangement & 5 & 4 & 80.0\% \\ 
Changing term length & 5 & 4 & 80.0\% \\ 
Reassigning power role & 4 & 2 & 50.0\% \\ 
Refusing election results & 3 & 0 & 0.0\% \\ 
Figurehead & 3 & 3 & 100.0\% \\ 
Cancelling elections & 3 & 3 & 100.0\% \\ 
Rigging elections & 1 & 0 & 0.0\% \\ 
Total & 83 & 64 & 77.1\% \\ 
\bottomrule
\end{tabular*}
\begin{minipage}{\linewidth}
\emph{Source: Autocoup dataset}\\
\end{minipage}

}

\end{table}%

Autocoups exhibit a notably high overall success rate of \(77\%\), in
stark contrast to the approximate \(50\%\) success rate observed in
classical military coups. Several factors may explain this discrepancy.
First, incumbents possess direct access to state resources and
institutional mechanisms, which can be deployed strategically to their
advantage. Second, in contrast to the abrupt and confrontational nature
of traditional coups, autocoups tend to unfold gradually and
deliberately, affording incumbents time to consolidate elite support and
cultivate public legitimacy. Third, many autocoup strategies are
implemented under the guise of legality---via constitutional amendments
or judicial rulings---which reduces overt resistance and complicates
efforts to mobilise effective opposition. Finally, incumbents typically
exercise considerable influence over key state institutions, including
the judiciary, legislature, and security services, which facilitates the
planning and consolidation of such actions.

However, success rates differ markedly depending on the method employed.
Certain strategies appear to be consistently effective. For example,
removing term limits, cancelling elections, declaring oneself leader for
life, delaying elections, and installing figurehead successors all
exhibit a \(100\%\) success rate within the dataset. Notably, these
approaches represent some of the most flagrant violations of
constitutional norms concerning executive succession. This pattern
suggests that outcomes are shaped less by the degree of legal or
constitutional transgression than by the underlying distribution of
political power. In other words, the success of an autocoup is
determined not by the legality of the act, but by the incumbent's
capacity to control coercive, judicial, and legislative institutions.
Leaders who command overwhelming authority are both willing and able to
disregard constitutional constraints precisely because their dominance
insulates them from meaningful resistance.

By contrast, refusal to accept electoral defeat exhibits the lowest
success rate, with only one of four such attempts proving successful.
Although the sample is limited, this pattern may reflect the greater
institutional resilience of electoral democracies, stronger civil
society mobilisation, more intensive international scrutiny, and the
inherently high-risk nature of overturning electoral outcomes. These
factors may increase the probability of failure for incumbents who
pursue this path.

Notably, in contrast to classical coups, which predominantly occur in
autocratic regimes, a substantial proportion of autocoups take place in
democratic settings. Of the 83 identified autocoup attempts, 30 took
place in democracies---of which 29 occurred in presidential
democracies---constituting approximately \(36\%\) of the total. By
comparison, traditional coups have been significantly less frequent in
democratic contexts, with only 99 out of 493 cases ( \(20\%\) ) taking
place in such regimes. This marked disparity will be examined in greater
depth in Chapter 3.

\section{Case studies}\label{case-studies}

\subsection*{High frequency and success rate of autocoups in
post-communist
regimes}\label{high-frequency-and-success-rate-of-autocoups-in-post-communist-regimes}
\addcontentsline{toc}{subsection}{High frequency and success rate of
autocoups in post-communist regimes}

The dataset reveals a notably high incidence and success rate of
autocoup attempts in post-communist states. These countries, which were
governed under communist rule prior to the collapse of the Soviet Union,
have predominantly transitioned into so-called `hybrid regimes'
(\citeproc{ref-nurumov2019}{Nurumov and Vashchanka 2019}), with only a
few---most notably China---retaining an overtly communist political
identity. Within these contexts, the dataset records 12 attempts by
incumbents to extend their tenure, of which only two were unsuccessful.
A closer examination of these cases reveals several shared structural
and political characteristics.

Firstly, many post-communist regimes inherited authoritarian
institutional legacies. While they formally transitioned away from
communism, these states often preserved core authoritarian features,
particularly the centralisation of executive authority.

Secondly, elite continuity has been a hallmark of post-communist
transitions. Rather than a clear break with the previous regime, many
transitions saw the retention of former communist elites, who
reconstituted themselves within ostensibly democratic frameworks,
frequently dominating newly formed political institutions.

Thirdly, democratic procedures have frequently been subverted in
post-communist contexts. Although democratic reforms introduced
elections and constitutional term limits, the enduring institutional
structures of communist rule have often facilitated the manipulation of
electoral processes and the circumvention of formal constraints on
executive power (\citeproc{ref-nurumov2019}{Nurumov and Vashchanka
2019}).

\subsubsection*{Lifelong ruler: Alexander Lukashenko in
Belarus}\label{lifelong-ruler-alexander-lukashenko-in-belarus}
\addcontentsline{toc}{subsubsection}{Lifelong ruler: Alexander
Lukashenko in Belarus}

Alexander Lukashenko, a former member of the Supreme Soviet of the
Byelorussian SSR, rose to national prominence as the head of Belarus's
interim anti-corruption committee following the dissolution of the
Soviet Union. In 1994, he was elected as the country's first president,
a position he has held continuously throughout the period under
examination. The original 1994 constitution imposed a two-term limit on
the presidency; however, this restriction was repealed in 2004 through a
constitutional amendment.

Since his initial election, international observers have consistently
found that Belarusian elections fall short of democratic standards.
Despite sustained domestic opposition and recurrent mass protests,
Lukashenko has claimed repeated re-election victories, frequently with
vote shares exceeding \(80\%\). This trajectory is emblematic of a
broader pattern across the post-Soviet space, particularly in Central
Asia, where former high-ranking communist officials transitioned into
presidential office and have retained power with limited institutional
constraints.

\subsubsection*{Dynastic succession: Nursultan Nazarbayev in
Kazakhstan}\label{dynastic-succession-nursultan-nazarbayev-in-kazakhstan}
\addcontentsline{toc}{subsubsection}{Dynastic succession: Nursultan
Nazarbayev in Kazakhstan}

Nursultan Nazarbayev served as the first president of independent
Kazakhstan from 1991 until 2019. Prior to independence, he was the First
Secretary of the Communist Party of Kazakhstan, thereby exercising de
facto leadership both before and after the Soviet collapse. Following
independence, Nazarbayev was elected president and remained in power
through a series of constitutional and legal modifications, including
the adoption of new constitutions that effectively reset term limits.

Importantly, Nazarbayev did not formally abolish term limits. Rather, a
constitutional exemption was created specifically for the ``First
President'', allowing him to circumvent term restrictions while
maintaining a veneer of legal continuity
(\citeproc{ref-nurumov2019}{Nurumov and Vashchanka 2019}). Unlike
Lukashenko, who has remained in office continuously since 1994 up to the
time of this study, Nazarbayev formally resigned in 2019, designating
Kassym-Jomart Tokayev as his successor. However, Nazarbayev continued to
exercise significant influence through his position as Chairman of the
Security Council, a role he retained until 2022. This illustrates the
persistence of informal executive dominance even after nominal power has
been relinquished.

\subsection*{Autocoups for immediate re-election: Cases of Latin
America}\label{autocoups-for-immediate-re-election-cases-of-latin-america}
\addcontentsline{toc}{subsection}{Autocoups for immediate re-election:
Cases of Latin America}

Latin America has a longstanding tradition of imposing constitutional
term limits on executive authority. Simón Bolívar, often regarded as a
founding father of several Latin American republics, initially endorsed
this principle, declaring in 1819 that ``nothing is as dangerous as
allowing the same citizen to remain in power for a long time\ldots{}
That is the origin of usurpation and tyranny''
(\citeproc{ref-ginsburg2019}{Ginsburg and Elkins 2019, 38}). Although
Bolívar later revised his stance---asserting in his 1826 address to the
Constitutional Assembly that ``a president for life with the right to
choose the successor is the most sublime inspiration for the republican
order''---the concept of term limitation became deeply ingrained in the
region's political culture. Indeed, approximately 81\% of Latin American
constitutions adopted between independence and 1985 included some form
of presidential term limit
(\citeproc{ref-marsteintredet2019a}{Marsteintredet 2019}).

An analysis of autocoup cases in the region reveals two noteworthy
patterns regarding re-election dynamics.

\subsubsection*{Frequent success in breaking non-re-election
rules}\label{frequent-success-in-breaking-non-re-election-rules}
\addcontentsline{toc}{subsubsection}{Frequent success in breaking
non-re-election rules}

Unlike many presidential systems in which two consecutive terms are the
norm, Latin American constitutions have historically favoured more
restrictive arrangements. According to Marsteintredet
(\citeproc{ref-marsteintredet2019a}{2019}), \(64.9\%\) of constitutions
in the region between independence and 1985 prohibited immediate
re-election, while \(5.9\%\) forbade re-election altogether.

Nevertheless, adherence to these rules has varied. Countries such as
Mexico, which enshrined a strict non-re-election clause in 1911 at the
outset of the Mexican Revolution, have consistently upheld this
restriction (\citeproc{ref-klesner2019}{Klesner 2019}). Panama and
Uruguay have similarly refrained from amending their re-election
provisions, while Costa Rica has permitted immediate re-election only
briefly (1897--1913) since its initial prohibition in 1859
(\citeproc{ref-marsteintredet2019a}{Marsteintredet 2019}). In contrast,
several states have frequently amended or circumvented their
constitutional term limits.

The pursuit of re-election or immediate re-election has often served as
a central motive for autocoup attempts aimed at consolidating executive
power. This study identifies 22 autocoup cases in Latin America, of
which 14 (over \(63\%\) ) involved efforts to enable re-election or
immediate re-election. Of these, 9 were successful, yielding a success
rate exceeding \(64\%\).

Importantly, not all such leaders sought indefinite tenure. Many stepped
down after securing and completing a second term. Notable examples
include Fernando Henrique Cardoso (Brazil, 1995--2003), Danilo Medina
(Dominican Republic, 2012--2020), and Juan Orlando Hernández (Honduras,
2014--2022) (\citeproc{ref-ginsburg2019}{Ginsburg and Elkins 2019};
\citeproc{ref-marsteintredet2019a}{Marsteintredet 2019};
\citeproc{ref-landau2019}{Landau, Roznai, and Dixon 2019};
\citeproc{ref-baturo2019}{Baturo 2019}; \citeproc{ref-neto2019}{Neto and
Acácio 2019}).

\subsubsection*{Resistance to further
extensions}\label{resistance-to-further-extensions}
\addcontentsline{toc}{subsubsection}{Resistance to further extensions}

The relative restraint exhibited by many leaders should not be
interpreted as a lack of ambition for further tenure. Rather, it
reflects the fact that additional extension attempts often failed, and
incumbents acquiesced to these outcomes without resorting to overt
manipulation or repression.

While autocoups that enable a single additional term tend to be
relatively successful, efforts to prolong tenure beyond two terms
encounter greater resistance and are more likely to fail. Two
contrasting cases illustrate these divergent trajectories.

\textbf{Unsuccessful extension--Carlos Menem (Argentina):} President
Menem secured a second term following a 1994 constitutional amendment
permitting one re-election, and he was re-elected in 1995. However, his
subsequent attempt to reset the term count---arguing that his first term
(1988--1995) had occurred under a previous constitutional
framework---was unanimously rejected by the Supreme Court in 1999
(\citeproc{ref-llanos2019}{Llanos 2019}). A comparable outcome was
observed in the case of President Álvaro Uribe in Colombia (2002--2010)
(\citeproc{ref-baturo2019}{Baturo 2019}).

\textbf{Successful extension--Daniel Ortega (Nicaragua):} In contrast,
President Daniel Ortega of Nicaragua succeeded in extending his tenure
through a series of judicial and legislative manoeuvres. In 2009, the
Supreme Court of Justice authorised his candidacy for the 2011 election.
Subsequently, in 2014, the National Assembly passed constitutional
amendments abolishing presidential term limits, thereby enabling Ortega
to pursue indefinite five-year terms. He has remained in office
continuously since 2007 (\citeproc{ref-close2019}{Close 2019}).

\subsection*{As common as classical coups: Autocoups in African
countries}\label{as-common-as-classical-coups-autocoups-in-african-countries}
\addcontentsline{toc}{subsection}{As common as classical coups:
Autocoups in African countries}

Classical coups have historically been widespread across the African
continent, accounting for approximately \(45\%\) of all coups
globally---219 out of 493 recorded incidents since 1950---affecting 45
of the 54 African states (\citeproc{ref-powell2011}{Powell and Thyne
2011}). While autocoups occur less frequently than traditional coups,
they nonetheless represent a significant political phenomenon within
Africa. Of the 83 documented cases of autocoups worldwide, \(43\%\) (36
cases) have taken place on the continent, spanning 29 countries.
Notably, the success rate of African autocoups exceeds \(77\%\) (28 out
of 36), a figure that significantly surpasses the regional success rate
for classical coups (approximately \(50\%\) ) and aligns with the global
average success rate for autocoups ( \(77\%\) ).

Identifying a clear and consistent pattern underpinning autocoups in
Africa remains a considerable challenge, reflecting the broader
analytical complexity long associated with classical coups.
Nevertheless, the literature has proposed several explanatory factors.

First, natural resource wealth has been identified as a key variable.
States endowed with oil, diamonds, or other strategic commodities often
present incumbents with both greater incentives and enhanced capacities
to pursue term extensions and entrench their authority
(\citeproc{ref-posner}{Posner and Young, n.d.};
\citeproc{ref-cheeseman2015}{Cheeseman 2015};
\citeproc{ref-cheeseman2019a}{Cheeseman and Klaas 2019}).

Second, the quality of democracy plays a crucial role. Higher levels of
democratic consolidation are strongly associated with greater adherence
to constitutional term limits (\citeproc{ref-reyntjens2016}{Reyntjens
2016}).

Third, international influence may act as a constraint. External
actors---including bilateral donors and international
organisations---can exert diplomatic or economic pressure to discourage
leaders from circumventing term limits (\citeproc{ref-brown2001}{S.
Brown 2001}; \citeproc{ref-tangri2010}{Tangri and Mwenda 2010}).

Finally, opposition strength and ruling party cohesion are critical
domestic factors. The effectiveness of opposition forces in coordinating
resistance, as well as the incumbent's ability to preserve unity within
the ruling party, significantly shapes the political feasibility of
tenure extensions (\citeproc{ref-cheeseman2019}{Cheeseman 2019}).

Drawing on the Africa Executive Term Limits (AETL) dataset, Cassani
(\citeproc{ref-cassani2020}{2020}) identifies human rights abuses and
the desire to evade legal or political accountability as key motivations
behind efforts to overstay in power. The more authoritarian a leader's
governing style, the more likely they are to challenge constitutional
constraints. Moreover, incumbents who can secure the loyalty of the
military---often through strategic use of public investment---are
significantly more likely to succeed in extending their tenure.

Although both classical coups and autocoups continue to be features of
African politics, a marked shift has occurred since the end of the Cold
War in 1991. While the incidence of traditional coups has declined,
autocoups have become increasingly prevalent. This trend is partly
attributable to the widespread adoption of multiparty electoral systems
in the 1990s, often accompanied by the formal institutionalisation of
presidential term limits (\citeproc{ref-cassani2020}{Cassani 2020};
\citeproc{ref-cheeseman2019}{Cheeseman 2019}).

Prior to 1991, personalist and military regimes predominated across much
of the continent, and constitutional term limits were seldom enshrined.
The post-Cold War expansion of democratic frameworks contributed to a
rise in the adoption of such provisions, and, correspondingly, in
attempts to circumvent them. However, it is important to stress that
this increase in challenges to term limits should not be interpreted as
indicative of declining compliance. On the contrary, overall turnover in
executive leadership has increased compared to earlier decades,
suggesting that while violations continue to attract attention,
adherence to constitutional rules has become more widespread.

\section{Summary}\label{summary}

This chapter has offered a comprehensive analysis of autocoups, focusing
on political episodes in which incumbent leaders extend their tenure
beyond constitutionally prescribed limits. By refining existing
definitions and distinguishing autocoups from related concepts---such as
self-coups, autogolpes, and executive takeovers---the study introduces a
novel dataset cataloguing 83 cases of autocoups from 1945 to 2023, of
which 64 were successful.

The revised conceptual framework, combined with the newly assembled
dataset, enables a more expansive and systematic analysis of irregular
leadership transitions. Whereas traditional scholarship has primarily
centred on the premature termination of leadership through coups, this
study broadens the analytical scope to encompass irregular tenure
extensions. This approach yields a more nuanced understanding of the
ways in which incumbents may subvert constitutional norms and democratic
procedures to consolidate their authority.

Nonetheless, several limitations must be acknowledged. First, the
proposed definition of an autocoup warrants further scholarly scrutiny
and debate before it can achieve broader consensus. Despite rigorous
efforts to ensure consistency, certain coding decisions---particularly
in ambiguous or borderline cases---may inevitably entail an element of
subjective judgement. Second, while this thesis concentrates on tenure
extensions, instances involving the expansion of executive power without
a formal extension of tenure also deserve deeper conceptual and
empirical investigation.

Despite these constraints, the research makes a substantive contribution
to the literature on political stability, leadership dynamics, and
democratic resilience. The dataset provides a valuable empirical
foundation for future inquiries into the mechanisms and motivations
underpinning autocoups.

Several promising directions for further research emerge. Subsequent
studies could utilise the dataset to examine the long-term institutional
consequences of autocoups, including their role in democratic
backsliding, authoritarian entrenchment, and the personalisation of
executive power. Additionally, comparative analyses between autocoups
and classical coups may illuminate the evolving strategies employed by
incumbents to consolidate authority across diverse regime types and
political environments.

\chapter{Power Dynamics and Autocoup Attempts}\label{sec-chapter2}

\section*{Abstract}\label{abstract-2}
\addcontentsline{toc}{section}{Abstract}

This chapter explores the determinants of autocoup attempts, aiming to
deepen understanding of the political dynamics that underpin tenure
extensions by incumbent leaders. Addressing a notable gap in the
existing literature, the study contends that the balance of power plays
a critical role in shaping the likelihood of autocoup events. In
contrast to classical coups---which are often triggered by unstable or
fragmented power structures---autocoups tend to arise in contexts
characterised by stable and concentrated power.

To operationalise the concept of power balance in an observable manner,
regime type is employed as a proxy, reflecting the structural
distribution of power between incumbents and potential institutional
constraints or elite challengers. Using a bias-reduced logistic
regression model, the analysis finds that regime type is a significant
predictor of autocoup attempts. Leaders operating within regimes marked
by concentrated power are more prone to extend their tenure
unconstitutionally. In particular, presidential democracies and
personalist autocracies are found to be significantly more susceptible
to autocoup attempts than dominant-party regimes.

The study contributes to the broader literature on irregular leadership
transitions by offering a more systematic and empirically grounded
account of the conditions under which incumbents seek to subvert
constitutional term limits.

\emph{\textbf{Keywords:} Autocoups, Coup, Regime types, Tenure
Extension, Authoritarianism}

\newpage

\section{Introduction}\label{introduction-2}

As outlined in Chapter 2, scholarly engagement with autocoups has been
hampered by conceptual ambiguity and the absence of systematic data,
thereby limiting the scope for rigorous empirical investigation. To
address this lacuna, the present chapter aims to make a substantive
contribution through a quantitative analysis of the determinants of
autocoup attempts. Following the methodological precedent established by
empirical studies of classical coups---which have primarily examined the
antecedents of coup initiation (\citeproc{ref-gassebner2016}{Gassebner,
Gutmann, and Voigt 2016})---this chapter similarly explores why some
incumbent leaders attempt to extend their tenure through autocoups,
while others do not.

There are three principal reasons for investigating the determinants of
autocoups. First, autocoups constitute one of the most prevalent forms
of irregular leadership transition, with over 80 documented cases since
1945 (as discussed in Chapter 2). Their frequency has increased notably
since 2000, coinciding with a marked global decline in classical coups
(\citeproc{ref-bermeo2016}{Bermeo 2016}; \citeproc{ref-thyne2019}{Thyne
and Powell 2019}). Second, autocoups exert profound effects on political
stability and democratic development, often resulting in enduring
institutional degradation. Third, identifying the drivers of autocoup
attempts is essential for future research into their consequences;
without a clear understanding of the conditions under which autocoups
occur, efforts to prevent them or mitigate their detrimental effects
remain constrained.

Although autocoups differ fundamentally from classical
coups---particularly in that they are instigated by incumbents rather
than external challengers---the two phenomena share key features as
disruptions to established political order. Accordingly, methodological
tools commonly applied in the study of traditional coups may be
fruitfully adapted to analyse autocoups. However, despite the extensive
literature on coup dynamics (\citeproc{ref-gassebner2016}{Gassebner,
Gutmann, and Voigt 2016}), regime type is frequently treated as a
background condition or control variable rather than a central
explanatory factor.

This chapter advances the argument that the likelihood of autocoup
attempts is shaped significantly by the structural distribution of power
inherent in regime type. In contrast to classical coups, which often
emerge from unstable or contested power structures, autocoups tend to
occur in regimes characterised by concentrated and stable authority.
Given the challenges of directly measuring internal power
configurations, regime type is employed as a proxy variable. The
underlying premise is that regime type reflects core institutional
arrangements, including the distribution of authority, the robustness of
constitutional constraints, and the capacity of incumbents to subvert
democratic norms. Analysing cross-regime variation thus facilitates a
deeper understanding of the institutional foundations that condition
autocoup risk. These power structures tend to be relatively stable over
time, as they both shape and are shaped by the regime's overarching
institutional design (\citeproc{ref-geddes2014}{Geddes, Wright, and
Frantz 2014}).

To empirically test this proposition, the chapter utilises both a
standard logistic regression model and a bias-reduced logistic
regression model to assess how regime type influences the likelihood of
incumbents extending their tenure through extra-constitutional means.

Given the paucity of quantitative research on autocoups, this study
offers a potentially pioneering contribution to the empirical literature
by providing a theoretically informed and methodologically rigorous
account of their determinants.

The remainder of the chapter is structured as follows. Section 2
examines the dynamics and outcomes of autocoup attempts. Section 3
outlines the research design, including the methodological approach and
variables employed. Section 4 presents and interprets the empirical
findings, highlighting key patterns and implications. Section 5
concludes by summarising the core insights and reflecting on their
broader significance for understanding and mitigating the risks posed by
autocoups.

\section{Dynamics of autocoup
attempts}\label{dynamics-of-autocoup-attempts}

Like traditional coup attempts, autocoups are shaped by two fundamental
elements: the disposition of incumbent leaders---referring to their
motivations and willingness to act---and their capability, defined by
the resources and opportunities at their disposal. However, autocoups
exhibit two notable features that distinguish them from classical coups.
First, whereas traditional coups occur predominantly in autocracies
(\citeproc{ref-thyne2014}{C. L. Thyne and Powell 2014}), over one-third
of documented autocoups have taken place in democratic regimes, as
outlined in Chapter 2. Second, while the success rate of traditional
coups hovers around \(50\%\), more than \(77\%\) of autocoup attempts
have resulted in success, according to the dataset introduced in Chapter
2. These distinctions indicate that the dynamics of disposition and
capability underlying autocoups differ significantly from those of
traditional coups.

This section explores the complex dynamics of autocoup attempts, with
particular emphasis on how the motivations of incumbents, the
determinants of success, and the institutional frameworks of various
regime types shape the vulnerability of states to such
extra-constitutional power extensions.

\subsection*{Motivations for autocoups}\label{motivations-for-autocoups}
\addcontentsline{toc}{subsection}{Motivations for autocoups}

Incumbents seeking to prolong their tenure may be driven by a range of
motivations, broadly falling into three principal categories: personal
ambition, appeals to national interest, and self-preservation.

First, the pursuit of personal power constitutes a compelling incentive
for many leaders. The capacity to govern free from institutional
constraints enables incumbents to exercise dominance over national
policy-making, access state resources, influence the judiciary and
legislature, and retain the prestige associated with holding high
office. For some, the aspiration to secure a lasting political
legacy---to be remembered as a transformative figure---further amplifies
the appeal of extended rule.

Second, tenure extensions are often justified by incumbents in the name
of the national interest. A commonly advanced rationale suggests that a
single term is insufficient for the completion of long-term reforms or
development initiatives. Within this narrative, remaining in power is
portrayed as essential to ensuring the continuity and success of ongoing
projects. The autocoup is thus framed not as an act of self-interest,
but as a necessary step for the greater good.

Third, autocoups may serve as mechanisms of self-preservation.
Incumbents facing the prospect of prosecution for corruption, human
rights violations, or other transgressions may view continued tenure as
a means of preserving legal immunity. Additionally, those who have
amassed significant political adversaries during their rule may fear
retribution upon leaving office. In such cases, the extension of power
is not merely a product of ambition but also a strategy for
survival---intended to shield the leader from legal or political
repercussions.

\subsection*{Power dynamics and
autocoups}\label{power-dynamics-and-autocoups}
\addcontentsline{toc}{subsection}{Power dynamics and autocoups}

While motivations may initiate an incumbent's decision to pursue an
autocoup, the decisive factor often lies in their ability to implement
and sustain such an action. The relatively high frequency and remarkable
success rate of autocoups---over \(77%
\), compared to approximately \(50\%\) for classical coups---suggest
that incumbents benefit from notable structural advantages when
attempting to consolidate power. These advantages are not limited to
autocracies but are also evident in democratic systems, underscoring the
variation in institutional leverage available to incumbents across
different regime types.

This reality necessitates a closer examination of state power
structures, particularly the allocation of control over the military.
The allegiance of the armed forces is a critical determinant of autocoup
outcomes. If the military remains loyal to the executive,
resistance---whether from civil society, the judiciary, or the
legislature---can be suppressed or marginalised. Conversely, open
defiance or refusal by the military to support the incumbent may render
an autocoup untenable.

Nevertheless, it is reductive to assume that formal authority as
commander-in-chief guarantees unqualified control. Just as it is overly
simplistic to attribute the success of traditional coups solely to the
presence of military force (\citeproc{ref-singh2016}{Singh 2016}), it is
equally erroneous to presume that incumbents invariably enjoy the
unconditional loyalty of the armed forces. Nominal titles often obscure
the complex and sometimes precarious dynamics underpinning military
allegiance.

In autocratic regimes, while the military may not be bound by
constitutional principles, it is not inherently loyal to the head of
state. Executives depend on military officers to execute their commands;
however, these officers may harbour independent political ambitions or
competing loyalties. A case in point is Uganda in 1971, when President
Milton Obote attempted to dismiss General Idi Amin. In response, Amin
exploited his influence within the armed forces to mount a successful
coup, ousting Obote (\citeproc{ref-sudduth2017}{Sudduth 2017}).

By contrast, in consolidated democracies, military loyalty is typically
institutionalised through allegiance to the constitution rather than to
individual officeholders. For example, in the United States, following
the 2020 presidential election, General Mark Milley, Chairman of the
Joint Chiefs of Staff, publicly reaffirmed the military's constitutional
commitment: ``We are unique among militaries. We do not take an oath to
a king or a queen, a tyrant or a dictator. We do not take an oath to an
individual. We take an oath to the Constitution.'' (US Army Museum, 12
November 2020\footnote{CNN. \emph{Top US General Stands Firm Amid
  Pentagon Turmoil.} 12 November 2020. Available at:
  \url{https://edition.cnn.com/2020/11/12/politics/mark-milley-pentagon-turmoil/index.html}
  {[}Accessed 24 April 2025{]}.})

In hybrid regimes or fragile democracies, attempts to prolong executive
tenure may entail significant political risks. In Niger, for example,
President Mamadou Tandja's attempt in 2009 to amend the constitution to
permit a third term precipitated a military coup in 2010
(\citeproc{ref-miller2016}{Miller 2016}). Similarly, in Honduras the
same year, President Manuel Zelaya was removed from office by the
military after seeking to alter the constitution to allow immediate
re-election (\citeproc{ref-muuxf1oz-portillo2019}{Muñoz-Portillo and
Treminio 2019}).

\subsection*{Regime types and
autocoups}\label{regime-types-and-autocoups}
\addcontentsline{toc}{subsection}{Regime types and autocoups}

Given the complexities discussed, a more effective analytical strategy
entails evaluating the broader balance of power within political
systems. As direct observation of this balance is inherently
challenging, this study adopts regime type as a proxy---an approach
consistent with established methodologies in comparative politics.
Regime types encapsulate the institutional architecture of power
distribution, particularly with respect to control over the military,
political appointments, and policy-making authority.

Following the typology developed by Geddes, Wright, and Frantz
(\citeproc{ref-geddes2014}{2014}), autocratic regimes can be categorised
as follows:

\textbf{Military regimes} are governed by a junta, typically comprising
senior military officers who collectively determine leadership and
policy direction. Notable examples include Brazil (1964--1985),
Argentina (1976--1983), and El Salvador (1948--1984)
(\citeproc{ref-geddes1999}{Geddes 1999b}).

\textbf{Personalist regimes} revolve around a dominant individual who
wields unchecked authority over the military, policy decisions, and
succession processes. Prominent instances include Rafael Trujillo in the
Dominican Republic (1930--1961), Idi Amin in Uganda (1971--1979), and
Jean-Bédel Bokassa in the Central African Republic (1966--1979)
(\citeproc{ref-geddes1999}{Geddes 1999b}).

\textbf{Dominant-party regimes} concentrate authority within a
structured political party, with the leader operating either as part of
or at the helm of the party apparatus. Illustrative cases include the
PRI in Mexico, CCM in Tanzania, and the Leninist parties of Eastern
Europe (\citeproc{ref-geddes1999}{Geddes 1999b}).

Among these regime types, personalist autocracies are particularly
conducive to autocoups. The concentration of power in a single
individual weakens institutional checks and fosters
loyalty---particularly from the military---through mechanisms of
personal patronage. While military regimes are rooted in coercive power,
they are often beset by internal factionalism, rendering them more
susceptible to traditional coups than to autocoups. Dominant-party
regimes occupy a more ambiguous position: although party structures can
constrain executive action, exceptionally powerful party leaders may
still initiate autocoups, as exemplified by Xi Jinping's constitutional
amendments in 2018 within a dominant-party framework.

Monarchies, though technically autocratic, generally render autocoups
redundant, as monarchs typically rule for life by constitutional design.

A key clarification is warranted at this juncture: why might leaders in
personalist regimes---already possessing extensive authority---feel
compelled to extend their tenure further? The answer lies in
distinguishing between the scope and duration of power. While such
leaders may exercise considerable de facto control over state
institutions, many initially assume office via legal or constitutional
channels, necessitating a gradual process of consolidation. In this
context, autocoups function as formal mechanisms to institutionalise
existing dominance---transforming informal power into legally sanctioned
permanence. This dynamic is exemplified by the repeated tenure
extensions pursued by Vladimir Putin and Alexander Lukashenko.

In post-Soviet Russia, President Boris Yeltsin presided over the
transformation of a parliamentary system into a personalist regime.
However, Yeltsin himself did not overstay his term; instead, he
designated Vladimir Putin as his successor. Upon assuming office in
2000, Putin progressively entrenched his authority, employing
constitutional amendments and legal strategies to circumvent term limits
and extend his rule indefinitely.

Likewise, in Belarus, Alexander Lukashenko was elected president in 1994
under a party-based system. Within a year, he dismantled the existing
institutional framework and established a personalist regime. Since
then, he has remained in power through successive tenure extensions,
steadily consolidating his control over the state apparatus.

In democratic contexts, autocoups are found exclusively in presidential
systems. This reflects the institutional leverage enjoyed by presidents,
who are directly elected, typically command the armed forces, and may
possess the capacity to override or circumvent legislative opposition.
By contrast, prime ministers in parliamentary systems are considerably
more constrained. Their tenure depends on maintaining legislative
confidence and they may be removed through votes of no confidence.
Moreover, they often lack direct control over the military, which is
institutionally separated from their office. As a result, prime
ministers are subject to more frequent leadership turnover and face
fewer opportunities to unilaterally extend their mandates. For instance,
the United Kingdom saw three prime ministers serve in 2022 alone, while
Japan has had 36 prime ministers since 1945---an average of one every
two years. In contrast, only 14 presidents have served in the United
States over the same period, reflecting greater institutional
continuity. These structural distinctions render presidential systems
more conducive to autocoups---even within well-established
democracies---due to their centralised executive authority and command
over the military.

From this analysis, the following hypothesis is proposed:

The likelihood of autocoup attempts is significantly shaped by regime
type, with regimes characterised by concentrated and stable executive
power---namely, personalist autocracies and presidential
democracies---being the most susceptible, relative to other regime
types.

\textbf{\emph{H3-1: The likelihood of autocoup attempts is significantly
shaped by regime type, with regimes characterised by concentrated and
stable executive power---namely, personalist autocracies and
presidential democracies---being the most susceptible, relative to other
regime types.}}

\section{Research design}\label{research-design}

\subsection*{Methodology}\label{methodology}
\addcontentsline{toc}{subsection}{Methodology}

Given the binary nature of the dependent variable---namely, whether an
autocoup is attempted in a given country-year---the study initially
employs a logistic regression model to investigate the determinants of
autocoup attempts. This method enables the identification of
statistically significant factors influencing the likelihood of such
events, as well as the direction and magnitude of their effects.

Nevertheless, the rarity of autocoup incidents---83 cases out of over
9,000 observations---poses a methodological challenge. Standard maximum
likelihood estimation techniques, including conventional logit and
probit models, are prone to underestimating the probability of rare
events. To mitigate this limitation and improve the robustness of
statistical inference, the analysis also employs Firth's Bias-Reduced
Penalised Maximum Likelihood Estimation (commonly referred to as
Bias-Reduced Logit), as outlined by FIRTH
(\citeproc{ref-firth1993}{1993}).

\subsection*{Data and variables}\label{data-and-variables}
\addcontentsline{toc}{subsection}{Data and variables}

The primary dataset, which incorporates information on autocoups and
regime types, spans the period from 1945 to 2023. However, due to data
alignment limitations, the usable data range extends from 1945 to 2018.
The dataset comprises approximately 9,400 country-year observations, of
which 83 represent recorded autocoup attempts.

\subsubsection*{Dependent variable}\label{dependent-variable}
\addcontentsline{toc}{subsubsection}{Dependent variable}

The analysis draws upon the autocoup dataset introduced in Chapter 2,
which covers the period from 1945 to 2023 and includes 83 documented
autocoup attempts. Summary statistics for these events, as well as the
corresponding regime classifications, are presented in Chapter 2.

\textbf{Autocoup attempt}: A binary variable indicating whether an
autocoup attempt occurred (coded as 1) or did not occur (coded as 0) in
each country-year observation.

\subsubsection*{Independent variables}\label{independent-variables}
\addcontentsline{toc}{subsubsection}{Independent variables}

The principal independent variable in this analysis is regime type,
reflecting the central analytical focus of the study. Regime
classifications are drawn from the typology developed by Geddes, Wright,
and Frantz (\citeproc{ref-geddes2014}{2014}) (GWF dataset), which
distinguishes among military, personalist, and dominant-party regimes
within autocratic systems. For democratic systems, regimes are
categorised as either parliamentary or presidential. A residual
category---labelled ``other''---captures regimes that are provisional,
transitional, or otherwise not easily classified within the primary
typology.

In addition to regime type, a range of control variables is included,
selected on the basis of established scholarship on the determinants of
coups. These controls account for factors such as economic performance,
political violence, and the tenure of incumbents. Further controls
comprise the level of democracy, population size, and a Cold War dummy
variable, which captures temporal variation in the global political
environment.

\textbf{Economic Level:} Measured by GDP per capita, this variable
reflects the overall economic wellbeing of a country. Data are sourced
from the V-Dem dataset (\citeproc{ref-fariss2022}{Fariss et al. 2022})
and are expressed in constant 2017 international dollars (PPP, per
thousand).

\textbf{Economic Performance:} Operationalised via the Current-Trend
(CT) ratio developed by Krishnarajan
(\citeproc{ref-krishnarajan2019}{2019}), this measure compares current
GDP per capita with the average of the previous five years. Higher CT
values indicate stronger economic growth. Formally:

\[
    \begin{aligned}
    CT_{i,t} = {GDP/cap_{i,t} \over {1 \over 5} {\sum_{k=1}^5GDP/cap_{i,t-k}}}
    \end{aligned}
\]

\textbf{Political violence:} Measured using a violence index based on
the ``actotal'' variable from the Major Episodes of Political Violence
dataset (\citeproc{ref-marshall2005current}{Monty G. Marshall 2005}),
this index captures both internal and interstate conflict. Scores range
from 0 (complete stability) to 18 (maximum instability).

\textbf{Days in office (log):} The natural logarithm of an incumbent
leader's cumulative days in office is included as a proxy for power
consolidation. Longer tenures are hypothesised to facilitate the
conditions necessary for an autocoup. Data are drawn from the Archigos
dataset (\citeproc{ref-goemans2009}{Goemans, Gleditsch, and Chiozza
2009}) and the Political Leaders' Affiliation Database (PLAD)
(\citeproc{ref-bomprezzi2024wedded}{Bomprezzi et al. 2024}).

\textbf{Democratic level:} This variable employs the Polity V score to
measure the degree of democracy in a country, ranging from -10 (fully
autocratic) to +10 (fully democratic). The index, developed by the
Centre for Systemic Peace, assesses regime characteristics such as the
competitiveness of political participation, executive recruitment, and
constraints on executive authority (\citeproc{ref-marshall}{Monty G.
Marshall, n.d.}).

\textbf{Population size:} The natural logarithm of a country's
population is included to account for the potential effects of
demographic scale on governance. Larger populations may present more
complex administrative challenges and generate greater opposition. Data
are sourced from the V-Dem dataset.

\textbf{Cold War:} Following the precedent of earlier studies
(\citeproc{ref-thyne2014}{C. L. Thyne and Powell 2014};
\citeproc{ref-derpanopoulos2016}{Derpanopoulos et al. 2016};
\citeproc{ref-dahl2023}{Dahl and Gleditsch 2023b}), a dummy variable is
included to distinguish the Cold War period (approximately 1960--1990)
from the post-Cold War era. This distinction reflects the relative
paucity of autocoup events during the Cold War and their increased
frequency thereafter.

\section{Results and discussions}\label{results-and-discussions}

This analysis employs a logistic regression framework to investigate the
factors influencing the likelihood of autocoup attempts. Given the
binary nature of the dependent variable---namely, whether an autocoup
attempt occurred in a given country-year---and the relative infrequency
of such events in certain categories, a bias-reduced logit model is used
alongside the standard logit model. The bias-reduced approach is
particularly appropriate for rare event data, as it mitigates estimation
bias associated with standard maximum likelihood techniques.
Accordingly, the discussion that follows focuses primarily on the
results derived from the bias-reduced logit model. These results are
presented in the form of Odds Ratios (ORs), which provide a more
intuitive understanding of the relationships between explanatory
variables and the probability of autocoup attempts than log-odds
coefficients.

The estimates obtained from the bias-reduced logit model are summarised
in Table~\ref{tbl-autocoupmodel}. The central hypothesis advanced in
this study posits that the likelihood of autocoup attempts is
significantly shaped by regime type, with personalist autocracies and
presidential democracies being particularly prone to such events in
comparison with other regime types. The model adopts ``dominant-party''
regimes as the reference category for comparison.

\begin{table}

\caption{\label{tbl-autocoupmodel}Determinants of autocoup
attempts(1945-2018)}

\centering{

\fontsize{9.8pt}{11.7pt}\selectfont
\begin{tabular*}{\linewidth}{@{\extracolsep{\fill}}lcccccccc}
\toprule
 & \multicolumn{5}{c}{Standard Logit} & \multicolumn{3}{c}{Bias-reduced Logit} \\ 
\cmidrule(lr){2-6} \cmidrule(lr){7-9}
\textbf{Characteristic} & \textbf{N} & \textbf{Event N} & \textbf{log(OR)}\textsuperscript{\textit{1}} & \textbf{OR}\textsuperscript{\textit{1}} & \textbf{SE} & \textbf{log(OR)}\textsuperscript{\textit{1}} & \textbf{OR}\textsuperscript{\textit{1}} & \textbf{SE} \\ 
\midrule\addlinespace[2.5pt]
Constant & 9,434 & 78 & -4.7** & 0.01** & 0.02 & -4.6*** & 0.01*** & 1.77 \\ 
Regime Type &  &  &  &  &  &  &  &  \\ 
    Dominant Party & 2,312 & 19 & — & — & — & — & — & — \\ 
    Personal & 1,308 & 26 & 0.74** & 2.10** & 0.65 & 0.73** & 2.08** & 0.30 \\ 
    Presidential Democracy & 1,642 & 27 & 1.6*** & 5.01*** & 2.42 & 1.6*** & 4.87*** & 0.47 \\ 
    Military & 630 & 2 & -0.80 & 0.45 & 0.34 & -0.62 & 0.54 & 0.67 \\ 
    Parliamentary Democracy & 2,368 & 1 & -1.7 & 0.18 & 0.20 & -1.4 & 0.26 & 0.92 \\ 
    Other & 1,174 & 3 & -1.2* & 0.30* & 0.19 & -1.1* & 0.34* & 0.58 \\ 
GDP per capita & 9,434 & 78 & -0.01 & 0.99 & 0.01 & -0.01 & 0.99 & 0.01 \\ 
GDP growth trend & 9,434 & 78 & 0.91 & 2.49 & 3.47 & 0.97 & 2.64 & 1.33 \\ 
Political violence & 9,434 & 78 & 0.01 & 1.01 & 0.07 & 0.03 & 1.03 & 0.06 \\ 
Log of Population & 9,434 & 78 & -0.14 & 0.87 & 0.08 & -0.15* & 0.86* & 0.09 \\ 
Polity V scores & 9,434 & 78 & -0.09*** & 0.91*** & 0.03 & -0.09*** & 0.91*** & 0.03 \\ 
Log of days in office & 9,434 & 78 & 0.01 & 1.01 & 0.13 & 0.00 & 1.00 & 0.12 \\ 
Cold war &  &  & -0.80*** & 0.45*** & 0.12 & -0.79*** & 0.45*** & 0.26 \\ 
\bottomrule
\end{tabular*}
\begin{minipage}{\linewidth}
\textsuperscript{\textit{1}}*p\textless{}0.1; **p\textless{}0.05; ***p\textless{}0.01\\
Abbreviations: CI = Confidence Interval, OR = Odds Ratio, SE = Standard Error\\
\end{minipage}

}

\end{table}%

The findings offer strong empirical support for this hypothesis.
Relative to dominant-party regimes, personalist autocracies exhibit an
estimated odds ratio of 2.08, which is statistically significant (\,
\(p<0.05\) \,). This suggests that the odds of an autocoup attempt are
just over twice as high in personalist autocracies, holding all other
variables constant. Presidential democracies demonstrate an even greater
propensity for autocoup attempts, with an estimated odds ratio of 4.87 (
\,\(p<0.01\) \,), indicating that the odds are nearly 3.8 times higher
than in dominant-party regimes, ceteris paribus. By contrast, military
regimes, parliamentary democracies, and regimes classified as ``other''
all exhibit lower odds of experiencing an autocoup relative to
dominant-party regimes. Among these, only the ``other'' category reaches
marginal statistical significance.

These results provide compelling evidence that both personalist
autocracies and presidential democracies are significantly more
susceptible to autocoup attempts than dominant-party regimes. The
magnitude of the odds ratios for personalist autocracies (2.08) and
presidential democracies (4.87) is substantially greater than those for
other regime types, thereby lending robust support to the central
hypothesis.

With respect to the control variables, several demonstrate statistically
significant relationships with the likelihood of autocoup attempts. The
logged Polity V score---reflecting the level of democratic
institutionalisation---is negatively associated with the probability of
an autocoup (\, \(OR = 0.91\), \,\(p<0.01\) \,), suggesting that higher
levels of democracy reduce the risk of such occurrences. Similarly, the
Cold War dummy variable is associated with a significantly reduced
likelihood of autocoup attempts ( \,\(OR 5 0.46\), \,\(p<0.01\) \,),
indicating that these events were comparatively rarer during the Cold
War period. The natural logarithm of population size is marginally
significant and exhibits a negative association with autocoup
likelihood, potentially reflecting the greater organisational complexity
and political constraints faced in more populous states.

By contrast, several variables---including GDP per capita, economic
performance (as measured by the GDP growth trend ratio), political
violence, and the logged number of days the incumbent has been in
office---do not exhibit statistically significant associations with the
likelihood of autocoup attempts in this model.

In sum, the bias-reduced logit model confirms that regime type is a
critical determinant of autocoup propensity. In line with the
theoretical expectations, personalist autocracies and presidential
democracies are significantly more vulnerable to autocoup attempts than
dominant-party regimes or other regime types. These findings underscore
the institutional fragility inherent in such systems, particularly in
contexts where executive authority is highly centralised. The analysis
also highlights the relevance of broader political and historical
factors, including democratic development and the Cold War period, in
shaping the incidence of autocoup events.

\section{Summary}\label{summary-1}

This chapter offers a quantitative examination of the determinants of
autocoup attempts, addressing a well-documented lacuna in the existing
literature, which has often been impeded by conceptual ambiguity and a
lack of systematic empirical data. The study advances the argument that
the likelihood of autocoup attempts is significantly influenced by the
structural distribution of power within political regimes,
operationalised through regime type. Drawing on both standard logistic
regression and Firth's bias-reduced logit model, the analysis
demonstrates that personalist autocracies and presidential democracies
are substantially more prone to autocoup attempts than dominant-party
regimes. Specifically, the odds of an autocoup are estimated to be
approximately three times higher in personalist autocracies and four
times higher in presidential democracies, relative to the reference
category.

These findings offer empirical support for the hypothesis that such
regime types possess structural vulnerabilities that render them
particularly susceptible to extra-constitutional power consolidation by
incumbents. In addition to regime type, the study identifies further
significant covariates: population size, the degree of democratisation,
and the broader historical context of the Cold War period all exert
measurable influence on the probability of autocoup occurrence. By
analysing the strategic incentives faced by incumbent leaders across
different institutional configurations, this research contributes to a
more comprehensive understanding of irregular leader transitions.

Nonetheless, the analysis also highlights conceptual and methodological
complexities that merit further scholarly attention. Unlike traditional
coups---which may occur at any point during a regime's lifespan and can
potentially recur within short intervals---autocoups appear to follow
distinct temporal logics. For example, the likelihood of an autocoup may
be lower during the early stages of a leader's tenure and increase as
the end of a constitutional term approaches. Moreover, while successful
tenure extensions may reduce the short-term likelihood of further
attempts, instances of repeated extensions---such as those undertaken by
Presidents Putin and Lukashenko---indicate that some incumbents may
engage in multiple autocoups over time.

To render the analysis tractable, this study adopts the simplifying
assumption that an autocoup attempt occurs only once per leadership
tenure. While analytically necessary, this assumption underscores the
need for future research to investigate the temporal dynamics and
sequencing of autocoup behaviour. Such inquiries would usefully
complement the present study by offering deeper insight into the
longitudinal patterns and institutional adaptations that shape
authoritarian resilience and democratic backsliding.

\chapter{Power Acquisition and Leadership Survival: A Comparative
Analysis of Coup-installed and Autocoup
Leaders}\label{power-acquisition-and-leadership-survival-a-comparative-analysis-of-coup-installed-and-autocoup-leaders}

\section*{Abstract}\label{abstract-3}
\addcontentsline{toc}{section}{Abstract}

This chapter investigates the impact of power acquisition methods on the
tenure of political leaders who assume office through irregular means,
with a particular focus on individuals installed via coups and those who
subsequently extend their rule through autocoups. It posits that leaders
who consolidate their authority through autocoups are likely to enjoy
greater longevity in office than those installed by coups alone.
However, this hypothesis is not supported by the empirical evidence. A
time-dependent Cox proportional hazards model indicates no statistically
significant difference in the risk of removal between coup-installed and
autocoup leaders once key covariates---especially regime type---are
taken into account.

Rather, the analysis underscores the critical role of regime
characteristics in shaping leadership survival. Leaders operating within
military or transitional regimes are found to face significantly higher
hazards of removal compared to their counterparts in dominant-party
systems. Furthermore, higher levels of GDP per capita are associated
with increased leadership stability, while greater levels of political
violence correspond with elevated risks of ousting.

These findings suggest that structural and institutional contexts exert
a stronger influence on the durability of irregular leadership than the
specific mechanism through which power is initially acquired. This study
contributes to the literature on political survival by highlighting the
significance of regime type and broader political conditions in
accounting for leadership longevity following non-conventional
ascensions to power.

\emph{\textbf{keywords}: Coups, Autocoups, Leadership Survival, Cox
Model}

\newpage

\section{Introduction}\label{introduction-3}

The enduring question of why some political leaders remain in power for
decades while others are deposed within months or even days has long
captivated scholars in political science. Despite the considerable body
of research on leadership longevity, a particular subset of
leaders---those who assume power through coups or consolidate it via
autocoups---has received comparatively limited scholarly attention.
Investigating the tenures of these leaders is essential, as it
illuminates the dynamics underpinning irregular transitions of power and
their broader implications for political stability and democratic
governance.

Leaders who rise through irregular means, such as coups and autocoups,
differ markedly from those who attain office via institutionalised,
constitutional processes. These irregular leaders often pose more
complex analytical challenges, given the uncertainties surrounding their
authority and legitimacy. Data from the Archigos dataset underscores the
prevalence of such irregular transitions: between 1945 and 2015, more
than half of the leaders who entered power irregularly also exited
through irregular means---a rate considerably higher than that observed
for leaders who entered office through regular procedures
(\citeproc{ref-goemans2009}{Goemans, Gleditsch, and Chiozza 2009}).

Coup-installed and autocoup leaders comprise a significant proportion of
these irregular cases. According to Archigos, 246 of the 374 leaders (
\(65.8\%\) ) who exited power irregularly were removed via coups
(\citeproc{ref-goemans2009}{Goemans, Gleditsch, and Chiozza 2009}).
Additionally, research by Frantz and Stein
(\citeproc{ref-frantz2016}{2016}) shows that coups constitute
approximately one-third of all exits in autocracies, representing the
most common form of leadership transition in such regimes. Complementing
this, the autocoup dataset introduced in Chapter 2 documents 83 autocoup
attempts between 1945 and 2023, of which 64 were successful.

Assessing the tenure of leaders who assume office through coups or
autocoups presents methodological challenges, due to the inherent
volatility and uncertainty associated with such irregular modes of
accession. Nonetheless, comparative evidence---excluding short-lived
leaders who remained in office for fewer than 180 days---suggests that
those who extend their rule through autocoups tend to experience longer
average post-autocoup tenures (approximately 9.8 years) than those who
initially assume power via coups (approximately 6.8 years), indicating a
potential tenure gap of around 3 years.

\begin{figure}

\centering{

\pandocbounded{\includegraphics[keepaspectratio]{_coups_and_autocoups_correction_files/figure-pdf/fig-logrank-1.pdf}}

}

\caption{\label{fig-logrank}Survival curves of autocoup and
coup-installed leaders}

\end{figure}%

Preliminary survival analysis, using a log-rank test illustrated in
Figure~\ref{fig-logrank}, reveals a statistically significant difference
in tenure length between these two groups. The survival curve for
autocoup leaders consistently lies above that of coup-installed leaders,
suggesting both a lower hazard of removal and longer durations in office
for the former.

This study posits that the method of power acquisition exerts a
significant influence on leadership survival. Coup-installed leaders may
encounter greater resistance or institutional fragility, contributing to
shorter average tenures than those who consolidate power through
autocoups. Employing Cox proportional hazards and time-dependent Cox
models, the analysis supports this hypothesis by demonstrating that
autocoup leaders tend to remain in office longer than their
coup-installed counterparts.

This research makes two key contributions to the literature on political
survival. First, it introduces an under-explored explanatory factor: the
method of accession to power. Second, by applying survival models, this
study provides robust empirical evidence of the significant disparity in
tenure length between autocoup and coup-installed leaders. These
insights may help account for the rising incidence of autocoup-driven
tenure extensions since 2000, as incumbents increasingly observe and
emulate successful precedents.

The remainder of this chapter is structured as follows: Section 2
reviews the existing literature on political survival, establishing the
theoretical context for the analysis. Section 3 examines the factors
influencing the longevity of coup-installed and autocoup leaders.
Section 4 details the methodological approach and data sources,
including the application of survival analysis techniques. Section 5
presents the empirical findings and interprets their implications.
Finally, Section 6 offers concluding reflections and considers the
broader consequences of the findings for political stability and
democratic development.

\section{Literature review}\label{literature-review}

The longevity of political leaders, which varies markedly across
regimes, countries, and historical periods, has long been a focal point
of inquiry within political science. Research in this field is generally
categorised into two interrelated strands: regime survival and
individual leader survival. While the former concerns the endurance of
political systems---such as monarchies, dominant parties, or ideological
frameworks---the latter focuses on the duration of individual leaders'
tenure in office.

Patterns of political survival differ significantly across regime types.
For instance, parliamentary democracies (e.g., Japan and the United
Kingdom) often witness sustained party dominance alongside frequent
leadership turnover. Similarly, communist regimes (e.g., China) are
typically characterised by stable party control but relatively frequent
changes in leadership. In contrast, presidential systems (e.g., the
United States) and many military regimes tend to exhibit more frequent
changes in both leadership and ruling entity.

The existing literature on leader survival is both extensive and
diverse. Some studies investigate mechanisms that influence leadership
durability within specific regime types, such as democracies
(\citeproc{ref-svolik2014}{Svolik 2014}) or autocracies
(\citeproc{ref-davenport2021}{Davenport, RezaeeDaryakenari, and Wood
2021}). Others attempt to formulate more general theoretical frameworks
applicable across various political systems
(\citeproc{ref-buenodemesquita2003}{Bueno de Mesquita et al. 2003}).
Despite these efforts, the ambition of constructing a universal theory
of leadership survival remains elusive due to the inherent complexities
across regime contexts.

Mechanisms of leadership transition vary substantially between
democracies and autocracies. In autocratic regimes, leadership selection
processes are often closed, with access restricted to a limited elite.
Even when elections are held, meaningful competition is frequently
constrained by structural or legal barriers. The opacity of leadership
transitions in autocracies complicates assessments of popular support
and renders concepts such as selectorates or winning coalitions, as
theorised by Bueno de Mesquita et al.
(\citeproc{ref-buenodemesquita2003}{2003}), difficult to operationalise.

Given these challenges, focusing research on specific categories of
leaders may yield more analytically fruitful outcomes. The study of
irregular leaders---those who ascend to power via coups or extend their
rule through autocoups---offers a compelling line of inquiry due to the
distinctive uncertainty and volatility that characterise their tenures.

Two dominant perspectives have emerged in the literature to explain
leader survival. The first emphasises objective structural factors and
material resources, such as individual competence
(\citeproc{ref-yu2016}{Yu and Jong-A-Pin 2016}), societal stability
(\citeproc{ref-arriola2009}{Arriola 2009}), economic development
(\citeproc{ref-palmer1999}{Palmer and Whitten 1999};
\citeproc{ref-williams2011}{Williams 2011}), natural resource wealth
(\citeproc{ref-smith2004}{Smith 2004};
\citeproc{ref-quirozflores2012}{Quiroz Flores and Smith 2012};
\citeproc{ref-wright2013}{Wright, Frantz, and Geddes 2013}), and
external support (\citeproc{ref-licht2009}{Licht 2009};
\citeproc{ref-wright2008}{Wright 2008}; \citeproc{ref-thyne2017}{C.
Thyne et al. 2017}). The second perspective focuses on subjective
dimensions and strategic choices, including policy decisions, management
of opposition, and mechanisms for consolidating authority
(\citeproc{ref-gandhi2007}{Gandhi and Przeworski 2007};
\citeproc{ref-morrison2009}{Morrison 2009};
\citeproc{ref-escribuxe0-folch2013}{Escribà-Folch 2013};
\citeproc{ref-davenport2021}{Davenport, RezaeeDaryakenari, and Wood
2021}).

Coups, a critical form of irregular leadership transition, have garnered
substantial scholarly attention. Research has examined strategies for
coup prevention (\citeproc{ref-powell2017}{J. Powell 2017};
\citeproc{ref-sudduth2017}{Sudduth 2017}; \citeproc{ref-debruin2020}{De
Bruin 2020}), as well as the effects of coups on leadership trajectories
and the subsequent behaviour of coup leaders
(\citeproc{ref-sudduth2017}{Sudduth 2017};
\citeproc{ref-sudduth2018}{Sudduth and Bell 2018};
\citeproc{ref-easton2018}{Easton and Siverson 2018}).

Despite this body of work, a significant lacuna remains in the
comparative analysis of leadership survival between coup-installed and
autocoup leaders. This study seeks to address this gap by examining and
comparing the tenure lengths of leaders emerging from these two distinct
forms of irregular power acquisition.

By centring its analysis on the survival of coup-installed versus
autocoup leaders, this research aims to enhance our understanding of
political longevity in the context of irregular leadership transitions.
Such a focus promises to yield important insights into the strategic and
structural conditions that underpin leadership durability in diverse
political environments.

\section{Survival dynamics of autocoup and coup-installed
leaders}\label{survival-dynamics-of-autocoup-and-coup-installed-leaders}

The study of leadership survival within political systems poses
significant methodological and conceptual challenges, owing to the
opaque and complex nature of power transitions. These very challenges,
however, underscore the importance of such inquiry, as it illuminates
the often-neglected dynamics of political leadership. While the survival
trajectories of individual leaders vary considerably, discernible
patterns can be identified. Leaders emerging from similar origins or
operating within comparable regime types frequently display analogous
characteristics, thereby enabling systematic and meaningful comparative
analysis.

\subsection*{Key definitions and scope}\label{key-definitions-and-scope}
\addcontentsline{toc}{subsection}{Key definitions and scope}

Prior to undertaking a comparative analysis, it is essential to
establish clear definitions of key terms to ensure conceptual clarity
and analytical coherence. The definitions employed in this chapter align
with those presented in Chapter 2.

Autocoup leaders are defined as incumbent rulers who utilise
extra-constitutional measures to prolong their tenure in office. In
contrast, coup-installed leaders are those who ascend to power following
a successful coup, irrespective of whether they personally orchestrated
or participated in the coup. This inclusive definition encompasses both
coup perpetrators and individuals subsequently appointed to lead,
thereby offering a comprehensive perspective on leadership following
violent or forceful regime change.

Three clarifications are warranted in delineating the analytical scope.
First is about the minimum tenure threshold. To facilitate a meaningful
and robust analysis, the study imposes a minimum threshold of six months
in office for both autocoup and coup-installed leaders. This criterion
serves to exclude brief or interim leadership episodes that are less
analytically relevant to the study of survival dynamics, thereby
enhancing the reliability of the findings.

Second is the potential overlap in leadership categories. Some cases may
present ambiguities due to overlapping leadership pathways. A notable
example is Zine El Abidine Ben Ali, who assumed the presidency of
Tunisia in 1987 following a bloodless coup that removed President Habib
Bourguiba on grounds of ill health. In 2002, Ben Ali further
consolidated power through a constitutional referendum that removed term
limits and raised the presidential age cap from 70 to 75 years
(\citeproc{ref-bonci2019}{Bonci and Cavatorta 2019}). This latter
manoeuvre could be construed as an autocoup. Nevertheless, since Ben Ali
initially came to power via the 1987 coup and remained in office
continuously, he is classified in this study as a coup-installed leader.
To preserve analytical consistency and prevent category overlap, this
study adopts the rule that any leader who initially acquires office
through a coup is categorised as coup-installed, even if they later
consolidate or extend their rule through elections or
extra-constitutional means.

Third is the focus on post-event tenure. The analysis compares the
post-autocoup tenure of autocoup leaders with the post-coup tenure of
coup-installed leaders. Any period served by autocoup leaders prior to
the tenure-extending manoeuvre is excluded. This approach ensures a
like-for-like comparison by focusing on the period of leadership
characterised by irregular legitimacy and heightened political
uncertainty. Both categories of leaders share key characteristics---such
as limited institutional legitimacy, increased exposure to instability,
and dependence on coercive or extra-legal mechanisms---which render the
comparison analytically fruitful.

\subsection*{Challenges in power
consolidation}\label{challenges-in-power-consolidation}
\addcontentsline{toc}{subsection}{Challenges in power consolidation}

Both autocoup and coup-installed leaders encounter distinct challenges
in consolidating power, largely arising from the differing intensity of
issues related to illegitimacy, uncertainty, and instability. These
disparities create an uneven political landscape, placing coup-installed
leaders at a marked disadvantage. Table~\ref{tbl-leaders} presents a
comparative overview of the principal characteristics of autocoup and
coup-installed leaders, highlighting these critical differences.

\blandscape

\begin{table}

\caption{\label{tbl-leaders}Main features of autocoup and coup-installed
leaders}

\centering{

\fontsize{12.0pt}{14.4pt}\selectfont
\begin{tabular*}{1\linewidth}{@{\extracolsep{\fill}}>{\raggedright\arraybackslash}p{\dimexpr 112.50pt -2\tabcolsep-1.5\arrayrulewidth}>{\raggedright\arraybackslash}p{\dimexpr 225.00pt -2\tabcolsep-1.5\arrayrulewidth}>{\raggedright\arraybackslash}p{\dimexpr 225.00pt -2\tabcolsep-1.5\arrayrulewidth}}
\toprule
Feature & Autocoup Leader & Coup Entry Leader \\ 
\midrule\addlinespace[2.5pt]
Illegitimacy & Normally attained through
lawful procedures, but
lacking consensus
legitimacy & Blatantly illegal \\ 
Uncertainty & Initially with some certainty, but decreases as the leader's age grows or health worsens & Significant uncertainty initially \\ 
Instability & Relatively stable & Unstable except when a strongman emerges or constitutional institutions are established \\ 
Balance of Power & Generally in a better position of power & Initially unclear and challenging to establish a balance \\ 
\bottomrule
\end{tabular*}

}

\end{table}%

\elandscape

\subsubsection*{Illegitimacy}\label{illegitimacy}
\addcontentsline{toc}{subsubsection}{Illegitimacy}

Although both categories of leaders face legitimacy deficits, the nature
and perception of this deficit vary considerably.

For coup-installed leaders, illegitimacy is overt and unequivocal,
stemming from the direct---often violent---seizure of power. Such abrupt
disruptions to established political norms and institutions elicit
immediate condemnation, both domestically and internationally, and cast
doubt on the regime's authority from the outset.

By contrast, autocoup leaders adopt a more covert and strategic
approach, utilising legal and institutional mechanisms to lend a veneer
of democratic legitimacy. Though often superficial, this legalistic
veneer can obscure the authoritarian nature of their actions, offering a
temporary shield from domestic opposition and international scrutiny
while they seek to consolidate power.

\subsubsection*{Uncertainty}\label{uncertainty}
\addcontentsline{toc}{subsubsection}{Uncertainty}

The irregular accession of both types of leaders generates uncertainty
regarding the durability of their rule and the modalities of succession.
However, the nature and sources of this uncertainty differ markedly.

Coup-installed leaders confront a triad of uncertainties. First, the
immediate post-coup environment frequently involves intense power
struggles within the military or ruling coalition, creating ambiguity
over who will ultimately prevail. Second, their tenure is intrinsically
unstable, threatened by internal rivalries, popular mobilisation, or the
prospect of counter-coups. Third, the absence of institutionalised
succession mechanisms exacerbates this unpredictability, heightening the
risk of future instability.

Autocoup leaders, while not entirely insulated from uncertainty,
typically face fewer ambiguities. As incumbents, they retain formal
authority post-autocoup, thereby eliminating immediate succession
questions. Moreover, autocoup leaders often articulate explicit
ambitions to prolong their rule indefinitely, or through gradual
extensions, cultivating an image of continuity. This perceived
stability---whether genuine or contrived---may foster a more predictable
political climate in the short term.

\subsubsection*{Instability}\label{instability}
\addcontentsline{toc}{subsubsection}{Instability}

The combination of legitimacy deficits and enduring uncertainty
inevitably fosters insecurity and a sense of political fragility.
Consequently, both autocoup and coup-installed leaders prioritise
strategies to stabilise their regimes. However, the scale and nature of
these challenges differ.

Coup-installed leaders typically face the formidable task of
reconfiguring political power from the ground up. This often involves
purging opponents, suppressing dissent, and restructuring institutional
frameworks. Such aggressive measures can provoke significant resistance,
alienate potential allies, and incite societal unrest. Moreover, the
imperative to appease powerful domestic and international actors may
force these leaders into precarious compromises that further undermine
their authority.

In contrast, autocoup leaders often benefit from a degree of
institutional continuity and regime loyalty. This relative stability
enables them to pursue consolidation incrementally, reducing the
likelihood of immediate backlash. While opposition may persist, autocoup
leaders are generally less exposed to existential threats in the early
stages of their extended rule, affording them greater latitude to
entrench their authority.

Understanding these contrasting challenges allows for a more refined
appreciation of the strategic environments in which irregular leaders
operate. This comparative lens provides a valuable framework for
analysing the divergent pathways to power consolidation, and the varied
tools and tactics employed by autocoup and coup-installed leaders in
navigating the precarious terrain of non-traditional political
ascension.

\subsection*{Empirical evidence and
hypothesis}\label{empirical-evidence-and-hypothesis}
\addcontentsline{toc}{subsection}{Empirical evidence and hypothesis}

Empirical evidence underscores the relative disadvantage faced by
coup-installed leaders, revealing a complex interplay between historical
patterns, difficulties in consolidating power, and variations in
leadership longevity. This section presents key empirical findings and
introduces the central hypothesis that guides this study.

Data analysis indicates a strong correlation between the frequency of
coup attempts within a given country and the likelihood of future coups.
Notably, more than one-third of all coups since 1950 have taken place in
the ten countries with the highest number of coup attempts
(\citeproc{ref-powell2011}{Powell and Thyne 2011}). This suggests a
self-reinforcing cycle of political instability, in which each
successful coup increases the probability of further attempts, thereby
cultivating an environment of persistent uncertainty for coup-installed
leaders.

The disparity in leadership duration between autocoup and coup-installed
leaders is clearly reflected in survival data. As illustrated in
Figure~\ref{fig-logrank}, leaders who extend their tenure through
autocoups remain in office, on average, approximately five years longer
than those who assume power via coups. This marked difference in tenure
highlights the distinct challenges these two categories of leaders
encounter in retaining power.

The divergent consolidation environments faced by autocoup and
coup-installed leaders contribute to a self-perpetuating cycle with
significant implications for tenure length. Coup-installed leaders
confront acute legitimacy deficits and heightened internal instability;
they often struggle to attract and retain durable support, rendering
them more susceptible to both internal dissent and external pressures.
Their comparatively shorter average tenures reinforce perceptions of
volatility and fragility. Autocoup leaders, by contrast, frequently
benefit from a superficial veneer of legality and enjoy a more
favourable starting position as incumbents. This allows them to
consolidate authority more effectively, cultivate elite and public
support, and reduce the immediate risk of displacement. Their longer
tenures further contribute to perceptions of regime stability. This
cyclical dynamic suggests that the initial method of acquiring or
extending power has long-term implications for a leader's capacity to
maintain their position.

Drawing upon these empirical observations and the theoretical framework
outlined in preceding sections, the following hypothesis is proposed:

\textbf{\emph{H4-1: Political leaders who successfully extend their
tenure through autocoups are more likely to enjoy longer extended
tenures than those who assume office through coups.}}

This hypothesis encapsulates the anticipated effects of the differing
challenges and advantages faced by coup-installed and autocoup leaders.
By empirically testing this claim, the study seeks to assess the impact
of irregular accession mechanisms on leadership survival, thereby
advancing a more nuanced understanding of political durability in
contexts of non-traditional transitions to power.

\section{Research design}\label{research-design-1}

This section outlines the methodological framework employed to test the
hypothesis that autocoup leaders exhibit longer survival times in office
than coup-installed leaders. Survival analysis is utilised to model
leadership tenure, with Cox proportional hazards models employed to
estimate the effects of leader type while controlling for relevant
covariates.

\subsection*{Methodology: Survival
analysis}\label{methodology-survival-analysis}
\addcontentsline{toc}{subsection}{Methodology: Survival analysis}

Two variants of the Cox model are employed to analyse the survival
durations of coup-installed and autocoup leaders.

\textbf{Cox proportional hazards (PH) model}: This model incorporates
only time-invariant covariates measured at the time of the leader's
entry into office. It assumes that the effects of these covariates on
the hazard rate remain constant over time.

\textbf{Time-dependent Cox model}: This model allows for the inclusion
of covariates whose values may vary over time, such as indicators of
economic performance and levels of political violence. By incorporating
temporal variation, this model offers a more dynamic and nuanced
analysis of leadership survival.

The Cox model is preferred over the Kaplan-Meier estimator due to its
capacity to account for multiple explanatory variables simultaneously.
Although the Cox model does not directly estimate the expected duration
of tenure, it estimates the hazard ratio, which reflects the relative
risk of being removed from office. A higher cumulative hazard
corresponds to a lower probability of survival, thereby capturing
critical dynamics of leadership vulnerability over time.

\subsection*{Data and variables}\label{data-and-variables-1}
\addcontentsline{toc}{subsection}{Data and variables}

The analysis relies on a set of dependent and independent variables,
complemented by a range of controls.

\textbf{Survival Time:} Measured in days, this variable captures the
length of a leader's tenure. For coup-installed leaders, the tenure is
measured from the date of their accession via coup. For autocoup
leaders, it begins on the date their original legitimate term would have
expired. For instance, Vladimir Putin assumed the presidency of Russia
in 2000, stepped down in 2008 after completing two terms, and assumed
the post of prime minister while continuing to exert de facto control.
His post-autocoup tenure, therefore, is coded as beginning in 2008.

\textbf{End Point Status:} This categorical variable indicates how a
leader's tenure ended:

\begin{itemize}
\item
  \textbf{0 = Censored:} Denotes leaders who exited office through
  regular or voluntary means, such as electoral defeat, term expiration,
  voluntary resignation due to health, or natural death.
\item
  \textbf{1 = Ousted:} Denotes leaders who were forcibly removed,
  including through coups, resignations under pressure, or
  assassination.
\end{itemize}

The key independent variable is the leader type, which categorizes
leaders into two distinct groups:

\begin{itemize}
\tightlist
\item
  \textbf{Group A = Autocoup leader}: An incumbent who extends their
  tenure through extra-constitutional means.
\item
  \textbf{Group B = Coup-installed leader}: A leader who assumes power
  through a coup, whether or not they personally participated in its
  execution.
\end{itemize}

This variable serves as the primary explanatory factor, enabling a
direct comparison of survival outcomes between the two categories of
irregular leaders.

Data for the dependent and independent variables are drawn from the
newly constructed autocoup dataset introduced in this study, as well as
the Archigos dataset (\citeproc{ref-goemans2009}{Goemans, Gleditsch, and
Chiozza 2009})and the Political Leaders and Alliances Dataset (PLAD)
(\citeproc{ref-bomprezzi2024wedded}{Bomprezzi et al. 2024}).

To isolate the effect of leader type on survival, the analysis
incorporates a set of control variables, as identified in the autocoup
analysis presented in Chapter 3. These include: regime type which is
categorised as democracy, hybrid regime, or autocracy, to account for
institutional differences that may influence leadership stability;
economic performance, measured through macroeconomic indicators such as
GDP growth, which may affect a leader's ability to retain support;
political violence, captures the extent of civil conflict, repression,
or unrest, which can threaten regime stability and leadership tenure;
population size, controls for structural differences across states that
may impact political dynamics; Polity V scores, reflects the
institutional characteristics and degree of democracy or autocracy
within a regime.

These control variables enhance the comparability and robustness of the
statistical models, ensuring that the estimated effects of leader type
are not confounded by broader political, economic, or demographic
conditions.

\section{Results and discussion}\label{results-and-discussion}

\subsection*{Model results}\label{model-results}
\addcontentsline{toc}{subsection}{Model results}

Regression results for both the Cox Proportional Hazards (PH) model and
the time-dependent Cox model, estimated using the survival package in R
(\citeproc{ref-survival}{Therneau 2024}), are presented in
Table~\ref{tbl-cox}.

\begin{table}

\caption{\label{tbl-cox}Cox models for survival time of different types
of leaders}

\centering{

\fontsize{12.0pt}{14.4pt}\selectfont
\begin{tabular*}{\linewidth}{@{\extracolsep{\fill}}lcccccccc}
\toprule
 & \multicolumn{4}{c}{\textbf{Cox PH Model}} & \multicolumn{4}{c}{\textbf{Time-dependent Cox Model}} \\ 
\cmidrule(lr){2-5} \cmidrule(lr){6-9}
\textbf{Characteristic} & \textbf{N} & \textbf{Event N} & \textbf{HR}\textsuperscript{\textit{1}} & \textbf{SE} & \textbf{N} & \textbf{Event N} & \textbf{HR}\textsuperscript{\textit{1}} & \textbf{SE} \\ 
\midrule\addlinespace[2.5pt]
{\bfseries Leader Type} &  &  &  &  &  &  &  &  \\ 
    Autocoup leaders & 61 & 21 & 1.00 & — & 559 & 21 & 1.00 & — \\ 
    Coup-installed leaders & 167 & 84 & 1.76** & 0.274 & 1,171 & 80 & 1.31 & 0.275 \\ 
{\bfseries Regime Types} &  &  &  &  &  &  &  &  \\ 
    dominant-party & 48 & 20 & 1.00 & — & 395 & 13 & 1.00 & — \\ 
    military & 38 & 19 & 2.26** & 0.351 & 356 & 36 & 2.06** & 0.356 \\ 
    personal & 64 & 30 & 1.67* & 0.296 & 749 & 43 & 1.55 & 0.327 \\ 
    presidential & 36 & 13 & 1.55 & 0.396 & 98 & 3 & 1.40 & 0.713 \\ 
    parliamentary & 18 & 9 & 2.00 & 0.448 & 27 & 1 & 1.91 & 1.07 \\ 
    other & 24 & 14 & 1.86* & 0.374 & 105 & 5 & 2.59** & 0.557 \\ 
{\bfseries GDP Growth Trend} & 228 & 105 & 0.91 & 1.86 & 1,730 & 101 & 0.07* & 1.72 \\ 
{\bfseries GDP per capita} & 228 & 105 & 0.98 & 0.010 & 1,730 & 101 & 0.98*** & 0.010 \\ 
{\bfseries Population: log} & 228 & 105 & 1.00 & 0.079 & 1,730 & 101 & 0.92 & 0.080 \\ 
{\bfseries Polity V score} & 228 & 105 & 1.00 & 0.031 & 1,730 & 101 & 1.01 & 0.027 \\ 
{\bfseries Political violence} & 228 & 105 & 0.95 & 0.051 & 1,730 & 101 & 1.09** & 0.046 \\ 
\bottomrule
\end{tabular*}
\begin{minipage}{\linewidth}
\textsuperscript{\textit{1}}*p\textless{}0.1; **p\textless{}0.05; ***p\textless{}0.01\\
Abbreviations: HR = Hazard Ratio, SE = Standard Error\\
\end{minipage}

}

\end{table}%

The two models yield divergent findings concerning the central question
of this study. The Cox PH model reveals a statistically significant
relationship between leadership type and the hazard of removal from
office ( \(p < 0.05\) ). Specifically, this model supports the
hypothesis that leaders installed through coups face a 1.76 times hazard
of removal compared to those who came to power via autocoups. However,
the time-dependent Cox model, which incorporates time-varying covariates
such as economic performance and political violence, does not find a
statistically significant relationship between leadership type and
survival in office. Given the greater robustness of the time-dependent
specification, the interpretation of the principal findings is grounded
in this model.

According to the time-dependent Cox model, and contrary to the initial
hypothesis and preliminary results, the mode of accession to power does
not significantly influence the tenure of irregularly inaugurated
political leaders once relevant covariates---particularly regime
type---are controlled for. Nevertheless, these results reinforce the
broader conclusion of Chapter 3: that the balance of power,
fundamentally shaped by regime characteristics, is central to both the
seizure and retention of political authority.

In particular, regime type emerges as a statistically significant
determinant of political survival. Leaders in military regimes exhibit a
hazard ratio of 2.06 relative to their counterparts in dominant-party
regimes, suggesting that military leaders are significantly more likely
to be ousted. This implies that, all else being equal, a military leader
faces a \(106\%\) greater risk of removal at any given point compared to
a leader within a dominant-party regime. Leaders operating within
regimes classified as ``Other''---typically encompassing transitional or
provisional arrangements---display an even higher hazard ratio of 2.59,
consistent with the inherent volatility of such political
configurations.

Economic development, proxied by GDP per capita, also exerts a
statistically significant influence. A hazard ratio of 0.98 indicates
that each additional \$10,000 in GDP per capita is associated with a
\(2\%\) reduction in the risk of removal, ceteris paribus. Conversely,
GDP growth trend only shows a marginal statistically significant
influence ( \(p<0.1\) ). Political violence, measured via the violence
index, demonstrates a positive relationship with leader removal: a
one-unit increase in the index raises the hazard of removal by
approximately \(9\%\).

Other control variables---including the logarithm of population size and
Polity V scores---do not reach statistical significance in the
time-dependent Cox model. Although these factors are theoretically
salient and frequently employed in studies of political survival, their
lack of significance in this context suggests that, under conditions of
irregular leadership transitions, more immediate variables such as
regime type and political violence may play a more decisive role. It is
plausible that the effects of structural economic growth, demographic
scale, and institutional quality are either mediated through more
proximate mechanisms or unfold over longer time horizons, rendering them
less visible in short- to medium-term analyses of leader tenure.

It is important to note that the results are contingent upon the
exclusion of very short-lived leaders---those who remained in office for
fewer than 180 days following a coup or autocoup. A significant number
of coup-installed leaders survive only for brief periods---often mere
days or months---a phenomenon that is comparatively rare among autocoup
leaders. Consequently, the exclusion criterion introduces a slight bias
in favour of coup-installed leaders. Nevertheless, this study contends
that the inclusion of such short-lived tenures would be methodologically
inappropriate. Although these leaders technically meet the minimal
threshold for a successful coup (i.e., retaining power for more than
seven days), their failure to consolidate authority suggests they did
not truly succeed in establishing post-coup rule in a meaningful or
sustained manner.

\subsection*{Discussion}\label{discussion}
\addcontentsline{toc}{subsection}{Discussion}

\begin{figure}

\begin{minipage}{0.50\linewidth}

\centering{

\pandocbounded{\includegraphics[keepaspectratio]{_coups_and_autocoups_correction_files/figure-pdf/fig-coxHR-1.pdf}}

}

\subcaption{\label{fig-coxHR-1}Cox PH Model}

\end{minipage}%
%
\begin{minipage}{0.50\linewidth}

\centering{

\pandocbounded{\includegraphics[keepaspectratio]{_coups_and_autocoups_correction_files/figure-pdf/fig-coxHR-2.pdf}}

}

\subcaption{\label{fig-coxHR-2}Time-dependent Cox Model}

\end{minipage}%

\caption{\label{fig-coxHR}Hazard ratios and 95\% CIs for Leader Ousting}

\end{figure}%

Figure~\ref{fig-coxHR} illustrates the hazard ratios and their
associated 95\% confidence intervals for the variables included in the
Cox proportional hazards model. The proximity of each hazard ratio
(represented by a dot) to 1 denotes minimal effect on the risk of
removal from office; a hazard ratio of 1 indicates no effect. The
horizontal lines denote the \(95\%\) confidence intervals, and variables
whose intervals cross the vertical reference line at 1 are not
statistically significant at the \(5\%\) level.

As previously discussed, the hazard ratios for leaders in military
regimes and ``Other'' regimes are both substantially above 1 and
statistically significant at the \(5\%\) level, confirming their
heightened vulnerability to removal. GDP per capita also attains
statistical significance, albeit with a hazard ratio very close to 1,
indicating a relatively modest substantive effect.

Although the hazard ratios for GDP growth and regime type (e.g.,
presidential or parliamentary) appear visually distant from 1, their
respective confidence intervals intersect the vertical line, indicating
a lack of statistical significance at conventional thresholds.

Most other variables display hazard ratios close to 1, suggesting that
marginal changes in these predictors do not substantially alter the
likelihood of political removal for leaders emerging from coups or
autocoups.

\subsection*{Assessing the proportional hazards
assumption}\label{assessing-the-proportional-hazards-assumption}
\addcontentsline{toc}{subsection}{Assessing the proportional hazards
assumption}

Evaluating the proportional hazards assumption is essential to ensure
the validity of the Cox model estimates. This assumption was tested
using a chi-squared test based on Schoenfeld residuals, which assesses
whether the effects of covariates remain constant over time. The results
indicate that neither the standard Cox PH model nor the time-dependent
Cox model violates this assumption. The global p-values---0.12 for the
standard model and 0.23 for the time-dependent model---exceed the
conventional \(5\%\) significance threshold, thereby confirming that the
proportional hazards assumption holds in both cases.

\section{Summary}\label{summary-2}

This chapter has examined the survival durations of political leaders
who assumed office through irregular means---specifically coups and
autocoups---by employing survival analysis techniques, including the Cox
proportional hazards model and a time-dependent Cox model. While the
standard Cox model indicated a marginally significant difference in the
risk of removal between autocoup and coup-installed leaders, this
association did not attain statistical significance in the
time-dependent model, which offers a more rigorous specification by
accounting for time-varying covariates.

The findings suggest that, once regime type and other pertinent
covariates are controlled for, the method of accession---whether via
coup or autocoup---does not independently determine leader survival.
Rather, regime type emerges as a key determinant. Leaders operating
within military or transitional (``other'') regimes face significantly
higher risks of removal than those in dominant-party systems.
Furthermore, economic development, as measured by GDP per capita, and
political violence significantly influence tenure length, whereas GDP
growth, population size, and democratic quality (as captured by Polity V
scores) do not exhibit statistically significant effects.

These results reinforce the central argument that institutional
context---particularly regime characteristics---plays a more decisive
role in shaping political longevity than the initial method of seizing
power. This conclusion is consistent with earlier qualitative
assessments and underscores the importance of integrating institutional
and structural variables into analyses of political survival.

Methodologically, the chapter illustrates the utility of time-dependent
modelling in political science, particularly where covariates evolve
over time. It also contributes to the emerging literature on autocoups
by offering one of the first systematic empirical assessments of their
implications for political survival. However, reliance on a newly
constructed dataset for autocoups introduces certain limitations,
underscoring the need for further refinement and expansion in future
research.

In sum, this chapter provides empirical evidence supporting the
proposition that regime characteristics, more than the mode of accession
alone, shape the durability of irregular political leadership. These
insights contribute to broader debates on authoritarian resilience,
democratic backsliding, and the institutional foundations of political
authority.

\chapter{Coups, Autocoups, and
Democracy}\label{coups-autocoups-and-democracy}

\section*{Abstract}\label{abstract-4}
\addcontentsline{toc}{section}{Abstract}

This chapter examines the impact of autocoups on political institutions,
drawing comparisons with conventional coups through an analysis of
changes in Polity V scores. It argues, first, that incumbent leaders
often consolidate power by systematically weakening institutional
constraints in the lead-up to an autocoup, resulting in a decline in
Polity scores even prior to the formal event. Second, in contrast to
coups---which exhibit varied outcomes with respect to
democratisation---autocoups almost invariably precipitate democratic
backsliding or the deepening of authoritarian rule. This is because
autocoup leaders, in seeking to extend their tenure beyond
constitutionally mandated limits, are explicitly motivated to dismantle
institutional checks and balances.

Utilising a country-fixed effects model and drawing on datasets
encompassing both coups and autocoups, the study demonstrates that
Polity scores tend to decline both before and after an autocoup. In
contrast, while coups typically lead to an immediate deterioration in
democratic indicators, they may also create conditions for partial
democratic recovery over time. These findings highlight the distinct
political trajectories engendered by coups and autocoups.

This research addresses a significant lacuna in the empirical study of
autocoups and contributes to scholarly and policy-oriented debates by
illuminating their deleterious effects, particularly in terms of
democratic erosion and the consolidation of authoritarianism.

\newpage

\section{Introduction}\label{introduction-4}

The preceding chapters have defined the concept of an autocoup,
introduced a novel dataset capturing such events, conducted empirical
analyses on the determinants of autocoup attempts, and compared the
post-event survival durations of coup-installed and autocoup leaders. A
logical and important extension of this inquiry concerns the broader
implications of autocoups. In particular, from a political science
perspective: how do autocoups affect democratisation processes?

As previously noted, the absence of a comprehensive and widely accepted
dataset on autocoups has meant that most discussions concerning their
consequences have relied primarily on case studies
(\citeproc{ref-baturo}{Baturo and Elgie, n.d.};
\citeproc{ref-baturo2022}{Baturo and Tolstrup 2022}). To move beyond
case-specific narratives and towards a more systematic and comparative
understanding, this chapter seeks to advance the first empirical
investigation into the democratic consequences of autocoups. The initial
objective, therefore, is to assess whether autocoups serve to entrench
authoritarian rule, facilitate democratisation, or exert no substantive
impact on regime type.

Given the conceptual and empirical parallels between coups and
autocoups, a second key aim is to compare their respective effects on
democratisation. While both phenomena disrupt existing political orders,
their immediate and longer-term consequences may diverge significantly.
Clarifying these differences is essential for evaluating their broader
political implications.

To address these questions, the study draws upon an established dataset
on coups in conjunction with a newly constructed dataset on autocoups.
Employing a fixed-effects model, it assesses their respective impacts on
levels of democracy, operationalised through the Polity Index. The
findings indicate that both coups and autocoups are associated with an
immediate decline in democratic quality. However, the short-term
negative impact of coups is more pronounced. Crucially, while
democracies affected by coups tend to show notable recovery within three
years, those subject to autocoups display no significant improvement
during the same period.

This study makes two principal contributions to the field of political
science. First, it offers the inaugural empirical analysis of the impact
of autocoups on democratisation, thereby addressing a significant gap in
the existing literature. Second, by juxtaposing the effects of coups and
autocoups---and demonstrating the more severe and enduring damage to
democratic institutions caused by the latter---it highlights the
necessity of treating autocoups as a distinct political phenomenon that
merits greater scholarly and policy-oriented attention.

The remainder of this chapter is structured as follows. Section 2
examines the impact of autocoups on democratisation, with particular
reference to their comparison with traditional coups. Section 3 outlines
the research design, methodological approach, and variables employed.
Section 4 presents the empirical findings and discusses their broader
implications. Section 5 concludes by summarising the key results and
considering their significance for understanding and responding to
autocoup dynamics.

\section{Impact of autocoups on political
change}\label{impact-of-autocoups-on-political-change}

As defined in Chapter 2, an autocoup refers to an incumbent leader
extending their tenure beyond constitutionally mandated limits through
extra-constitutional means. While the title or official position of the
leader may change, the individual in power remains the same. In contrast
to traditional coups, therefore, an autocoup does not result in genuine
leadership turnover, elite restructuring, or regime transformation. The
fundamental structure of rule remains intact.

This distinction bears significant implications. Since regime change
seldom follows an autocoup, its political impact cannot be adequately
assessed using conventional approaches. Typically, studies on coups and
democratisation evaluate outcomes by estimating the likelihood of regime
transitions---from autocracy to democracy or vice versa---as
demonstrated in earlier research on the consequences of coups
(\citeproc{ref-thyne2014}{C. L. Thyne and Powell 2014};
\citeproc{ref-derpanopoulos2016}{Derpanopoulos et al. 2016};
\citeproc{ref-miller2016}{Miller 2016}). However, this framework is
ill-suited for autocoups, which do not typically induce formal regime
transfers.

The absence of regime change does not, however, imply political stasis.
On the contrary, autocoups invariably alter political dynamics and, in
some instances, may precipitate substantial shifts. Consequently, a more
suitable method for examining their political effects is through the
analysis of democracy indices, such as those provided by the Polity5
dataset (\citeproc{ref-p1}{Monty G. Marshall and Gurr 2020}). The Polity
V score---ranging from -10 (full autocracy) to +10 (full
democracy)---captures gradual changes in regime characteristics,
enabling the assessment of more nuanced shifts in political openness and
institutional constraints, even in the absence of formal regime change.
Similar analytical approach has been adopted in prior studies
(\citeproc{ref-dahl2023}{Dahl and Gleditsch 2023b}).

Although autocoups rarely bring about overt regime transformation, their
implications for democratisation warrant serious consideration.
Importantly, their effects differ from those of traditional coups in at
least two key respects.

\subsection*{The pre-event effects versus post-event
effects}\label{the-pre-event-effects-versus-post-event-effects}
\addcontentsline{toc}{subsection}{The pre-event effects versus
post-event effects}

First, unlike coups---typically characterised by sudden, definitive
events such as the removal of a sitting leader---autocoups often unfold
incrementally rather than through a singular, dramatic act. Incumbents
intent on extending their rule generally begin preparing well in advance
of the decisive action. This preparation may involve purging political
elites, suppressing opposition parties, cracking down on dissent and
protest, and curtailing media freedom. Without such pre-emptive
measures, an autocoup would likely face robust resistance and, in the
worst case, provoke a backlash that could result in the leader's rapid
removal.

However, once the incumbent has successfully extended their tenure,
continued repression is not always required. On the contrary, some
leaders ease political restrictions in order to placate internal dissent
and minimise international condemnation. This adaptive strategy is aimed
at stabilising the post-autocoup political order.

As a result, the most significant political changes associated with
autocoups often occur prior to the final act itself. Once the autocoup
is formalised, additional changes may be limited. By contrast, coup
plotters---who are not incumbent leaders---lack the institutional power
to influence political structures beforehand. Hence, the political
consequences of coups tend to manifest in the aftermath rather than
beforehand.

This distinction is borne out by empirical examples of autocoups.

One of the most frequently cited cases of an autocoup is Peru's 1992
episode, in which President Alberto Fujimori dissolved Congress,
temporarily suspended the 1979 Constitution, and governed by decree
until November of that year, when a Democratic Constituent Congress was
elected to draft a new constitution (\citeproc{ref-cameron1998}{Maxwell
A. Cameron 1998b}). These actions, however, did not immediately result
in an extension of Fujimori's tenure. Under the provisions of the 1979
Constitution, immediate presidential re-election was prohibited. To
bypass this restriction, Fujimori initiated a process of constitutional
reform, culminating in the adoption of a new constitution in 1993 that
permitted re-election. This institutional change enabled him to secure a
second term in 1995 (\citeproc{ref-baturo2019}{Baturo 2019}).

Peru's Polity V scores illustrate the political repercussions of these
developments. When Fujimori assumed office in 1990, the score stood at 8
and remained unchanged in 1991. However, following the dissolution of
Congress in 1992, the score declined sharply to -4. Notably, with the
adoption of the new constitution in 1993, the score recovered modestly
to -1. This figure remained constant throughout the remainder of
Fujimori's presidency until his resignation in 2000, indicating neither
further deterioration nor significant recovery in democratic
institutions as reflected in the Polity V index following the
constitutional revision.

A comparable pattern is evident in Belarus under Alexander Lukashenko.
Upon taking office in 1994, Belarus recorded a Polity V score of 8. In
1995, however, Lukashenko bypassed parliamentary opposition by calling a
controversial referendum and threatening to dissolve the legislature,
prompting a drop in the score to 0. Following the 1996 referendum, which
extended his term, the score fell further to -7, where it has remained
ever since---despite two further term extensions
(\citeproc{ref-ash2014}{Ash 2014}; \citeproc{ref-baturo}{Baturo and
Elgie, n.d.}).

These cases illustrate a broader trend: the political effects of
autocoups are often manifest before the final stage of the autocoup is
executed, whereas the political consequences of traditional coups emerge
predominantly after the event. This distinction has been explored in
depth in previous studies.

Based on this analysis, the following hypothesis is proposed:

Autocoups primarily influence political change in advance of their
execution, whereas coups tend to drive political change predominantly in
their aftermath.

\textbf{\emph{H5-1: Autocoups significantly reduce Polity V scores prior
to their execution, whereas coups primarily decrease Polity V scores
immediately following the event.}}

\subsection*{Consistent autocoup outcomes versus the varied impact of
coups}\label{consistent-autocoup-outcomes-versus-the-varied-impact-of-coups}
\addcontentsline{toc}{subsection}{Consistent autocoup outcomes versus
the varied impact of coups}

Secondly, in contrast to the ambiguous effects of coups
(\citeproc{ref-dahl2023}{Dahl and Gleditsch 2023b}), autocoups seldom
contribute to democratisation.

The relationship between coups and democratisation has been widely
explored in the existing literature. Some scholars contend that
coups---or even the credible threat thereof---can act as catalysts for
democratic transition. One line of argument posits that coups generate a
political ``shock'' which may create openings for liberalisation that
would otherwise not arise (\citeproc{ref-thyne2014}{C. L. Thyne and
Powell 2014}). In a critical reappraisal, Derpanopoulos et al.
(\citeproc{ref-derpanopoulos2016}{2016}) questioned the presumed
democratising effects of coups, prompting a series of scholarly
exchanges with Miller (\citeproc{ref-miller2016}{2016}). More recently,
Dahl and Gleditsch (\citeproc{ref-dahl2023}{2023b}) advanced this debate
by suggesting that coups may be followed by either democratic or
autocratic transitions, depending in large part on the scale and
character of popular mobilisation.

A frequently cited example of a so-called ``pro-democracy coup''
occurred in Niger in February 2010, when military forces removed
President Mamadou Tandja after he had unconstitutionally extended his
rule. The Supreme Council for the Restoration of Democracy (CSRD)
assumed power, pledging a return to democratic governance. Their actions
were widely welcomed, both by the domestic opposition and international
observers, as a potential restoration of constitutional rule. In
fulfilment of this commitment, the CSRD organised competitive elections
in 2011, which brought Mahamadou Issoufou to the presidency
(\citeproc{ref-miller2016}{Miller 2016}).

While debate continues over the long-term democratic implications of
coups, it is evident that their outcomes are varied and
context-dependent. In contrast, autocoups almost never facilitate
democratic progress, nor do they lead to even marginal gains in
political freedoms. This pattern is attributable to the very nature of
autocoups, which undermine constitutional provisions for leadership
succession---most notably, term limits.

Term limits are institutional mechanisms designed to curtail the
monopolisation of political power. In both democratic and autocratic
settings, they serve as critical safeguards against executive overreach.
In democracies, term limits foster political accountability, leadership
renewal, and reduce the risk of authoritarian backsliding. In
autocracies, their enforcement can mitigate succession crises and offer
rare windows for political transition. Conversely, when term limits are
circumvented, political entrenchment becomes more likely, to the
detriment of institutional development and democratic norms.

As outlined in Table~\ref{tbl-autocoup_method} in Chapter 2, autocoups
are executed through either pseudo-legal reforms or overtly
unconstitutional means. These include amending or disregarding term
limits, postponing or annulling elections, manipulating electoral
outcomes, or refusing to accept adverse results. Although many autocoups
retain a façade of legality, they are fundamentally characterised by
their breach of constitutional constraints intended to prevent
indefinite rule.

As discussed earlier, incumbent leaders often engage in institutional
weakening well before they formally extend their tenure. Moreover, once
entrenched, they seldom reverse repressive measures or meaningfully
restore democratic freedoms, even if political pressures are temporarily
relaxed.

Case studies from Peru and Belarus underscore the tendency of autocoups
to precipitate declines in Polity scores, indicative of democratic
erosion. However, it is important to note that most autocoups (two
thirds) occur in regimes that are already authoritarian, where Polity
scores are already low. This pattern mirrors broader trends observed in
the coup literature, where coups disproportionately occur in autocratic
contexts.

For example, in China's 2018 constitutional amendment, President Xi
Jinping abolished presidential term limits, thereby allowing indefinite
tenure\footnote{\textbf{BBC News,} ``China's Xi Allowed to Remain
  `President for Life' as Term Limits Removed,'' \emph{BBC News}, March
  11, 2018, \url{https://www.bbc.co.uk/news/world-asia-china-43361276},
  accessed March 14, 2025.}. Nonetheless, China's Polity score remained
constant at -7, both before and after the amendment. This reflects the
broader tendency in highly autocratic regimes (Polity scores below -6),
where minimal democratic institutions exist and little further decline
is statistically possible.

While most autocoups occur in low-scoring regimes, some have been
recorded in relatively more democratic systems where the Polity score
remained stable despite term limit circumvention. This is particularly
evident in Latin America, where several presidents have amended ``no
immediate re-election'' provisions to allow a second consecutive term,
but later relinquished power voluntarily. Examples include Argentina
(1993), where the Polity score remained at 7; Brazil (1997), where it
held at 8; and Colombia (2004), where it stayed at 7. In these
instances, term extensions were achieved within a functioning
institutional framework, without further undermining democratic
structures (\citeproc{ref-baturo2019}{Baturo 2019}).

Across all cases mentioned above---whether in Peru, Belarus, China,
Argentina, Brazil, or Colombia---there is no instance in which the
Polity score increased following an autocoup. Within the autocoup
dataset introduced in Chapter 3, only four cases---Guinea-Bissau (1988),
Burkina Faso (1997), Congo-Brazzaville (2001), and Lebanon
(2004)---exhibited slight increases in Polity scores, but these changes
were negligible.

Unlike some coup leaders who claim democratic intent---as in Niger's
2010 example---autocoup leaders seldom, if ever, articulate such
justifications. If the advancement of democracy were truly their
objective, they would relinquish power in accordance with constitutional
limits rather than dismantle them.

Based on this analysis, the second hypothesis is proposed:

Autocoups typically result in a decline in the Polity Index and rarely
see democratic recovery, whereas coups often allow for recovery and may
even facilitate transitions from autocracy to democracy.

\textbf{\emph{H5-2: Autocoups significantly reduce Polity V scores with
minimal democratic recovery, whereas coups often allow recovery and
sometimes improving democratic conditions.}}

\section{Methodology and variables}\label{methodology-and-variables}

\subsection*{Methodology}\label{methodology-1}
\addcontentsline{toc}{subsection}{Methodology}

As previously discussed, autocoups are less likely to result in full
regime transitions---whether from democracy to autocracy or vice versa.
Consequently, it is inappropriate to evaluate their effects solely in
terms of regime shifts or changes that surpass critical thresholds.
Instead, this study assesses political change through variations in
Polity V scores.

In contrast to conventional analyses of coups, which primarily focus on
post-event consequences, this study examines both pre- and post-event
effects of autocoups. Specifically, the pre-event impact is measured by
the change in Polity V scores over the two-year period preceding the
autocoup, calculated as:

\[
Polity_{t-1} - Polity_{t-3}
\]

Similarly, the post-event effect is measured by the change in Polity V
scores over the three years following the autocoup, expressed as:

\[
Polity_{t+3} - Polity_t
\]

A three-year time frame is adopted for two principal reasons. First,
political changes preceding an autocoup generally occur incrementally,
as incumbents gradually consolidate power rather than taking abrupt
action. Second, post-event analysis is designed to capture medium-term
developments rather than immediate shocks, as autocoups rarely produce
rapid regime transitions. Instead, they typically serve to reinforce
existing political structures. Short-term fluctuations, therefore, may
be insufficient to reflect meaningful institutional transformation.

To estimate how political institutions evolve before and after
autocoups, a linear model with country-case fixed effects is employed.
For the analysis of pre-event effects, all attempted autocoups are
included, on the grounds that prior to the event, it is unknown whether
the attempt will succeed or fail.

Conversely, the analysis of post-event effects is restricted to
successful autocoups, for three main reasons. First, the vast majority
of autocoups are successful (64 out of 83 cases). Second, failed
autocoups tend to produce immediate political disturbances rather than
sustained medium- or long-term consequences. Third, failed attempts are
frequently followed by significant disruptive events---such as coups,
uprisings, or mass protests---which complicate efforts to isolate the
effects of the autocoup itself. For example, in Niger, a failed autocoup
in 2009 was swiftly followed by a coup in 2010, thereby entangling the
political consequences of both episodes. In contrast, successful
autocoups offer a more coherent analytical framework, as their effects
are less likely to be confounded by subsequent disruptions, thus
allowing for a more systematic evaluation of institutional change.

\subsection*{Variables}\label{variables}
\addcontentsline{toc}{subsection}{Variables}

This study utilises a global sample of all country-year data from 1950
to 2020, employing a linear model to examine the effects of autocoups on
political change. The dependent variable is the change in Polity scores,
while the principal independent variable is the occurrence of an
autocoup. The dataset comprises approximately 9,100 observations.

The dependent variable captures political change through two-year or
three-year differences in Polity scores. Specifically, Model 1
(pre-event effects) is specified as: \(Polity_{t-1} − Polity_{t − 3}\).
Model 2 (post-event effects) is defined as: \(Polity_{t+3} - Polity_t\).

Polity scores range from -10 (full autocracy) to +10 (full democracy).
Certain values within the dataset---namely -66, -77, and -88---represent
transitional regimes or periods of political uncertainty. To minimise
data loss, these values are replaced with the closest valid Polity
scores, thereby ensuring the analysis captures continuous changes in
institutional quality rather than focusing exclusively on transitions
that cross democratic thresholds.

The primary independent variable is the autocoup, as defined in Chapter
2. The dataset includes 83 attempted autocoups (used in the pre-event
analysis) and 64 successful autocoups (used in the post-event analysis).
For pre-event analysis, the autocoup variable is binary, coded as 1 if
an attempted autocoup occurred and 0 otherwise.

In the post-event analysis, a decay function is applied to capture both
immediate and delayed effects, following the approach of Dahl and
Gleditsch (\citeproc{ref-dahl2023}{2023b}). To assess the persistence of
autocoup impacts, a half-life specification of five years is adopted,
evaluating effects from the year of the autocoup ( \(y_t\) ) through to
four years after ( \(y_{t+4}\) ).

Traditional coups are also included as a secondary independent variable
for two main reasons. First, for comparative significance: a robust
empirical comparison requires assessing whether autocoups differ
meaningfully from coups in their political consequences. Second, due to
overlapping events: autocoups and coups often occur in close proximity
or causal sequence, making it necessary to distinguish their respective
effects. The coup dataset employed is drawn from Powell and Thyne
(\citeproc{ref-powell2011}{2011}). To ensure consistency, the same
methodological framework used for autocoups is applied to coups---binary
coding for pre-event effects and decay function coding for post-event
effects.

Control variables include economic performance, political violence, and
population size, all of which have been examined in previous chapters.
Two additional dummy variables are incorporated. The first,
`non\_democracy', accounts for regime type: countries with Polity scores
below -6 are already autocratic and thus less susceptible to further
decline, while those above 6 are more resilient to deterioration. The
second, `cold\_war', reflects temporal context, following prior research
on the effects of coups on democratisation (\citeproc{ref-thyne2014}{C.
L. Thyne and Powell 2014};
\citeproc{ref-derpanopoulos2016}{Derpanopoulos et al. 2016};
\citeproc{ref-dahl2023}{Dahl and Gleditsch 2023b}). This variable
captures broader geopolitical trends, notably the general decline in
Polity scores during the Cold War (1960s--1990), followed by a tendency
towards democratisation after 1990.

\section{Results and discussion}\label{results-and-discussion-1}

\subsection*{Pre-event effects}\label{pre-event-effects}
\addcontentsline{toc}{subsection}{Pre-event effects}

This section begins by analysing the trajectory of Polity scores in the
lead-up to autocoup events. As shown in Models 1 to 4 in
Table~\ref{tbl-demomodel}, Column 1 presents the empirical results for
pre-event effects, focusing on changes observed over the two-year period
preceding the event. Consistent with the first hypothesis, Polity V
scores demonstrate a statistically significant decline in the two years
prior to an autocoup. Notably, traditional coups exhibit no
statistically significant effects on Polity V scores in the period
preceding the events.

\begin{table}

\caption{\label{tbl-demomodel}The Impacts on
Democratization(1950--2018): Autocoups vs coups}

\centering{

\begin{tabular}{@{\extracolsep{30pt}}lcccc} 
\\[-1.8ex]\hline 
\hline \\[-1.8ex] 
 & \multicolumn{4}{c}{Dependent variable: Differences of Polity scores} \\ 
\cline{2-5} 
\\[-1.8ex] & Pre-event & Event-year & \multicolumn{2}{c}{Post-event} \\ 
 & (1) & (2) & (3) & (4) \\ 
\hline \\[-1.8ex] 
 Autocoup & $-$0.666$^{**}$ & $-$1.276$^{***}$ & $-$0.338 & $-$0.130 \\ 
  & (0.297) & (0.201) & (0.322) & (0.360) \\ 
  & & & & \\ 
 Coup & 0.106 & $-$1.312$^{***}$ & 1.203$^{***}$ & 1.868$^{***}$ \\ 
  & (0.133) & (0.091) & (0.127) & (0.183) \\ 
  & & & & \\ 
 GDP per Capita & $-$0.006$^{***}$ & $-$0.003$^{**}$ & $-$0.009$^{***}$ & $-$0.010$^{***}$ \\ 
  & (0.002) & (0.001) & (0.002) & (0.002) \\ 
  & & & & \\ 
 Economic Trend & $-$0.519 & $-$0.428 & $-$0.563 & $-$0.635 \\ 
  & (0.401) & (0.277) & (0.480) & (0.480) \\ 
  & & & & \\ 
 Log Population & 0.502$^{***}$ & 0.178$^{**}$ & 0.755$^{***}$ & 0.734$^{***}$ \\ 
  & (0.102) & (0.070) & (0.122) & (0.122) \\ 
  & & & & \\ 
 Political Violence & 0.025 & 0.015 & 0.033 & 0.033 \\ 
  & (0.020) & (0.014) & (0.024) & (0.024) \\ 
  & & & & \\ 
 Non-Democracy & 0.087 & 0.809$^{***}$ & $-$0.776$^{***}$ & $-$0.775$^{***}$ \\ 
  & (0.090) & (0.062) & (0.109) & (0.109) \\ 
  & & & & \\ 
 Cold War & $-$0.133 & $-$0.235$^{***}$ & $-$0.092 & $-$0.116 \\ 
  & (0.091) & (0.063) & (0.109) & (0.109) \\ 
  & & & & \\ 
\hline \\[-1.8ex] 
Observations & 9,104 & 9,104 & 9,104 & 9,104 \\ 
R$^{2}$ & 0.009 & 0.047 & 0.028 & 0.030 \\ 
Adjusted R$^{2}$ & $-$0.011 & 0.029 & 0.009 & 0.011 \\ 
F Statistic & 9.998$^{***}$ & 55.436$^{***}$ & 32.690$^{***}$ & 34.462$^{***}$ \\ 
\hline 
\hline \\[-1.8ex] 
\textit{Note:}  & \multicolumn{4}{r}{$^{*}$p$<$0.1; $^{**}$p$<$0.05; $^{***}$p$<$0.01} \\ 
\end{tabular}

}

\end{table}%

Model 1 specifically examines the change in Polity V scores between year
\(t−3\) (three years before the event) and year \(t−1\) (the year
immediately preceding the event), expressed as
\(Polity_{t−1} - Polity_{t−3}\) . The results indicate a statistically
significant average decline of 0.67 points in Polity scores during this
period prior to an autocoup, after controlling for other variables. By
contrast, the analysis reveals no significant change in Polity scores
prior to traditional coups, suggesting that the periods leading up to
such coups do not exert a measurable influence on levels of
democratisation.

\subsection*{Event-year effects}\label{event-year-effects}
\addcontentsline{toc}{subsection}{Event-year effects}

Model 2 evaluates the immediate impact by comparing Polity scores in the
year of the autocoup or coup with those of the preceding year,
calculated as \(Polity_{t} - Polity_{t-1}\). As previously noted, the
year in which an autocoup or coup occurs typically registers a
substantial disruption to political institutions. Accordingly, both
types of events are associated with a notable decline in Polity scores
relative to the prior year.

The magnitude of the negative effect is comparable across both event
types: on average, Polity scores decrease by 1.3 points following either
an autocoup or a coup, holding other variables constant. These findings
suggest that, although autocoups have received less scholarly and public
attention than coups, they nonetheless exert similarly significant
immediate effects on institutional democratic quality as measured by the
Polity index.

\subsection*{Mid-term effects: comparing attempted and successful
autocoups}\label{mid-term-effects-comparing-attempted-and-successful-autocoups}
\addcontentsline{toc}{subsection}{Mid-term effects: comparing attempted
and successful autocoups}

Models 3 and 4 present the empirical findings on the mid-term effects of
autocoups, assessing changes in Polity scores three years after the
events. These effects are measured as
\(Polity_{t+3} - Polity_{t}\)\hspace{0pt}, with Model 3 focusing on
attempted autocoups and Model 4 on successful ones.

Model 3 evaluates the impact of attempted autocoups. The analysis
reveals no statistically significant change in Polity scores following
such attempts. This finding contrasts with attempted coups, which are
associated with an average increase of 1.2 points in Polity scores over
the same period, controlling for other variables.

Model 4 considers the consequences of successful autocoups. Similar to
attempted autocoups, the results do not indicate any statistically
significant changes in Polity scores three years after the event. In
contrast, successful coups are linked to a positive trajectory in
democratisation, with an average increase of 1.87 points in Polity
scores within three years, all else being equal.

These findings offer several important insights. First, neither
attempted nor successful autocoups produce discernible changes in Polity
scores over the mid-term. This suggests that the effects of autocoups
are primarily front-loaded, exerting their influence in the lead-up to
the event rather than in its aftermath. Coups, by contrast, tend to
generate significant post-event dynamics.

Second, autocoups generally have a negative and unidirectional impact on
democratic institutions, primarily eroding democratic quality before the
event without subsequent recovery. In contrast, coups exhibit a more
complex temporal pattern: they often trigger an immediate shock and
decline in democratic indicators, but this is frequently followed by
recovery---or even overshooting---where the democracy index not only
returns to its pre-coup level but may surpass it within two to three
years.

\subsection*{Effects of control
variables}\label{effects-of-control-variables}
\addcontentsline{toc}{subsection}{Effects of control variables}

To ensure the robustness of the empirical findings, the models
incorporate a range of control variables. While economic trends,
political violence, and the Cold War (which is only statistically
significant in the event year) do not exhibit consistent or significant
effects on Polity scores, several other variables merit further
discussion.

The effects of GDP per capita, population size, and regime type reveal a
more nuanced pattern. Counterintuitively, higher levels of GDP per
capita are associated with lower Polity scores, whereas non-democratic
regimes and larger populations are linked to positive changes in Polity
scores. This seemingly paradoxical result can be understood by
considering the baseline differences between democratic and
non-democratic regimes.

In established democracies, Polity scores are already close to the upper
end of the scale, leaving limited scope for further increases. These
states also tend to exhibit higher GDP per capita and lower population
growth rates, factors that are typically associated with political and
institutional stability. As a result, their Polity scores tend to remain
stable over time.

By contrast, non-democratic regimes begin from a lower baseline in the
Polity index, thus allowing for greater potential improvement. These
regimes often experience weaker economic performance and more rapid
population growth, but may also be more likely to undergo political
liberalisation or reform, contributing to upward shifts in Polity
scores.

Taken together, these findings lend empirical support to both core
hypotheses. First, the principal effects of autocoups on Polity scores
materialise prior to the events themselves. Second, while the
consequences of coups for democratisation are mixed and
context-dependent, the impact of autocoups is both consistent and
unidirectional---uniformly negative.

\subsection*{Robustness tests}\label{robustness-tests}
\addcontentsline{toc}{subsection}{Robustness tests}

To evaluate the sensitivity of the main findings to alternative model
specifications, I conduct a series of robustness tests. The results
indicate that the core conclusions remain stable across these
variations.

First, I assess the effects of autocoups over a period ranging from one
to five years following the event. The analysis, as shown in Models 5 to
9 in Table~\ref{tbl-demomodel1}, reveals that the effects of coups
remain positive and statistically significant throughout the five-year
period, with a general trend of increasing magnitude over time. In
contrast, the effects of autocoups remain negative across all five years
but do not attain statistical significance at any point. This finding
corroborates the earlier hypothesis that autocoups do not lead to
increases in Polity scores.

\begin{table}

\caption{\label{tbl-demomodel1}The Impact of Autocoups on
Democratization: one to five years}

\centering{

\begin{tabular}{@{\extracolsep{10pt}}lccccc} 
\\[-1.8ex]\hline 
\hline \\[-1.8ex] 
 & \multicolumn{5}{c}{Dependent variable: Differences of Polity scores} \\ 
\cline{2-6} 
\\[-1.8ex] & \multicolumn{5}{c}{Years after the event} \\ 
 & (5) & (6) & (7) & (8) & (9) \\ 
\hline \\[-1.8ex] 
 Autocoup & $-$0.212 & $-$0.139 & $-$0.130 & $-$0.069 & 0.088 \\ 
  & (0.207) & (0.298) & (0.360) & (0.415) & (0.457) \\ 
  & & & & & \\ 
 Coup & 0.452$^{***}$ & 1.250$^{***}$ & 1.868$^{***}$ & 2.121$^{***}$ & 2.334$^{***}$ \\ 
  & (0.108) & (0.153) & (0.183) & (0.206) & (0.227) \\ 
  & & & & & \\ 
 GDP per Capita & $-$0.004$^{***}$ & $-$0.007$^{***}$ & $-$0.010$^{***}$ & $-$0.014$^{***}$ & $-$0.018$^{***}$ \\ 
  & (0.001) & (0.002) & (0.002) & (0.003) & (0.003) \\ 
  & & & & & \\ 
 Economic Trend & $-$0.211 & $-$0.392 & $-$0.635 & $-$0.902$^{*}$ & $-$1.488$^{**}$ \\ 
  & (0.280) & (0.400) & (0.480) & (0.541) & (0.593) \\ 
  & & & & & \\ 
 Log Population & 0.297$^{***}$ & 0.554$^{***}$ & 0.734$^{***}$ & 0.899$^{***}$ & 1.062$^{***}$ \\ 
  & (0.071) & (0.101) & (0.122) & (0.137) & (0.150) \\ 
  & & & & & \\ 
 Political Violence & 0.010 & 0.022 & 0.033 & 0.051$^{*}$ & 0.063$^{**}$ \\ 
  & (0.014) & (0.020) & (0.024) & (0.027) & (0.030) \\ 
  & & & & & \\ 
 Non-Democracy & 0.758$^{***}$ & 0.034 & $-$0.775$^{***}$ & $-$1.499$^{***}$ & $-$2.128$^{***}$ \\ 
  & (0.064) & (0.090) & (0.109) & (0.122) & (0.134) \\ 
  & & & & & \\ 
 Cold War & $-$0.225$^{***}$ & $-$0.143 & $-$0.116 & $-$0.130 & $-$0.179 \\ 
  & (0.064) & (0.091) & (0.109) & (0.123) & (0.135) \\ 
  & & & & & \\ 
\hline \\[-1.8ex] 
Observations & 9,104 & 9,104 & 9,104 & 9,104 & 9,104 \\ 
R$^{2}$ & 0.023 & 0.016 & 0.030 & 0.048 & 0.065 \\ 
Adjusted R$^{2}$ & 0.003 & $-$0.004 & 0.011 & 0.029 & 0.047 \\ 
F Statistic & 25.940$^{***}$ & 17.745$^{***}$ & 34.462$^{***}$ & 55.683$^{***}$ & 77.997$^{***}$ \\ 
\hline 
\hline \\[-1.8ex] 
\textit{Note:}  & \multicolumn{5}{r}{$^{*}$p$<$0.1; $^{**}$p$<$0.05; $^{***}$p$<$0.01} \\ 
\end{tabular}

}

\end{table}%

\begin{table}

\caption{\label{tbl-demomodel2}The Impact of Autocoups on
Democratization: Dummy autocoups}

\centering{

\begin{tabular}{@{\extracolsep{20pt}}lcccc} 
\\[-1.8ex]\hline 
\hline \\[-1.8ex] 
 & \multicolumn{4}{c}{Dependent variable: Differences of Polity scores} \\ 
\cline{2-5} 
\\[-1.8ex] & \multicolumn{2}{c}{Attempted} & \multicolumn{2}{c}{Succeeded} \\ 
 & (10) & (11) & (12) & (13) \\ 
\hline \\[-1.8ex] 
 Autocoup & $-$0.093 & $-$0.148 & $-$0.043 & $-$0.057 \\ 
  & (0.298) & (0.359) & (0.335) & (0.402) \\ 
  & & & & \\ 
 Coup & 0.814$^{***}$ & 1.240$^{***}$ & 0.959$^{***}$ & 1.712$^{***}$ \\ 
  & (0.132) & (0.157) & (0.180) & (0.215) \\ 
  & & & & \\ 
 GDP per Capita & $-$0.006$^{***}$ & $-$0.010$^{***}$ & $-$0.006$^{***}$ & $-$0.010$^{***}$ \\ 
  & (0.002) & (0.002) & (0.002) & (0.002) \\ 
  & & & & \\ 
 Economic Trend & $-$0.337 & $-$0.569 & $-$0.405 & $-$0.629 \\ 
  & (0.402) & (0.482) & (0.402) & (0.482) \\ 
  & & & & \\ 
 Log Population & 0.667$^{***}$ & 0.890$^{***}$ & 0.650$^{***}$ & 0.879$^{***}$ \\ 
  & (0.105) & (0.126) & (0.105) & (0.126) \\ 
  & & & & \\ 
 Political Violence & 0.027 & 0.044$^{*}$ & 0.028 & 0.046$^{*}$ \\ 
  & (0.020) & (0.024) & (0.020) & (0.024) \\ 
  & & & & \\ 
 Regime: Military & $-$0.122 & $-$0.545$^{***}$ & $-$0.118 & $-$0.584$^{***}$ \\ 
  & (0.148) & (0.177) & (0.148) & (0.178) \\ 
  & & & & \\ 
 \hspace{1.5cm} Personal & $-$0.343$^{**}$ & $-$0.532$^{***}$ & $-$0.330$^{**}$ & $-$0.526$^{***}$ \\ 
  & (0.136) & (0.164) & (0.136) & (0.164) \\ 
  & & & & \\ 
 \hspace{1.5cm} Presidential & $-$0.153 & 0.399$^{**}$ & $-$0.152 & 0.381$^{**}$ \\ 
  & (0.132) & (0.158) & (0.132) & (0.158) \\ 
  & & & & \\ 
 \hspace{1.5cm} Parliamentary & 0.131 & 0.965$^{***}$ & 0.136 & 0.966$^{***}$ \\ 
  & (0.152) & (0.182) & (0.152) & (0.182) \\ 
  & & & & \\ 
 \hspace{1.5cm} Other & 1.077$^{***}$ & 1.094$^{***}$ & 1.086$^{***}$ & 1.115$^{***}$ \\ 
  & (0.165) & (0.199) & (0.165) & (0.199) \\ 
  & & & & \\ 
 Cold War & $-$0.048 & $-$0.002 & $-$0.054 & $-$0.011 \\ 
  & (0.092) & (0.111) & (0.092) & (0.111) \\ 
  & & & & \\ 
\hline \\[-1.8ex] 
Observations & 9,036 & 9,036 & 9,036 & 9,036 \\ 
R$^{2}$ & 0.020 & 0.033 & 0.019 & 0.033 \\ 
Adjusted R$^{2}$ & 0.0004 & 0.014 & $-$0.001 & 0.014 \\ 
F Statistic & 15.237$^{***}$ & 25.244$^{***}$ & 14.407$^{***}$ & 25.364$^{***}$ \\ 
\hline 
\hline \\[-1.8ex] 
\textit{Note:}  & \multicolumn{4}{r}{$^{*}$p$<$0.1; $^{**}$p$<$0.05; $^{***}$p$<$0.01} \\ 
\end{tabular}

}

\end{table}%

Subsequently, I refine the operationalisation of the autocoup variable
by replacing the decay function with a binary indicator distinguishing
between attempted and successful autocoups. Additionally, I disaggregate
the `non-democracy' category into more specific regime types---namely
dominant-party, military, personalist, presidential, parliamentary, and
``other'' regimes---in line with the autocoup determinants analysis,
using dominant-party as the reference category. The results of these
models are presented Models 10 to 13 in Table~\ref{tbl-demomodel2}.

Models 10 and 11 in Table~\ref{tbl-demomodel2} assess the effects of
attempted autocoups, measured two and three years after the event,
respectively, while Models 12 and 13 focus on successful autocoups. As
with the primary models, these specifications do not alter the main
findings: autocoups continue to exhibit no statistically significant
effects on Polity scores two or three years post-event, whereas coups
consistently produce positive and statistically significant increases in
the same periods. The control variables GDP per capita and population
size likewise yield results consistent with the baseline models.

However, political violence and regime type display differing effects
depending on the time horizon. Political violence exhibits marginal
statistical significance in the three-year post-event models (Models 11
and 13) but not in the two-year models (Models 10 and 12) . Although
modest in magnitude, a one-point increase in the political violence
index is associated with a 0.045 increase in the Polity score after
three years.

Among the disaggregated regime types, military regimes consistently
correlate with declines in Polity scores, though statistical
significance emerges only in the three-year period and does not differ
between attempted and successful events. Personalist and ``other''
regimes demonstrate consistent and statistically significant effects
across all four models, albeit in opposite directions: personalist
regimes are associated with negative effects, while ``other'' regimes
(typically transitional or provisional regimes) show positive effects.
This suggests that transitional regimes frequently move towards greater
levels of democracy. Finally, both presidential and parliamentary
democracies display positive and statistically significant effects in
the three-year models, though not in the two-year models.

\section{Summary}\label{summary-3}

This chapter investigates the impact of autocoups on political
institutions---particularly in comparison to coups---by analysing their
effects on changes in Polity V scores. It tests two central hypotheses.
First, that unlike coups, the effects of autocoups on Polity scores
primarily manifest prior to the event, suggesting that leaders
consolidate power in anticipation of executing an autocoup. Second,
whereas coups tend to produce a ``U-shaped'' effect---marked by an
initial decline in Polity scores followed by a recovery to prior levels
or beyond---autocoups are consistently associated with sustained
declines, indicating democratic backsliding or authoritarian
entrenchment.

To evaluate these hypotheses, the chapter conducts a series of
robustness checks, including variations in the time horizon, alternative
model specifications, and different variable operationalisations. The
core finding is that these alternative models yield results consistent
with the main analysis.

The implications of these findings are significant for both academic
scholarship and policy-making. While coups have traditionally received
considerable attention in studies of democratisation, this chapter
contends that autocoups merit equal---if not greater---focus due to
their systematic role in reversing democratic progress. Unlike coups,
which may occasionally serve as catalysts for reform and transitions
from autocracy to democracy, autocoups almost invariably consolidate
authoritarian rule, weaken political institutions, and erode democratic
governance. Furthermore, as demonstrated in Chapter 4, leaders who carry
out autocoups tend to remain in power for nearly ten years on average,
compared to less than seven years for coup-installed leaders. This
suggests that the detrimental effects of autocoups may be more enduring
than those of coups.

This chapter also makes a methodological contribution to the study of
democratic processes by highlighting the importance of pre-event
dynamics in analysing the effects of political shocks. Previous research
has largely concentrated on post-event consequences; by contrast, this
chapter demonstrates that significant political changes can precede
events such as autocoups.

Nevertheless, certain limitations remain that future research should
address. Coups and autocoups sometimes occur in close
succession---either in the same year or within a short interval---which
complicates efforts to isolate their respective impacts on Polity
scores. Disentangling these overlapping effects remains a methodological
challenge and represents a valuable avenue for future inquiry.

In conclusion, this chapter strengthens the argument that autocoups
represent a critical yet under-explored mechanism of democratic
backsliding. As such, they warrant further investigation to better
understand their implications for global patterns of democratisation and
authoritarian resilience.

\chapter{Conclusion}\label{conclusion}

This study has systematically compared coups and autocoups, two distinct
forms of irregular leadership transitions, to deepen our understanding
of how power dynamics shape leadership change, survival, and their
impact on political institutions. Through conceptual refinement, the
construction of a novel dataset, and multi-faceted empirical analysis,
this thesis has shed light on the similarities and differences between
coups and autocoups in terms of their drivers, the fate of leaders, and
their democratic consequences.

\section{Main findings}\label{main-findings}

A central contribution of this research lies in the incorporation of
autocoups---an often overlooked phenomenon---into the analytical
framework of irregular transfers of power. Based on this framework, the
study yields the following principal findings:

First, in terms of conceptualisation and empirical grounding, this study
addresses the prevailing conceptual fragmentation marked by the
proliferation of overlapping and inconsistently applied terms such as
self-coup, autogolpe, and executive aggrandisement, alongside a dearth
of systematic data (\citeproc{ref-marsteintredet2019}{Marsteintredet and
Malamud 2019}; \citeproc{ref-baturo2022}{Baturo and Tolstrup 2022}). To
resolve this, \textbf{Chapter 2} proposes a more analytically rigorous
definition of an autocoup: the act of an incumbent leader extending
their constitutionally mandated term through extra-constitutional means.
Building upon this definition, this chapter introduces and makes
publicly available the first global dataset on autocoup events,
encompassing 83 attempted (64 successful) cases from 1945 to 2023. This
provides a robust empirical foundation for future quantitative analyses.

Second, to address the paucity of large-N empirical studies on the
determinants of autocoup attempts, \textbf{Chapter 3} presents
pioneering research which finds that regime type significantly affects
the likelihood of autocoup occurrence. In contrast to traditional
coups---which typically emerge in unstable or fragmented political
systems such as military regimes (\citeproc{ref-powell2012}{J. Powell
2012}; \citeproc{ref-frantz2017}{Frantz and Stein 2017a};
\citeproc{ref-powell2018}{Powell et al. 2018};
\citeproc{ref-thyne2019}{Thyne and Powell 2019};
\citeproc{ref-kim2021}{Kim and Sudduth 2021})---autocoups are more
prevalent in regimes characterised by concentrated and relatively stable
executive power. The empirical analysis demonstrates that presidential
democracies and personalist authoritarian regimes are significantly more
susceptible to autocoup attempts than dominant-party regimes. This
pattern reflects the fragility of institutional constraints and the
centralisation of authority in such systems.

Third, although leadership survival has been extensively examined in the
literature (\citeproc{ref-buenodemesquita2003}{Bueno de Mesquita et al.
2003}; \citeproc{ref-svolik2014}{Svolik 2014};
\citeproc{ref-frantz2016}{Frantz and Stein 2016};
\citeproc{ref-sudduth2018}{Sudduth and Bell 2018};
\citeproc{ref-davenport2021}{Davenport, RezaeeDaryakenari, and Wood
2021}), few studies have focused on post-autocoup leader survival.
\textbf{Chapter 4} addresses this gap by investigating tenure
differences between coup-installed and autocoup leaders. While
preliminary analysis suggested that autocoup leaders tend to remain in
office longer, a time-dependent Cox proportional hazards
model---controlling for key covariates such as regime type---found no
statistically significant difference in the risk of removal based solely
on the method of power acquisition. Rather, regime characteristics
emerged as the critical determinant of leadership survival: leaders in
military and transitional regimes face a significantly higher risk of
removal than their counterparts in dominant-party regimes. These
findings suggest that the durability of irregular leadership depends
more on institutional context than on the particular mode of power
seizure.

Finally, whereas the existing literature has predominantly examined the
effects of coups on democratic outcomes (\citeproc{ref-thyne2014}{C. L.
Thyne and Powell 2014}; \citeproc{ref-derpanopoulos2016}{Derpanopoulos
et al. 2016}; \citeproc{ref-miller2016}{Miller 2016};
\citeproc{ref-dahl2023a}{Dahl and Gleditsch 2023a}), \textbf{Chapter 5}
undertakes a novel analysis of the democratic consequences of autocoups.
The findings reveal that autocoups differ markedly from traditional
coups in their democratic impact. Specifically, autocoups are frequently
preceded by a gradual erosion of democratic institutions (pre-event
effects) and are consistently associated with sustained democratic
backsliding or the consolidation of authoritarian rule (post-event
effects). In contrast, while traditional coups often cause an immediate
decline in democratic indicators, they may, in some cases, facilitate
conditions conducive to democratic recovery or transition. This
underscores the uniquely detrimental and enduring role that autocoups
play in undermining democratic governance.

\section{Policy implications}\label{policy-implications-1}

The findings of this study not only enhance academic understanding of
irregular transfers of power but also offer significant insights for
policy-makers, particularly in addressing global democratic backsliding
and reinforcing the resilience of political institutions.

First, the importance of institutional design is underscored.
\textbf{Chapter 3} demonstrates that regimes characterised by highly
centralised executive authority are more vulnerable to autocoups,
echoing earlier scholarship that highlights the pivotal role of regime
type and institutional architecture (\citeproc{ref-geddes1999a}{Geddes
1999a}; \citeproc{ref-frantz2017a}{Frantz and Stein 2017b}).
Strengthening mechanisms of horizontal accountability---such as
independent legislatures, judiciaries, and oversight bodies---is thus
essential for limiting executive overreach and preventing incumbents
from circumventing constitutional constraints. Key institutional
safeguards include robust and enforceable term limits, a vibrant and
empowered civil society, and codified, transparent procedures for
political succession. These components are critical in constructing
institutional bulwarks against autocoups and the entrenchment of
authoritarian rule.

Second, international and regional responses must become more nuanced
and proactive. While the global community has developed relatively
standardised mechanisms for addressing military coups, \textbf{Chapters
4 and 5} highlight that autocoups often unfold more subtly and
incrementally. Consequently, international and regional
organisations---such as the African Union (AU), the Organisation of
American States (OAS), and the European Union (EU)---must adopt a more
vigilant posture (\citeproc{ref-wobig2014}{Wobig 2014};
\citeproc{ref-shannon2014}{Shannon et al. 2014};
\citeproc{ref-thyne2017}{C. Thyne et al. 2017}). Beyond condemning and
sanctioning overt military takeovers, these bodies should exert
sustained diplomatic and economic pressure against attempts to subvert
constitutional term limits through formal amendments, manipulated
electoral processes, or other means that undermine democratic integrity.
In doing so, they can play a more active role in upholding democratic
norms and deterring executive aggrandisement.

Finally, as \textbf{Chapter 5} illustrates, the use of gradual and
incremental monitoring tools---such as Polity scores---warrants greater
attention (\citeproc{ref-dahl2023a}{Dahl and Gleditsch 2023a}).
Autocoups lack the abrupt and visible character of military coups;
instead, they often proceed under the guise of legality and
institutional continuity, making them more difficult to detect and
counter in real time. Accordingly, policy-makers, scholars, and civil
society actors should prioritise the systematic observation of
democratic indicators, including nuanced shifts in Polity scores,
Freedom House ratings, and V-Dem indices. This form of monitoring is
essential for identifying early warning signs of executive overreach and
the gradual erosion of democratic safeguards, thereby enabling the
timely implementation of preventative interventions.

\section{Limitations and directions for future
research}\label{limitations-and-directions-for-future-research}

While this study has made progress in the conceptualization, data, and
empirical analysis of irregular leadership transitions, it also has
limitations that suggest avenues for future research.

Firstly, there is still room for refinement of concepts and datasets.
The concept of autocoup proposed in \textbf{Chapter 2} and the dataset
constructed are preliminary; further discussion and consensus within the
academic community are needed to refine the boundaries of autocoups.
Specifically, how to handle ``borderline cases'' where term limits are
circumvented through legal or quasi-legal means requires more detailed
coding rules and judgement criteria. Future research can further refine
the autocoup dataset by adding cases and cross-validating with existing
datasets.

Secondly, methodological challenges need to be addressed. In analysing
the determinants of autocoup attempts, \textbf{Chapter 3} treated
autocoup attempts as a binary outcome. However, autocoups can be a
continuous process, and leaders may make multiple attempts. Future
research could explore more refined time-series analysis methods to
capture the dynamics and sequencing of autocoup behaviour. Furthermore,
coups and autocoups are sometimes closely linked, and effectively
isolating their respective impacts remains a persistent methodological
challenge.

Thirdly, expanding the scope of research is important. This study
primarily focused on term extension as a specific form of executive
power expansion, but leadership power expansion also includes other
forms that do not involve overt term extension (e.g., weakening the
legislature, controlling the judiciary)
(\citeproc{ref-marsteintredet2019}{Marsteintredet and Malamud 2019};
\citeproc{ref-baturo2022}{Baturo and Tolstrup 2022}). These forms of
power expansion themselves impact institutional resilience and
democratic quality and may be related to attempts at term prolongation.
Future research could attempt to construct a more comprehensive dataset
on executive power expansion and integrate it into the analytical
framework to more fully understand the mechanisms of authoritarian
consolidation and democratic backsliding.

Through the comparative analysis of coups and autocoups, this study has
enhanced our understanding of the complexities of irregular power
transfers. Autocoups, as a distinct strategy of authoritarian
consolidation, with their gradual, hidden nature and continuous erosion
of democratic institutions, represent a significant challenge to
contemporary global democracy. This research provides an empirical
foundation for understanding the drivers and consequences of autocoups
and aims to stimulate further research to collectively address this
important political phenomenon. Through continued conceptual
clarification, data accumulation, and mechanism exploration, we can
better understand the resilience of authoritarianism, the fragility of
democratic institutions, and the complex pathways of political change.

\chapter*{References}\label{references}
\addcontentsline{toc}{chapter}{References}

\phantomsection\label{refs}
\begin{CSLReferences}{1}{0}
\bibitem[\citeproctext]{ref-albrecht2014}
Albrecht, Holger. 2014a. {``Does Coup-Proofing Work?
Political{\textendash}Military Relations in Authoritarian Regimes Amid
the Arab Uprisings.''} \emph{Mediterranean Politics} 20 (1): 36--54.
\url{https://doi.org/10.1080/13629395.2014.932537}.

\bibitem[\citeproctext]{ref-albrecht2014a}
---------. 2014b. {``The Myth of Coup-Proofing.''} \emph{Armed Forces \&
Society} 41 (4): 659--87.
\url{https://doi.org/10.1177/0095327x14544518}.

\bibitem[\citeproctext]{ref-antonio2021}
Antonio, Robert J. 2021. {``Democracy and Capitalism in the Interregnum:
Trump{'}s Failed Self-Coup and After.''} \emph{Critical Sociology} 48
(6): 937--65. \url{https://doi.org/10.1177/08969205211049499}.

\bibitem[\citeproctext]{ref-arriola2009}
Arriola, Leonardo R. 2009. {``Patronage and Political Stability in
Africa.''} \emph{Comparative Political Studies} 42 (10): 1339--62.
\url{https://doi.org/10.1177/0010414009332126}.

\bibitem[\citeproctext]{ref-ash2014}
Ash, Konstantin. 2014. {``The Election Trap: The Cycle of Post-Electoral
Repression and Opposition Fragmentation in Lukashenko's Belarus.''}
\emph{Democratization} 22 (6): 1030--53.
\url{https://doi.org/10.1080/13510347.2014.899585}.

\bibitem[\citeproctext]{ref-baturo2019}
Baturo, Alexander. 2019. {``Continuismo in Comparison.''} In, 75--100.
Oxford University Press.
\url{https://doi.org/10.1093/oso/9780198837404.003.0005}.

\bibitem[\citeproctext]{ref-baturo}
Baturo, Alexander, and Robert Elgie. n.d. {``The Politics of
Presidential Term Limits.''}

\bibitem[\citeproctext]{ref-baturo2022}
Baturo, Alexander, and Jakob Tolstrup. 2022. {``Incumbent Takeovers.''}
\emph{Journal of Peace Research} 60 (2): 373--86.
\url{https://doi.org/10.1177/00223433221075183}.

\bibitem[\citeproctext]{ref-bermeo2016}
Bermeo, Nancy. 2016. {``On Democratic Backsliding.''} \emph{Journal of
Democracy} 27 (1): 5--19. \url{https://doi.org/10.1353/jod.2016.0012}.

\bibitem[\citeproctext]{ref-bomprezzi2024wedded}
Bomprezzi, Pietro, Axel Dreher, Andreas Fuchs, Teresa Hailer, Andreas
Kammerlander, Lennart Kaplan, Silvia Marchesi, Tania Masi, Charlotte
Robert, and Kerstin Unfried. 2024. {``Wedded to Prosperity? Informal
Influence and Regional Favoritism.''} Discussion Paper. CEPR.

\bibitem[\citeproctext]{ref-bonci2019}
Bonci, Alessandra, and Francesco Cavatorta. 2019. {``The Politics of
Presidential Term Limits in Tunisia.''} In, 179--98. Oxford University
PressOxford. \url{https://doi.org/10.1093/oso/9780198837404.003.0010}.

\bibitem[\citeproctext]{ref-brown2015}
Brown, Cameron S., Christopher J. Fariss, and R. Blake McMahon. 2015.
{``Recouping After Coup-Proofing: Compromised Military Effectiveness and
Strategic Substitution.''} \emph{International Interactions} 42 (1):
1--30. \url{https://doi.org/10.1080/03050629.2015.1046598}.

\bibitem[\citeproctext]{ref-brown2001}
Brown, Stephen. 2001. {``Authoritarian Leaders and Multiparty Elections
in Africa: How Foreign Donors Help to Keep Kenya's Daniel Arap Moi in
Power.''} \emph{Third World Quarterly} 22 (5): 725--39.
\url{https://doi.org/10.1080/01436590120084575}.

\bibitem[\citeproctext]{ref-buenodemesquita2003}
Bueno de Mesquita, Bruce, Alastair Smith, Randolph M. Siverson, and
James D. Morrow. 2003. \emph{The Logic of Political Survival}. The MIT
Press. \url{https://doi.org/10.7551/mitpress/4292.001.0001}.

\bibitem[\citeproctext]{ref-cameron1998a}
Cameron, Maxwell A. 1998a. {``Latin American Autogolpes : Dangerous
Undertows in the Third Wave of Democratisation.''} \emph{Third World
Quarterly} 19 (2): 219--39.
\url{https://doi.org/10.1080/01436599814433}.

\bibitem[\citeproctext]{ref-cameron1998}
Cameron, Maxwell A. 1998b. {``Self-Coups: Peru, Guatemala, and
Russia.''} \emph{Journal of Democracy} 9 (1): 125--39.
\url{https://doi.org/10.1353/jod.1998.0003}.

\bibitem[\citeproctext]{ref-carey2015}
Carey, Sabine C., Michael P. Colaresi, and Neil J. Mitchell. 2015.
{``Risk Mitigation, Regime Security, and Militias: Beyond
Coup-Proofing.''} \emph{International Studies Quarterly}, August, n/a--.
\url{https://doi.org/10.1111/isqu.12210}.

\bibitem[\citeproctext]{ref-cassani2020}
Cassani, Andrea. 2020. {``Autocratisation by Term Limits Manipulation in
Sub-Saharan Africa.''} \emph{Africa Spectrum} 55 (3): 228--50.
\url{https://doi.org/10.1177/0002039720964218}.

\bibitem[\citeproctext]{ref-chaisty2019}
Chaisty, Paul. 2019. {``The Uses and Abuses of Presidential Term Limits
in Russian Politics.''} In, 385--402. Oxford University PressOxford.
\url{https://doi.org/10.1093/oso/9780198837404.003.0019}.

\bibitem[\citeproctext]{ref-cheeseman2015}
Cheeseman, Nic. 2015. {``Democracy in Africa,''} March.
\url{https://doi.org/10.1017/cbo9781139030892}.

\bibitem[\citeproctext]{ref-cheeseman2019}
---------. 2019. {``Should I Stay or Should I Go? Term Limits,
Elections, and Political Change in Kenya, Uganda, and Zambia.''} In,
311--38. Oxford University PressOxford.
\url{https://doi.org/10.1093/oso/9780198837404.003.0016}.

\bibitem[\citeproctext]{ref-cheeseman2019a}
Cheeseman, Nic, and Brian Klaas. 2019. \emph{How to Rig an Election}.
Yale University Press. \url{https://doi.org/10.12987/9780300235210}.

\bibitem[\citeproctext]{ref-clayton2000}
Clayton, Anthony, and Chuka Onwumechili. 2000. {``African
Democratization and Military Coups.''} \emph{The International Journal
of African Historical Studies} 33 (1): 187.
\url{https://doi.org/10.2307/220297}.

\bibitem[\citeproctext]{ref-close2019}
Close, David. 2019. {``Presidential Term Limits in Nicaragua.''} In,
159--78. Oxford University PressOxford.
\url{https://doi.org/10.1093/oso/9780198837404.003.0009}.

\bibitem[\citeproctext]{ref-dahl2023a}
Dahl, Marianne, and Kristian Skrede Gleditsch. 2023a. {``Clouds with
Silver Linings: How Mobilization Shapes the Impact of Coups on
Democratization.''} \emph{European Journal of International Relations}
29 (4): 1017--40. \url{https://doi.org/10.1177/13540661221143213}.

\bibitem[\citeproctext]{ref-dahl2023}
---------. 2023b. {``Clouds with Silver Linings: How Mobilization Shapes
the Impact of Coups on Democratization.''} \emph{European Journal of
International Relations}, January, 135406612211432.
\url{https://doi.org/10.1177/13540661221143213}.

\bibitem[\citeproctext]{ref-davenport2021}
Davenport, Christian, Babak RezaeeDaryakenari, and Reed M Wood. 2021.
{``Tenure Through Tyranny? Repression, Dissent, and Leader Removal in
Africa and Latin America, 1990{\textendash}2006.''} \emph{Journal of
Global Security Studies} 7 (1).
\url{https://doi.org/10.1093/jogss/ogab023}.

\bibitem[\citeproctext]{ref-debruin2020}
De Bruin, Erica. 2020. {``Preventing Coups d{'}état.''} In, 1--12.
Cornell University Press.
\url{https://doi.org/10.7591/cornell/9781501751912.003.0001}.

\bibitem[\citeproctext]{ref-derpanopoulos2016}
Derpanopoulos, George, Erica Frantz, Barbara Geddes, and Joseph Wright.
2016. {``Are Coups Good for Democracy?''} \emph{Research \& Politics} 3
(1): 205316801663083. \url{https://doi.org/10.1177/2053168016630837}.

\bibitem[\citeproctext]{ref-easton2018}
Easton, Malcolm R, and Randolph M Siverson. 2018. {``Leader Survival and
Purges After a Failed Coup d{'}état.''} \emph{Journal of Peace Research}
55 (5): 596--608. \url{https://doi.org/10.1177/0022343318763713}.

\bibitem[\citeproctext]{ref-escribuxe0-folch2013}
Escribà-Folch, Abel. 2013. {``Repression, Political Threats, and
Survival Under Autocracy.''} \emph{International Political Science
Review} 34 (5): 543--60. \url{https://doi.org/10.1177/0192512113488259}.

\bibitem[\citeproctext]{ref-ezrow2019}
Ezrow, Natasha. 2019. {``Term Limits and Succession in Dictatorships.''}
In, 269--88. Oxford University PressOxford.
\url{https://doi.org/10.1093/oso/9780198837404.003.0014}.

\bibitem[\citeproctext]{ref-fariss2022}
Fariss, Christopher J., Therese Anders, Jonathan N. Markowitz, and
Miriam Barnum. 2022. {``New Estimates of Over 500 Years of Historic GDP
and Population Data.''} \emph{Journal of Conflict Resolution} 66 (3):
553--91. \url{https://doi.org/10.1177/00220027211054432}.

\bibitem[\citeproctext]{ref-firth1993}
FIRTH, DAVID. 1993. {``Bias Reduction of Maximum Likelihood
Estimates.''} \emph{Biometrika} 80 (1): 27--38.
\url{https://doi.org/10.1093/biomet/80.1.27}.

\bibitem[\citeproctext]{ref-frantz2016}
Frantz, Erica, and Elizabeth A. Stein. 2016. {``Countering Coups:
Leadership Succession Rules in Dictatorships.''} \emph{Comparative
Political Studies} 50 (7): 935--62.
\url{https://doi.org/10.1177/0010414016655538}.

\bibitem[\citeproctext]{ref-frantz2017}
---------. 2017a. {``Countering Coups: Leadership Succession Rules in
Dictatorships.''} \emph{Comparative Political Studies} 50 (7): 935--62.
\url{https://doi.org/10.1177/0010414016655538}.

\bibitem[\citeproctext]{ref-frantz2017a}
---------. 2017b. {``Countering Coups: Leadership Succession Rules in
Dictatorships.''} \emph{Comparative Political Studies} 50 (7): 935--62.
\url{https://doi.org/10.1177/0010414016655538}.

\bibitem[\citeproctext]{ref-freedomhouse2024freedom}
Freedom House. 2024. {``Freedom in the World 2024.''}
\url{https://freedomhouse.org/sites/default/files/2024-02/FIW_2024_DigitalBooklet.pdf}.

\bibitem[\citeproctext]{ref-gandhi2007}
Gandhi, Jennifer, and Adam Przeworski. 2007. {``Authoritarian
Institutions and the Survival of Autocrats.''} \emph{Comparative
Political Studies} 40 (11): 1279--1301.
\url{https://doi.org/10.1177/0010414007305817}.

\bibitem[\citeproctext]{ref-gassebner2016}
Gassebner, Martin, Jerg Gutmann, and Stefan Voigt. 2016. {``When to
Expect a Coup d{'}état? An Extreme Bounds Analysis of Coup
Determinants.''} \emph{Public Choice} 169 (3-4): 293--313.
\url{https://doi.org/10.1007/s11127-016-0365-0}.

\bibitem[\citeproctext]{ref-geddes1999a}
Geddes, Barbara. 1999a. {``What Do We Know About Democratization After
Twenty Years?''} \emph{Annual Review of Political Science} 2 (1):
115--44. \url{https://doi.org/10.1146/annurev.polisci.2.1.115}.

\bibitem[\citeproctext]{ref-geddes1999}
---------. 1999b. {``What Do We Know About Democratization After Twenty
Years?''} \emph{Annual Review of Political Science} 2 (1): 115--44.
\url{https://doi.org/10.1146/annurev.polisci.2.1.115}.

\bibitem[\citeproctext]{ref-geddes2014}
Geddes, Barbara, Joseph Wright, and Erica Frantz. 2014. {``Autocratic
Breakdown and Regime Transitions: A New Data Set.''} \emph{Perspectives
on Politics} 12 (2): 313--31.
\url{https://doi.org/10.1017/s1537592714000851}.

\bibitem[\citeproctext]{ref-ginsburg2019}
Ginsburg, Tom, and Zachary Elkins. 2019. {``One Size Does Not Fit
All.''} In, 37--52. Oxford University Press.
\url{https://doi.org/10.1093/oso/9780198837404.003.0003}.

\bibitem[\citeproctext]{ref-ginsburg2011evasion}
Ginsburg, Tom, James Melton, and Zachary Elkins. 2011. {``On the Evasion
of Executive Term Limits.''} \emph{William and Mary Law Review} 52:
1807.

\bibitem[\citeproctext]{ref-goemans2009}
Goemans, Henk E., Kristian Skrede Gleditsch, and Giacomo Chiozza. 2009.
{``Introducing Archigos: A Dataset of Political Leaders.''}
\emph{Journal of Peace Research} 46 (2): 269--83.
\url{https://doi.org/10.1177/0022343308100719}.

\bibitem[\citeproctext]{ref-haynes2022d}
Haynes, Jeffrey. 2022. {``Revolution and Democracy in Ghana,''}
December. \url{https://doi.org/10.4324/9781003229773}.

\bibitem[\citeproctext]{ref-helmke2017}
Helmke, Gretchen. 2017. {``Institutions on the Edge,''} January.
\url{https://doi.org/10.1017/9781139031738}.

\bibitem[\citeproctext]{ref-hiroi2013}
Hiroi, Taeko, and Sawa Omori. 2013. {``Causes and Triggers of
{\emph{Coups d'état}}: An Event History Analysis.''} \emph{Politics \&
Policy} 41 (1): 39--64. \url{https://doi.org/10.1111/polp.12001}.

\bibitem[\citeproctext]{ref-kim2021}
Kim, Nam Kyu, and Jun Koga Sudduth. 2021. {``Political Institutions and
Coups in Dictatorships.''} \emph{Comparative Political Studies} 54 (9):
1597--1628. \url{https://doi.org/10.1177/0010414021997161}.

\bibitem[\citeproctext]{ref-klesner2019}
Klesner, Joseph L. 2019. {``The Politics of Presidential Term Limits in
Mexico.''} In, 141--58. Oxford University Press.
\url{https://doi.org/10.1093/oso/9780198837404.003.0008}.

\bibitem[\citeproctext]{ref-kokkonen2019}
Kokkonen, Andrej, and Anders Sundell. 2019. {``Leader Succession and
Civil War.''} \emph{Comparative Political Studies} 53 (3-4): 434--68.
\url{https://doi.org/10.1177/0010414019852712}.

\bibitem[\citeproctext]{ref-krishnarajan2019}
Krishnarajan, Suthan. 2019. {``Economic Crisis, Natural Resources, and
Irregular Leader Removal in Autocracies.''} \emph{International Studies
Quarterly} 63 (3): 726--41. \url{https://doi.org/10.1093/isq/sqz006}.

\bibitem[\citeproctext]{ref-landau2019}
Landau, David, Yaniv Roznai, and Rosalind Dixon. 2019. {``Term Limits
and the Unconstitutional Constitutional Amendment Doctrine.''} In,
53--74. Oxford University PressOxford.
\url{https://doi.org/10.1093/oso/9780198837404.003.0004}.

\bibitem[\citeproctext]{ref-licht2009}
Licht, Amanda A. 2009. {``Coming into Money: The Impact of Foreign Aid
on Leader Survival.''} \emph{Journal of Conflict Resolution} 54 (1):
58--87. \url{https://doi.org/10.1177/0022002709351104}.

\bibitem[\citeproctext]{ref-llanos2019}
Llanos, Mariana. 2019. {``The Politics of Presidential Term Limits in
Argentina.''} In, 473--94. Oxford University Press.
\url{https://doi.org/10.1093/oso/9780198837404.003.0023}.

\bibitem[\citeproctext]{ref-londregan1995}
Londregan, John, Henry Bienen, and Nicolas van de Walle. 1995.
{``Ethnicity and Leadership Succession in Africa.''} \emph{International
Studies Quarterly} 39 (1): 1. \url{https://doi.org/10.2307/2600721}.

\bibitem[\citeproctext]{ref-marshall2005current}
Marshall, Monty G. 2005. {``Current Status of the World's Major Episodes
of Political Violence.''} \emph{Report to Political Instability Task
Force.(3 February)}.

\bibitem[\citeproctext]{ref-marshall}
---------. n.d. {``Center for Systemic Peace and Societal-Systems
Research Inc.''}

\bibitem[\citeproctext]{ref-p1}
Marshall, Monty G., and Ted Robert Gurr. 2020. {``Polity v Project,
Political Regime Characteristics and Transitions, 1800-2018.''} Center
for Systemic Peace.

\bibitem[\citeproctext]{ref-marsteintredet2019a}
Marsteintredet, Leiv. 2019. {``Presidential Term Limits in Latin
America: {\emph{C}}.1820{\textendash}1985.''} In, 103--22. Oxford
University PressOxford.
\url{https://doi.org/10.1093/oso/9780198837404.003.0006}.

\bibitem[\citeproctext]{ref-marsteintredet2019}
Marsteintredet, Leiv, and Andrés Malamud. 2019. {``Coup with Adjectives:
Conceptual Stretching or Innovation in Comparative Research?''}
\emph{Political Studies} 68 (4): 1014--35.
\url{https://doi.org/10.1177/0032321719888857}.

\bibitem[\citeproctext]{ref-mauceri1995}
Mauceri, Philip. 1995. {``State Reform, Coalitions, and The Neoliberal
{\emph{Autogolpe}} in Peru.''} \emph{Latin American Research Review} 30
(1): 7--37. \url{https://doi.org/10.1017/s0023879100017155}.

\bibitem[\citeproctext]{ref-demesquita1995}
Mesquita, Bruce Bueno de, and Randolph M. Siverson. 1995. {``War and the
Survival of Political Leaders: A Comparative Study of Regime Types and
Political Accountability.''} \emph{American Political Science Review} 89
(4): 841--55. \url{https://doi.org/10.2307/2082512}.

\bibitem[\citeproctext]{ref-miller2012}
Miller, Michael K. 2012. {``Economic Development, Violent Leader
Removal, and Democratization.''} \emph{American Journal of Political
Science} 56 (4): 1002--20.
\url{https://doi.org/10.1111/j.1540-5907.2012.00595.x}.

\bibitem[\citeproctext]{ref-miller2016}
---------. 2016. {``Reanalysis: Are Coups Good for Democracy?''}
\emph{Research \& Politics} 3 (4): 205316801668190.
\url{https://doi.org/10.1177/2053168016681908}.

\bibitem[\citeproctext]{ref-morrison2009}
Morrison, Kevin M. 2009. {``Oil, Nontax Revenue, and the
Redistributional Foundations of Regime Stability.''} \emph{International
Organization} 63 (1): 107--38.
\url{https://doi.org/10.1017/s0020818309090043}.

\bibitem[\citeproctext]{ref-muuxf1oz-portillo2019}
Muñoz-Portillo, Juan, and Ilka Treminio. 2019. {``The Politics of
Presidential Term Limits in Central America.''} In, 495--516. Oxford
University PressOxford.
\url{https://doi.org/10.1093/oso/9780198837404.003.0024}.

\bibitem[\citeproctext]{ref-neto2019}
Neto, Octavio Amorim, and Igor P. Acácio. 2019. {``Presidential Term
Limits as a Credible-Commitment Mechanism.''} In, 123--40. Oxford
University PressOxford.
\url{https://doi.org/10.1093/oso/9780198837404.003.0007}.

\bibitem[\citeproctext]{ref-nurumov2019}
Nurumov, Dmitry, and Vasil Vashchanka. 2019. {``Presidential Terms in
Kazakhstan.''} In, 221--46. Oxford University PressOxford.
\url{https://doi.org/10.1093/oso/9780198837404.003.0012}.

\bibitem[\citeproctext]{ref-palmer1999}
Palmer, Harvey D., and Guy D. Whitten. 1999. {``The Electoral Impact of
Unexpected Inflation and Economic Growth.''} \emph{British Journal of
Political Science} 29 (4): 623--39.
\url{https://doi.org/10.1017/s0007123499000307}.

\bibitem[\citeproctext]{ref-pieterse1982}
Pieterse, Jan. 1982. {``Rawlings and the 1979 Revolt in Ghana.''}
\emph{Race \& Class} 23 (4): 251--73.
\url{https://doi.org/10.1177/030639688202300402}.

\bibitem[\citeproctext]{ref-pilster2012}
Pilster, Ulrich, and Tobias Böhmelt. 2012. {``Do Democracies Engage Less
in Coup-Proofing? On the Relationship Between Regime Type and
Civil-Military Relations{\textsuperscript{1}}.''} \emph{Foreign Policy
Analysis} 8 (4): 355--72.
\url{https://doi.org/10.1111/j.1743-8594.2011.00160.x}.

\bibitem[\citeproctext]{ref-pion-berlin2022}
Pion-Berlin, David, Thomas Bruneau, and Richard B. Goetze. 2022. {``The
Trump Self-Coup Attempt: Comparisons and Civil{\textendash}Military
Relations.''} \emph{Government and Opposition} 58 (4): 789--806.
\url{https://doi.org/10.1017/gov.2022.13}.

\bibitem[\citeproctext]{ref-posner}
Posner, Daniel N., and Daniel J. Young. n.d. {``Term Limits: Leadership,
Political Competition and the Transfer of Power.''} In, 260--78.
Cambridge University Press.
\url{https://doi.org/10.1017/9781316562888.011}.

\bibitem[\citeproctext]{ref-powell2012}
Powell, Jonathan. 2012. {``Determinants of the Attempting and Outcome of
Coups d{'}état.''} \emph{Journal of Conflict Resolution} 56 (6):
1017--40. \url{https://doi.org/10.1177/0022002712445732}.

\bibitem[\citeproctext]{ref-powell2017}
---------. 2017. {``Leader Survival Strategies and the Onset of Civil
Conflict: A Coup-Proofing Paradox.''} \emph{Armed Forces \& Society} 45
(1): 27--44. \url{https://doi.org/10.1177/0095327x17728493}.

\bibitem[\citeproctext]{ref-powell}
Powell, Jonathan M. n.d. {``Coups and Conflict: The Paradox of
Coup-Proofing.''}

\bibitem[\citeproctext]{ref-powell2014a}
Powell, Jonathan M. 2014. {``An Assessment of the {`}Democratic{'} Coup
Theory.''} \emph{African Security Review} 23 (3): 213--24.
\url{https://doi.org/10.1080/10246029.2014.926949}.

\bibitem[\citeproctext]{ref-powell2018}
Powell, Christopher Faulkner, William Dean, and Kyle Romano. 2018.
{``Give Them Toys? Military Allocations and Regime Stability in
Transitional Democracies.''} \emph{Democratization} 25 (7): 1153--72.
\url{https://doi.org/10.1080/13510347.2018.1450389}.

\bibitem[\citeproctext]{ref-powell2011}
Powell, and Thyne. 2011. {``Global Instances of Coups from 1950 to 2010:
A New Dataset.''} \emph{Journal of Peace Research} 48 (2): 249--59.
\url{https://doi.org/10.1177/0022343310397436}.

\bibitem[\citeproctext]{ref-przeworski2000}
Przeworski, Adam, Michael E. Alvarez, Jose Antonio Cheibub, and Fernando
Limongi. 2000. {``Democracy and Development,''} August.
\url{https://doi.org/10.1017/cbo9780511804946}.

\bibitem[\citeproctext]{ref-quinlivan1999}
Quinlivan, James. 1999. \emph{Coup-Proofing: Its Practice and
Consequences in the Middle East}. MIT Press.
\url{https://doi.org/10.7249/rp844}.

\bibitem[\citeproctext]{ref-quirozflores2012}
Quiroz Flores, Alejandro, and Alastair Smith. 2012. {``Leader Survival
and Natural Disasters.''} \emph{British Journal of Political Science} 43
(4): 821--43. \url{https://doi.org/10.1017/s0007123412000609}.

\bibitem[\citeproctext]{ref-reiter2020}
Reiter, Dan. 2020. {``Avoiding the Coup-Proofing Dilemma: Consolidating
Political Control While Maximizing Military Power.''} \emph{Foreign
Policy Analysis} 16 (3): 312--31.
\url{https://doi.org/10.1093/fpa/oraa001}.

\bibitem[\citeproctext]{ref-reyntjens2016}
Reyntjens, Filip. 2016. {``A New Look at the Evidence.''} \emph{Journal
of Democracy} 27 (3): 61--68.
\url{https://doi.org/10.1353/jod.2016.0044}.

\bibitem[\citeproctext]{ref-schiel2019}
Schiel, Rebecca E. 2019. {``An Assessment of Democratic Vulnerability:
Regime Type, Economic Development, and Coups d{'}état.''}
\emph{Democratization} 26 (8): 1439--57.
\url{https://doi.org/10.1080/13510347.2019.1645652}.

\bibitem[\citeproctext]{ref-shannon2014}
Shannon, Megan, Clayton Thyne, Sarah Hayden, and Amanda Dugan. 2014.
{``The International Community's Reaction to Coups.''} \emph{Foreign
Policy Analysis} 11 (4): 363--76.
\url{https://doi.org/10.1111/fpa.12043}.

\bibitem[\citeproctext]{ref-singh2016}
Singh, Naunihal. 2016. \emph{Seizing Power}. Johns Hopkins University
Press. \url{https://doi.org/10.1353/book.31450}.

\bibitem[\citeproctext]{ref-smith2004}
Smith, Benjamin. 2004. {``Oil Wealth and Regime Survival in the
Developing World, 1960{\textendash}1999.''} \emph{American Journal of
Political Science} 48 (2): 232--46.
\url{https://doi.org/10.1111/j.0092-5853.2004.00067.x}.

\bibitem[\citeproctext]{ref-stinnett2002}
Stinnett, Douglas M., Jaroslav Tir, Paul F. Diehl, Philip Schafer, and
Charles Gochman. 2002. {``The Correlates of War (Cow) Project Direct
Contiguity Data, Version 3.0.''} \emph{Conflict Management and Peace
Science} 19 (2): 59--67.
\url{https://doi.org/10.1177/073889420201900203}.

\bibitem[\citeproctext]{ref-sudduth2017}
Sudduth, Jun Koga. 2017. {``Strategic Logic of Elite Purges in
Dictatorships.''} \emph{Comparative Political Studies} 50 (13):
1768--1801. \url{https://doi.org/10.1177/0010414016688004}.

\bibitem[\citeproctext]{ref-sudduth2018}
Sudduth, Jun Koga, and Curtis Bell. 2018. {``The Rise Predicts the Fall:
How the Method of Leader Entry Affects the Method of Leader Removal in
Dictatorships.''} \emph{International Studies Quarterly} 62 (1):
145--59. \url{https://doi.org/10.1093/isq/sqx075}.

\bibitem[\citeproctext]{ref-svolik2009}
Svolik, Milan W. 2009. {``Power Sharing and Leadership Dynamics in
Authoritarian Regimes.''} \emph{American Journal of Political Science}
53 (2): 477--94. \url{https://doi.org/10.1111/j.1540-5907.2009.00382.x}.

\bibitem[\citeproctext]{ref-svolik2014}
---------. 2014. {``Which Democracies Will Last? Coups, Incumbent
Takeovers, and the Dynamic of Democratic Consolidation.''} \emph{British
Journal of Political Science} 45 (4): 715--38.
\url{https://doi.org/10.1017/s0007123413000550}.

\bibitem[\citeproctext]{ref-tangri2010}
Tangri, Roger, and Andrew M. Mwenda. 2010. {``President Museveni and the
Politics of Presidential Tenure in Uganda.''} \emph{Journal of
Contemporary African Studies} 28 (1): 31--49.
\url{https://doi.org/10.1080/02589000903542574}.

\bibitem[\citeproctext]{ref-survival}
Therneau, Terry M. 2024. {``A Package for Survival Analysis in r.''}
\url{https://CRAN.R-project.org/package=survival}.

\bibitem[\citeproctext]{ref-thyne2014}
Thyne, Clayton L., and Jonathan M. Powell. 2014. {``Coup d{'}état or
Coup d'Autocracy? How Coups Impact Democratization, 1950-2008.''}
\emph{Foreign Policy Analysis}, April, n/a--.
\url{https://doi.org/10.1111/fpa.12046}.

\bibitem[\citeproctext]{ref-thyne2020}
Thyne, Clayton, and Kendall Hitch. 2020. {``Democratic Versus
Authoritarian Coups: The Influence of External Actors on States{'}
Postcoup Political Trajectories.''} \emph{Journal of Conflict
Resolution} 64 (10): 1857--84.
\url{https://doi.org/10.1177/0022002720935956}.

\bibitem[\citeproctext]{ref-thyne2017}
Thyne, Clayton, Powell, Sarah Parrott, and Emily VanMeter. 2017. {``Even
Generals Need Friends.''} \emph{Journal of Conflict Resolution} 62 (7):
1406--32. \url{https://doi.org/10.1177/0022002716685611}.

\bibitem[\citeproctext]{ref-thyne2019}
Thyne, and Powell. 2019. {``Coup Research,''} October.
\url{https://doi.org/10.1093/acrefore/9780190846626.013.369}.

\bibitem[\citeproctext]{ref-williams2011}
Williams, Laron K. 2011. {``Pick Your Poison: Economic Crises,
International Monetary Fund Loans and Leader Survival.''}
\emph{International Political Science Review} 33 (2): 131--49.
\url{https://doi.org/10.1177/0192512111399006}.

\bibitem[\citeproctext]{ref-wobig2014}
Wobig, Jacob. 2014. {``Defending Democracy with International Law:
Preventing Coup Attempts with Democracy Clauses.''}
\emph{Democratization} 22 (4): 631--54.
\url{https://doi.org/10.1080/13510347.2013.867948}.

\bibitem[\citeproctext]{ref-wright2008}
Wright, Joseph. 2008. {``To Invest or Insure?''} \emph{Comparative
Political Studies} 41 (7): 971--1000.
\url{https://doi.org/10.1177/0010414007308538}.

\bibitem[\citeproctext]{ref-wright2013}
Wright, Joseph, Erica Frantz, and Barbara Geddes. 2013. {``Oil and
Autocratic Regime Survival.''} \emph{British Journal of Political
Science} 45 (2): 287--306.
\url{https://doi.org/10.1017/s0007123413000252}.

\bibitem[\citeproctext]{ref-yu2016}
Yu, Shu, and Richard Jong-A-Pin. 2016. {``Political Leader Survival:
Does Competence Matter?''} \emph{Public Choice} 166 (1-2): 113--42.
\url{https://doi.org/10.1007/s11127-016-0317-8}.

\end{CSLReferences}

\chapter*{\texorpdfstring{Appendix\textbf{:
Datasets}}{Appendix: Datasets}}\label{appendix-datasets}
\addcontentsline{toc}{chapter}{Appendix\textbf{: Datasets}}

\begin{itemize}
\item
  \textbf{Coup Model Dataset}

  \begin{itemize}
  \item
    \textbf{Dataset Name:} \textbf{\texttt{coup\_model.csv}}
  \item
    \textbf{Description:} This dataset is specifically cleaned for the
    coup model and contains the relevant data points necessary for
    analysis.
  \end{itemize}
\item
  \textbf{Autocoup Dataset}

  \begin{itemize}
  \item
    \textbf{Dataset Name:} \textbf{\texttt{autocoup.csv}}
  \item
    \textbf{Description:} This original dataset was compiled to support
    the study's empirical objectives.
  \end{itemize}
\item
  \textbf{Autocoup Model Dataset}

  \begin{itemize}
  \item
    \textbf{Dataset Name:} \textbf{\texttt{autocoup\_model.csv}}
  \item
    \textbf{Description:} This dataset is cleaned for the autocoup model
    and includes the data required for the modelling process.
  \end{itemize}
\item
  \textbf{Cox Proportional Hazards (Cox PH) Model Dataset}

  \begin{itemize}
  \item
    \textbf{Dataset Name:}
    \textbf{\texttt{survival\_cox\_ph\_model.csv}}
  \item
    \textbf{Description:} This dataset is used for the Cox Proportional
    Hazards model and contains the data necessary for analysing survival
    rates and hazard ratios.
  \end{itemize}
\item
  \textbf{Time-Dependent Cox Model Dataset}

  \begin{itemize}
  \item
    \textbf{Dataset Name:}
    \textbf{\texttt{survival\_cox\_td\_model.csv}}
  \item
    \textbf{Description:} This dataset is cleaned for the time-dependent
    Cox model, incorporating variables that account for time-dependent
    effects in survival analysis.
  \end{itemize}
\end{itemize}




\end{document}

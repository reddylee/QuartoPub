% Options for packages loaded elsewhere
\PassOptionsToPackage{unicode}{hyperref}
\PassOptionsToPackage{hyphens}{url}
\PassOptionsToPackage{dvipsnames,svgnames,x11names}{xcolor}
%
\documentclass[
  12pt,
]{article}

\usepackage{amsmath,amssymb}
\usepackage{setspace}
\usepackage{iftex}
\ifPDFTeX
  \usepackage[T1]{fontenc}
  \usepackage[utf8]{inputenc}
  \usepackage{textcomp} % provide euro and other symbols
\else % if luatex or xetex
  \usepackage{unicode-math}
  \defaultfontfeatures{Scale=MatchLowercase}
  \defaultfontfeatures[\rmfamily]{Ligatures=TeX,Scale=1}
\fi
\usepackage{lmodern}
\ifPDFTeX\else  
    % xetex/luatex font selection
    \setmainfont[]{Times New Roman}
\fi
% Use upquote if available, for straight quotes in verbatim environments
\IfFileExists{upquote.sty}{\usepackage{upquote}}{}
\IfFileExists{microtype.sty}{% use microtype if available
  \usepackage[]{microtype}
  \UseMicrotypeSet[protrusion]{basicmath} % disable protrusion for tt fonts
}{}
\usepackage{xcolor}
\usepackage[top=30mm,left=1in,right=1in,bottom=25mm]{geometry}
\setlength{\emergencystretch}{3em} % prevent overfull lines
\setcounter{secnumdepth}{5}
% Make \paragraph and \subparagraph free-standing
\makeatletter
\ifx\paragraph\undefined\else
  \let\oldparagraph\paragraph
  \renewcommand{\paragraph}{
    \@ifstar
      \xxxParagraphStar
      \xxxParagraphNoStar
  }
  \newcommand{\xxxParagraphStar}[1]{\oldparagraph*{#1}\mbox{}}
  \newcommand{\xxxParagraphNoStar}[1]{\oldparagraph{#1}\mbox{}}
\fi
\ifx\subparagraph\undefined\else
  \let\oldsubparagraph\subparagraph
  \renewcommand{\subparagraph}{
    \@ifstar
      \xxxSubParagraphStar
      \xxxSubParagraphNoStar
  }
  \newcommand{\xxxSubParagraphStar}[1]{\oldsubparagraph*{#1}\mbox{}}
  \newcommand{\xxxSubParagraphNoStar}[1]{\oldsubparagraph{#1}\mbox{}}
\fi
\makeatother


\providecommand{\tightlist}{%
  \setlength{\itemsep}{0pt}\setlength{\parskip}{0pt}}\usepackage{longtable,booktabs,array}
\usepackage{calc} % for calculating minipage widths
% Correct order of tables after \paragraph or \subparagraph
\usepackage{etoolbox}
\makeatletter
\patchcmd\longtable{\par}{\if@noskipsec\mbox{}\fi\par}{}{}
\makeatother
% Allow footnotes in longtable head/foot
\IfFileExists{footnotehyper.sty}{\usepackage{footnotehyper}}{\usepackage{footnote}}
\makesavenoteenv{longtable}
\usepackage{graphicx}
\makeatletter
\def\maxwidth{\ifdim\Gin@nat@width>\linewidth\linewidth\else\Gin@nat@width\fi}
\def\maxheight{\ifdim\Gin@nat@height>\textheight\textheight\else\Gin@nat@height\fi}
\makeatother
% Scale images if necessary, so that they will not overflow the page
% margins by default, and it is still possible to overwrite the defaults
% using explicit options in \includegraphics[width, height, ...]{}
\setkeys{Gin}{width=\maxwidth,height=\maxheight,keepaspectratio}
% Set default figure placement to htbp
\makeatletter
\def\fps@figure{htbp}
\makeatother
% definitions for citeproc citations
\NewDocumentCommand\citeproctext{}{}
\NewDocumentCommand\citeproc{mm}{%
  \begingroup\def\citeproctext{#2}\cite{#1}\endgroup}
\makeatletter
 % allow citations to break across lines
 \let\@cite@ofmt\@firstofone
 % avoid brackets around text for \cite:
 \def\@biblabel#1{}
 \def\@cite#1#2{{#1\if@tempswa , #2\fi}}
\makeatother
\newlength{\cslhangindent}
\setlength{\cslhangindent}{1.5em}
\newlength{\csllabelwidth}
\setlength{\csllabelwidth}{3em}
\newenvironment{CSLReferences}[2] % #1 hanging-indent, #2 entry-spacing
 {\begin{list}{}{%
  \setlength{\itemindent}{0pt}
  \setlength{\leftmargin}{0pt}
  \setlength{\parsep}{0pt}
  % turn on hanging indent if param 1 is 1
  \ifodd #1
   \setlength{\leftmargin}{\cslhangindent}
   \setlength{\itemindent}{-1\cslhangindent}
  \fi
  % set entry spacing
  \setlength{\itemsep}{#2\baselineskip}}}
 {\end{list}}
\usepackage{calc}
\newcommand{\CSLBlock}[1]{\hfill\break\parbox[t]{\linewidth}{\strut\ignorespaces#1\strut}}
\newcommand{\CSLLeftMargin}[1]{\parbox[t]{\csllabelwidth}{\strut#1\strut}}
\newcommand{\CSLRightInline}[1]{\parbox[t]{\linewidth - \csllabelwidth}{\strut#1\strut}}
\newcommand{\CSLIndent}[1]{\hspace{\cslhangindent}#1}

\usepackage{booktabs}
\usepackage{caption}
\usepackage{longtable}
\usepackage{colortbl}
\usepackage{array}
\usepackage{anyfontsize}
\usepackage{multirow}
\makeatletter
\@ifpackageloaded{caption}{}{\usepackage{caption}}
\AtBeginDocument{%
\ifdefined\contentsname
  \renewcommand*\contentsname{Table of contents}
\else
  \newcommand\contentsname{Table of contents}
\fi
\ifdefined\listfigurename
  \renewcommand*\listfigurename{List of Figures}
\else
  \newcommand\listfigurename{List of Figures}
\fi
\ifdefined\listtablename
  \renewcommand*\listtablename{List of Tables}
\else
  \newcommand\listtablename{List of Tables}
\fi
\ifdefined\figurename
  \renewcommand*\figurename{Figure}
\else
  \newcommand\figurename{Figure}
\fi
\ifdefined\tablename
  \renewcommand*\tablename{Table}
\else
  \newcommand\tablename{Table}
\fi
}
\@ifpackageloaded{float}{}{\usepackage{float}}
\floatstyle{ruled}
\@ifundefined{c@chapter}{\newfloat{codelisting}{h}{lop}}{\newfloat{codelisting}{h}{lop}[chapter]}
\floatname{codelisting}{Listing}
\newcommand*\listoflistings{\listof{codelisting}{List of Listings}}
\makeatother
\makeatletter
\makeatother
\makeatletter
\@ifpackageloaded{caption}{}{\usepackage{caption}}
\@ifpackageloaded{subcaption}{}{\usepackage{subcaption}}
\makeatother

\ifLuaTeX
  \usepackage{selnolig}  % disable illegal ligatures
\fi
\usepackage{bookmark}

\IfFileExists{xurl.sty}{\usepackage{xurl}}{} % add URL line breaks if available
\urlstyle{same} % disable monospaced font for URLs
\hypersetup{
  pdftitle={Autocoups Redefined: A Clarification of Concept and Introduction of A Novel Dataset},
  pdfkeywords={Coups, Autocoups, Political Leadership, Power
Transitions},
  colorlinks=true,
  linkcolor={blue},
  filecolor={Maroon},
  citecolor={Blue},
  urlcolor={blue},
  pdfcreator={LaTeX via pandoc}}


\title{Autocoups Redefined: A Clarification of Concept and Introduction
of A Novel Dataset}
\author{}
\date{2024-10-15}

\begin{document}


\def\spacingset#1{\renewcommand{\baselinestretch}%
{#1}\small\normalsize} \spacingset{1}


%%%%%%%%%%%%%%%%%%%%%%%%%%%%%%%%%%%%%%%%%%%%%%%%%%%%%%%%%%%%%%%%%%%%%%%%%%%%%%

\date{2024-10-15}
\title{\bf Autocoups Redefined: A Clarification of Concept and
Introduction of A Novel Dataset}
\author{
}

\maketitle

\bigskip
\bigskip
\begin{abstract}
This article provides a critical reexamination of autocoups, an
under-explored phenomenon in which incumbent leaders manipulate
institutions to extend their tenure. By distinctly separating autocoups
from the broader and more ambiguous concepts of self-coups or executive
takeovers, the research introduces a refined and more nuanced definition
of autocoups. Building on this clarified conceptual framework, the study
unveils a novel dataset of autocoup events from 1945 to 2023, greatly
enhancing the empirical foundation for future research. A robust
mixed-methods approach is employed, combining three types of qualitative
case studies that offer in-depth insights into the dynamics of
autocoups, alongside a comprehensive quantitative analysis of the
determinants influencing both autocoup attempts and their outcomes. This
rigorous empirical inquiry not only underscores the utility and
versatility of the new dataset but also makes a significant contribution
to the existing body of literature.
\end{abstract}

\noindent%
{\it Keywords:} Coups, Autocoups, Political Leadership, Power
Transitions
\vfill

\newpage
\spacingset{1.9} % DON'T change the spacing!

\setstretch{1.618}
\subsection{Introduction}\label{introduction}

While the study of irregular leadership transitions has predominantly
focused on coups due to their frequency and significant impact, another
form---the incumbent leader's refusal to relinquish power---has received
comparatively less attention despite its importance. Recent decades,
particularly since the end of the Cold War, have witnessed a decline in
classic coups and a concomitant rise in this incumbent retention or
overstay type of irregular leadership transition
(\citeproc{ref-ginsburg2010evasion}{Ginsburg, Melton, and Elkins 2010};
\citeproc{ref-baturo2014}{Baturo 2014};
\citeproc{ref-versteeg2020law}{Versteeg et al. 2020}).

This research aims to redefine and clarify this type of irregular
leadership transition, where leaders overstay their mandated term
limits, as an \textbf{\emph{autocoup}}. Although analyses related to
autocoups are not uncommon, the existing literature exhibits several
notable shortcomings:

\begin{itemize}
\item
  \textbf{Terminological ambiguity:} The use of terms like
  ``self-coups'', ``autocoups'', ``autogolpes'', ``incumbent
  takeovers'', ``executive aggrandizement'', ``overstay'', and
  ``continuismo'' in different literature lacks clear, universally
  accepted definitions, leading to confusion and inconsistent
  application (\citeproc{ref-marsteintredet2019}{Marsteintredet and
  Malamud 2019}; \citeproc{ref-baturo2022}{Baturo and Tolstrup 2022}).
  This terminological ambiguity hinders accurate analysis and comparison
  across studies
\item
  \textbf{Limited dataset:} Due to the conceptual ambiguity surrounding
  autocoups, data collection remains in its nascent stages compared to
  the rich datasets available for classic coups.
\item
  \textbf{Methodological gaps:} The study of autocoups has been hindered
  by a limited dataset, resulting in a reliance on in-depth case studies
  (\citeproc{ref-cameron1998}{Maxwell A. Cameron 1998b};
  \citeproc{ref-antonio2021}{Antonio 2021};
  \citeproc{ref-pion-berlin2022}{Pion-Berlin, Bruneau, and Goetze 2022})
  to explore this phenomenon. Notably, quantitative analysis has been
  underutilized in this field, with few studies employing statistical
  methods to examine autocoups.
\end{itemize}

More importantly, analyses of autocoups are often not integrated with
those of classic coups, despite their interconnected nature. As a
distinct category of coup, autocoups lack a clear and differentiated
definition in relation to traditional coups. Classic coups are typically
defined as the complete removal of incumbent leaders, with a focus on
the termination of their tenure. In contrast, autocoups often focus more
on incumbent leaders consolidating power by seizing control from other
state institutions, rather than on extending their tenure. As a result,
coups and autocoups are frequently analysed in isolation. This
separation has led to a dearth of comparative analyses, hindering a more
comprehensive understanding of the complex dynamics between these two
types of irregular leadership transitions.

Examining autocoups, particularly in conjunction with classic coups, is
essential for several compelling reasons. Firstly, both coups and
autocoups represent significant and frequent means of irregular
leadership transitions, underscoring the need for a comprehensive
understanding of these phenomena. Secondly, autocoups, much like coups,
have a profoundly detrimental impact on governance, as they undermine
the rule of law, erode institutional capacity, and contribute to
democratic backsliding or the personalization of authoritarian power.
Thirdly, successful autocoups, akin to successful coups, create a
precedent that increases the likelihood of future irregular power
transitions, thereby perpetuating a cycle of instability. For instance,
since 1945, a striking 62\% of leaders who extended their terms through
autocoups in non-democratic countries ultimately met a tumultuous end,
either being ousted or assassinated while in office
(\citeproc{ref-baturo2019}{Baturo 2019}). Lastly, failed autocoups,
similar to failed coups, often precipitate instability, inciting
widespread protests, violence, and even civil wars, which can have
far-reaching and devastating consequences for the affected country and
its citizens.

This chapter addresses these gaps by focusing on autocoups, aiming to
clarify terminology, refine concepts and definitions, enhance data
collection, and explore determinants through empirical analysis,
contributing in three key areas:

\begin{itemize}
\item
  \textbf{Conceptual clarification:} The term autocoup will be redefined
  and clarified, with a focus on power extension.
\item
  \textbf{Data collection:} A new dataset of autocoups since 1945 will
  be introduced based on this refined definition.
\item
  \textbf{Empirical analysis:} Utilizing this dataset, a quantitative
  analysis of the factors influencing leaders' decisions to attempt
  autocoups will be conducted.
\end{itemize}

The structure of this chapter is as follows: Section 2 will review
definitions related to power expansions and extensions, leading to a
precise definition of autocoups. Section 3 will present the new autocoup
dataset. Sections 4 and 5 will explore the determinants of autocoup
attempts through case studies and demonstrate the application of the
dataset in empirical analysis. The conclusion will summarize key
findings and suggest directions for future research.

\subsection{Autocoups: A literature review and clarification of
definitions}\label{autocoups-a-literature-review-and-clarification-of-definitions}

A significant limitation in the study of irregular leadership transition
is the conspicuous lack of integration between research on autocoups and
classic coups. Despite both coups and autocoups being crucial mechanisms
of irregular leadership change, the existing literature has largely
treated these two phenomena in isolation, neglecting to explore their
interconnectedness and the nuanced dynamics that govern their
occurrence.

This separation is attributable to two primary factors. Firstly,
previous research has often overlooked autocoups as a distinct form of
irregular leadership transition. Secondly, a persistent conceptual
ambiguity has hindered the development of a clear and consistent
definition and categorization of autocoups.

Classic coups are typically characterized by the abrupt and
comprehensive removal of incumbent leaders, focusing on the swift
termination of their tenure. In contrast, autocoups are often defined as
events wherein incumbent leaders consolidate their authority by
systematically usurping power from other state institutions, rather than
merely extending their own tenure.

The absence of a clear, differentiated definition of autocoups in
relation to their classic counterparts has exacerbated this divide.
While classic coups have been extensively studied and well-defined,
autocoups remain a distinct yet under-explored category. This
definitional ambiguity has resulted in a dearth of comparative analyses
that could illuminate the complex interplay between these two forms of
power subversion.

Integrating the study of autocoups with that of classic coups is crucial
for bridging this analytical gap. Such an approach would offer a more
nuanced and comprehensive perspective on the various mechanisms by which
political power can be subverted or consolidated. By examining these
phenomena in tandem, researchers can better understand the full spectrum
of irregular power transitions, from outright removal to internal power
grabs, providing valuable insights into the nature of political
instability and regime change.

To strengthen the analysis of autocoups, a crucial first step is to
establish a clear and consistent terminology, followed by a refinement
of the definition of autocoup to mitigate ambiguity and clarify its
distinct characteristics.

\subsubsection{Terminology}\label{terminology}

The most prevalent term in autocoup literature is ``self-coup,'' or
``autogolpe'' in Spanish (\citeproc{ref-przeworski2000}{Przeworski et
al. 2000}; \citeproc{ref-cameron1998a}{Maxwell A. Cameron 1998a};
\citeproc{ref-bermeo2016}{Bermeo 2016}; \citeproc{ref-helmke2017}{Helmke
2017}; \citeproc{ref-marsteintredet2019}{Marsteintredet and Malamud
2019}). This term gained academic prominence following Peruvian
President Alberto Fujimori's actions in 1992, when he dissolved
Congress, temporarily suspended the constitution, and ruled by decree
(\citeproc{ref-mauceri1995}{Mauceri 1995};
\citeproc{ref-cameron1998}{Maxwell A. Cameron 1998b}). However, as
Marsteintredet and Malamud (\citeproc{ref-marsteintredet2019}{2019})
astutely points out, the term ``self-coup'' can be misleading, as it
implies a coup against oneself, which is inaccurate since the action
typically targets other state institutions or apparatus.

Another approach to describing coups staged by incumbents involves using
terms with adjectives or modifiers, such as ``presidential coup,''
``executive coup,'' ``constitutional coup,'' ``electoral coup,''
``judicial coup,'' ``slow-motion coup,'' ``soft coup,'' and
``parliamentary coup'' (\citeproc{ref-marsteintredet2019}{Marsteintredet
and Malamud 2019}). While these terms can be useful in specific
instances, their proliferation often creates more confusion than
clarity. Most of these terms focus on the specific methods employed by
coup perpetrators but fail to clearly identify the perpetrator,
necessitating further explanation. Moreover, many of these methods could
be employed either by or against executive leaders, further muddying the
waters.

A third alternative involves terms like ``incumbent takeover,''
``executive takeover,'' or ``overstay.'' Incumbent takeover refers to
``an event perpetuated by a ruling executive that significantly reduces
the formal and/or informal constraints on his/her power''
(\citeproc{ref-baturo2022}{Baturo and Tolstrup 2022, 374}), building on
earlier research by (\citeproc{ref-svolik2014}{Svolik 2014}). Meanwhile,
overstay is defined as ``staying longer than the maximum term as it
stood when the candidate originally came into office''
(\citeproc{ref-ginsburg2011evasion}{Ginsburg, Melton, and Elkins 2011,
1844}). These terms effectively identify the perpetrator (the incumbent)
and/or the nature of the event (overstaying/extending power). However,
they fall short in highlighting the illegality or illegitimacy of these
actions. Consequently, they cannot serve as a direct counterpart to
``coup,'' which clearly denotes the illegality of leadership ousters,
while ``takeover'' or ``overstay'' diminish the severity of the act.

Given that these terms often lack precision, focusing on specific
methods rather than the core act of power usurpation, this study
proposes ``autocoup'' as the most suitable term for this phenomenon.
Unlike other terms, `autocoup' clearly identifies the perpetrator and
the illegitimate nature of the power grab, distinguishing it from
classic coups while maintaining a parallel structure in terminology.

\subsubsection{Definition}\label{definition}

While precise terminology is undoubtedly crucial, another issue arises
with previous definitions of autocoups: what is the primary
emphasis---power expansion, power extension, or a combination of the
two?

Definitions of power expansion and power extension within the field of
political science can often be ambiguous or overlapping, presenting a
potential source of confusion. To ensure greater clarity in the study of
autocoups, it is necessary to distinguish these two distinct conceptual
frameworks more clearly:

\begin{itemize}
\item
  \textbf{Power expansion:} This refers to the process by which an
  incumbent leader acquires additional authority or control over state
  apparatuses beyond their original mandate. This may involve
  centralizing power, reducing checks and balances, or encroaching on
  the authority of other branches such as the legislature or judiciary.
\item
  \textbf{Power extension:} This describes situations where a leader
  prolongs their tenure beyond the originally mandated term in office,
  often through constitutional amendments, cancellation of elections, or
  other means of circumventing term limits.
\end{itemize}

Existing definitions of autocoups or related concepts often suffer from
ambiguity between power expansion and extension, or they focus more on
power expansion, which has several drawbacks.

Firstly, defining autocoups primarily in terms of power expansion does
not align well with the traditional definition of a coup. A classical
coup is clearly focused on the ouster of the current leader, not merely
a limitation or restriction on their power. Using the same logic, a more
appropriate definition of an autocoup should prioritize the tenure
extension of executive leadership. Power restrictions on incumbents
would not be coded as a coup as long as they remain in office.
Similarly, an executive leader acquiring more power from other branches
could be coded as power aggrandizement but not an autocoup, as long as
they step down when their term expires.

Secondly, emphasizing power expansion in autocoups often neglects the
ultimate purpose of incumbents. It is irrational for an incumbent to
expand executive power only to pass the powerful role to future leaders.
Although the term ``self-coup'' gained prominence from the 1992 Fujimori
case in Peru, which initially involved seizing power from other
institutions, it is important to note that Fujimori ultimately extended
his term limits through constitutional amendments. The 1993 Constitution
allowed Fujimori to run for a second term, which he won in April 1995.
Shortly after Fujimori began his second term, his supporters in Congress
passed a law of ``authentic interpretation'' that effectively allowed
him to run for another term in 2000, which he won amid suspicions and
rumors. However, he did not survive the third term; in 2000, facing
charges of corruption and human rights abuses, Fujimori fled Peru and
took refuge in Japan (\citeproc{ref-ezrow2019}{Ezrow 2019}).

Thirdly, measuring the extent of power expansion to qualify as an
autocoup can be challenging. As Maxwell A. Cameron
(\citeproc{ref-cameron1998a}{1998a}) defined, a self-coup is ``a
temporary suspension of the constitution and dissolution of congress by
the executive, who rules by decree until new legislative elections and a
referendum can be held to ratify a political system with broader
executive power'' (p.~220). However, defining ``broader executive
power'' is inherently problematic and disputable.

Therefore, this study argues that a more accurate definition of
autocoups should prioritize power extension as its core characteristic.
This approach is straightforward and easy to identify in practice. In
most cases, autocoups involving power extension also involve power
expansions as their prerequisite and foreshadowing.

Based on these criteria, I define \textbf{an autocoup as the
illegitimate extension of an incumbent leader's term in office beyond
the originally mandated limits through unconstitutional means}. This
definition emphasizes the core characteristic of power extension while
acknowledging the potential for power expansion as a related phenomenon:

\begin{itemize}
\item
  \textbf{Leadership Focus}: This definition refers to the actual
  leaders of the country, regardless of their official titles.
  Typically, this would be the president; however, in some cases, such
  as in Germany, the primary leader is the premier, as the president
  serves as a nominal head of state.
\item
  \textbf{Primary Characteristic}: While the primary characteristic of
  an autocoup is extending the term in office, this definition does not
  exclude instances of power expansion. Both aspects can coexist, but
  the extension of the term is the central element.
\item
  \textbf{Illegitimacy}: Autocoups, by their nature, subvert legal norms
  and established leadership transfer mechanisms. No matter how
  legitimate they claim to be, their illegitimacy is not beyond a
  reasonable doubt as long as the incumbents are the direct
  beneficiaries. This critical aspect will be explored further in
  Section 3.
\end{itemize}

By clarifying these definitions, this study aims to provide a more
precise and consistent framework for understanding and analysing
autocoups, thereby enhancing the clarity and rigor of research in this
field.

\subsection{Introduction to the autocoup
dataset}\label{introduction-to-the-autocoup-dataset}

\subsubsection{Defining the scope}\label{defining-the-scope}

Classifying political events as autocoups often involves addressing
borderline cases. To maintain consistency and avoid ambiguity, this
study adopts a broad coding approach: All instances of incumbents
extending their original mandated term in office are coded as autocoups,
regardless of the apparent legality of the extension.

This approach is justified because truly legitimate amendments to power
transition institutions should apply only to subsequent leaders, not the
incumbent. Even when extension procedures appear legal, the legitimacy
is questionable when the incumbent is the direct beneficiary.

\subsubsection{Classifying autocoups}\label{sec-classify}

To maintain consistency, autocoups are categorized based on several key
factors.

\begin{itemize}
\item
  \textbf{Methods employed}: Specific strategies used by incumbents
  (e.g., constitutional amendments, election cancellation).
\item
  \textbf{Degree of legality}: Extent of deviation from established
  legal norms.
\item
  \textbf{Duration of extension}: Length of time the incumbent remains
  in office beyond designated term limits.
\item
  \textbf{Outcomes}: Whether the autocoup attempt succeeds or fails.
\end{itemize}

This study primarily focuses on the methods employed, while coding for
other aspects when information is available.

\paragraph{Evasion of term limits}\label{evasion-of-term-limits}

Evasion of term limits is a common tactic employed in autocoups.
Incumbents often resort to seemingly legal manoeuvres to extend their
hold on power. These manoeuvres primarily involve manipulating
constitutional provisions through various means. The incumbents may
pressure legislative bodies (congress) or judicial institutions (Supreme
Court) to reinterpret existing term limits, amend the constitution to
extend terms, or even replace the constitution altogether. This might
also involve popular vote through referendums, or a combination of these
approaches. The extension can range from a single term to indefinite
rule.

These manoeuvres primarily involve manipulating constitutional
provisions through various means.

\begin{itemize}
\item
  \textbf{Changing term length:} Incumbents might lengthen the official
  term duration (e.g., from 4 to 6 years) to stay in office longer, even
  if the number of allowed terms remains unchanged. Examples, in the
  dataset, include Presidents Dacko (CAR, 1962), Kayibanda (Rwanda,
  1973), and Pinochet (Chile, 1988).
\item
  \textbf{Enabling re-election:} This approach involves incumbents
  modifying legal or constitutional frameworks to permit themselves to
  run for leadership again, despite initial restrictions. These
  restrictions might include prohibitions on re-election, bans on
  immediate re-election, or term limits that the incumbents have already
  reached. An illustrative example is President Menem of Argentina in
  1993, who leveraged this tactic to extend his tenure.
\item
  \textbf{Removing term limits altogether:} This approach was
  implemented by President Paul Biya of Cameroon in 2008. Biya, who had
  been in power since 1982, successfully pushed for a constitutional
  amendment that abolished presidential term limits. This change allowed
  him to run for re-election indefinitely, effectively opening the
  possibility for him to rule for life.
\item
  \textbf{Declaring} \textbf{leader for life:} This differs from
  removing term limits as the leader still faces elections (although
  potentially rigged or uncontested). An example is Indonesia's
  President Sukarno, who attempted to declare himself president for life
  in 1963 (ultimately unsuccessful).
\end{itemize}

These methods are often used in combination. Initially, the duration of
a term is extended, followed by amendments to allow re-election, then
the removal of term limits, and finally, the declaration of the leader
for life. For example, Haitian President François Duvalier amended the
constitution in 1961 to permit immediate re-election and then declared
himself president for life in 1964.

\paragraph{Election manipulation or
rigging}\label{election-manipulation-or-rigging}

Election manipulation or rigging is the second most commonly used tactic
to extend an incumbent's tenure.

\begin{itemize}
\item
  \textbf{Delaying or removing elections:} Delaying or removing
  scheduled elections without legitimate justification is a frequent
  method used by incumbents to maintain power. For instance, Chadian
  President François Tombalbaye delayed general elections until 1969
  after assuming power in 1960. Similarly, Angolan President José
  Eduardo dos Santos suspended elections throughout his rule from 1979
  to 2017.
\item
  \textbf{Refusing unfavourable election results:} Incumbents may refuse
  to accept unfavourable election results and attempt to overturn them
  through illegitimate means. For example, President Donald Trump of the
  United States refused to accept the results of the 2020 election and
  tried to overturn them.
\item
  \textbf{Rigging elections:} Winning elections with an extraordinarily
  high percentage of votes is highly questionable. This study will code
  elections where the incumbent wins more than 90\% of the vote as
  autocoups. For instance, President Teodoro Obiang of Equatorial Guinea
  has consistently won elections with over 95\% of the vote in
  multi-party elections since 1996, indicating election rigging.
\item
  \textbf{Excluding opposition in elections:} Manipulating the electoral
  process by excluding opposition parties or candidates from
  participation, effectively creating a one-candidate race, clearly
  signifies an autocoup.
\end{itemize}

\paragraph{Figurehead Installation}\label{figurehead-installation}

One strategy employed by incumbents to evade term limits is to install a
trusted associate as a figurehead, allowing the incumbent to maintain de
facto control while formally relinquishing office. This can be achieved
through the creation of seemingly subordinate positions, which in
reality serve as conduits for the incumbent's continued influence.

A notable example of this tactic is the 2008 Russian presidential
transition. Confronted with constitutional term limits, President
Vladimir Putin hand-picked Dmitry Medvedev to succeed him as president.
Following Medvedev's election, he appointed Putin as Prime Minister,
ostensibly reversing their roles. However, most observers and analysts
concur that Putin continued to wield significant behind-the-scenes
influence, effectively rendering Medvedev a proxy leader.

\paragraph{Reassigning supreme authority to a new
role}\label{reassigning-supreme-authority-to-a-new-role}

This tactic involves an incumbent leader manipulating the constitution
or legal framework to create a new position of power, or elevate an
existing one, before stepping down from their current role. They then
strategically take on this new position, effectively retaining
significant control despite appearing to relinquish power. For example,
in 2017, Recep Tayyip Erdoğan, the Prime Minister of Turkey, spearheaded
a constitutional referendum that transitioned the country from a
parliamentary system to a presidential one. This new system concentrated
significant executive power in the presidency. Following the
referendum's approval, Erdoğan successfully ran for the newly
established presidency, effectively retaining control under a different
title.

\paragraph{One-time arrangement for current
leaders}\label{one-time-arrangement-for-current-leaders}

This strategy involves special arrangements that extend the term or
tenure of current leaders without altering the underlying institutions.
For example, Lebanon extended President Émile Lahoud's term by three
years in 2004 through a one-time arrangement.

\subsubsection{Data coding}\label{data-coding}

The autocoup dataset is built upon existing studies and datasets,
ensuring a comprehensive and reliable foundation. Table~\ref{tbl-source}
outlines the main sources used for coding the autocoup dataset.

The Archigos dataset (\citeproc{ref-goemans2009}{Goemans, Gleditsch, and
Chiozza 2009}) and the Political Leaders' Affiliation Database (PLAD)
(\citeproc{ref-bomprezzi2024wedded}{Bomprezzi et al. 2024}) provide
comprehensive data on all leaders from 1875 to 2023, although our coding
only includes autocoups since 1945. These datasets are invaluable for
identifying actual rulers, distinguishing them from nominal heads of
state.

The Incumbent Takeover dataset (\citeproc{ref-baturo2022}{Baturo and
Tolstrup 2022}) integrates data from 11 related datasets, offering a
broad spectrum of cases where leaders significantly reduced constraints
on their power. This dataset includes both power expansions and
extensions, necessitating cross-referencing with Archigos to verify
qualifications for autocoups.

\begin{longtable}[]{@{}llrr@{}}

\caption{\label{tbl-source}Main Data Sources for Coding the Autocoup
Dataset}

\tabularnewline

\toprule\noalign{}
Dataset & Authors & Coverage & Obervations \\
\midrule\noalign{}
\endhead
\bottomrule\noalign{}
\endlastfoot
Archigos & Goemans et al (2009) & 1875-2015 & 3409 \\
PLAD & Bomprezzi et al. (2024) & 1989-2023 & 1334 \\
Incumbent Takeover & Baturo and Tolstrup (2022) & 1913-2019 & 279 \\

\end{longtable}

In total, 110 observations were coded, with 95 overlapping with the
candidate data from Incumbent Takeover. The remaining 15 events were
newly coded by the author through verification with other sources such
as Archigos, PLAD and news reports.

The main deviation from the Incumbent Takeover dataset arises from
excluding power expansions that do not involve attempts to extend
tenure.

The dataset encompasses a total of 14 variables along with the
\emph{notes} field.

\begin{itemize}
\item
  \textbf{Country identification:} Country code (\emph{ccode}) and
  country name (\emph{country}) from Correlates of War project
  (\citeproc{ref-stinnett2002}{Stinnett et al. 2002}).
\item
  \textbf{Leader information:} Name of the de facto leader
  (\emph{leader\_name},coded following Archigos and PLAD datasets).
\item
  \textbf{Timeline variables:} Date the leader assumed power
  (\emph{entry\_date}), date the leader left office(\emph{exit\_date}),
  date of the significant event marking the autocoup
  (\emph{autocoup\_date}), and Start date of the leader's additional
  term acquired through the autocoup (\emph{extending\_date}).
\item
  \textbf{Power transition methods:} Categorical variable for how the
  leader entered power (\emph{entry\_method}), categorical variable for
  how the leader exited power (\emph{exit\_method}), dummy variable
  indicating regular (1) or irregular (0) entry (\emph{entry\_regular}),
  and dummy variable indicating regular (1) or irregular (0) exit
  (\emph{exit\_regular}).
\item
  \textbf{Autocoup details:} Key variable capturing methods used to
  extend power (\emph{autocoup\_method}) and outcome of the autocoup
  attempt (\emph{autocoup\_outcome}, ``fail and lose power'', ``fail but
  complete original tenure'', or ``successful''). For successful coups,
  the additional term length can be calculated from the difference
  between \emph{exit\_date} and \emph{extending\_date}.
\item
  \textbf{Data source:} Identifies the dataset source used for coding
  (\emph{source}).
\item
  \textbf{Additional notes:} Provides context for exceptional cases
  (\emph{notes}).
\end{itemize}

There are a few coding challenges and decisions worth mention. For cases
where extensions happen incrementally, the \emph{autocoup\_date}
reflects a significant event marking the extension, such as a
legislative vote or successful referendum. In cases where a leader
undertook multiple autocoup attempts, details are recorded in the notes
field. Care was taken to differentiate between cases of power expansion
and actual attempts to extend tenure, which required cross-referencing
multiple sources. Determining the success or failure of an autocoup
attempt often required in-depth research, especially for less documented
cases.

\subsubsection{Data descriptions}\label{data-descriptions}

The primary coding has identified 110 autocoup cases from 1945 to 2023,
involving 73 countries. This comprehensive dataset provides a rich
source of information for analysing trends and patterns in autocoup
attempts across different political contexts.

Table~\ref{tbl-autocoup_method} presents a breakdown of the autocoup
methods employed by leaders:

\begin{table}

\caption{\label{tbl-autocoup_method}Autocoup methods and success rates
(1945-2021)}

\centering{

\fontsize{12.0pt}{14.4pt}\selectfont
\begin{tabular*}{0.99\linewidth}{@{\extracolsep{\fill}}lccr}
\toprule
Autocoup Method & Attempted & Succeeded & Success Rate \\ 
\midrule\addlinespace[2.5pt]
Enabling re-election & 46 & 33 & 71.7\% \\ 
Removing term limits & 14 & 14 & 100.0\% \\ 
Delaying elections & 9 & 9 & 100.0\% \\ 
Leader for life & 9 & 9 & 100.0\% \\ 
Changing term length & 7 & 5 & 71.4\% \\ 
Figurehead & 6 & 5 & 83.3\% \\ 
One-time arrangement & 5 & 4 & 80.0\% \\ 
Refusing election results & 4 & 1 & 25.0\% \\ 
Reassigning power role & 4 & 2 & 50.0\% \\ 
Rigging elections & 3 & 2 & 66.7\% \\ 
Cancelling elections & 3 & 3 & 100.0\% \\ 
Total & 110 & 87 & 79.1\% \\ 
\bottomrule
\end{tabular*}
\begin{minipage}{\linewidth}
\emph{Source: Autocoup dataset}\\
\end{minipage}

}

\end{table}%

The most common autocoup method is ``enabling re-election'', accounting
for 46 events. This is followed by ``removing term limits'' (14 cases),
and then ``delaying elections'' and ``declaring the leader for life''
(each with 9 cases).

Autocoups have a success rate of 79\%, compared to the 50\% success rate
of classical coups. This high success rate can be attributed to several
factors:

\begin{itemize}
\item
  Incumbent Advantage: Leaders already in power have access to resources
  and institutional mechanisms that can be leveraged to their advantage.
\item
  Gradual Implementation: Unlike sudden coups, autocoups can be
  implemented gradually, allowing leaders to build support and
  legitimacy over time.
\item
  Legal Facade: Many autocoup methods operate within a veneer of
  legality, making them harder to oppose openly.
\item
  Control of State Apparatus: Incumbents often have significant control
  over state institutions, which can be used to facilitate their
  autocoup attempts.
\end{itemize}

However, success rates vary significantly across different methods.

\begin{itemize}
\item
  \textbf{100\% success rate}: Removing term limits, delaying elections,
  declaring the leader for life, and cancelling elections all have
  perfect success rates. This suggests that once these processes are set
  in motion, they are difficult to reverse.
\item
  \textbf{Lower success rates:} Refusing to accept election results has
  the lowest success rate, with only 1 out of 4 attempts succeeding.
  Although the sample size is limited (only 4 cases in total), this
  trend might suggest several factors at play. These include greater
  democratic resilience in systems where general elections are regularly
  held, heightened international scrutiny and pressure in response to
  blatant manipulation of election results, and stronger domestic
  opposition to such overt power grabs.
\end{itemize}

\subsection{Determinants of autocoup attempts: Case
studies}\label{determinants-of-autocoup-attempts-case-studies}

\subsubsection{High frequency and success rate of autocoups in
post-communist
regimes}\label{high-frequency-and-success-rate-of-autocoups-in-post-communist-regimes}

Our dataset shows a high frequency and success rate of autocoups in
post-communist countries. These nations, formerly communist regimes
prior to the collapse of the Soviet Union, have largely evolved into
`hybrid regimes' (\citeproc{ref-nurumov2019}{Nurumov and Vashchanka
2019}), with only a few retaining their communist status. The data
documents 12 cases of autocoups aimed at prolonging incumbency in these
countries, with only two attempts failing. Examination of these cases
highlights several distinctive characteristics:

\begin{itemize}
\item
  \textbf{Inherited authoritarian systems}: Despite most of these 12
  countries transitioning from communist to non-communist governments
  (with the exception of China), they retained many authoritarian
  systems from their communist past.
\item
  \textbf{Continuity of former elites}: The transitions did not result
  in the removal or overthrow of previous ruling groups. Instead, former
  communist elites often maintained their positions of power.
\item
  \textbf{Subverted democratic processes}: While general elections and
  term limits were introduced in most of these countries, the legacy of
  former communist regimes frequently led to the circumvention of term
  limits and manipulation of elections
  (\citeproc{ref-nurumov2019}{Nurumov and Vashchanka 2019}).
\end{itemize}

\paragraph{Case 1: Lifelong ruler--Alexander Lukashenko in
Belarus}\label{case-1-lifelong-ruleralexander-lukashenko-in-belarus}

Alexander Lukashenko, a former member of the Supreme Soviet of the
Byelorussian Soviet Socialist Republic, became the head of the interim
anti-corruption committee of the Supreme Council of Belarus following
the dissolution of the Soviet Union. Elected as Belarus's first
president in 1994, he has maintained this position ever since.
Initially, the 1994 constitution limited presidents to two successive
terms. However, Lukashenko removed this restriction in 2004.
International monitors have not regarded Belarusian elections as free
and fair since his initial victory. Despite significant protests,
Lukashenko has consistently claimed to win with a high vote share, often
exceeding 80\% in each election. This pattern is evident across all five
Central Asian countries of the former Soviet Union, where
post-dissolution leaders were typically high officials or heads of the
former Soviet republics who continued their leadership in the
presidency.

\paragraph{Case 2: Transferring power to a handpicked
successor--Nursultan Nazarbayev in
Kazakhstan}\label{case-2-transferring-power-to-a-handpicked-successornursultan-nazarbayev-in-kazakhstan}

Nursultan Nazarbayev served as the first president of Kazakhstan from
1991 until 2019. Prior to the dissolution of the Soviet Union, he held
de facto leadership as the First Secretary of the Communist Party of
Kazakhstan. Following independence, he was elected as the first
president and retained office until 2019 through various means,
including resetting term limits due to the implementation of new
constitutions. Notably, Nazarbayev did not officially eliminate term
limits but instead created an exemption for the ``First President''
(\citeproc{ref-nurumov2019}{Nurumov and Vashchanka 2019}). Unlike
Lukashenko, who remains the incumbent of Belarus, Nazarbayev transferred
the presidency to a designated successor, Kassym-Jomart Tokayev, in
2019. However, he retained significant influence as the Chairman of the
Security Council of Kazakhstan until 2022.

\subsubsection{Autocoups for immediate re-election: Cases of Latin
America}\label{autocoups-for-immediate-re-election-cases-of-latin-america}

Latin America has a long-standing tradition of maintaining term limit
conventions. Simón Bolívar, the founding father of Bolivia, was
initially a strong advocate for term limits, stating in 1819, ``Nothing
is as dangerous as allowing the same citizen to remain in power for a
long time\ldots{} That's the origin of usurpation and tyranny''
(\citeproc{ref-ginsburg2019}{Ginsburg and Elkins 2019, 38}). Although
Bolívar eventually modified his stance, arguing in his 1826 Constitution
Assembly speech that ``a president for life with the right to choose the
successor is the most sublime inspiration for the republican order,''
term limits became a convention in Latin America. Approximately 81\% of
Latin American constitutions between independence and 1985 imposed some
form of term limits on the presidency
(\citeproc{ref-marsteintredet2019a}{Marsteintredet 2019}).

An analysis of cases in Latin American countries reveals two notable
patterns.

\paragraph{Often successful at breaking non-re-election or non-immediate
re-election
restrictions}\label{often-successful-at-breaking-non-re-election-or-non-immediate-re-election-restrictions}

Unlike other presidential systems where two terms are more common,
non-re-election or non-immediate re-election used to be prevalent in
Latin America. According to Marsteintredet
(\citeproc{ref-marsteintredet2019a}{2019}), non-consecutive re-election
was mandated in about 64.9\% of all constitutions between independence
and 1985, while 5.9\% banned re-election entirely.

However, adherence to these conventions has varied across the region.
Since Mexico introduced non-re-election institutions in 1911 at the
start of the Mexican Revolution, they have remained inviolate
(\citeproc{ref-klesner2019}{Klesner 2019}). Similarly, Panama and
Uruguay have never altered their re-election rules, and Costa Rica has
only experienced a brief period (1897-1913) permitting immediate
presidential re-election since prohibiting it in 1859
(\citeproc{ref-marsteintredet2019a}{Marsteintredet 2019}). In many other
countries, however, constitutions have been frequently amended or
violated.

The pursuit of re-election or consecutive re-election, therefore, has
been a significant trigger for autocoups aimed at power extension in
this region. Our research documents 32 autocoup cases, with over 50\%
(17 cases) attempting to enable re-election or immediate re-election,
and about 59\% (10 cases out of 17) being successful.

Unlike those who attempt to overstay in office indefinitely, many Latin
American leaders exit after their second term expires. Examples include
President Fernando Henrique Cardoso of Brazil (1995-2003), President
Danilo Medina of the Dominican Republic (2012-2020), and President Juan
Orlando Hernández of Honduras (2014-2022)
(\citeproc{ref-ginsburg2019}{Ginsburg and Elkins 2019};
\citeproc{ref-marsteintredet2019a}{Marsteintredet 2019};
\citeproc{ref-landau2019}{Landau, Roznai, and Dixon 2019};
\citeproc{ref-baturo2019}{Baturo 2019}; \citeproc{ref-neto2019}{Neto and
Acácio 2019}).

\paragraph{Failing to further extend
tenure}\label{failing-to-further-extend-tenure}

This trend does not imply that none of these leaders attempted further
extensions, but rather that most accepted their unsuccessful outcomes
without abusing their power to manipulate the process. While autocoups
aimed at securing one additional term are often successful, attempts to
overstay beyond this are frequently unsuccessful.

In contrast to the previous examples, two contrasting cases illustrate
the varied outcomes of term limit challenges:

\begin{itemize}
\item
  \textbf{Unsuccessful extension -- Carlos Menem (Argentina)}: President
  Menem successfully extended his tenure by one term through a 1994
  constitutional amendment allowing one executive re-election. He was
  subsequently re-elected in 1995. However, his attempt to reset his
  term count, arguing that his first term (1988-1995) should not count
  as it was under previous constitutions, was unanimously rejected by
  the Supreme Court in March 1999 (\citeproc{ref-llanos2019}{Llanos
  2019}). A similar scenario unfolded with President Álvaro Uribe of
  Colombia (2002-2010) (\citeproc{ref-baturo2019}{Baturo 2019}).
\item
  \textbf{Successful extension -- Daniel Ortega (Nicaragua)}: In
  contrast, Daniel Ortega, the incumbent president of Nicaragua,
  successfully extended his presidency. In 2009, the Supreme Court of
  Justice of Nicaragua permitted his re-election in 2011. Subsequently,
  in 2014, the National Assembly of Nicaragua approved constitutional
  amendments abolishing presidential term limits, allowing Ortega to run
  for an unlimited number of five-year terms. As a result, he has held
  the presidency since 2007 (\citeproc{ref-close2019}{Close 2019}).
\end{itemize}

\subsubsection{As common as classical coups: Cases of African
countries}\label{as-common-as-classical-coups-cases-of-african-countries}

Classical coups have been prevalent in Africa, accounting for
approximately 45\% of all global coups (219 out of 491 cases) since
1950, involving 45 out of 54 African countries (GIC dataset). While
autocoups are less frequent compared to traditional coups, they maintain
a significant presence in Africa. Among 110 documented autocoup cases
globally, 46\% (51 cases) occurred in Africa, involving 36 countries.
Notably, the success rate of autocoups in Africa is over 84\% (43 out of
51 attempts), which surpasses both the success rate of classical coups
in the region (roughly 50\%) and the global average success rate of
autocoups (79\%).

Identifying a clear pattern of autocoups in Africa is challenging,
mirroring the complexity observed with classical coups. Various factors
have been proposed to explain this phenomenon:

\begin{itemize}
\item
  \textbf{Natural Resources}: Countries rich in natural resources,
  particularly oil or diamonds, may see leaders more likely to attempt
  and succeed in extending their terms (\citeproc{ref-posner}{Posner and
  Young, n.d.}; \citeproc{ref-cheeseman2015}{Cheeseman 2015};
  \citeproc{ref-cheeseman2019a}{Cheeseman and Klaas 2019}).
\item
  \textbf{Quality of democracy}: The quality of democracy is a critical
  factor influencing respect for term limits
  (\citeproc{ref-reyntjens2016}{Reyntjens 2016}).
\item
  \textbf{International influence}: International aid or donor influence
  can play a significant role in discouraging attempts at power
  extension (\citeproc{ref-brown2001}{Brown 2001};
  \citeproc{ref-tangri2010}{Tangri and Mwenda 2010}).
\item
  \textbf{Organized opposition and party unity}: The extent of organized
  opposition and the president's ability to enforce unity within the
  ruling party are crucial factors
  (\citeproc{ref-cheeseman2019}{Cheeseman 2019}).
\end{itemize}

Utilizing the Africa Executive Term Limits (AETL) dataset, Cassani
(\citeproc{ref-cassani2020}{2020}) highlights human rights abuses and
the desire for impunity as main drivers for incumbents to cling to
power. The more authoritarian a leader, the more likely they are to
attempt to break term limits and overstay in office. A leader's ability
to secure the loyalty of the armed forces through public investment
increases the chances of success in overstaying.

Despite both coups and autocoups being prevalent, there has been a
noticeable shift since the end of the Cold War in 1991: Traditional
coups have decreased in frequency while autocoups have become more
prevalent.

This trend can be partially attributed to the introduction of
multi-party elections in Africa in the 1990s, which also brought in term
limits for executives (\citeproc{ref-cassani2020}{Cassani 2020};
\citeproc{ref-cheeseman2019}{Cheeseman 2019}). Before 1991, personal or
military rule was more common, and term limits were less frequent.
Post-1991, with more term limits introduced, challenges to these limits
have increased. However, it is crucial to note that this increase in
challenges does not necessarily imply that violations are more common
than adherence to term limits, because total power transitions have
increased compared to the past.

\subsection{Empirical analysis: An example of utilizing the autocoup
dataset}\label{empirical-analysis-an-example-of-utilizing-the-autocoup-dataset}

The autocoup dataset opens up opportunities for quantitative analysis
that extend beyond traditional case studies. This section presents a
clear example of how to effectively leverage this dataset. To examine
the factors driving autocoup attempts, I employ a probit regression
model, demonstrating its applicability for investigating the
determinants of such events. This example highlights the dataset's
potential for robust empirical inquiry and provides a foundation for
future research on the conditions that influence autocoup attempts.

\subsubsection{Dependent variables}\label{dependent-variables}

\begin{itemize}
\item
  \textbf{Autocoup attempt}: Binary variable indicating whether an
  autocoup attempt occurred (1) or not (0) during the tenure of an
  incumbent leader.
\item
  \textbf{Autocoup success}: Binary variable indicating whether an
  autocoup attempt was successful (1) or failed (0), conditional on an
  autocoup attempt occurring.
\end{itemize}

\subsubsection{Key independent variable: regime
type}\label{key-independent-variable-regime-type}

I categorize regime types following Geddes, Wright, and Frantz
(\citeproc{ref-geddes2014}{2014}) (GWF), focusing on military,
personalist, and dominant-party regimes, with democracies and monarchies
included for comparison. Descriptive statistics for regime types are
presented in Table~\ref{tbl-regimes}.

\begin{longtable}[]{@{}
  >{\raggedright\arraybackslash}p{(\columnwidth - 4\tabcolsep) * \real{0.3333}}
  >{\raggedleft\arraybackslash}p{(\columnwidth - 4\tabcolsep) * \real{0.3333}}
  >{\raggedleft\arraybackslash}p{(\columnwidth - 4\tabcolsep) * \real{0.3333}}@{}}

\caption{\label{tbl-regimes}Regime types since 1950}

\tabularnewline

\toprule\noalign{}
\begin{minipage}[b]{\linewidth}\raggedright
Regime Type
\end{minipage} & \begin{minipage}[b]{\linewidth}\raggedleft
Country Year
\end{minipage} & \begin{minipage}[b]{\linewidth}\raggedleft
Share
\end{minipage} \\
\midrule\noalign{}
\endhead
\midrule\noalign{}
\multicolumn{3}{@{}>{\raggedright\arraybackslash}p{(\columnwidth - 4\tabcolsep) * \real{1.0000} + 4\tabcolsep}@{}}{%
\begin{minipage}[t]{\linewidth}\raggedright
\emph{Source: REIGN Datasets}
\end{minipage}} \\
\bottomrule\noalign{}
\endlastfoot
Democracy & 5303 & 46.7\% \\
Dominant-Party & 2569 & 22.6\% \\
Personal & 1477 & 13.0\% \\
Monarchy & 1056 & 9.3\% \\
Military & 638 & 5.6\% \\
Other & 322 & 2.8\% \\
Total & 11365 & 100.0\% \\

\end{longtable}

\subsubsection{Control variables}\label{control-variables}

The control variables are chosen based on the research of Gassebner,
Gutmann, and Voigt (\citeproc{ref-gassebner2016}{2016}). They analysed
66 factors potentially influencing coups and found that slow economic
growth, prior coup attempts, and other forms of political violence are
particularly significant factors. Therefore, we include economic
performance, political violence, and the number of previous coups as our
main control variables.

\begin{itemize}
\tightlist
\item
  \textbf{Economic Level:} Represented by GDP per capita. This measure
  provides an indication of the overall economic health and standard of
  living in a country. We use GDP per capita data (in constant 2017
  international 1000 dollars, PPP) from the V-Dem dataset by Fariss et
  al. (\citeproc{ref-fariss2022}{2022}).
\item
  \textbf{Economic Performance:} Measured using the current-trend
  (\(CT\)) ratio developed by Krishnarajan
  (\citeproc{ref-krishnarajan2019}{2019}). This ratio compares a
  country's current GDP per capita to the average GDP per capita over
  the previous five years. A higher \(CT\) ratio indicates stronger
  economic performance. For a country \(i\) at year \(t\), the \(CT\)
  ratio is calculated as follows:
\end{itemize}

\begin{equation}\phantomsection\label{eq-eq6}{
    \begin{aligned}
    CT_{i,t} = {GDP/cap_{i,t} \over {1 \over 5} {\sum_{k=1}^5GDP/cap_{i,t-k}}}
    \end{aligned}
}\end{equation}

\begin{itemize}
\item
  \textbf{Political stability:} This variable captures overall regime
  stability by including a violence index that encompasses all types of
  internal and interstate wars and violence. The data for this index is
  sourced from the variable ``actotal'' in the Major Episodes of
  Political Violence dataset
  (\citeproc{ref-marshall2005current}{Marshall 2005}), with 0
  representing the most stable conditions (no violence at all) and 18
  representing the most unstable.
\item
  \textbf{Previous coups:} Included in the selection equation as either:
  a) The number of previous coups in a country (Model 1), or b) The time
  since the last coup attempt (Model 2 for robustness check).
\item
  \textbf{Population size:} To account for its potential impact on
  leaders' tenures, we consider the log of the population size. This
  transformation helps in managing the wide range of population sizes
  across different countries. The data is sourced from the V-Dem dataset
  and is evaluated to understand its influence on power transitions.
  Larger populations may present more governance challenges and
  potential sources of opposition, thereby affecting the stability and
  longevity of a leader's tenure.
\item
  \textbf{Leader's age:} The age of the leader is included as an
  additional variable in the analysis, offering insights into potential
  correlations with leadership strength. Older leaders may have
  different experiences, networks, and health considerations that could
  influence their ability to maintain power. This data is sourced from
  Archigos and PLAD datasets.
\end{itemize}

Unlike the analysis of classic coup determinants, which could
theoretically occur in any given year, I assume that an autocoup happens
only once during an incumbent leader's tenure, as a successful autocoup
negates the need for another attempt. However, this assumption does not
always reflect reality, as leaders might attempt further extensions or
try again after a failed attempt. For simplicity, I overlook these
possibilities in our analysis.

Therefore, in our probit model, the unit of analysis for autocoups is
the entire tenure of a leader, rather than a country-year. I establish a
base year for the variables: for leaders who staged an autocoup, we use
the year of their first attempt as the base year; for leaders who did
not attempt to overstay, I use the middle year of their tenure as the
base year.

\subsubsection{Results and discussions}\label{results-and-discussions}

\begin{table}

\caption{\label{tbl-autocoupmodel}Determinants of autocoup attempts and
success (1945-2018)}

\centering{

\begin{tabular}{@{\extracolsep{50pt}}lcc} 
\\[-1.8ex]\hline 
\hline \\[-1.8ex] 
 & Autocoup Attempts & Autocoup Outcome \\ 
\\[-1.8ex] & (1) & (2)\\ 
\hline \\[-1.8ex] 
 Constant & $-$1.674$^{***}$ & $-$0.888 \\ 
  & (0.624) & (1.935) \\ 
  & & \\ 
 Regime: Dominant-party & 0.070 & 0.672$^{*}$ \\ 
  & (0.145) & (0.402) \\ 
  & & \\ 
 \hspace{1.6cm}Military & $-$0.255 & 0.615 \\ 
  & (0.189) & (0.541) \\ 
  & & \\ 
 \hspace{1.6cm}Personalist & 0.737$^{***}$ & 1.609$^{***}$ \\ 
  & (0.157) & (0.448) \\ 
  & & \\ 
 GDP per capita & $-$0.009 & 0.064 \\ 
  & (0.011) & (0.045) \\ 
  & & \\ 
 Economic trend & 0.653 & 0.197 \\ 
  & (0.533) & (1.772) \\ 
  & & \\ 
 Political stability & $-$0.044 & 0.126 \\ 
  & (0.036) & (0.130) \\ 
  & & \\ 
 Age & $-$0.001 & 0.004 \\ 
  & (0.001) & (0.017) \\ 
  & & \\ 
 Population(log) & $-$0.048 & 0.029 \\ 
  & (0.042) & (0.144) \\ 
  & & \\ 
\hline \\[-1.8ex] 
Observations & 1,028 & 102 \\ 
Log Likelihood & $-$308.494 & $-$43.651 \\ 
Akaike Inf. Crit. & 634.988 & 105.302 \\ 
\hline 
\hline \\[-1.8ex] 
\textit{Note:}  & \multicolumn{2}{r}{$^{*}$p$<$0.1; $^{**}$p$<$0.05; $^{***}$p$<$0.01} \\ 
\end{tabular}

}

\end{table}%

Table~\ref{tbl-autocoupmodel} summarizes the findings from the probit
regression models based on our analysis of the determinants of autocoup
attempts and their success.

Model 1, which examines autocoup attempts, reveals only one significant
predictor besides the constant term. Among the regime types, personalist
regimes significantly increase the likelihood of autocoup attempts, all
else being equal. This suggests that leaders in personalist regimes are
more prone to attempt to extend their power through autocoups compared
to leaders in democratic regimes (reference regime). Leaders in
dominant-party and military regimes, however, show no significant
difference in the likelihood of attempting an autocoup compared to
democratic leaders.

The model for autocoup success (Model 2) shows similar dynamics.
Personalist regimes again have a strong positive and significant effect
on the success of autocoups compared to democratic leaders.
Dominant-party regimes also show a positive and marginally significant
effect. However, a detailed examination reveals that about half of the
successful autocoups in dominant-party regimes (9 out of 20) exhibit a
personalist style, such as ``party-personal-military'' regimes.

This outcome is logical since personalist leaders are typically much
more powerful than other types of leaders, making them more inclined and
capable of overstaying in power.

Other factors play an insignificant role in determining the attempts and
outcomes of autocoups. This aligns with our conclusions on the
determinants of classic coups. Both coups and autocoups are
significantly affected by power dynamics. As power transitions involve
the struggle between seizing and maintaining power, the balance of power
status quo inevitably matters in both coups and autocoups. This also
explains the high success rate of autocoups. Compared to power
challengers, incumbents are in an obviously advantageous position.
Incumbent leaders can use state power to their benefit, which is
difficult to counteract. Even the abuse of power is often unchecked
under a powerful leader's rule.

The empirical analysis of autocoups yields significant implications for
real-world politics. In particular, the high overall success rate of
autocoups highlights the vulnerability of democratic institutions to
gradual erosion by incumbent leaders. The threshold for ousting or
impeaching an incumbent leader through constitutional means is
exceptionally high, with success often requiring more than a simple
majority and substantial support across various sectors. Resorting to
illegal means, such as a coup, presents even greater challenges due to
high costs, severe consequences, and a low likelihood of success.

Conversely, political dynamics, whether in democracies or autocracies,
tend to favour incumbents even when they act unconstitutionally.
Incumbents can leverage state resources to achieve their political
ambitions, benefiting from a high probability of success and minimal
consequences in case of failure. This asymmetry in power and risk
creates a concerning scenario: for incumbents who do not respect
constitutional institutions, the opportunity to launch an autocoup
appears sufficiently low-risk to warrant an attempt.

\subsection{Conclusion}\label{conclusion}

This study conducts a thorough and comprehensive analysis of autocoups,
with a specific focus on political events where incumbent leaders
illegitimately extend their tenure in power. By refining the existing
definition and distinguishing autocoups from related concepts such as
``self-coups'', ``autogolpes'', and ``executive takeovers'', this
research introduces a novel dataset that catalogues autocoups from 1945
to 2023. This refined definition and the accompanying dataset enable the
study to broaden its analysis of irregular leadership transitions. While
traditional analyses often concentrate on the abrupt termination of
tenure through coups, this research expands the scope to include the
irregular extension of tenure through autocoups. This approach provides
a more comprehensive and nuanced understanding of the phenomenon,
highlighting the various mechanisms by which incumbent leaders can
subvert democratic processes to maintain their power.

The findings reveal that personalist regimes are significantly more
likely to experience autocoup attempts and succeed in these attempts
compared to democracies. Dominant-party systems, often exhibiting
personalist characteristics, also show an association with successful
autocoups. While regime type significantly influences autocoups, other
factors appear less impactful, mirroring classic coups where the balance
of power is a more essential determinant. The high success rate of
autocoups can be attributed to the inherent advantages incumbents
possess, such as control over or abuse of state power and the difficulty
of removing or impeaching them through legal or illegal means.

However, several limitations warrant consideration for future research.
Firstly, the definition of an autocoup requires further commentary and
discussion to gain wider acceptance in the academic community. Despite
efforts to maintain objectivity, some coding decisions may involve
subjective judgments, particularly in borderline cases. Secondly, due to
the nature of autocoups, which are less frequent than classic coups (491
coups versus 110 autocoups during the same period), the quantitative
analysis cannot be conducted as a country-year variable as in coup
studies. This raises the issue of choosing an appropriate base year for
the analysis, which requires further discussion and potentially
sensitivity analyses.

Despite these limitations, this research significantly enhances our
understanding of the mechanisms and motivations behind autocoups,
contributing to the literature on political stability and democratic
resilience. The findings highlight the vulnerability of political
systems, particularly democracies, to erosion from within by incumbent
leaders.

Future studies could build on this work by employing the dataset to
explore more nuanced power dynamics or examine the long-term impacts of
these events on political systems. Particularly fruitful areas for
investigation include the relationship between autocoups and democratic
backsliding, democratic breakdown, and the personalization of power.
Additionally, comparative analyses between autocoups and traditional
coups could yield insights into the evolving nature of power
consolidation strategies in different political contexts.

In conclusion, this study enhances our understanding of autocoups by
clarifying terminology, refining definitions, and providing a
comprehensive dataset. Future research could explore the relationship
between autocoups and democratic backsliding, democratic breakdown, and
the personalization of power.

\newpage

\subsection*{References}\label{references}
\addcontentsline{toc}{subsection}{References}

\phantomsection\label{refs}
\begin{CSLReferences}{1}{0}
\bibitem[\citeproctext]{ref-antonio2021}
Antonio, Robert J. 2021. {``Democracy and Capitalism in the Interregnum:
Trump{'}s Failed Self-Coup and After.''} \emph{Critical Sociology} 48
(6): 937--65. \url{https://doi.org/10.1177/08969205211049499}.

\bibitem[\citeproctext]{ref-baturo2014}
Baturo, Alexander. 2014. {``Democracy, Dictatorship, and Term Limits.''}
\url{https://doi.org/10.3998/mpub.4772634}.

\bibitem[\citeproctext]{ref-baturo2019}
---------. 2019. {``Continuismo in Comparison.''} In, 75--100. Oxford
University Press.
\url{https://doi.org/10.1093/oso/9780198837404.003.0005}.

\bibitem[\citeproctext]{ref-baturo2022}
Baturo, Alexander, and Jakob Tolstrup. 2022. {``Incumbent Takeovers.''}
\emph{Journal of Peace Research} 60 (2): 373--86.
\url{https://doi.org/10.1177/00223433221075183}.

\bibitem[\citeproctext]{ref-bermeo2016}
Bermeo, Nancy. 2016. {``On Democratic Backsliding.''} \emph{Journal of
Democracy} 27 (1): 5--19. \url{https://doi.org/10.1353/jod.2016.0012}.

\bibitem[\citeproctext]{ref-bomprezzi2024wedded}
Bomprezzi, Pietro, Axel Dreher, Andreas Fuchs, Teresa Hailer, Andreas
Kammerlander, Lennart Kaplan, Silvia Marchesi, Tania Masi, Charlotte
Robert, and Kerstin Unfried. 2024. {``Wedded to Prosperity? Informal
Influence and Regional Favoritism.''} Discussion Paper. CEPR.

\bibitem[\citeproctext]{ref-brown2001}
Brown, Stephen. 2001. {``Authoritarian Leaders and Multiparty Elections
in Africa: How Foreign Donors Help to Keep Kenya's Daniel Arap Moi in
Power.''} \emph{Third World Quarterly} 22 (5): 725--39.
\url{https://doi.org/10.1080/01436590120084575}.

\bibitem[\citeproctext]{ref-cameron1998a}
Cameron, Maxwell A. 1998a. {``Latin American Autogolpes : Dangerous
Undertows in the Third Wave of Democratisation.''} \emph{Third World
Quarterly} 19 (2): 219--39.
\url{https://doi.org/10.1080/01436599814433}.

\bibitem[\citeproctext]{ref-cameron1998}
Cameron, Maxwell A. 1998b. {``Self-Coups: Peru, Guatemala, and
Russia.''} \emph{Journal of Democracy} 9 (1): 125--39.
\url{https://doi.org/10.1353/jod.1998.0003}.

\bibitem[\citeproctext]{ref-cassani2020}
Cassani, Andrea. 2020. {``Autocratisation by Term Limits Manipulation in
Sub-Saharan Africa.''} \emph{Africa Spectrum} 55 (3): 228--50.
\url{https://doi.org/10.1177/0002039720964218}.

\bibitem[\citeproctext]{ref-cheeseman2015}
Cheeseman, Nic. 2015. {``Democracy in Africa,''} March.
\url{https://doi.org/10.1017/cbo9781139030892}.

\bibitem[\citeproctext]{ref-cheeseman2019}
---------. 2019. {``Should I Stay or Should I Go? Term Limits,
Elections, and Political Change in Kenya, Uganda, and Zambia.''} In,
311--38. Oxford University PressOxford.
\url{https://doi.org/10.1093/oso/9780198837404.003.0016}.

\bibitem[\citeproctext]{ref-cheeseman2019a}
Cheeseman, Nic, and Brian Klaas. 2019. \emph{How to Rig an Election}.
Yale University Press. \url{https://doi.org/10.12987/9780300235210}.

\bibitem[\citeproctext]{ref-close2019}
Close, David. 2019. {``Presidential Term Limits in Nicaragua.''} In,
159--78. Oxford University PressOxford.
\url{https://doi.org/10.1093/oso/9780198837404.003.0009}.

\bibitem[\citeproctext]{ref-ezrow2019}
Ezrow, Natasha. 2019. {``Term Limits and Succession in Dictatorships.''}
In, 269--88. Oxford University PressOxford.
\url{https://doi.org/10.1093/oso/9780198837404.003.0014}.

\bibitem[\citeproctext]{ref-fariss2022}
Fariss, Christopher J., Therese Anders, Jonathan N. Markowitz, and
Miriam Barnum. 2022. {``New Estimates of Over 500 Years of Historic GDP
and Population Data.''} \emph{Journal of Conflict Resolution} 66 (3):
553--91. \url{https://doi.org/10.1177/00220027211054432}.

\bibitem[\citeproctext]{ref-gassebner2016}
Gassebner, Martin, Jerg Gutmann, and Stefan Voigt. 2016. {``When to
Expect a Coup d{'}état? An Extreme Bounds Analysis of Coup
Determinants.''} \emph{Public Choice} 169 (3-4): 293--313.
\url{https://doi.org/10.1007/s11127-016-0365-0}.

\bibitem[\citeproctext]{ref-geddes2014}
Geddes, Barbara, Joseph Wright, and Erica Frantz. 2014. {``Autocratic
Breakdown and Regime Transitions: A New Data Set.''} \emph{Perspectives
on Politics} 12 (2): 313--31.
\url{https://doi.org/10.1017/s1537592714000851}.

\bibitem[\citeproctext]{ref-ginsburg2019}
Ginsburg, Tom, and Zachary Elkins. 2019. {``One Size Does Not Fit
All.''} In, 37--52. Oxford University Press.
\url{https://doi.org/10.1093/oso/9780198837404.003.0003}.

\bibitem[\citeproctext]{ref-ginsburg2010evasion}
Ginsburg, Tom, James Melton, and Zachary Elkins. 2010. {``On the Evasion
of Executive Term Limits.''} \emph{Wm. \& Mary L. Rev.} 52: 1807.

\bibitem[\citeproctext]{ref-ginsburg2011evasion}
---------. 2011. {``On the Evasion of Executive Term Limits.''}
\emph{William and Mary Law Review} 52: 1807.

\bibitem[\citeproctext]{ref-goemans2009}
Goemans, Henk E., Kristian Skrede Gleditsch, and Giacomo Chiozza. 2009.
{``Introducing Archigos: A Dataset of Political Leaders.''}
\emph{Journal of Peace Research} 46 (2): 269--83.
\url{https://doi.org/10.1177/0022343308100719}.

\bibitem[\citeproctext]{ref-helmke2017}
Helmke, Gretchen. 2017. {``Institutions on the Edge,''} January.
\url{https://doi.org/10.1017/9781139031738}.

\bibitem[\citeproctext]{ref-klesner2019}
Klesner, Joseph L. 2019. {``The Politics of Presidential Term Limits in
Mexico.''} In, 141--58. Oxford University Press.
\url{https://doi.org/10.1093/oso/9780198837404.003.0008}.

\bibitem[\citeproctext]{ref-krishnarajan2019}
Krishnarajan, Suthan. 2019. {``Economic Crisis, Natural Resources, and
Irregular Leader Removal in Autocracies.''} \emph{International Studies
Quarterly} 63 (3): 726--41. \url{https://doi.org/10.1093/isq/sqz006}.

\bibitem[\citeproctext]{ref-landau2019}
Landau, David, Yaniv Roznai, and Rosalind Dixon. 2019. {``Term Limits
and the Unconstitutional Constitutional Amendment Doctrine.''} In,
53--74. Oxford University PressOxford.
\url{https://doi.org/10.1093/oso/9780198837404.003.0004}.

\bibitem[\citeproctext]{ref-llanos2019}
Llanos, Mariana. 2019. {``The Politics of Presidential Term Limits in
Argentina.''} In, 473--94. Oxford University Press.
\url{https://doi.org/10.1093/oso/9780198837404.003.0023}.

\bibitem[\citeproctext]{ref-marshall2005current}
Marshall, Monty G. 2005. {``Current Status of the World's Major Episodes
of Political Violence.''} \emph{Report to Political Instability Task
Force.(3 February)}.

\bibitem[\citeproctext]{ref-marsteintredet2019a}
Marsteintredet, Leiv. 2019. {``Presidential Term Limits in Latin
America: {\emph{C}}.1820{\textendash}1985.''} In, 103--22. Oxford
University PressOxford.
\url{https://doi.org/10.1093/oso/9780198837404.003.0006}.

\bibitem[\citeproctext]{ref-marsteintredet2019}
Marsteintredet, Leiv, and Andrés Malamud. 2019. {``Coup with Adjectives:
Conceptual Stretching or Innovation in Comparative Research?''}
\emph{Political Studies} 68 (4): 1014--35.
\url{https://doi.org/10.1177/0032321719888857}.

\bibitem[\citeproctext]{ref-mauceri1995}
Mauceri, Philip. 1995. {``State Reform, Coalitions, and The Neoliberal
{\emph{Autogolpe}} in Peru.''} \emph{Latin American Research Review} 30
(1): 7--37. \url{https://doi.org/10.1017/s0023879100017155}.

\bibitem[\citeproctext]{ref-neto2019}
Neto, Octavio Amorim, and Igor P. Acácio. 2019. {``Presidential Term
Limits as a Credible-Commitment Mechanism.''} In, 123--40. Oxford
University PressOxford.
\url{https://doi.org/10.1093/oso/9780198837404.003.0007}.

\bibitem[\citeproctext]{ref-nurumov2019}
Nurumov, Dmitry, and Vasil Vashchanka. 2019. {``Presidential Terms in
Kazakhstan.''} In, 221--46. Oxford University PressOxford.
\url{https://doi.org/10.1093/oso/9780198837404.003.0012}.

\bibitem[\citeproctext]{ref-pion-berlin2022}
Pion-Berlin, David, Thomas Bruneau, and Richard B. Goetze. 2022. {``The
Trump Self-Coup Attempt: Comparisons and Civil{\textendash}Military
Relations.''} \emph{Government and Opposition} 58 (4): 789--806.
\url{https://doi.org/10.1017/gov.2022.13}.

\bibitem[\citeproctext]{ref-posner}
Posner, Daniel N., and Daniel J. Young. n.d. {``Term Limits: Leadership,
Political Competition and the Transfer of Power.''} In, 260--78.
Cambridge University Press.
\url{https://doi.org/10.1017/9781316562888.011}.

\bibitem[\citeproctext]{ref-przeworski2000}
Przeworski, Adam, Michael E. Alvarez, Jose Antonio Cheibub, and Fernando
Limongi. 2000. {``Democracy and Development,''} August.
\url{https://doi.org/10.1017/cbo9780511804946}.

\bibitem[\citeproctext]{ref-reyntjens2016}
Reyntjens, Filip. 2016. {``A New Look at the Evidence.''} \emph{Journal
of Democracy} 27 (3): 61--68.
\url{https://doi.org/10.1353/jod.2016.0044}.

\bibitem[\citeproctext]{ref-stinnett2002}
Stinnett, Douglas M., Jaroslav Tir, Paul F. Diehl, Philip Schafer, and
Charles Gochman. 2002. {``The Correlates of War (Cow) Project Direct
Contiguity Data, Version 3.0.''} \emph{Conflict Management and Peace
Science} 19 (2): 59--67.
\url{https://doi.org/10.1177/073889420201900203}.

\bibitem[\citeproctext]{ref-svolik2014}
Svolik, Milan W. 2014. {``Which Democracies Will Last? Coups, Incumbent
Takeovers, and the Dynamic of Democratic Consolidation.''} \emph{British
Journal of Political Science} 45 (4): 715--38.
\url{https://doi.org/10.1017/s0007123413000550}.

\bibitem[\citeproctext]{ref-tangri2010}
Tangri, Roger, and Andrew M. Mwenda. 2010. {``President Museveni and the
Politics of Presidential Tenure in Uganda.''} \emph{Journal of
Contemporary African Studies} 28 (1): 31--49.
\url{https://doi.org/10.1080/02589000903542574}.

\bibitem[\citeproctext]{ref-versteeg2020law}
Versteeg, Mila, Timothy Horley, Anne Meng, Mauricio Guim, and Marilyn
Guirguis. 2020. {``The Law and Politics of Presidential Term Limit
Evasion.''} \emph{Colum. L. Rev.} 120: 173.

\end{CSLReferences}




\end{document}

% Options for packages loaded elsewhere
% Options for packages loaded elsewhere
\PassOptionsToPackage{unicode}{hyperref}
\PassOptionsToPackage{hyphens}{url}
\PassOptionsToPackage{dvipsnames,svgnames,x11names}{xcolor}
%
\documentclass[
  12pt,
]{article}
\usepackage{xcolor}
\usepackage[top = 3cm,bottom = 3cm,left = 3cm,right = 2.7cm]{geometry}
\usepackage{amsmath,amssymb}
\setcounter{secnumdepth}{5}
\usepackage{iftex}
\ifPDFTeX
  \usepackage[T1]{fontenc}
  \usepackage[utf8]{inputenc}
  \usepackage{textcomp} % provide euro and other symbols
\else % if luatex or xetex
  \usepackage{unicode-math} % this also loads fontspec
  \defaultfontfeatures{Scale=MatchLowercase}
  \defaultfontfeatures[\rmfamily]{Ligatures=TeX,Scale=1}
\fi
\usepackage{lmodern}
\ifPDFTeX\else
  % xetex/luatex font selection
  \setmainfont[]{Times New Roman}
  \setsansfont[]{Arial}
  \setmonofont[]{Courier New}
\fi
% Use upquote if available, for straight quotes in verbatim environments
\IfFileExists{upquote.sty}{\usepackage{upquote}}{}
\IfFileExists{microtype.sty}{% use microtype if available
  \usepackage[]{microtype}
  \UseMicrotypeSet[protrusion]{basicmath} % disable protrusion for tt fonts
}{}
\usepackage{setspace}
% Make \paragraph and \subparagraph free-standing
\makeatletter
\ifx\paragraph\undefined\else
  \let\oldparagraph\paragraph
  \renewcommand{\paragraph}{
    \@ifstar
      \xxxParagraphStar
      \xxxParagraphNoStar
  }
  \newcommand{\xxxParagraphStar}[1]{\oldparagraph*{#1}\mbox{}}
  \newcommand{\xxxParagraphNoStar}[1]{\oldparagraph{#1}\mbox{}}
\fi
\ifx\subparagraph\undefined\else
  \let\oldsubparagraph\subparagraph
  \renewcommand{\subparagraph}{
    \@ifstar
      \xxxSubParagraphStar
      \xxxSubParagraphNoStar
  }
  \newcommand{\xxxSubParagraphStar}[1]{\oldsubparagraph*{#1}\mbox{}}
  \newcommand{\xxxSubParagraphNoStar}[1]{\oldsubparagraph{#1}\mbox{}}
\fi
\makeatother


\usepackage{longtable,booktabs,array}
\usepackage{calc} % for calculating minipage widths
% Correct order of tables after \paragraph or \subparagraph
\usepackage{etoolbox}
\makeatletter
\patchcmd\longtable{\par}{\if@noskipsec\mbox{}\fi\par}{}{}
\makeatother
% Allow footnotes in longtable head/foot
\IfFileExists{footnotehyper.sty}{\usepackage{footnotehyper}}{\usepackage{footnote}}
\makesavenoteenv{longtable}
\usepackage{graphicx}
\makeatletter
\newsavebox\pandoc@box
\newcommand*\pandocbounded[1]{% scales image to fit in text height/width
  \sbox\pandoc@box{#1}%
  \Gscale@div\@tempa{\textheight}{\dimexpr\ht\pandoc@box+\dp\pandoc@box\relax}%
  \Gscale@div\@tempb{\linewidth}{\wd\pandoc@box}%
  \ifdim\@tempb\p@<\@tempa\p@\let\@tempa\@tempb\fi% select the smaller of both
  \ifdim\@tempa\p@<\p@\scalebox{\@tempa}{\usebox\pandoc@box}%
  \else\usebox{\pandoc@box}%
  \fi%
}
% Set default figure placement to htbp
\def\fps@figure{htbp}
\makeatother


% definitions for citeproc citations
\NewDocumentCommand\citeproctext{}{}
\NewDocumentCommand\citeproc{mm}{%
  \begingroup\def\citeproctext{#2}\cite{#1}\endgroup}
\makeatletter
 % allow citations to break across lines
 \let\@cite@ofmt\@firstofone
 % avoid brackets around text for \cite:
 \def\@biblabel#1{}
 \def\@cite#1#2{{#1\if@tempswa , #2\fi}}
\makeatother
\newlength{\cslhangindent}
\setlength{\cslhangindent}{1.5em}
\newlength{\csllabelwidth}
\setlength{\csllabelwidth}{3em}
\newenvironment{CSLReferences}[2] % #1 hanging-indent, #2 entry-spacing
 {\begin{list}{}{%
  \setlength{\itemindent}{0pt}
  \setlength{\leftmargin}{0pt}
  \setlength{\parsep}{0pt}
  % turn on hanging indent if param 1 is 1
  \ifodd #1
   \setlength{\leftmargin}{\cslhangindent}
   \setlength{\itemindent}{-1\cslhangindent}
  \fi
  % set entry spacing
  \setlength{\itemsep}{#2\baselineskip}}}
 {\end{list}}
\usepackage{calc}
\newcommand{\CSLBlock}[1]{\hfill\break\parbox[t]{\linewidth}{\strut\ignorespaces#1\strut}}
\newcommand{\CSLLeftMargin}[1]{\parbox[t]{\csllabelwidth}{\strut#1\strut}}
\newcommand{\CSLRightInline}[1]{\parbox[t]{\linewidth - \csllabelwidth}{\strut#1\strut}}
\newcommand{\CSLIndent}[1]{\hspace{\cslhangindent}#1}



\setlength{\emergencystretch}{3em} % prevent overfull lines

\providecommand{\tightlist}{%
  \setlength{\itemsep}{0pt}\setlength{\parskip}{0pt}}



 


\usepackage{sectsty}
\chapterfont{\centering}
\usepackage{lscape}
\newcommand{\blandscape}{\begin{landscape}}
\newcommand{\elandscape}{\end{landscape}}
\makeatletter
\@ifpackageloaded{caption}{}{\usepackage{caption}}
\AtBeginDocument{%
\ifdefined\contentsname
  \renewcommand*\contentsname{Table of contents}
\else
  \newcommand\contentsname{Table of contents}
\fi
\ifdefined\listfigurename
  \renewcommand*\listfigurename{List of Figures}
\else
  \newcommand\listfigurename{List of Figures}
\fi
\ifdefined\listtablename
  \renewcommand*\listtablename{List of Tables}
\else
  \newcommand\listtablename{List of Tables}
\fi
\ifdefined\figurename
  \renewcommand*\figurename{Figure}
\else
  \newcommand\figurename{Figure}
\fi
\ifdefined\tablename
  \renewcommand*\tablename{Table}
\else
  \newcommand\tablename{Table}
\fi
}
\@ifpackageloaded{float}{}{\usepackage{float}}
\floatstyle{ruled}
\@ifundefined{c@chapter}{\newfloat{codelisting}{h}{lop}}{\newfloat{codelisting}{h}{lop}[chapter]}
\floatname{codelisting}{Listing}
\newcommand*\listoflistings{\listof{codelisting}{List of Listings}}
\makeatother
\makeatletter
\makeatother
\makeatletter
\@ifpackageloaded{caption}{}{\usepackage{caption}}
\@ifpackageloaded{subcaption}{}{\usepackage{subcaption}}
\makeatother
\usepackage{bookmark}
\IfFileExists{xurl.sty}{\usepackage{xurl}}{} % add URL line breaks if available
\urlstyle{same}
\hypersetup{
  pdftitle={Autocoups and Democracy},
  pdfauthor={Zhu Qi},
  colorlinks=true,
  linkcolor={blue},
  filecolor={Maroon},
  citecolor={Blue},
  urlcolor={blue},
  pdfcreator={LaTeX via pandoc}}


\title{Autocoups and Democracy}
\author{Zhu Qi}
\date{2025-11-24}
\begin{document}
\maketitle


\setstretch{1.618}
\section*{Abstract}\label{abstract}
\addcontentsline{toc}{section}{Abstract}

This chapter investigates the impact of autocoups on political
institutions, comparing them with traditional coups through an analysis
of variations in Polity V scores. It advances two primary hypotheses:
first, that incumbent leaders frequently consolidate power by
systematically undermining institutional constraints in the period
leading up to an autocoup, resulting in a decline in Polity V scores
attributable to the autocoup. Second, unlike traditional coups, which
exhibit a ``U-shaped'' trajectory in Polity V scores, autocoups
precipitate a persistent decline in these scores without subsequent
recovery. This is attributed to autocoup leaders' deliberate intent to
suppress opposition and dismantle institutional checks and balances to
secure prolonged tenure. Employing a country-fixed effects model, this
study demonstrates that Polity V scores typically decline following
autocoups, mirroring the magnitude of decline observed after traditional
coups. However, while traditional coups often lead to an immediate
reduction in Polity V scores followed by conditions conducive to
recovery over time, autocoups result in sustained democratic erosion.
These findings highlight the divergent political trajectories induced by
coups and autocoups. This research addresses a critical gap in the
empirical analysis of autocoups and contributes to academic and policy
discussion by elucidating their detrimental effects, particularly in
terms of democratic backsliding and the entrenchment of authoritarian
governance.

\textbf{Keywords:} \emph{Coups, Autocoups, Democratization}

\newpage

\section{Introduction}\label{introduction}

The global decline in political rights and civil liberties---now
extending for eighteen consecutive years
(\citeproc{ref-freedomhouse2024freedom}{Freedom House 2024})---has
renewed debate over the mechanisms driving contemporary democratic
backsliding. Historically, the coup d'état has been viewed as a central
threat to democratic governance, implicated in episodes of repression
and the breakdown of constitutional order. Yet recent empirical patterns
complicate this assumption. The incidence of traditional coups has
fallen sharply: the decade from 2008 to 2017 witnessed the lowest number
of attempts since 1960 (\citeproc{ref-powell2011}{Powell and Thyne
2011}; \citeproc{ref-thyne2019}{Thyne and Powell 2019}). Moreover, a
growing literature argues that coups can, under certain conditions,
catalyze democratization by removing entrenched autocrats or resolving
political deadlock (\citeproc{ref-powell2014a}{J. Powell 2014};
\citeproc{ref-thyne2014}{C. Thyne and Powell 2014};
\citeproc{ref-dahl2023}{Dahl and Gleditsch 2023}). If coup frequency is
declining and their long-term effects increasingly mixed---or even
beneficial---they are unlikely to account for today's widespread erosion
of democratic norms.

In contrast, another type of irregular power grab has become markedly
more common since 2000: the autocoup. Unlike coups, which typically
involve external attempts to unseat an incumbent
(\citeproc{ref-powell2011}{Powell and Thyne 2011}), autocoups are
perpetrated by incumbents themselves. In this article, an autocoup
refers to the extension of a leader's tenure beyond constitutional
limits through extra-legal manipulation of institutional rules
(\citeproc{ref-zhu2024}{Zhu 2024}). Autocoups represent a form of
internally driven subversion in which executives dismantle checks and
balances from within---often incrementally and under the guise of
legality. Despite their rising prevalence
(\citeproc{ref-bermeo2016}{Bermeo 2016};
\citeproc{ref-baturo2022}{Baturo and Tolstrup 2022};
\citeproc{ref-zhu2024}{Zhu 2024}) and clear relevance to democratic
backsliding, autocoups remain under-examined in comparative politics.

Autocoups should have systematically negative institutional effects
because they combine two reinforcing mechanisms: revealed authoritarian
intent and structural incentives for continued repression. By violating
constitutional limits, incumbents demonstrate a willingness to override
institutional constraints---signaling to elites that formal rules are
malleable. At the same time, once a leader has illegally extended
tenure, stepping down becomes personally risky, creating powerful
incentives to further weaken oversight bodies, politicize security
forces, and suppress opposition. These mechanisms predict a
unidirectional, persistent erosion of democratic institutions following
an autocoup.

This article provides the first systematic empirical assessment of these
institutional consequences. Specifically, it compares the impact of
coups and autocoups on democratic quality, operationalized through
changes in Polity V scores. The analysis addresses two core questions:
(1) How autocoups impact democratic backsliding? and (2) How do their
effects differ from those of traditional coups? While both forms of
irregular power seizure generate immediate institutional disruption, I
argue that their trajectories diverge sharply over time. Coups often
produce a U-shaped pattern---an initial decline in democratic quality
followed by medium-term recovery---whereas autocoups yield persistent
erosion with no rebound.

Using a country fixed-effects design and an original dataset of 83
autocoups (64 successful), the findings confirm these expectations.
Autocoups produce significant declines in Polity V scores in the event
year, and unlike coups, these declines do not reverse within three
years. Coups, by contrast, exhibit immediate negative effects but are
associated with significant democratic improvement in the medium term.
These patterns remain robust across alternative time horizons, different
specifications of irregular events, and disaggregated regime types.

The article makes two main contributions. First, it introduces autocoups
as a distinct category of irregular power seizure with unique
institutional consequences, filling a notable gap in the scholarship on
democratic backsliding. Second, by directly comparing the effects of
coups and autocoups, it clarifies why contemporary forms of executive
overreach---rather than traditional military coups---may be the
predominant drivers of democratic decline today.

\section{Autocoups: Definition and Measurement of Democratic
Impact}\label{autocoups-definition-and-measurement-of-democratic-impact}

\subsection{Defining the Irregular
Transition}\label{defining-the-irregular-transition}

Traditional political science has focused on the \textbf{coup d'état},
defined by Powell and Thyne (\citeproc{ref-powell2011}{2011}) as
``illegal and overt attempts by the military or other elites within the
state apparatus to unseat the sitting executive.'' This definition
focuses on \textbf{leadership turnover} and \textbf{external
disruption}.

In contrast, the \textbf{autocoup} refers to the phenomenon of an
incumbent leader's refusal to relinquish power and has received
comparatively less scholarly attention, despite its growing frequency.
Since the end of the Cold War, while classic coups have declined, these
``incumbent retention'' or ``overstay'' strategies have become more
frequent (\citeproc{ref-ginsburg2010evasion}{Ginsburg, Melton, and
Elkins 2010}; \citeproc{ref-baturo2014}{Baturo 2014}).

To conduct a valuable pioneer analysis, this study formally defines the
\textbf{autocoup as the extension of an incumbent leader's tenure in
office beyond the originally mandated limit, achieved through
extra-constitutional means.}

\subsection{Data Compilation}\label{data-compilation}

To operationalize this definition, \textbf{I compile a new autocoup
dataset} based on three complementary sources:

\begin{itemize}
\item
  \textbf{Leader Identification:} I utilize the Archigos dataset
  (\citeproc{ref-goemans2009}{Goemans, Gleditsch, and Chiozza 2009}) and
  the Political Leaders' Affiliation Database (PLAD)
  (\citeproc{ref-bomprezzi2024wedded}{Bomprezzi et al. 2024}) to
  identify \emph{de facto} national leaders and their precise time in
  office (1945--2023).
\item
  \textbf{Event Inventory:} I use the Incumbent Takeover dataset
  (\citeproc{ref-baturo2022}{Baturo and Tolstrup 2022}) as the primary
  inventory of potential autocoup events.
\item
  \textbf{Cross-Referencing:} Entries from the Incumbent Takeover
  dataset are cross-referenced with Archigos and PLAD to isolate only
  those events that meet the strict definition of an autocoup (illegal
  tenure \textbf{extension}).
\end{itemize}

\subsection{Measuring Impact on
Democracy}\label{measuring-impact-on-democracy}

The scholarly literature on the impact of traditional coups on
democratization often evaluates political outcomes through
\textbf{binary regime classifications} (e.g., democratize, autocratize)
(\citeproc{ref-thyne2014}{C. Thyne and Powell 2014};
\citeproc{ref-derpanopoulos2016}{Derpanopoulos et al. 2016}). This
framework is appropriate for coups, which trigger abrupt leadership
replacement and thus easily captured dichotomous shifts.

However, this binary framework is \textbf{inadequate for measuring and
comparing the impact of autocoups}. An autocoup retains the existing
leader, meaning it rarely triggers immediate changes in regime labels.
The absence of nominal transition thus \textbf{obscures the true
consequence: the subversion of institutional constraints that regulate
executive power.}

For this reason, to evaluate the political impact of autocoups, a
\textbf{more sensitive and continuous approach is required}. This study
employs the \textbf{Polity V score} (from the Polity5 dataset), which
ranges from \(\text{–10}\) (full autocracy) to \(+\text{10}\) (full
democracy). This continuous measure enables the detection of
\textbf{incremental degradation} in executive constraints and political
participation, aligning with recent research on subtle democratic
backsliding (\citeproc{ref-dahl2023}{Dahl and Gleditsch 2023}).

\section{Theory Framework: Autocoups and Democratic
Erosion}\label{theory-framework-autocoups-and-democratic-erosion}

Research on democratization has long centered on two broad explanatory
traditions. The first, rooted in modernization theory, posits that
rising levels of socio-economic development foster democratic outcomes
by expanding access to education, cultivating a burgeoning middle class,
and enhancing individual economic autonomy
(\citeproc{ref-lipset1959}{Lipset 1959}). For instance, cross-national
analyses conducted by Robert Barro, a Nobel laureate in economics,
demonstrate that elevations in GDP per capita, levels of schooling, and
broader human development indicators are positively correlated with
enhanced democratic rights and freedoms (\citeproc{ref-barro1999}{Barro
1999}). This perspective has been influential, suggesting that economic
progress creates societal pressures that inevitably push regimes toward
greater openness and accountability.

However, subsequent scholarship has introduced significant nuances and
complications to these foundational claims. Utilizing extensive data
from 167 countries spanning the period from 1875 to 2004, Miller
(\citeproc{ref-miller2012}{2012}) illustrates that while economic
development may indeed promote certain democratic tendencies, it can
paradoxically entrench authoritarian leaders by diminishing the
likelihood of their violent removal from office. In other words,
modernization does not uniformly facilitate democratic transitions;
instead, it may stabilize non-democratic regimes just as effectively as
it bolsters democratic consolidation in others. These contradictory
findings underscore the inherent limitations of purely economic
explanations when accounting for the contemporary global drift toward
illiberal governance, where hybrid regimes and electoral autocracies
persist despite economic advancements.

A second major strand of literature emphasizes coups d'état as pivotal
junctures that profoundly shape regime trajectories. Historically, coups
have been associated with heightened political instability and
pronounced democratic backsliding, often ushering in periods of military
rule or authoritarian retrenchment. Yet, evidence from the post--Cold
War era reveals a more nuanced and conditional picture. Several scholars
argue that coups---whether successful or failed---can occasionally
create unexpected openings for democratization, especially when they
dislodge deeply entrenched autocrats or compel elites to initiate
meaningful reforms (\citeproc{ref-thyne2014}{C. Thyne and Powell 2014};
\citeproc{ref-marinov2013}{Marinov and Goemans 2013}). This potential
for positive change is evidenced by the emergence of competitive
multi-party elections in the aftermath of numerous post--Cold War coups,
which sometimes lead to transitional governments or electoral pacts.

Nevertheless, these democratizing outcomes are far from guaranteed; they
depend heavily on factors such as sustained international pressure from
organizations like the United Nations or regional bodies
(\citeproc{ref-arbatli2014}{Arbatli and Arbatli 2014}), the formation of
robust domestic coalitions among opposition groups and civil society,
and the underlying ambitions of the coup orchestrators themselves.
Consequently, the long-term consequences of coups remain highly
inconsistent, context-dependent, and challenging to predict with
precision, highlighting the need for disaggregated analyses that
consider regional variations and specific historical contingencies
(\citeproc{ref-marinov2013}{Marinov and Goemans 2013};
\citeproc{ref-derpanopoulos2016}{Derpanopoulos et al. 2016},
\citeproc{ref-derpanopoulos2017}{2017}; \citeproc{ref-miller2016}{Miller
2016}; \citeproc{ref-bell2016}{Bell 2016}; \citeproc{ref-thyne2016a}{C.
L. Thyne and Powell 2016}; \citeproc{ref-tansey2016}{Tansey 2016};
\citeproc{ref-lumjiak2018}{Lumjiak et al. 2018};
\citeproc{ref-schiel2019}{Schiel 2019}; \citeproc{ref-thyne2020}{C.
Thyne and Hitch 2020}).

Although economic development and coups continue to influence regime
dynamics in meaningful ways, neither explanatory tradition fully
captures the systematic pattern of institutional erosion that has
proliferated across diverse regions since the early 2000s---particularly
in light of the concurrent decline in the frequency of conventional
coups (\citeproc{ref-powell2011}{Powell and Thyne 2011};
\citeproc{ref-thyne2020}{C. Thyne and Hitch 2020}). This
theoretical-empirical mismatch prompts a reevaluation of existing
frameworks, revealing the necessity to integrate additional mechanisms
that better explain the subtle, incremental processes of democratic
decline observed in contemporary politics. Factors such as judicial
capture, media manipulation, and electoral engineering, which operate
below the threshold of overt violence, demand greater attention in
models of regime change.

One particularly salient and under-explored development in this regard
is the rise of the autocoup. Rather than seizing power through explicit
military intervention or external force, an increasing number of leaders
have prolonged their rule by subverting institutional and constitutional
procedures from positions of incumbency. Prominent examples include
Vladimir Putin's constitutional maneuvers in Russia, Aleksandr
Lukashenko's referenda in Belarus, Xi Jinping's removal of term limits
in China, Recep Tayyip Erdoğan's shift to a presidential system in
Turkey, and Nayib Bukele's judicial purges in El Salvador
(\citeproc{ref-baturo2019politics}{Baturo and Elgie 2019};
\citeproc{ref-baturo2022}{Baturo and Tolstrup 2022}). Additional cases
from the 21st century encompass Viktor Orbán's constitutional reforms in
Hungary, which centralized executive authority and weakened checks and
balances, and Nicolás Maduro's creation of a parallel legislative body
in Venezuela to sideline opposition (\citeproc{ref-zhu2024}{Zhu 2024}).
Even in long-standing democracies, vulnerabilities have surfaced: Donald
Trump's efforts to overturn the 2020 U.S. election results ignited
extensive debates regarding the potential for an American self-coup,
underscoring the fragility of democratic norms in advanced economies
(\citeproc{ref-antonio2021}{Antonio 2021};
\citeproc{ref-pion-berlin2022}{Pion-Berlin, Bruneau, and Goetze 2022}).
Empirically, datasets document 83 autocoups globally from 1945 to 2023,
with 64 succeeding in extending incumbent tenure; notably, 46 of
these---representing more than half---have transpired since 2000,
aligning temporally with the broader global downturn in democratic
quality during this era (\citeproc{ref-zhu2024}{Zhu 2024}).

In this evolving context, the autocoup emerges as a critical yet
understudied driver of democratic erosion. Investigating this phenomenon
is imperative for comprehending the defining pattern of regime change in
the 21st-century: the gradual, strategic, and internally orchestrated
dismantling of democratic institutions by incumbents determined to
perpetuate their power. Unlike abrupt upheavals, autocoups often unfold
under a veneer of legality, making them insidious and resistant to
immediate international condemnation.

\subsection{Why Autocoups Matter: Leadership Agency and Authoritarian
Intent}\label{why-autocoups-matter-leadership-agency-and-authoritarian-intent}

Autocoups deserve scholarly scrutiny not merely due to their marked
increase since the early 2000s but also because they illuminate a
dimension that democratization research has historically
under-emphasized: the agency, strategic calculations, and personal
ambitions of political leaders. Much of the extant literature
prioritizes structural determinants---such as economic performance,
natural resource endowments, colonial legacies, or regime
characteristics---while affording relatively less attention to the
individuals who actively initiate, navigate, and shape political events.
Yet, leadership choices frequently prove decisive, particularly in
hybrid regimes or fragile democracies where institutional constraints
are porous, and political trajectories pivot on elite behavior and
decision-making. This is nowhere more evident than in autocoup
scenarios, where episodes of autocratic consolidation often stem from
deliberate strategies employed by incumbents to extend their rule,
shield themselves from legal accountability or political reprisals, or
fundamentally reshape the political order to align with their
preferences and ideologies.

Despite this evident role of agency, scholarship frequently assumes that
political leaders harbor uniform preferences---chief among them, the
imperative to maximize tenure in office
(\citeproc{ref-buenodemesquita2003}{Bueno de Mesquita et al. 2003}).
This assumption, while parsimonious, overlooks substantial variation in
leader behavior and motivations. Some leaders elect to relinquish power
voluntarily even when they retain the institutional capacity to
overstay, as exemplified by George Washington's precedent-setting
retirement after two terms, which embodied restraint and a commitment to
democratic principles (\citeproc{ref-baturo2019politics}{Baturo and
Elgie 2019}). Similarly, leaders like Nelson Mandela in South Africa or
Ellen Johnson Sirleaf in Liberia stepped down despite opportunities for
extension, prioritizing institutional integrity over personal gain. The
analytical challenge arises because leaders seldom articulate their
intentions transparently; authoritarian aspirations are typically masked
behind rhetorical commitments to stability or national interest, and
even overtly autocratic regimes maintain democratic facades to bolster
domestic and international legitimacy.

Autocoups, however, furnish an unusually clear and observable indicator
of authoritarian intent. When an incumbent pursues tenure extension via
unconstitutional channels---such as manipulating referenda, co-opting
judiciaries, or dissolving oppositional bodies---the action represents
an unequivocal breach of democratic norms and procedures. A leader
genuinely committed to democratic governance can adhere to term limits
and facilitate orderly succession; conversely, only a leader harboring
authoritarian ambitions opts to circumvent or dismantle the
constitutional framework. This distinction is crucial, as it separates
rhetorical posturing from behavioral evidence.

This renders autocoups analytically significant in dual respects. First,
they manifest as a form of revealed preference: a tangible signal of a
leader's predisposition to subvert democratic institutions for personal
or partisan advantage, stripping away ambiguity in motivational
assessments. Second, they function as an institutional mechanism that
initiates or exacerbates processes of democratic erosion, often
triggering downstream effects like voter suppression or media
censorship. Whereas many political actions remain ambiguous or subject
to interpretive debate, an autocoup embodies a deliberate and overt
repudiation of the institutional rules that govern democratic succession
and power distribution.

By encapsulating both intent and institutional repercussions, autocoups
elucidate a pivotal facet of contemporary democratic decline---one that
structural factors alone cannot adequately explain. They highlight how
individual agency interacts with systemic vulnerabilities to produce
enduring authoritarian shifts.

\subsection{The Causal Mechanisms: How Autocoups Damage Democratic
Institutions}\label{the-causal-mechanisms-how-autocoups-damage-democratic-institutions}

Autocoups propel democratic erosion through a cumulative, multi-stage
process that commences well before the formal extension of tenure and
persists long afterward. This sequence unfolds via three interconnected
and mutually reinforcing mechanisms: preparation, execution, and
consolidation, each contributing to the progressive weakening of
democratic safeguards.

\subsubsection{Pre-Autocoup Preparation: Weakening Constraints and
Neutralizing
Opposition}\label{pre-autocoup-preparation-weakening-constraints-and-neutralizing-opposition}

Democratic deterioration frequently predates the autocoup event itself,
setting the stage for its success. To orchestrate an uncontested
violation of constitutional limits, incumbents must preemptively
neutralize potential blockers, including rival elites, independent
institutions, and societal actors. This preparatory phase commonly
entails purging disloyal officials and reshuffling leadership within
security forces to ensure allegiance; imposing restrictions on media
freedoms while targeting independent journalists and outlets;
undermining judicial autonomy through appointments of partisan judges or
budgetary controls; harassing, disqualifying, or outright banning
opposition parties; and leveraging state resources to fortify patronage
networks that reward loyalty and punish dissent.

These maneuvers systematically dismantle the institutional foundations
essential for oversight, transparency, and public accountability. For
example, Alberto Fujimori's 1992 self-coup in Peru---which precipitated
a drastic plunge in the Polity V score from +8 to --4---was preceded by
calculated efforts to undermine watchdog institutions, sideline
political adversaries, and consolidate control over the military and
judiciary (\citeproc{ref-cameron1998}{Cameron 1998}). Such pre-emptive
strategies represent the initial wave of democratic degradation:
intentional, incremental actions that reduce the political costs of an
unconstitutional power grab while elevating its probability of success.
This phase often escapes immediate scrutiny, as measures are framed as
administrative reforms or responses to crises, further eroding public
trust over time.

\subsubsection{Execution: Violating the Constitutional Order and
Destroying Succession
Norms}\label{execution-violating-the-constitutional-order-and-destroying-succession-norms}

The autocoup proper signifies a decisive rupture in the democratic
order. By overriding term limits, dissolving legislatures, suspending
constitutions, or usurping judicial authority, incumbents flagrantly
violate the foundational norm of peaceful leadership rotation. Term
limits, in particular, serve as one of the most vital bulwarks against
personalist rule; their dismantlement undermines the credibility of the
entire constitutional edifice, signaling that rules are negotiable
rather than inviolable.

This rupture yields two primary consequences. First, it delegitimizes
the broader institutional framework: both elites and ordinary citizens
internalize that legal constraints are malleable, contingent upon the
incumbent's whims and power dynamics. Second, it establishes a perilous
precedent, normalizing future violations---whether by the same leader or
successors---and rendering them more justifiable and less resistible.
The outcome is a cascading erosion, wherein uncertainty about future
successions heightens the risk of irregular, destabilizing transitions,
potentially inviting further interventions or unrest.

This execution stage embodies the most visible and dramatic
manifestation of democratic breakdown, often accompanied by public
spectacles like referenda or emergency declarations that lend a patina
of popular endorsement.

\subsubsection{Post-Autocoup Consolidation: Structural Incentives for
Repression}\label{post-autocoup-consolidation-structural-incentives-for-repression}

Following the unconstitutional extension of power, the autocoup leader
confronts amplified vulnerability. Having transgressed democratic rules,
they face prospective prosecution, exile, or retaliation upon leaving
office, incentivizing them to entrench authoritarian control as a
protective mechanism.

In this consolidation phase, leaders intensify efforts to suppress
opposition through surveillance, arrests, or extrajudicial measures;
deter elite defections via co-optation or purges; and foreclose any
institutional apertures that could jeopardize their hold, such as
independent electoral commissions or civil society organizations. These
dynamics foster a self-reinforcing authoritarian drift, marked by
escalated repression, the hyper-concentration of executive authority,
the politicization of security apparatuses, the permanent debilitation
of horizontal accountability mechanisms (e.g., parliaments and courts),
and the indefinite deferral of democratic restoration prospects.

Democratic erosion thus accelerates, as the leader's survival becomes
inextricably linked to the perpetuation of a system that systematically
subverts democratic institutions. Consolidation transforms what might
have been a transient breach into a durable authoritarian
reconfiguration.

Given this multi-stage sequence, autocoups engender a sharp and
immediate institutional discontinuity. Although incumbents retain office
continuity, the magnitude of democratic rupture parallels that induced
by traditional coups, as both phenomena fundamentally disrupt
established norms of succession and governance.

\textbf{\emph{H1: Autocoups will lead to a significant decline in Polity
V scores immediately following their occurrence, comparable in magnitude
to the effects observed after traditional coups.}}

This hypothesis captures the inherent severity of the constitutional
breach: an autocoup constitutes an acute violation of democratic order,
precipitating an immediate deterioration in formal institutional
quality, as measured by indices like Polity V.

\subsection{Divergent Long-Term Trajectories: Why Autocoups Are More
Damaging Than
Coups}\label{divergent-long-term-trajectories-why-autocoups-are-more-damaging-than-coups}

Although autocoups and traditional coups may induce analogous short-term
institutional disruptions, their long-term trajectories diverge
substantially. Extensive literature on coups reveals considerable
variation in post-event outcomes: some deepen authoritarianism through
repressive consolidation, others inadvertently trigger liberalization
via elite pacts, and a notable proportion pave avenues for
democratization, especially in contexts of popular mobilization
(\citeproc{ref-thyne2014}{C. Thyne and Powell 2014};
\citeproc{ref-miller2016}{Miller 2016}). Autocoups, by contrast, exhibit
a more uniform pattern of protracted democratic decline, with minimal
evidence of reversal or recovery.

A primary explanation resides in their disparate success rates.
Autocoups achieve success at markedly higher levels---approximately 77
percent globally, based on datasets covering 1945--2023
(\citeproc{ref-zhu2024}{Zhu 2024})---compared to traditional coups,
which succeed in only about half of attempts
(\citeproc{ref-powell2011}{Powell and Thyne 2011}). Moreover, the
repercussions of failure differ starkly: failed coup plotters commonly
endure imprisonment, exile, or execution, whereas failed autocoup
leaders often preserve their positions or complete their terms,
leveraging incumbency advantages (\citeproc{ref-baturo2019}{Baturo
2019}). This asymmetry affords autocoup perpetrators greater latitude to
iteratively reshape political institutions, even after partial setbacks.

Survival in office further exacerbates this disparity. Leaders who
execute autocoups endure in power, on average, five years longer than
those installed via traditional coups, providing an extended temporal
horizon to embed loyalists, reconfigure elite coalitions, and erode
competing power centers (\citeproc{ref-zhu2024}{Zhu 2024}). In contrast,
many coup-installed leaders face swift ousters before institutionalizing
changes, curtailing their long-term imprint.

Traditional coups also yield heterogeneous institutional results.
Successful coups occasionally displace entrenched authoritarians,
fostering democratic openings, as seen in cases where military
interventions respond to executive overreach
(\citeproc{ref-thyne2014}{C. Thyne and Powell 2014}). Failed coups can
compel incumbents to concede reforms under pressure. Critically, coup
plotters often pursue domestic and international legitimacy, generating
incentives to pledge---or at least gesture toward---democratic
restoration. Illustrative examples include Niger's 2010 military coup,
which ousted President Mamadou Tandja following his unconstitutional
third-term bid and subsequently facilitated multiparty elections
(\citeproc{ref-miller2016}{Miller 2016}), and Honduras' 2009 coup
against President Manuel Zelaya, prompted by his attempts to alter
re-election rules (\citeproc{ref-muuxf1oz-portillo2019}{Muñoz-Portillo
and Treminio 2019}). These instances demonstrate how coups can serve as
corrective mechanisms against incipient authoritarianism.

Autocoups adhere to a diametrically opposed logic. Having flagrantly
violated the constitutional order, the incumbent cannot credibly pledge
liberalization without risking personal peril, such as legal
accountability or reprisals from empowered opponents. The structural
incentives engendered by an autocoup---anchored in self-preservation and
institutional dominance---compel leaders toward intensified
authoritarian consolidation rather than reformist concessions.
Liberalization thus becomes not only improbable but existentially
hazardous for the perpetrator.

Because autocoups are intrinsically motivated by self-preservation and
executed from entrenched institutional vantage points, their enduring
effects are consistently detrimental. They entrench political systems in
authoritarian pathways that prove resilient to reversal, often
perpetuating cycles of repression and institutional decay.

\textbf{\emph{H2: Autocoups generate sustained declines in Polity V
scores that do not rebound, whereas traditional coups often follow a
U-shaped trajectory characterized by an initial decline followed by
gradual democratic improvement.}}

Thus, while coups may, under specific conditions, catalyze democratizing
impulses or corrective reforms, autocoups predominantly yield
enduring---and frequently irreversible---democratic erosion. This
distinction underscores the need for tailored policy responses, such as
targeted sanctions or diplomatic isolation, to mitigate the risks posed
by incipient autocoups in vulnerable regimes.

\section{Methodology and variables}\label{methodology-and-variables}

\subsection{Methodology}\label{methodology}

As outlined above, autocoups are less likely to result in full regime
transitions---whether from democracy to autocracy or vice versa.
Consequently, evaluating their effects solely in terms of regime change
or shifts across democratic thresholds is analytically inappropriate.
Instead, this study assesses political change by examining variations in
Polity V scores, which capture more subtle shifts in institutional
quality and democratic performance.

To differentiate between immediate and medium-term effects, the analysis
considers both event-year and two-year impacts of autocoups. The
event-year effect is measured as the change in Polity V score in the
year of the autocoup relative to the preceding year:

\[
Polity_{t} - Polity_{t-1}
\]

The three-year effect captures the change in Polity V score two years
after the event, relative to the year of the autocoup:

\[
Polity_{t+3} - Polity_t
\]

This three-year specification is intended to capture medium-term
political developments, as autocoups typically entrench existing power
structures rather than inducing immediate systemic change. Short-term
fluctuations may not fully reflect the institutional consequences of
such events.

To empirically test the hypotheses, the study employs a linear
fixed-effects model at the country level. To distinguish between
attempted and successful autocoups, separate models are estimated using
binary variables that code for autocoup attempts and successes,
respectively.

\subsection{Variables}\label{variables}

The analysis draws upon a global panel of country-year observations
spanning from 1950 to 2020, resulting in approximately 9,100
observations. The primary dependent variable is the change in Polity V
score, calculated either as a one-year or three-year difference,
depending on the model specification. Polity V scores range from −10
(full autocracy) to +10 (full democracy). To address missing data caused
by transitional codes (−66, −77, −88), these values are replaced with
the nearest valid Polity score to preserve temporal continuity and
reduce bias associated with listwise deletion.

The primary independent variable is the occurrence of an autocoup, as
defined in Chapter 2. The dataset includes 83 attempted and 64
successful autocoups. For models analysing attempted autocoups, the
variable is coded as 1 in the year of the attempt and 0 otherwise. In
the three-year specification, a decay function is applied to measure the
persistence of effects, following the approach of Dahl and Gleditsch
(\citeproc{ref-dahl2023}{2023}). To account for temporal diffusion, a
half-life of five years is specified, allowing the model to capture both
immediate and delayed consequences from the year of the autocoup (
\(y_t\) ) through to four years post-event ( \(y_{t+4}\) ).

In addition, traditional coups are included as a secondary independent
variable for two reasons. First, they enable a comparative evaluation of
the political consequences of coups versus autocoups. Second, coups and
autocoups may occur in close proximity or in causal sequence,
necessitating analytical disaggregation. The coup data are drawn from
Powell and Thyne (\citeproc{ref-powell2011}{2011}), and are coded in a
manner consistent with the autocoup variables---using a binary indicator
for one-year effects and a decay function for three-year impacts.

A set of control variables is included to account for alternative
explanations. These comprise: economic performance, proxied by GDP
growth and GDP per capita; political violence, to capture variations in
political stability; and the logarithm of population size, which serves
as a proxy for state capacity and scale effects. To mitigate concerns
regarding reverse causality, all control variables are lagged by one
year, ensuring that their values precede the outcome being measured.

Two additional dummy variables are incorporated:

\textbf{Non-democracy:} This variable captures regime type by
distinguishing cases with Polity V scores below −6 (already autocratic
and less prone to further decline) and above +6 (institutionally
resilient to democratic erosion).

\textbf{Cold War:} A temporal dummy variable to account for the
geopolitical context, in line with previous studies on the relationship
between coups and democratisation (\citeproc{ref-thyne2014}{C. Thyne and
Powell 2014}; \citeproc{ref-derpanopoulos2016}{Derpanopoulos et al.
2016}; \citeproc{ref-dahl2023}{Dahl and Gleditsch 2023}). It captures
broad international trends, such as the stagnation or decline in
democratic scores during the Cold War (1960s--1990) and the more
pronounced democratising trend after 1990.

\section{Results and discussion}\label{results-and-discussion}

This section examines the democratic implications of autocoups by
analysing their effects on Polity V scores, both in the immediate
aftermath and in the medium term. Table~\ref{tbl-demomodel} presents
four models: Models 1 and 2 report results for attempted autocoups,
while Models 3 and 4 pertain to successful autocoups. Within each group,
Models 1 and 3 assess immediate effects (in the event year), whereas
Models 2 and 4 evaluate medium-term effects (three years after the
event).

\subsection{Immediate democratic
impact}\label{immediate-democratic-impact}

Consistent with the first hypothesis, autocoups and coups are associated
with significant immediate declines in Polity V scores. In both Models 1
and 3, autocoups---whether attempted or successful---lead to a
statistically significant reduction of approximately 1.3 points in
Polity V scores in the event year, all else equal. These effects are
comparable in magnitude across both attempted and successful autocoups,
suggesting that the democratic damage materialises irrespective of
whether the attempt fully succeeds.

Traditional coups are associated with larger immediate declines. Model 1
shows that attempted coups reduce Polity V scores by 1.31 points, while
successful coups, in Model 3, lead to a drop of 2.12 points, both
significant at the \(1\%\) level. These findings confirm that both types
of irregular power grabs deliver immediate shocks to democratic
institutions, though coups---especially successful ones---inflict
greater disruption.

\begin{table}

\caption{\label{tbl-demomodel}The Impacts on
Democratization(1950--2018): Autocoups vs Coups}

\centering{

\begin{tabular}{@{\extracolsep{30pt}}lcccc} 
\\[-1.8ex]\hline 
\hline \\[-1.8ex] 
 & \multicolumn{4}{c}{Dependent variable: Differences of Polity V scores} \\ 
\cline{2-5} 
\\[-1.8ex] & \multicolumn{2}{c}{Attempted} & \multicolumn{2}{c}{Succeeded} \\ 
 & (1) & (2) & (3) & (4) \\ 
\hline \\[-1.8ex] 
 Autocoup & $-$1.276$^{***}$ & $-$0.338 & $-$1.290$^{***}$ & $-$0.130 \\ 
  & (0.201) & (0.322) & (0.226) & (0.360) \\ 
  & & & & \\ 
 Coup & $-$1.312$^{***}$ & 1.203$^{***}$ & $-$2.120$^{***}$ & 1.868$^{***}$ \\ 
  & (0.091) & (0.127) & (0.124) & (0.183) \\ 
  & & & & \\ 
 GDP per Capita & $-$0.003$^{**}$ & $-$0.009$^{***}$ & $-$0.003$^{**}$ & $-$0.010$^{***}$ \\ 
  & (0.001) & (0.002) & (0.001) & (0.002) \\ 
  & & & & \\ 
 Economic Trend & $-$0.428 & $-$0.563 & $-$0.329 & $-$0.635 \\ 
  & (0.277) & (0.480) & (0.275) & (0.480) \\ 
  & & & & \\ 
 Log Population & 0.178$^{**}$ & 0.755$^{***}$ & 0.188$^{***}$ & 0.734$^{***}$ \\ 
  & (0.070) & (0.122) & (0.070) & (0.122) \\ 
  & & & & \\ 
 Political Violence & 0.015 & 0.033 & 0.012 & 0.033 \\ 
  & (0.014) & (0.024) & (0.014) & (0.024) \\ 
  & & & & \\ 
 Non-Democracy & 0.809$^{***}$ & $-$0.776$^{***}$ & 0.797$^{***}$ & $-$0.775$^{***}$ \\ 
  & (0.062) & (0.109) & (0.062) & (0.109) \\ 
  & & & & \\ 
 Cold War & $-$0.235$^{***}$ & $-$0.092 & $-$0.224$^{***}$ & $-$0.116 \\ 
  & (0.063) & (0.109) & (0.063) & (0.109) \\ 
  & & & & \\ 
\hline \\[-1.8ex] 
Observations & 9,104 & 9,104 & 9,104 & 9,104 \\ 
R$^{2}$ & 0.047 & 0.028 & 0.055 & 0.030 \\ 
Adjusted R$^{2}$ & 0.029 & 0.009 & 0.036 & 0.011 \\ 
F Statistic & 55.436$^{***}$ & 32.690$^{***}$ & 64.970$^{***}$ & 34.462$^{***}$ \\ 
\hline 
\hline \\[-1.8ex] 
\textit{Note:}  & \multicolumn{4}{r}{$^{*}$p$<$0.1; $^{**}$p$<$0.05; $^{***}$p$<$0.01} \\ 
\end{tabular}

}

\end{table}%

\subsection{Medium-term divergence: coups
vs.~autocoups}\label{medium-term-divergence-coups-vs.-autocoups}

In the medium term, however, the political trajectories begin to
diverge: while coups are followed by significant improvements in Polity
V scores, autocoups continue to exert a negative effect, albeit one that
does not reach statistical significance.

Models 2 and 4 evaluate changes in Polity V scores three years after the
event. The results indicate that autocoups have no statistically
significant effect in the medium term---whether attempted or
successful---implying that the initial democratic decline is not
followed by subsequent institutional reform or recovery. In contrast,
attempted coups are associated with a significant increase of 1.2
points, and successful coups show a particularly strong rebound of 1.87
points, both at the \(1\%\) significance level.

These findings provide clear support for the second hypothesis. Whereas
coups tend to exhibit a ``U-shaped'' pattern---with democratic erosion
followed by recovery---autocoups demonstrate a consistent,
unidirectional decline in democratic quality, with no evidence of
rebound.

The results suggest that autocoups exert their impact primarily in the
short term, as reflected in the immediate drop in Polity V scores, while
offering no potential for democratic revitalisation in the medium term.
This contrasts with coups, which, although initially disruptive,
sometimes serve as catalysts for institutional renewal, particularly in
cases where they are followed by electoral processes or popular
mobilisation.

These findings reinforce the notion that autocoups function to entrench
incumbents, undermining constitutional safeguards and consolidating
executive power. By contrast, coups---particularly those that displace
entrenched regimes---may open space for institutional realignment or
liberalisation, depending on the post-coup political context.

The models incorporate a range of control variables to isolate the
effects of coups and autocoups:

GDP per capita is negatively and significantly associated with changes
in Polity V scores across all models. This counterintuitive negative
association may reflect the limited potential for democratic gains in
already high-income democracies, where Polity V scores are near their
ceiling.

Log of population size is positively and significantly associated with
Polity score changes, suggesting that larger states may possess greater
institutional adaptability or reform potential.

The results for non-democratic regimes (defined as those with Polity V
scores below −6) reveal a temporal asymmetry in their effects on
democratic outcomes. In the event-year models (Models 1 and 3),
non-democratic regimes are associated with significant positive changes
in Polity V scores. This likely reflects cases where short-term
liberalisation or reform efforts follow leadership crises or
institutional ruptures, producing modest democratic gains even within
authoritarian contexts. By contrast, in the three-year models (Models 2
and 4), the effect reverses direction: non-democratic regimes are
associated with significant declines in Polity V scores over the medium
term. This pattern suggests that early signs of liberalisation often
fail to consolidate and may be followed by renewed authoritarian
entrenchment. In essence, while non-democratic regimes may exhibit
initial democratic openings---whether symbolic or procedural---these
gains are frequently short-lived, with longer-term trajectories
reverting to autocratic norms. This dynamic underscores the fragility of
democratic progress in authoritarian contexts, where reforms introduced
in the aftermath of institutional disruption are often superficial or
strategically instrumental, lacking the structural support required for
sustained democratisation.

Cold War context is statistically significant only in the event-year
models, where it correlates with a decline in Polity V scores,
reflecting the broader global pattern of democratic suppression during
the Cold War period.

Political violence and economic growth do not show consistent or
significant effects, indicating that immediate democratic outcomes are
more sensitive to regime characteristics and structural factors than to
short-term economic or security conditions.

Overall, the empirical results offer robust support for both hypotheses.
Autocoups and coups both lead to significant immediate declines in
democratic quality, with coups inflicting greater short-term damage. In
the medium term, coups are often followed by democratic recovery,
whereas autocoups result in persistent democratic erosion with no
evidence of rebound.

These findings suggest that autocoups represent a particularly insidious
form of democratic backsliding, less dramatic than coups but ultimately
more damaging in their long-term effects. They reinforce the need for
greater scholarly and policy attention to constitutional manipulations
by incumbents, which, although often gradual and legally framed, can
produce lasting democratic decay.

\subsection{Robustness tests}\label{robustness-tests}

To assess the robustness of the main findings, a series of alternative
model specifications were estimated. The results confirm that the core
conclusions remain stable under these variations.

First, the operationalisation of the autocoup variable was modified: the
decay function used in the baseline analysis was replaced with a binary
indicator distinguishing between attempted and successful autocoups.
Additionally, the broad `non-democracy' category was disaggregated into
more specific regime types---military, personalist, presidential,
parliamentary, and `other'---with dominant-party regimes serving as the
reference category. This classification mirrors the approach used in the
determinants analysis of autocoups presented in earlier chapters. The
results of these robustness models are presented in Models 5 to 8 in
Table~\ref{tbl-demomodel2}.

\begin{table}

\caption{\label{tbl-demomodel2}The Impact of Autocoups on
Democratization: Binary Autocoups}

\centering{

\begin{tabular}{@{\extracolsep{20pt}}lcccc} 
\\[-1.8ex]\hline 
\hline \\[-1.8ex] 
 & \multicolumn{4}{c}{Dependent variable: Differences of Polity V scores} \\ 
\cline{2-5} 
\\[-1.8ex] & \multicolumn{2}{c}{Attempted} & \multicolumn{2}{c}{Succeeded} \\ 
 & (5) & (6) & (7) & (8) \\ 
\hline \\[-1.8ex] 
 Autocoup & $-$1.236$^{***}$ & $-$0.148 & $-$1.234$^{***}$ & $-$0.057 \\ 
  & (0.200) & (0.359) & (0.226) & (0.402) \\ 
  & & & & \\ 
 Coup & $-$1.366$^{***}$ & 1.240$^{***}$ & $-$2.190$^{***}$ & 1.712$^{***}$ \\ 
  & (0.091) & (0.157) & (0.123) & (0.215) \\ 
  & & & & \\ 
 GDP per Capita & $-$0.003$^{**}$ & $-$0.010$^{***}$ & $-$0.003$^{**}$ & $-$0.010$^{***}$ \\ 
  & (0.001) & (0.002) & (0.001) & (0.002) \\ 
  & & & & \\ 
 Economic Trend & $-$0.387 & $-$0.569 & $-$0.282 & $-$0.629 \\ 
  & (0.277) & (0.482) & (0.276) & (0.482) \\ 
  & & & & \\ 
 Log Population & 0.247$^{***}$ & 0.890$^{***}$ & 0.262$^{***}$ & 0.879$^{***}$ \\ 
  & (0.072) & (0.126) & (0.072) & (0.126) \\ 
  & & & & \\ 
 Political Violence & 0.015 & 0.044$^{*}$ & 0.012 & 0.046$^{*}$ \\ 
  & (0.014) & (0.024) & (0.014) & (0.024) \\ 
  & & & & \\ 
 Regime: Military & 0.602$^{***}$ & $-$0.545$^{***}$ & 0.574$^{***}$ & $-$0.584$^{***}$ \\ 
  & (0.101) & (0.177) & (0.101) & (0.178) \\ 
  & & & & \\ 
 \hspace{1.5cm} Personal & $-$0.042 & $-$0.532$^{***}$ & $-$0.065 & $-$0.526$^{***}$ \\ 
  & (0.094) & (0.164) & (0.094) & (0.164) \\ 
  & & & & \\ 
 \hspace{1.5cm} Presidential & $-$0.576$^{***}$ & 0.399$^{**}$ & $-$0.578$^{***}$ & 0.381$^{**}$ \\ 
  & (0.091) & (0.158) & (0.090) & (0.158) \\ 
  & & & & \\ 
 \hspace{1.5cm} Parliamentary & $-$0.475$^{***}$ & 0.965$^{***}$ & $-$0.468$^{***}$ & 0.966$^{***}$ \\ 
  & (0.105) & (0.182) & (0.104) & (0.182) \\ 
  & & & & \\ 
 \hspace{1.5cm} Other & 0.999$^{***}$ & 1.094$^{***}$ & 1.013$^{***}$ & 1.115$^{***}$ \\ 
  & (0.114) & (0.199) & (0.114) & (0.199) \\ 
  & & & & \\ 
 Cold War & $-$0.168$^{***}$ & $-$0.002 & $-$0.156$^{**}$ & $-$0.011 \\ 
  & (0.064) & (0.111) & (0.063) & (0.111) \\ 
  & & & & \\ 
\hline \\[-1.8ex] 
Observations & 9,036 & 9,036 & 9,036 & 9,036 \\ 
R$^{2}$ & 0.060 & 0.033 & 0.068 & 0.033 \\ 
Adjusted R$^{2}$ & 0.041 & 0.014 & 0.049 & 0.014 \\ 
F Statistic & 47.043$^{***}$ & 25.244$^{***}$ & 53.742$^{***}$ & 25.364$^{***}$ \\ 
\hline 
\hline \\[-1.8ex] 
\textit{Note:}  & \multicolumn{4}{r}{$^{*}$p$<$0.1; $^{**}$p$<$0.05; $^{***}$p$<$0.01} \\ 
\end{tabular}

}

\end{table}%

Consistent with the main models, autocoups remain significantly
associated with negative changes in Polity V scores in the short term
(Models 5 and 7), with coefficients of −1.236 and −1.234, respectively
(both significant at the \(1\%\) level). However, in the three-year
models (Models 6 and 8), the effect becomes statistically insignificant,
indicating that the negative effect of autocoups is immediate but not
sustained over time.

By contrast, coups continue to show a distinct ``U-shaped'' effect. In
the event-year models (Models 5 and 7), coups are associated with
significant declines in Polity V scores (−1.366 and −2.190), both at the
\(1\%\) level. Yet in the three-year models (Models 6 and 8), the effect
reverses direction: coups are now associated with large positive changes
in Polity V scores (+1.240 and +1.712, also significant at the \(1\%\)
level). This confirms the earlier interpretation that while coups may
cause immediate democratic disruption, they are often followed by
democratic recovery in the medium term.

The disaggregated regime type variables provide additional insights.
Military regimes show significant positive effects in the event-year
models (Models 5 and 7), with coefficients of +0.602 and +0.574, but
become negative and significant in the three-year models (−0.545 and
−0.584 in Models 6 and 8). This reversal suggests that initial
post-event liberalisation in military regimes is not sustained, and may
even regress.

Personalist regimes are consistently associated with negative and
significant effects in the three-year models (Models 6 and 8: −0.532 and
−0.526), but not in the two-year models, suggesting that their
democratic erosion becomes more evident over time.

Presidential and parliamentary democracies follow a similar pattern:
both show significant negative effects in the short term (Models 5 and
7), and positive, statistically significant effects in the medium term
(Models 6 and 8). For example, parliamentary democracies are associated
with a drop of −0.475/−0.468 in the short term but show a gain of
+0.965/0.966 over three years. This pattern supports the idea that
democratic institutions may initially be shaken by political disruption
but recover when institutional mechanisms are strong.

``Other'' regimes (likely transitional or provisional systems) show
consistently large and positive effects across all models, ranging from
+0.999 to +1.115, all significant at the \(1\%\) level. This implies
that these regimes tend to transition toward more democratic forms over
both short and medium time frames.

Several control variables also behave consistently with the baseline
models. GDP per capita is negatively and significantly associated with
changes in Polity V scores across all models, again likely reflecting
ceiling effects in advanced democracies with limited room for
improvement. Log of population is positively and significantly related
to Polity changes, reinforcing earlier interpretations that larger
states may possess greater reform potential or be more likely to
register changes in democratic performance. Political violence becomes
statistically significant only in the three-year models (Models 6 and
8), where it has a small positive effect (+0.044, +0.046), suggesting
that prolonged unrest may precede some form of institutional response or
democratic opening. The Cold War variable is significant only in the
event-year models (Models 5 and 7), where it is associated with small
negative effects (−0.168 and −0.156), consistent with broader patterns
of democratic suppression during the Cold War period.

These robustness models confirm the main findings while offering
additional nuance. These results underscore the importance of both
regime context and temporal scope in evaluating the consequences of
irregular power grabs. Autocoups, unlike coups, represent a consistently
negative force for democratic institutions---one that undermines without
paving the way for recovery.

\section{Conclusion}\label{conclusion}

This article has examined the institutional consequences of autocoups
through a comparative analysis of their effects on Polity V scores
relative to traditional coups. Guided by two expectations---that
autocoups produce immediate and consistent democratic decline, and that
coups often follow a U-shaped trajectory marked by medium-term
recovery---the findings offer clear empirical support for both.

Across multiple model specifications, autocoups---attempted or
successful---generate significant reductions in democratic quality in
the event year, with no subsequent rebound. Coups, by contrast, yield
initial democratic deterioration but are associated with measurable
improvement within three years. These patterns persist across
alternative lag structures, extended temporal windows, and models
disaggregating regime types.

The analysis also highlights meaningful variation across political
systems. Military and personalist regimes occasionally exhibit
short-term gains but ultimately slip toward renewed authoritarianism.
Presidential and parliamentary democracies tend to recover from initial
declines, reflecting their institutional resilience. Transitional
regimes experience the most substantial democratic gains, suggesting
heightened reform potential during moments of political flux.

These results have important implications for the study of regime
change. While coups remain central to debates on authoritarian
durability and democratic transitions, autocoups warrant far greater
scholarly attention. They represent a systematically anti-democratic
strategy---one rooted in incumbents' efforts to evade constraints and
extend their tenure. That autocoup leaders remain in office
significantly longer than coup-installed leaders underscores their
capacity to reshape institutions in enduring ways.

The article also underscores the importance of temporal framing in
analyses of democratic erosion. Many of the dynamics associated with
autocoups---elite purges, legal manipulation, and the weakening of
oversight institutions---begin well before the formal event, reinforcing
the need for longitudinal approaches that capture cumulative processes
rather than discrete shocks.

Some limitations remain. In particular, autocoups and coups sometimes
occur in close succession, complicating causal identification. Future
research should develop finer-grained event sequencing or mixed-method
designs to better disentangle overlapping effects.

Taken together, the findings demonstrate that autocoups constitute a
distinctive and especially damaging mechanism of democratic backsliding.
Their legality-tinged execution obscures their profound institutional
consequences, yet their impact is durable, asymmetric, and
systematically negative. As autocoups become increasingly prominent in
global politics, understanding their causes, dynamics, and consequences
is essential for both scholarly analysis and policy efforts aimed at
defending constitutional governance.

\phantomsection\label{refs}
\begin{CSLReferences}{1}{0}
\bibitem[\citeproctext]{ref-antonio2021}
Antonio, Robert J. 2021. {``Democracy and Capitalism in the Interregnum:
Trump{'}s Failed Self-Coup and After.''} \emph{Critical Sociology} 48
(6): 937--65. \url{https://doi.org/10.1177/08969205211049499}.

\bibitem[\citeproctext]{ref-arbatli2014}
Arbatli, Cemal Eren, and Ekim Arbatli. 2014. {``External Threats and
Political Survival: Can Dispute Involvement Deter Coup Attempts?''}
\emph{Conflict Management and Peace Science} 33 (2): 115--52.
\url{https://doi.org/10.1177/0738894214545956}.

\bibitem[\citeproctext]{ref-barro1999}
Barro, Robert~J. 1999. {``Determinants of Democracy.''} \emph{Journal of
Political Economy} 107 (S6): S158--83.
\url{https://doi.org/10.1086/250107}.

\bibitem[\citeproctext]{ref-baturo2014}
Baturo, Alexander. 2014. {``Democracy, Dictatorship, and Term Limits.''}
\url{https://doi.org/10.3998/mpub.4772634}.

\bibitem[\citeproctext]{ref-baturo2019}
---------. 2019. {``Continuismo in Comparison.''} In, 75--100. Oxford
University Press.
\url{https://doi.org/10.1093/oso/9780198837404.003.0005}.

\bibitem[\citeproctext]{ref-baturo2019politics}
Baturo, Alexander, and Robert Elgie. 2019. \emph{The Politics of
Presidential Term Limits}. Oxford University Press.

\bibitem[\citeproctext]{ref-baturo2022}
Baturo, Alexander, and Jakob Tolstrup. 2022. {``Incumbent Takeovers.''}
\emph{Journal of Peace Research} 60 (2): 373--86.
\url{https://doi.org/10.1177/00223433221075183}.

\bibitem[\citeproctext]{ref-bell2016}
Bell, Curtis. 2016. {``Coup d{'}État and Democracy.''} \emph{Comparative
Political Studies} 49 (9): 1167--1200.
\url{https://doi.org/10.1177/0010414015621081}.

\bibitem[\citeproctext]{ref-bermeo2016}
Bermeo, Nancy. 2016. {``On Democratic Backsliding.''} \emph{Journal of
Democracy} 27 (1): 5--19. \url{https://doi.org/10.1353/jod.2016.0012}.

\bibitem[\citeproctext]{ref-bomprezzi2024wedded}
Bomprezzi, Pietro, Axel Dreher, Andreas Fuchs, Teresa Hailer, Andreas
Kammerlander, Lennart Kaplan, Silvia Marchesi, Tania Masi, Charlotte
Robert, and Kerstin Unfried. 2024. {``Wedded to Prosperity? Informal
Influence and Regional Favoritism.''} Discussion Paper. CEPR.

\bibitem[\citeproctext]{ref-buenodemesquita2003}
Bueno de Mesquita, Bruce, Alastair Smith, Randolph M. Siverson, and
James D. Morrow. 2003. \emph{The Logic of Political Survival}. The MIT
Press. \url{https://doi.org/10.7551/mitpress/4292.001.0001}.

\bibitem[\citeproctext]{ref-cameron1998}
Cameron, Maxwell A. 1998. {``Self-Coups: Peru, Guatemala, and Russia.''}
\emph{Journal of Democracy} 9 (1): 125--39.
\url{https://doi.org/10.1353/jod.1998.0003}.

\bibitem[\citeproctext]{ref-dahl2023}
Dahl, Marianne, and Kristian Skrede Gleditsch. 2023. {``Clouds with
Silver Linings: How Mobilization Shapes the Impact of Coups on
Democratization.''} \emph{European Journal of International Relations},
January, 135406612211432.
\url{https://doi.org/10.1177/13540661221143213}.

\bibitem[\citeproctext]{ref-derpanopoulos2016}
Derpanopoulos, George, Erica Frantz, Barbara Geddes, and Joseph Wright.
2016. {``Are Coups Good for Democracy?''} \emph{Research \& Politics} 3
(1): 205316801663083. \url{https://doi.org/10.1177/2053168016630837}.

\bibitem[\citeproctext]{ref-derpanopoulos2017}
---------. 2017. {``Are Coups Good for Democracy? A Response to Miller
(2016).''} \emph{Research \& Politics} 4 (2): 205316801770735.
\url{https://doi.org/10.1177/2053168017707355}.

\bibitem[\citeproctext]{ref-freedomhouse2024freedom}
Freedom House. 2024. {``Freedom in the World 2024.''}
\url{https://freedomhouse.org/sites/default/files/2024-02/FIW_2024_DigitalBooklet.pdf}.

\bibitem[\citeproctext]{ref-ginsburg2010evasion}
Ginsburg, Tom, James Melton, and Zachary Elkins. 2010. {``On the Evasion
of Executive Term Limits.''} \emph{Wm. \& Mary L. Rev.} 52: 1807.

\bibitem[\citeproctext]{ref-goemans2009}
Goemans, Henk E., Kristian Skrede Gleditsch, and Giacomo Chiozza. 2009.
{``Introducing Archigos: A Dataset of Political Leaders.''}
\emph{Journal of Peace Research} 46 (2): 269--83.
\url{https://doi.org/10.1177/0022343308100719}.

\bibitem[\citeproctext]{ref-lipset1959}
Lipset, Seymour Martin. 1959. {``Some Social Requisites of Democracy:
Economic Development and Political Legitimacy.''} \emph{American
Political Science Review} 53 (1): 69--105.
\url{https://doi.org/10.2307/1951731}.

\bibitem[\citeproctext]{ref-lumjiak2018}
Lumjiak, Sutsarun, Nguyen Thi Thieu Quang, Christopher Gan, and Sirimon
Treepongkaruna. 2018. {``Good Coups, Bad Coups: Evidence from
Thailand{'}s Financial Markets.''} \emph{Investment Management and
Financial Innovations} 15 (2): 68--86.
\url{https://doi.org/10.21511/imfi.15(2).2018.07}.

\bibitem[\citeproctext]{ref-marinov2013}
Marinov, Nikolay, and Hein Goemans. 2013. {``Coups and Democracy.''}
\emph{British Journal of Political Science} 44 (4): 799--825.
\url{https://doi.org/10.1017/s0007123413000264}.

\bibitem[\citeproctext]{ref-miller2012}
Miller, Michael K. 2012. {``Economic Development, Violent Leader
Removal, and Democratization.''} \emph{American Journal of Political
Science} 56 (4): 1002--20.
\url{https://doi.org/10.1111/j.1540-5907.2012.00595.x}.

\bibitem[\citeproctext]{ref-miller2016}
---------. 2016. {``Reanalysis: Are Coups Good for Democracy?''}
\emph{Research \& Politics} 3 (4): 205316801668190.
\url{https://doi.org/10.1177/2053168016681908}.

\bibitem[\citeproctext]{ref-muuxf1oz-portillo2019}
Muñoz-Portillo, Juan, and Ilka Treminio. 2019. {``The Politics of
Presidential Term Limits in Central America.''} In, 495--516. Oxford
University PressOxford.
\url{https://doi.org/10.1093/oso/9780198837404.003.0024}.

\bibitem[\citeproctext]{ref-pion-berlin2022}
Pion-Berlin, David, Thomas Bruneau, and Richard B. Goetze. 2022. {``The
Trump Self-Coup Attempt: Comparisons and Civil{\textendash}Military
Relations.''} \emph{Government and Opposition} 58 (4): 789--806.
\url{https://doi.org/10.1017/gov.2022.13}.

\bibitem[\citeproctext]{ref-powell2014a}
Powell, Jonathan. 2014. {``An Assessment of the {`}Democratic{'} Coup
Theory.''} \emph{African Security Review} 23 (3): 213--24.
\url{https://doi.org/10.1080/10246029.2014.926949}.

\bibitem[\citeproctext]{ref-powell2011}
Powell, and Thyne. 2011. {``Global Instances of Coups from 1950 to 2010:
A New Dataset.''} \emph{Journal of Peace Research} 48 (2): 249--59.
\url{https://doi.org/10.1177/0022343310397436}.

\bibitem[\citeproctext]{ref-schiel2019}
Schiel, Rebecca E. 2019. {``An Assessment of Democratic Vulnerability:
Regime Type, Economic Development, and Coups d{'}état.''}
\emph{Democratization} 26 (8): 1439--57.
\url{https://doi.org/10.1080/13510347.2019.1645652}.

\bibitem[\citeproctext]{ref-tansey2016}
Tansey, Oisín. 2016. {``The Limits of the {``}Democratic Coup{''}
Thesis: International Politics and Post-Coup Authoritarianism: Table
1.''} \emph{Journal of Global Security Studies} 1 (3): 220--34.
\url{https://doi.org/10.1093/jogss/ogw009}.

\bibitem[\citeproctext]{ref-thyne2016a}
Thyne, Clayton L., and Jonathan M. Powell. 2016. {``Coup d{'}état or
Coup d{'}autocracy? How Coups Impact Democratization,
1950{\textendash}2008.''} \emph{Foreign Policy Analysis} 12 (2):
192--213. \url{https://www.jstor.org/stable/26168099}.

\bibitem[\citeproctext]{ref-thyne2020}
Thyne, Clayton, and Kendall Hitch. 2020. {``Democratic Versus
Authoritarian Coups: The Influence of External Actors on States{'}
Postcoup Political Trajectories.''} \emph{Journal of Conflict
Resolution} 64 (10): 1857--84.
\url{https://doi.org/10.1177/0022002720935956}.

\bibitem[\citeproctext]{ref-thyne2014}
Thyne, Clayton, and Jonathan Powell. 2014. {``Coup d{'}état or Coup
d'Autocracy? How Coups Impact Democratization, 1950-2008.''}
\emph{Foreign Policy Analysis}, April, n/a--.
\url{https://doi.org/10.1111/fpa.12046}.

\bibitem[\citeproctext]{ref-thyne2019}
Thyne, and Powell. 2019. {``Coup Research,''} October.
\url{https://doi.org/10.1093/acrefore/9780190846626.013.369}.

\bibitem[\citeproctext]{ref-zhu2024}
Zhu, Qi. 2024. {``Leadership Transitions and Survival: Coups, Autocoups,
and Power Dynamics.''} PhD thesis, University of Essex.

\end{CSLReferences}




\end{document}

% Options for packages loaded elsewhere
\PassOptionsToPackage{unicode}{hyperref}
\PassOptionsToPackage{hyphens}{url}
\PassOptionsToPackage{dvipsnames,svgnames,x11names}{xcolor}
%
\documentclass[
  12pt,
]{report}

\usepackage{amsmath,amssymb}
\usepackage{setspace}
\usepackage{iftex}
\ifPDFTeX
  \usepackage[T1]{fontenc}
  \usepackage[utf8]{inputenc}
  \usepackage{textcomp} % provide euro and other symbols
\else % if luatex or xetex
  \usepackage{unicode-math}
  \defaultfontfeatures{Scale=MatchLowercase}
  \defaultfontfeatures[\rmfamily]{Ligatures=TeX,Scale=1}
\fi
\usepackage{lmodern}
\ifPDFTeX\else  
    % xetex/luatex font selection
  \setmainfont[]{Times New Roman}
  \setsansfont[]{Arial}
  \setmonofont[]{Courier New}
\fi
% Use upquote if available, for straight quotes in verbatim environments
\IfFileExists{upquote.sty}{\usepackage{upquote}}{}
\IfFileExists{microtype.sty}{% use microtype if available
  \usepackage[]{microtype}
  \UseMicrotypeSet[protrusion]{basicmath} % disable protrusion for tt fonts
}{}
\usepackage{xcolor}
\usepackage[top=30mm,left=1in,right=1in,bottom=25mm,top = 3cm,bottom =
3cm,left = 3cm,right = 2.7cm]{geometry}
\setlength{\emergencystretch}{3em} % prevent overfull lines
\setcounter{secnumdepth}{5}
% Make \paragraph and \subparagraph free-standing
\ifx\paragraph\undefined\else
  \let\oldparagraph\paragraph
  \renewcommand{\paragraph}[1]{\oldparagraph{#1}\mbox{}}
\fi
\ifx\subparagraph\undefined\else
  \let\oldsubparagraph\subparagraph
  \renewcommand{\subparagraph}[1]{\oldsubparagraph{#1}\mbox{}}
\fi


\providecommand{\tightlist}{%
  \setlength{\itemsep}{0pt}\setlength{\parskip}{0pt}}\usepackage{longtable,booktabs,array}
\usepackage{calc} % for calculating minipage widths
% Correct order of tables after \paragraph or \subparagraph
\usepackage{etoolbox}
\makeatletter
\patchcmd\longtable{\par}{\if@noskipsec\mbox{}\fi\par}{}{}
\makeatother
% Allow footnotes in longtable head/foot
\IfFileExists{footnotehyper.sty}{\usepackage{footnotehyper}}{\usepackage{footnote}}
\makesavenoteenv{longtable}
\usepackage{graphicx}
\makeatletter
\def\maxwidth{\ifdim\Gin@nat@width>\linewidth\linewidth\else\Gin@nat@width\fi}
\def\maxheight{\ifdim\Gin@nat@height>\textheight\textheight\else\Gin@nat@height\fi}
\makeatother
% Scale images if necessary, so that they will not overflow the page
% margins by default, and it is still possible to overwrite the defaults
% using explicit options in \includegraphics[width, height, ...]{}
\setkeys{Gin}{width=\maxwidth,height=\maxheight,keepaspectratio}
% Set default figure placement to htbp
\makeatletter
\def\fps@figure{htbp}
\makeatother
% definitions for citeproc citations
\NewDocumentCommand\citeproctext{}{}
\NewDocumentCommand\citeproc{mm}{%
  \begingroup\def\citeproctext{#2}\cite{#1}\endgroup}
\makeatletter
 % allow citations to break across lines
 \let\@cite@ofmt\@firstofone
 % avoid brackets around text for \cite:
 \def\@biblabel#1{}
 \def\@cite#1#2{{#1\if@tempswa , #2\fi}}
\makeatother
\newlength{\cslhangindent}
\setlength{\cslhangindent}{1.5em}
\newlength{\csllabelwidth}
\setlength{\csllabelwidth}{3em}
\newenvironment{CSLReferences}[2] % #1 hanging-indent, #2 entry-spacing
 {\begin{list}{}{%
  \setlength{\itemindent}{0pt}
  \setlength{\leftmargin}{0pt}
  \setlength{\parsep}{0pt}
  % turn on hanging indent if param 1 is 1
  \ifodd #1
   \setlength{\leftmargin}{\cslhangindent}
   \setlength{\itemindent}{-1\cslhangindent}
  \fi
  % set entry spacing
  \setlength{\itemsep}{#2\baselineskip}}}
 {\end{list}}
\usepackage{calc}
\newcommand{\CSLBlock}[1]{\hfill\break\parbox[t]{\linewidth}{\strut\ignorespaces#1\strut}}
\newcommand{\CSLLeftMargin}[1]{\parbox[t]{\csllabelwidth}{\strut#1\strut}}
\newcommand{\CSLRightInline}[1]{\parbox[t]{\linewidth - \csllabelwidth}{\strut#1\strut}}
\newcommand{\CSLIndent}[1]{\hspace{\cslhangindent}#1}

\usepackage{booktabs}
\usepackage{caption}
\usepackage{longtable}
\usepackage{colortbl}
\usepackage{array}
\usepackage{anyfontsize}
\usepackage{sectsty}
\chapterfont{\centering}
\usepackage{lscape}
\newcommand{\blandscape}{\begin{landscape}}
\newcommand{\elandscape}{\end{landscape}}
\makeatletter
\@ifpackageloaded{caption}{}{\usepackage{caption}}
\AtBeginDocument{%
\ifdefined\contentsname
  \renewcommand*\contentsname{Table of contents}
\else
  \newcommand\contentsname{Table of contents}
\fi
\ifdefined\listfigurename
  \renewcommand*\listfigurename{Figures}
\else
  \newcommand\listfigurename{Figures}
\fi
\ifdefined\listtablename
  \renewcommand*\listtablename{Tables}
\else
  \newcommand\listtablename{Tables}
\fi
\ifdefined\figurename
  \renewcommand*\figurename{Figure}
\else
  \newcommand\figurename{Figure}
\fi
\ifdefined\tablename
  \renewcommand*\tablename{Table}
\else
  \newcommand\tablename{Table}
\fi
}
\@ifpackageloaded{float}{}{\usepackage{float}}
\floatstyle{ruled}
\@ifundefined{c@chapter}{\newfloat{codelisting}{h}{lop}}{\newfloat{codelisting}{h}{lop}[chapter]}
\floatname{codelisting}{Listing}
\newcommand*\listoflistings{\listof{codelisting}{List of Listings}}
\makeatother
\makeatletter
\makeatother
\makeatletter
\@ifpackageloaded{caption}{}{\usepackage{caption}}
\@ifpackageloaded{subcaption}{}{\usepackage{subcaption}}
\makeatother
\ifLuaTeX
  \usepackage{selnolig}  % disable illegal ligatures
\fi
\usepackage{bookmark}

\IfFileExists{xurl.sty}{\usepackage{xurl}}{} % add URL line breaks if available
\urlstyle{same} % disable monospaced font for URLs
\hypersetup{
  pdftitle={Leadership Transitions and Survival: Coups, Autocoups, and Power Dynamics},
  pdfauthor={Zhu Qi},
  colorlinks=true,
  linkcolor={blue},
  filecolor={Maroon},
  citecolor={Blue},
  urlcolor={blue},
  pdfcreator={LaTeX via pandoc}}

\title{Leadership Transitions and Survival: Coups, Autocoups, and Power
Dynamics}
\author{Zhu Qi}
\date{}

\begin{document}


\def\spacingset#1{\renewcommand{\baselinestretch}%
{#1}\small\normalsize} \spacingset{1}


%%%%%%%%%%%%%%%%%%%%%%%%%%%%%%%%%%%%%%%%%%%%%%%%%%%%%%%%%%%%%%%%%%%%%%%%%%%%%%

\title{\bf Leadership Transitions and Survival: Coups, Autocoups, and
Power Dynamics}
\author{
Zhu Qi\\University of
Essex\\\href{mailto:qz21485@essex.ac.uk}{qz21485@essex.ac.uk}
}

\maketitle

\bigskip
\bigskip
\begin{abstract}

\end{abstract}


\newpage
\spacingset{1.9} % DON'T change the spacing!
\renewcommand*\contentsname{Contents}
{
\hypersetup{linkcolor=}
\setcounter{tocdepth}{2}
\tableofcontents
}
\listoffigures
\listoftables
\setstretch{1.618}
\chapter*{Acknowledgements}\label{acknowledgements}
\addcontentsline{toc}{chapter}{Acknowledgements}

The completion of this thesis has been a significant journey, filled
with hard work, learning, and moments of joy. Throughout this time, I
have received support and encouragement from many individuals, without
whom this dissertation would not have been possible.

First and foremost, I would like to express my deepest gratitude to my
great supervisor, Professor Kristian Skrede Gleditsch, for his
invaluable guidance, unwavering support, and insightful feedback
throughout this journey. His expertise and encouragement have been
instrumental in shaping this dissertation. I would also like to extend
my heartfelt thanks to the chair of my board panel, Professor Han
Dorussen, for his continuous support and constructive criticism, which
have significantly enhanced the quality of my research.

I am profoundly grateful for the comments, advice, and suggestions from
several esteemed scholars who have contributed to this work. Dr.~Brian J
Phillips, Dr. Prabin Khadka, and Dr.~Winnie Xia, their expertise and
thoughtful input have been greatly appreciated and have enriched this
dissertation.

Finally, I want to thank my family for their unwavering support and
love. To my beloved wife, Ji Zhi, who has been my rock throughout this
journey, and to my dear daughter, Siyan, and son, Sisheng, who have been
my source of joy and motivation. I am deeply thankful to my father for
his enduring support, and to the memory of my late mother, whose love
and guidance continue to inspire me.

All errors and faults are my own.

\chapter*{Abstract}\label{abstract}
\addcontentsline{toc}{chapter}{Abstract}

This dissertation examines the dynamics of irregular power transitions,
particularly coups and autocoups, and their influence on leader
survival. It highlights the critical role of power dynamics, shaped by
\textbf{regime type}, in determining coup success rates and attempt
frequency. Utilizing a \textbf{double probit model with sample
selection}, the study reveals that expected coup success significantly
influences attempts, with military regimes facing a heightened
vulnerability due to their power structure.

While often understudied, autocoups are shown to have a substantial
impact on democratic trends. This research introduces a refined
definition of autocoups alongside a novel dataset encompassing events
from 1945 to 2022, enabling a more robust quantitative analysis.

Employing survival analysis, the study compares the longevity of leaders
who rise to power through coups versus autocoups. The findings
demonstrate that coup-installed leaders face a significantly shorter
tenure and higher risk of removal. This contrasts with autocoup leaders
who manipulate the system to extend their rule, suggesting the potential
for autocoups to incentivize power grabs and contribute to democratic
backsliding.

This work contributes significantly to the political science literature
by:

\begin{itemize}
\item
  Defining key concepts: It establishes a clear definition of autocoups,
  a previously understudied phenomenon.
\item
  Introducing a novel dataset: This dataset enables researchers to
  conduct more comprehensive quantitative analyses of autocoups.
\item
  Establishing a general framework: The framework provides a comparative
  approach to studying the dynamics of irregular power transitions and
  their impact on democratic stability.
\end{itemize}

\textbf{keywords:} \emph{Coups, Autocoups, Power transitions, Leadership
Survival}

\chapter{Introduction}\label{introduction}

\section{Research question}\label{research-question}

Irregular power transitions, marked by a disregard for constitutional
procedures, are a critical area of study in political science. They not
only disrupt established rules but often require unconstitutional
tactics to secure power. Furthermore, these transitions can inspire
copycat behaviour among other ambitious leaders.

Despite their central role in political science and the extensive
research conducted on irregular power transitions, a long-standing
question continues to intrigue political scientists: \textbf{\emph{Why
are some leaders ousted before their terms expire, while others complete
their full terms or even overstay beyond their originally mandated
limits?}} In other words, why do some leaders survive for decades while
others last for only years, months, or even days? This dissertation
focuses on this question and seeks to provide a comprehensive analysis,
dedicated to understanding how leaders come to power through
unconstitutional means and what factors determine the duration of a
leader's rule following an irregular ascent.

\section{Analyses on coups and autocoups in a general
framework}\label{analyses-on-coups-and-autocoups-in-a-general-framework}

When discussing irregular power transitions, the concepts that often
come to mind are irregular entries or exits, such as coups,
assassinations, rebellions, protests, and foreign interventions. Among
these methods, coups hold a prominent position due to their frequent
occurrence. According to the Archigos dataset
(\citeproc{ref-goemans2009}{Goemans, Gleditsch, and Chiozza 2009}), from
1945 to 2015, there were approximately 145 instances of irregular leader
exits, with coups\footnote{According to the Archigos dataset, ``Removed
  by Military, without Foreign Support'' and ``Removed by Other
  Government Actors, without Foreign Support'' in the variable
  \texttt{exitcode} are classified as coups.} accounting for more than
half (79 leaders). The often-cited Global Instances of Coups
(GIC)\footnote{According to the Archigos dataset, ``Removed by Military,
  without Foreign Support'' and ``Removed by Other Government Actors,
  without Foreign Support'' in the variable \texttt{exitcode} are
  classified as coups.} dataset (\citeproc{ref-powell2011}{J. M. Powell
and Thyne 2011}) records even more leaders (245 cases) removed by coups
from 1950 to 2023.

Given their prevalence and substantial influence on political systems,
coups have been extensively studied, particularly since 2000
(\citeproc{ref-thyne2019}{Thyne and Powell 2019}). Consequently, the
concept of a coup is comparatively clear and widely accepted in academic
circles. Many scholars, including this study, follow the definition by
J. M. Powell and Thyne (\citeproc{ref-powell2011}{2011}), which
describes coups as ``illegal and overt attempts by the military or other
elites within the state apparatus to unseat the sitting
executive\ldots{} {[}a coup is successful{]} if the perpetrators seize
and hold power for at least seven days'' (P. 252). Although debates
persist, two elements are clear: first, the perpetrators are elites
within the ruling group, and the victims of coups are incumbent
executive leaders. Second, the strategy or aim of a coup involves
completely removing the incumbents, not merely seizing part of their
power or forcing them to concede on specific policies. Beyond defining
coups, several datasets have been developed for quantitative analyses,
such as the Global Instances of Coups (\citeproc{ref-powell2011}{J. M.
Powell and Thyne 2011}), the Cline Centre Coup d'État Project Dataset
(\citeproc{ref-peyton2024}{Peyton et al. 2024}), and the Colpus Dataset
(\citeproc{ref-chin2021}{Chin, Carter, and Wright 2021}). These datasets
are well-developed and frequently used in political science research.

However, irregular power transitions are not limited to irregular
entries and exits but should also include irregular ``overstays.'' Using
illegal means to overthrow an incumbent leader before their term expires
is undoubtedly an irregular power transition. Similarly, an incumbent
using illegitimate means to extend their term beyond term limits is also
an irregular power transition.

Although academic attention to irregular retention of power has
increased since the 1990s, especially after Peru's President Alberto
Fujimori's self-coup in 1992, it remains comparatively understudied and
has several shortcomings. First, there is no universally accepted
terminology for this ``overstaying in power'' type of irregular power
transition, unlike the clear term ``coup.'' Consequently, various terms
such as self-coup, autogolpe, and executive coup are used by different
scholars. This dissertation will use `autocoup' to refer to this type of
irregular power transition, which will be thoroughly discussed in
Chapter 3. Second, there is no consensus on the definition of an
autocoup. Existing definitions remain vague, often conflating power
expansions and power extensions\footnote{The definitions and concepts of
  power expansion and power extension can be vague. In this study, we
  define power expansion as an incumbent acquiring additional authority
  from other state apparatuses, whereas power extension refers to an
  incumbent prolonging their tenure beyond the designated term in
  office.}. For example, Maxwell A. Cameron
(\citeproc{ref-cameron1998a}{1998a}) defines an autogolpe as a temporary
suspension of the constitution and dissolution of Congress by the
executive, who then rules by decree. This definition focuses on power
expansion instead of power extension, leading to conceptual confusion
and misalignment with the definition of a classic coup. Third, a
consensus autocoup dataset is lacking. While several related datasets
exist, as discussed by Baturo and Tolstrup
(\citeproc{ref-baturo2022}{2022}) in coding their Incumbent Takeover
dataset, the terminologies, definitions, and coverage years vary,
lacking wide acknowledgement and extensive academic exploration. In
summary, autocoup has not been analysed in a comparative manner
connected with coups.

Analysing coups and autocoups separately is less problematic. However,
from a comprehensive framework perspective on irregular power
transitions and leader survival, coups and autocoups should be, and can
be, analysed within the same framework. Both coup and autocoup
significantly influence democratic backsliding and are the most frequent
means of irregular power transition. Furthermore, as both are called
``coups,'' classic coups and autocoups are very similar since a coup is
launched to replace the current leader, while an autocoup is staged to
replace the future leader.

\section{Academic Contributions}\label{academic-contributions}

This study addresses a critical gap in the literature by offering a
comprehensive framework for analysing both coups and autocoups, which
are the most common forms of irregular power transitions. While existing
research often examines these topics separately with varying
terminologies, definitions, methods, and datasets, this dissertation
integrates these elements to provide a unified perspective on irregular
power transitions and leader survival.

Our contributions are threefold:

\begin{itemize}
\item
  \textbf{Emphasis on power dynamics and regime types}: We highlight the
  significant role of power dynamics, particularly the influence of
  regime types, in determining the success and frequency of coup
  attempts. Our analysis underscores how the expected chances of coup
  success motivate such attempts, with military regimes being notably
  susceptible.
\item
  \textbf{Refined definition and novel dataset for autocoups:} We
  introduce a refined definition of autocoups and develop a novel
  dataset covering events from 1945 to 2022. This enables a comparative
  analysis with classic coups, providing clearer insights into the
  nature and impact of autocoups on political systems.
\item
  \textbf{Survival analysis of leaders from different entry modes:} By
  applying survival analysis to existing coup data and our new autocoup
  dataset, we demonstrate how different modes of entry into power
  significantly affect leader survival. Our findings reveal that leaders
  who come to power through coups typically have shorter tenures and
  face higher removal risks compared to those who extend their rule
  through autocoups.
\end{itemize}

Our analysis of irregular power transitions is particularly relevant to
understanding democratic backsliding. These transitions violate
democratic norms and disrupt the path towards stable democracy. Leaders
who gain power through irregular means often employ undemocratic
tactics, such as suppressing opposition, to consolidate their
illegitimate hold on power. This creates a vicious cycle where the
erosion of democratic institutions is both a cause and consequence of
efforts to maintain power.

\section{Overview of the thesis}\label{overview-of-the-thesis}

This study is structured into three main chapters beyond the
introduction, each addressing key aspects of irregular power transitions
and their implications for political stability and democratic processes.

\textbf{Chapter 2} examines the determinants of classic coup attempts.
While extensive research exists on coups, most studies focus on
observable factors before coups, such as economic performance, political
stability, previous coups, and coup-proofing strategies. This chapter,
however, emphasizes the less observable but crucial factor of expected
chances of coup success, which have been often overlooked. Utilizing the
double probit model with sample selection, the analysis reveals that
expected success rates significantly influence coup attempts. These
success rates are primarily shaped by the balance of power between
incumbents and challengers, which is largely determined by regime type.
The findings indicate that military regimes face a much higher risk of
coups compared to dominant-party regimes.

\textbf{Chapter 3} focuses on the concept of autocoups, specifically on
power extensions by incumbent leaders. It distinguishes autocoups from
broader concepts like self-coups or executive coups by redefining them
as instances where incumbent leaders refuse to transition power as
mandated, thereby overstaying in office. Based on this refined
definition, a novel dataset of autocoup events from 1945 to 2022 is
introduced, encompassing 110 attempts and 87 successes. The chapter
includes case studies and empirical analyses that demonstrate the
utility of this dataset for quantitative research, providing a basis for
empirical analysis on autocoups.

\textbf{Chapter 4} investigates how the method of power acquisition
impacts the longevity of leaders who come to power through coups versus
those who extend their rule through autocoups. The hypothesis is that
the method of accession significantly affects leader tenure. Using the
Cox proportional hazards model and a time-dependent Cox model, the
chapter provides evidence of differing survival times between these two
types of leaders. The results indicate that leaders who come to power
through coups face a significantly higher risk of removal compared to
those who extend their rule through autocoups. This finding highlights
the implications for political stability and democratic processes,
suggesting that the relatively low cost and high returns of autocoups
could incentivize incumbents to seize power in this manner, potentially
leading to democratic backsliding and the personalization of power.

In \textbf{Chapter 5}, the study concludes by summarizing the main
findings, discussing policy implications, and acknowledging the
limitations of the research. It also outlines directions for future
research, emphasizing the need for further exploration of irregular
power transitions, particularly coups and autocoups.

\chapter{Power Dynamics and Coup Attempts: A Selection Mechanism
Analysis}\label{power-dynamics-and-coup-attempts-a-selection-mechanism-analysis}

\section{Introduction}\label{introduction-1}

Coups, defined as ``illegal and overt attempts by the military or other
elites within the state apparatus to unseat the sitting executive''
(\citeproc{ref-powell2011}{J. M. Powell and Thyne 2011, 252}), occur
with varying frequency across countries, with some experiencing them
more often than others. According to GIC dataset, Latin American
countries such as Bolivia witnessed 23 coups between 1950 and 1984,
while Argentina experienced 20 during a similar time frame. However,
Mexico's authoritarian period from 1917 to 2000 saw no coups at all. In
Africa, Sudan endured 17 coups between 1955 and 2023, whereas South
Africa has not experienced any coup since 1950. Similar patterns are
observed in the Middle East and South Asia.

The varying frequency of coup attempts has captivated political
scientists for decades, leading to extensive research on the subject. As
highlighted by Gassebner, Gutmann, and Voigt
(\citeproc{ref-gassebner2016}{2016}), despite approximately one hundred
potential determinants of coups being suggested, no consensus has been
reached. In an effort to address this issue, they have tested 66 factors
proposed in previous literature using three million model permutations
in an extreme bounds analysis (ibid.).

Examining previous research, which has tested around 100 variables as
potential determinants of coups, raises an important question beyond
simply understanding why coups are more frequent in some countries than
others. The critical question is: Can we establish a framework to help
scholars focus on the most relevant factors of coups, rather than
sifting through over 100 variables without reaching a consensus?

Reviewing previously proposed variables of coups, it is evident that all
focus on pre-coup conditions, with no consideration given to post-coup
factors. This means more attention is paid to factors observable before
a coup, such as military force or spending, economic performance,
political stability, and previous coups.

\begingroup
\setlength\LTleft{0.05\linewidth}
\setlength\LTright{0.05\linewidth}\fontsize{12.0pt}{14.4pt}\selectfont
\setlength{\LTpost}{0mm}

\begin{longtable}{@{\extracolsep{\fill}}lccr}

\caption{\label{tbl-coups}Top 10 countries with the most coup attempts}

\tabularnewline

\toprule
Country & Coup Attempted & Coup Succeeded & Success Rate \\ 
\midrule\addlinespace[2.5pt]
Bolivia & 23 & 11 & 47.8\% \\ 
Argentina & 20 & 7 & 35.0\% \\ 
Sudan & 17 & 6 & 35.3\% \\ 
Haiti & 13 & 9 & 69.2\% \\ 
Venezuela & 13 & 0 & 0.0\% \\ 
Iraq & 12 & 4 & 33.3\% \\ 
Syria & 12 & 8 & 66.7\% \\ 
Thailand & 12 & 8 & 66.7\% \\ 
Ecuador & 11 & 5 & 45.5\% \\ 
Burundi & 11 & 5 & 45.5\% \\ 
Guatemala & 10 & 5 & 50.0\% \\ 
Total & 491 & 245 & 49.9\% \\ 
\bottomrule

\end{longtable}

\begin{minipage}{\linewidth}
\emph{Source: GIC dataset}\\
\end{minipage}
\endgroup

However, coups are high-stakes gambles with an all-or-nothing nature.
Due to its illegality, the consequences of a failed coup can be severe,
with perpetrators risking imprisonment, exile, or even death. In some
instances, repercussions extend to their families, which means coup
plotters would not take the risk without some assurance of success.

Historical coup attempts and their success rates provide valuable hints
about coup plotters' decision-making processes. Despite the significant
risks associated with coups, as shown in Table~\ref{tbl-coups}, there
have been 491 coups worldwide since 1950. Importantly, about half of
these coups have been successful. At first glance, coups appear to be a
high-success-rate political venture. However, compared to over 12,000
country-years since 1950, the occurrence of 491 coups is relatively
rare, accounting for only about 4\% (GIC).

The low occurrence rate and high success rate clearly indicate that the
initiation of coups is highly selective. In other words, the likelihood
of a coup occurring depends greatly on its probability of success.
However, the probability of success is not a factor that can be observed
before a coup, at least not to outsiders and researchers, but coup
plotters surely have more information on the possible outcome of coups.

Since coup plotters meticulously assess potential outcomes before
staging a coup, coup researchers should also analyse what factors might
affect the outcome of coups. Otherwise, we ran the risk of selection
bias. When considering the factors that most affect the outcomes of
coups, the current literature predominantly identifies military power as
the decisive factor in the success of coups. Ultimately, military power
is determined by the power structure within the regime.

Unlike coup plotters, who have a clearer understanding of power
structures and dynamics, including their supporters, followers,
bystanders, and opponents, researchers are not able to accurately
observe this information. However, we can analyse the most relevant
factors which determine the balance of power and are observable to us.
This leads us to \textbf{regime type}, which reflects the distribution
of power within a government, encompassing who controls the military,
sets policy, and appoints officials. Since coup plotters consider the
balance of power within the regime before acting, this study argues that
regime type plays a crucial role in shaping coup attempts. Analysing
which regime types are more susceptible to coups can offer valuable
insights.

We employ a \textbf{double probit model with sample selection} to
address the selection bias. This model allows us to analyze both the
factors influencing coup success and the factors related to the decision
to initiate a coup attempt in the first place.

This study contributes to the literature in two key ways. First, it
emphasizes the importance of focusing on expected success rates as a
driver of coup attempts, offering a more targeted approach compared to
past studies. Second, it highlights the significance of regime type as a
factor influencing coup likelihood, even when researchers lack perfect
knowledge of a regime's internal power dynamics.

The subsequent section (Section 2) explores the dynamics of coup
attempts and their outcomes. Section 3 delves into the research design,
outlining the methodology and variables used in the analysis. Section 4
presents and discusses the empirical findings. Finally, Section 5
concludes the chapter by summarizing the key insights and their
implications.

\section{Dynamics of coup attempts and
outcomes}\label{dynamics-of-coup-attempts-and-outcomes}

Coup attempts are driven by a complex interplay of factors, including
motivations (\textbf{disposition}) and the resources and opportunities
available to succeed (\textbf{capability}).

\subsection{Motivations for coups}\label{motivations-for-coups}

This section focuses on the motivations that compel challengers to
undertake coups. We can categorize coup motivations into three main
types:

\textbf{Personal Ambition:} Personal ambition is a significant motivator
for the majority of coup plotters, driven by the allure of absolute
power, prestige, and wealth. The prospect of seizing absolute power
provides an unparalleled opportunity to shape national policies, control
resources, and make significant decisions without constraints. The
pursuit of prestige and recognition, along with the potential for
immense economic gain and wealth, further incentivizes individuals.
Additionally, the desire to leave a lasting legacy and make a historical
impact can compel individuals to undertake the risky venture of staging
a coup.

\textbf{Purported National Interest:} Sometimes coups are justified as
necessary interventions to address national crises, uphold the
constitution, or facilitate a transition to democracy. While the
motivations behind such claims require scrutiny, genuine examples do
exist. For instance, the 2010 coup in Niger ousted President Tandja, who
attempted an unconstitutional third term by dissolving the opposing
court and calling a self-serving referendum
(\citeproc{ref-ginsburg2019}{Ginsburg and Elkins 2019}).

\textbf{Self-Preservation:} In some cases, coups serve as pre-emptive
strikes against perceived threats. Coup leaders might not necessarily
seek power, but rather fear elimination or political persecution by the
incumbent regime. An example is Idi Amin's 1971 coup against Ugandan
President Obote, who was attempting to remove Amin from his military
command position (\citeproc{ref-sudduth2017}{Sudduth 2017}).

These motivations are often most prevalent in autocratic regimes, where
justifications under the guise of national interest or self-preservation
can be used to mask personal agendas. Stable democracies rarely face the
same level of constitutional crises or political persecution that might
necessitate a coup. However, new established democracies can be
vulnerable to instability, economic downturns, and democratic
backsliding, creating opportunities for coup plotters to exploit these
weaknesses and justify their actions.

Despite the potential motivations outlined above, coups remain
relatively uncommon, occurring in only about 4\% of country-years since
1950. The main reason is that coup perpetrators face inherent
disadvantages compared to incumbent leaders, highlighting the importance
of capability. Even the most motivated plotters need the resources and
opportunities to succeed. The next section will explore the concept of
capability in greater detail.

\subsection{Capability for coups}\label{capability-for-coups}

For coup plotters, the decision to act hinges not only on their
motivations but also on a calculated assessment of their chances of
success. Several factors can influence this threshold:

\begin{itemize}
\item
  \textbf{Military Strength}: A clear advantage in military capabilities
  compared to the incumbent regime significantly increases the odds of a
  successful coup.
\item
  \textbf{Internal Divisions within the Regime}: Existing fractures
  within the government's power structure can present opportunities for
  coup plotters to exploit weaknesses and gain support from disgruntled
  factions.
\item
  \textbf{Public Support}: Widespread discontent with the incumbent
  regime, especially within the military or key sectors of society, can
  create an environment ripe for a successful coup.
\item
  \textbf{Foreign Backing}: External support from powerful nations can
  provide resources, legitimacy, and even direct military intervention
  to tip the scales in favour of the coup plotters.
\end{itemize}

The high success rate of coups since 1950, as shown in
Table~\ref{tbl-coups}, might suggest that coups are not particularly
difficult to succeed. However, it is important to consider selection
bias. We only observe attempted coups, not the numerous dispositions and
conspiracies that never came to light. Analysing launched coup data
alone can be misleading.

To gain a more comprehensive understanding of coup attempts and their
likelihood, we need to move beyond historical data and employ a
theoretical framework that accounts for selection bias. This framework
will allow us to analyze the factors influencing both the success of
coups and the decision to attempt a coup in the first place. The next
section will delve into such a framework and its implications for our
understanding of coup attempts.

\subsection{Framework of coup success}\label{framework-of-coup-success}

An oft-cited framework (\citeproc{ref-gassebner2016}{Gassebner, Gutmann,
and Voigt 2016}; \citeproc{ref-aidt2019}{Aidt and Leon 2019}) provides a
structured approach to assess the disposition and capability of coup
attempts by evaluating the anticipated benefits for coup plotters. The
expected pay-off of coups can be represented by the equation:

\begin{equation}\phantomsection\label{eq-eq1}{
\begin{aligned}
E(U) = p \times B + (1 - p) \times (-C)
\end{aligned}
}\end{equation}

Here, \(\mathbf B\) represents the return of a successful coup,
\(\mathbf C\) signifies the cost of a failed coup, and \(p\) represents
the probability of coup success. The condition for staging a coup is
when the expected benefit is positive, meaning that the expected pay-off
is greater than 0. Rearranging the equation, we get:

\begin{equation}\phantomsection\label{eq-eq2}{
\begin{aligned}
p \times B > (1 - p) \times C
\end{aligned}
}\end{equation}

Equation~\ref{eq-eq2} implies that for Equation~\ref{eq-eq1} to hold,
the expected benefits earned from successful coups must outweigh the
expected cost of failed coups.

While seemingly clear, the equation faces practical challenges.
Quantifying \(\mathbf B\) (the value of a successful coup) and
\(\mathbf C\) (the cost of failure) is difficult. The loss of life,
freedom, or loved ones after a failed coup, as well as the value of
assuming leadership after a successful coup, are intangible concepts
that defy precise measurement. As evidenced by the 1979 coup in
Ghana\footnote{According to the Archigos dataset, ``Removed by Military,
  without Foreign Support'' and ``Removed by Other Government Actors,
  without Foreign Support'' in the variable \texttt{exitcode} are
  classified as coups.}, the fate of the coup leader(s) hangs in the
balance; they are high likely to be killed if the coup fails, or to
execute others if the coup succeeds.

However, these challenges do not render the framework useless. Firstly,
its core logic remains valuable, offering insights into how coup
plotters might assess the return and cost of their actions. Secondly,
given the significant and elusive nature of precise values for
\(\mathbf B\) and \(\mathbf C\), they can be treated as roughly equal.
Consequently, there is no need to fret over how to measure and compare
these values precisely. Instead, we can shift our focus from
\(\mathbf B\) and \(\mathbf C\), to the probability of success (\(p\)),
simplifying Equation~\ref{eq-eq2} to:

\begin{equation}\phantomsection\label{eq-eq3}{
\begin{aligned}
p > (1-p)
\end{aligned}
}\end{equation}

Equation~\ref{eq-eq3} suggests that, to hold Equation~\ref{eq-eq2} true,
a success probability greater than \(50\%\) is necessary. Interestingly,
empirical data on coups since 1950 somewhat supports this notion. As
shown in Table~\ref{tbl-coups}, the overall success rate is \(49.9\%\).
While this falls short of the \(50\%\) threshold, it's important to
consider two factors. Firstly, this is an average rate, not necessarily
reflective of the probabilities assessed by coup plotters beforehand.
Secondly, outliers such as irrational actors and coups driven by
self-preservation may not prioritize success probabilities. Taking these
points into account, we can propose our first hypothesis:

\begin{quote}
\textbf{\emph{H1: The fundamental determinant of a coup attempt is the
perceived chance of success. Coup plotters likely require a success
threshold of at least 50\%.}}
\end{quote}

This leads us to the next crucial question: What factors determine coup
success, influencing the very decision to attempt one? We will discuss
it in the next section.

\subsection{Regime types and power
dynamics}\label{regime-types-and-power-dynamics}

The high success rate in historical data is merely a post-event
statistical description, not indicative of the probability of any
specific coup. Importantly, coup plotters do not rely on past success
rates as their guide; instead, they assess their chances based on their
unique context.

Military strength undeniably plays a critical role in coup attempts.
Control of the armed forces offers a significant advantage, explaining
why military coups dominate discussions on the topic. Much of the
literature treats ``coup'' and ``military coup'' interchangeably, with
scholars like J. M. Powell and Thyne (\citeproc{ref-powell2011}{2011})
finding that half of 14 studies attribute coups solely to the military.
Consequently, significant focus, from both researchers and
policy-makers, centres on the balance of power between civilian and
military authorities, or among military factions themselves. Strategies
like ``keeping the military content'' (\citeproc{ref-aidt2019}{Aidt and
Leon 2019}) or ``providing them with resources''
(\citeproc{ref-huntington1991democratization}{Huntington 1991}) aim to
reduce military intervention. Empirical research informs coup-proofing
strategies that either decrease the military's desire for coups or raise
barriers to success (\citeproc{ref-leon2013a}{Leon 2013};
\citeproc{ref-powell2018}{J. Powell et al. 2018}).

However, while military power is decisive, it complicates the analysis.
As Table~\ref{tbl-regimes} demonstrates, military regimes, despite
concentrated military control, exhibit surprising instability,
experiencing the most frequent coup attempts. This highlights a crucial
issue: the intra-military component. Treating the military as a
monolithic entity ignores its complex internal dynamics
(\citeproc{ref-singh2016}{Singh 2016}). Any military comprises diverse
groups with their own hierarchies, fostering suspicion, competition, and
vigilance rather than unity. The clandestine nature of coups
necessitates small, secretive groups. Plotters are unsure of other
factions' stances and fear their opposition or intervention, as
exemplified by the swiftly thwarted 2021 Niger coup\footnote{The
  definitions and concepts of power expansion and power extension can be
  vague. In this study, we define power expansion as an incumbent
  acquiring additional authority from other state apparatuses, whereas
  power extension refers to an incumbent prolonging their tenure beyond
  the designated term in office.}. The success of a coup hinges heavily
on other military factions' reactions (\citeproc{ref-geddes1999}{Geddes
1999}).

Moreover, military force is not the only factor shaping the balance of
power. Internal divisions within the ruling elites, public support, and
foreign backing also play important roles. Since the balance of power is
not a clearly observable variable, especially difficult to measure for
outsiders and academic scholars, we can shift our focus from who
controls power in a specific context to what factors shape the balance
of power. This leads us to regime type, as the classification of regime
type is based mainly on the power structure. For example,
``\ldots definition of regimes emphasizes the rules that identify the
group from which leaders can come and determine who influences
leadership choice and policy'' (\citeproc{ref-geddes2014}{Geddes,
Wright, and Frantz 2014, 314}). We will leverage this framework to
categorize autocracies based on leadership origin and decision-making,
classifying regimes into three main categories: military, personalist,
and dominant-party.

\begin{itemize}
\item
  \textbf{Military Regimes:} Characterized by the dominance of a
  junta---a group of military officers who control the regime's power
  structure, including leadership selection and policy formulation.
  Examples include the Brazilian regime (1964-1985), the Argentine
  regime (1976-1983), and the Salvadoran regime (1948-1984)
  (\citeproc{ref-geddes1999}{Geddes 1999}).
\item
  \textbf{Personalist Regimes:} Power resides with a single, charismatic
  leader who controls policy, the military, and succession. Regimes like
  Rafael Trujillo's in the Dominican Republic (1930-1961), Idi Amin's in
  Uganda (1971-1979), and Jean-Bédel Bokassa's in the Central African
  Republic (1966-1979) exemplify personalist rule (ibid.).
\item
  \textbf{Dominant-Party Regimes:} Power rests within a well-organized
  ruling party, with leaders acting as its representatives. The party
  structure and ideology foster internal cohesion and a long-term
  vision. Examples include the Partido Revolucionario Institucional
  (PRI) in Mexico, the Revolutionary Party of Tanzania (CCM), and
  Leninist parties in various Eastern European countries (ibid.).
\end{itemize}

The critical distinction between regime types lies in the unique power
balance established during their seizure of power, often through
irregular and violent means, frequently involving military
confrontations. This power struggle typically results in the emergence
of the most competent group---be it a military junta, a political party,
or a strongman---as the core leadership. Furthermore, the early stages
of new regimes are often marked by internal purges to consolidate power
and eliminate potential rivals (\citeproc{ref-sudduth2017}{Sudduth
2017}; \citeproc{ref-roessler2011}{Roessler 2011}).

Following these external challenges and internal purges, new power
dynamics emerge in three primary forms: dominant-party regimes,
personalist regimes, and military regimes. These contrasting power
dynamics significantly influence a regime's susceptibility to coups.
Dominant-party regimes, characterized by a well-organized party with
clear ideology, hierarchy, and discipline, exhibit the greatest
resilience against coups due to their institutionalized structures and
unified leadership. Personalist regimes, where power centers around a
single strong leader, are relatively stable during the leader's tenure.
However, they face a higher risk of coups, particularly due to the lack
of clear succession plans and vulnerabilities associated with the
leader's personal weaknesses, health issues, and finite lifespan.
Military regimes, characterized by power struggles within a junta, are
the most vulnerable to coups. The absence of a clear final authority in
military regimes often leads to internal conflicts, and the presence of
multiple military factions increases the likelihood of resolving
disputes through force, making these regimes the least stable.
Table~\ref{tbl-regimes1} summarises the key characteristics of the three
main regime types.

These contrasting power dynamics significantly influence a regime's
susceptibility to coups. As Table~\ref{tbl-regimes} confirms, military
regimes, despite representing only 5.6\% of country-years, experience a
disproportionate share of coups, accounting for over 22\% of all
occurrences. Personalist regimes face a similarly high coup risk,
constituting 23\% of coups while representing only 13\% of
country-years. Conversely, dominant-party regimes, with their
institutionalized structures and unified leadership, exhibit the
greatest resilience. They represent 22.6\% of country-years but account
for only 16.7\% of coups. The column ``Coup Likelihood'' clearly
illustrates the varying risks: military regimes have the highest
likelihood at 17.2\%, followed by personalist regimes at 7.7\%, and
dominant-party regimes at 3.2\% (excluding \emph{Other} regime types).

\begin{quote}
\textbf{\emph{H2: Due to their balance of power dynamics, military
regimes are more prone to coups, followed by personalist regimes, while
dominant-party regimes are the least likely to experience coups among
the three.}}
\end{quote}

\newpage

\blandscape

\begingroup
\setlength\LTleft{0\linewidth}
\setlength\LTright{0\linewidth}\fontsize{12.0pt}{14.4pt}\selectfont
\setlength{\LTpost}{0mm}

\begin{longtable}{@{\extracolsep{\fill}}>{\raggedright\arraybackslash}p{75pt}>{\raggedright\arraybackslash}p{75pt}>{\raggedright\arraybackslash}p{75pt}>{\raggedright\arraybackslash}p{75pt}>{\raggedright\arraybackslash}p{75pt}>{\raggedright\arraybackslash}p{75pt}}

\caption{\label{tbl-regimes1}Main features of different types of
regimes}

\tabularnewline

\toprule
Regime Type & Power Concentration & Succession & Military Alignment & Stability & Examples \\ 
\midrule\addlinespace[2.5pt]
Military & Junta & Unclear & May have significant influence & Low & Brazil (1964-1985), Argentina (1976-1983) \\ 
Personalist & Single Leader & Unclear or dependent on leader's will & Subordinated to leader & Moderate (initially), Low (long-term) & Dominican Republic (Trujillo, 1930-1961) \\ 
Dominant-Party & Party Leadership & Institutionalized & Aligned with the party & High & Mexico (PRI), China (CPC) \\ 
\bottomrule

\end{longtable}

\begin{minipage}{\linewidth}
\emph{Source: GWF \& Author}\\
\end{minipage}
\endgroup

\newpage

\begingroup
\setlength\LTleft{0\linewidth}
\setlength\LTright{0\linewidth}\fontsize{12.0pt}{14.4pt}\selectfont
\setlength{\LTpost}{0mm}

\begin{longtable}{@{\extracolsep{\fill}}lrrrrrr}

\caption{\label{tbl-regimes}Regime types and coups since 1950}

\tabularnewline

\toprule
Regime Type & Country Year & Share & Num of Coups & Percent of Coups & Success Rate & Coup Likelihood \\ 
\midrule\addlinespace[2.5pt]
Democracy & 5303 & 46.7\% & 122 & 24.8\% & 51.6\% & 2.3\% \\ 
Dominant-Party & 2569 & 22.6\% & 82 & 16.7\% & 53.7\% & 3.2\% \\ 
Personal & 1477 & 13.0\% & 113 & 23.0\% & 44.2\% & 7.7\% \\ 
Monarchy & 1056 & 9.3\% & 25 & 5.1\% & 56.0\% & 2.4\% \\ 
Military & 638 & 5.6\% & 110 & 22.4\% & 48.2\% & 17.2\% \\ 
Other & 322 & 2.8\% & 39 & 7.9\% & 53.8\% & 12.1\% \\ 
Total & 11365 & 100.0\% & 491 & 100.0\% & 49.9\% & 4.3\% \\ 
\bottomrule

\end{longtable}

\begin{minipage}{\linewidth}
\emph{Source: REIGN and GIC Datasets}\\
\end{minipage}
\endgroup

\elandscape

\section{Research Design}\label{research-design}

\subsection{Double probit with sample selection
model}\label{double-probit-with-sample-selection-model}

This study employs a sophisticated statistical approach to account for
the selective nature of coup attempts. While coup attempt rates vary
across regimes (as discussed previously), success rates tend to be
surprisingly consistent, hovering around 50\% (as shown in
Table~\ref{tbl-regimes}). This suggests that coup attempts are not
random acts, but rather strategically planned and undertaken only when
the odds of success appear favourable. A standard statistical model
would not account for this selectivity, potentially leading to biased
results.

To address this issue, we utilize a two-stage sample selection model,
similar to the approach used by J. Powell
(\citeproc{ref-powell2012}{2012}). This model has two parts:

\begin{itemize}
\item
  \textbf{Selection Equation (Stage 1):} This stage analyzes the factors
  influencing whether a coup attempt occurs in a particular regime. The
  primary explanatory variable here is regime type, as previously
  discussed. Additional control variables may also be included, denoted
  by \(\mathbf{XB}\).
\item
  \textbf{Outcome Equation (Stage 2):} This stage focuses on the
  probability of success for those coup attempts that actually take
  place.
\end{itemize}

The primary explanatory variables are regime types, as previously
discussed. Control variables are included in \(\mathbf{XB}\). The
selection equation (first stage) models the probability that a coup
attempt occurs and can be expressed as follows:

\begin{equation}\phantomsection\label{eq-eq4}{
\begin{aligned}
y_1^*=\alpha_0 + \alpha_1 Regime_i + \mathbf {XA} + \mu_{1i}
\end{aligned}
}\end{equation}

Here, \({y_1}^*\) is an unobserved variable, which may be known to coup
plotters. \(Regime_i\) is a categorical variable (\emph{military},
\emph{personalist}, or \emph{dominant-party}). \(\mathbf{XB}\) captures
other control variables, such as the economic crisis index, previous
coups, military expenditure, etc.

The observed binary outcome \(\mathbf{y_1}\) is:

\[
\begin{aligned}
y_1 = 
\begin{cases} 
1 &\text{if $y_1^*>0$ (coup attempt occurs)} \\
0 &\text{if $y_1^*\le0$ (no coup attempt)}
\end{cases}
\end{aligned}
\]\\
In the first stage, if \(y_1^*\le0\), no coup attempt occurs in a given
country-year, indicating that the unobserved variable does not reach the
threshold. If \(y_1^*>0\), at least one coup attempt is made in a
country-year, indicating that the unobserved variable surpasses the
threshold. The probability is expressed as:

\begin{equation}\phantomsection\label{eq-eq4a}{
\begin{aligned}
Prob(y_1 =1)&=Prob(y_{1}^*>0) \\
&=\Phi(\alpha_0 + \alpha_1 Regime_i + \mathbf{XA})
\end{aligned}
}\end{equation}

Similarly, the outcome equation (second stage) models the probability
that a coup attempt is successful, given that it occurs:

\begin{equation}\phantomsection\label{eq-eq5}{
\begin{aligned}
y_2^*=\beta_0 + \beta_1 Regime_i + \mathbf {XB} + \mu_{2i}
\end{aligned}
}\end{equation}

The observed outcome \(y_2\) is:

\[
\begin{aligned}
y_2 = 
\begin{cases} 
1 &\text{if $y_2^*>0$ (coup succeeds)} \\
0 &\text{if $y_2^*\le0$ (coup fails)}
\end{cases}
\end{aligned}
\]

The probability equations is:

\begin{equation}\phantomsection\label{eq-eq5a}{
\begin{aligned}
Prob(y_2=1|y_1 =1)=\Phi(\beta_0 + \beta_1 Regime_i + \mathbf{XB})
\end{aligned}
}\end{equation}

\subsection{Variables}\label{variables}

\begin{itemize}
\tightlist
\item
  Dependent variable
\end{itemize}

Our analysis utilizes data on coup attempts and outcomes from J. M.
Powell and Thyne (\citeproc{ref-powell2011}{2011}). A successful coup is
defined as one where the incumbent leader is removed from power for more
than seven days. The dataset covers the period from 1950 to 2023 and
includes information on 491 coup attempts, with roughly half (245) being
successful. Descriptive statistics for these coup attempts and regime
types can be found in Table~\ref{tbl-coups} and Table~\ref{tbl-regimes}.

\begin{itemize}
\tightlist
\item
  Key Independent Variable: Regime Type
\end{itemize}

The core variable of interest is regime type, categorized following the
classification system of Geddes, Wright, and Frantz
(\citeproc{ref-geddes2014}{2014}) (GWF). We focus on military,
personalist, and dominant-party regimes, with democracies and monarchies
included for comparison. Descriptive statistics for regime types are
presented in Table~\ref{tbl-regimes}.

\begin{itemize}
\tightlist
\item
  Control variables
\end{itemize}

Our control variables are chosen based on the research of Gassebner,
Gutmann, and Voigt (\citeproc{ref-gassebner2016}{2016}). They analyzed
66 factors potentially influencing coups and found that slow economic
growth, prior coup attempts, and other forms of political violence are
particularly significant factors. Therefore, we include economic
performance, political violence, and the number of previous coups as our
main control variables.

\textbf{Economic Performance:} We measure economic performance using the
current-trend (\(CT\)) ratio developed by Krishnarajan
(\citeproc{ref-krishnarajan2019}{2019}). This ratio compares a country's
current GDP per capita to the average GDP per capita over the previous
five years. A higher \(CT\) ratio indicates stronger economic
performance. We use GDP per capita data (in constant 2017 international
1000 dollars, PPP) from the V-Dem dataset by Fariss et al.
(\citeproc{ref-fariss2022}{2022}), lagged by one year to reflect the
prior year's economic impact. For a country \(i\) at year \(t\), the
\(CT\) ratio is calculated as follows: \[
\begin{aligned}
CT_{i,t} = {GDP/cap_{i,t} \over {1 \over 5} {\sum_{k=1}^5GDP/cap_{i,t-k}}}
\end{aligned}
\]

\textbf{Political Violence:} We capture overall regime stability by
including a violence index that encompasses all types of internal and
interstate wars and violence. This data comes from the Major Episodes of
Political Violence dataset by Marshall
(\citeproc{ref-marshall2005current}{Marshall 2005}).

\textbf{Previous coups:} The number of previous coups in a country is
included in the first-stage (selection) model to assess its influence on
the likelihood of a coup attempt. However, it is excluded from the
second-stage model (outcome) because the number of past coups may not
directly impact the outcome of a specific coup attempt. Recognizing that
previous coups from a long time ago might not significantly affect the
current situation, I ran another regression using the time distance
since the last coup instead of the number of previous coups for
comparison. This approach accounts for the diminishing impact of past
events over time, providing a potentially more accurate assessment of
how historical coup activity influences the likelihood of current coup
attempts.

\section{Results and Discussion}\label{results-and-discussion}

\begin{table}[!htbp] \centering 
  \caption{Sample Selection Model of Regime Types and Coups, 1950-2019} 
  \label{results} 
\small 
\begin{tabular}{@{\extracolsep{7pt}}lcccc} 
\\[-1.8ex]\hline 
\hline \\[-1.8ex] 
\\[-1.8ex] & \multicolumn{2}{c}{Model 1} & \multicolumn{2}{c}{Model 2} \\ 
 & Coup Attempts & Coup Outcome & Coup Attempts & Coup Outcome \\ 
\\[-1.8ex] & (1) & (2) & (3) & (4)\\ 
\hline \\[-1.8ex] 
 Constant & $-$1.774$^{***}$ & $-$1.803$^{***}$ & $-$1.663$^{***}$ & $-$0.653 \\ 
  & (0.058) & (0.360) & (0.088) & (0.518) \\ 
  & & & & \\ 
 Regime: Democracy & 0.056 & 0.068 & 0.043 & 0.042 \\ 
  & (0.072) & (0.121) & (0.075) & (0.192) \\ 
  & & & & \\ 
 \hspace{1.6cm}Military & 0.687$^{***}$ & 0.596$^{***}$ & 0.345$^{***}$ & 0.247 \\ 
  & (0.084) & (0.170) & (0.091) & (0.229) \\ 
  & & & & \\ 
 \hspace{1.6cm}Monarchy & 0.282$^{**}$ & 0.178 & 0.234$^{*}$ & 0.088 \\ 
  & (0.118) & (0.201) & (0.123) & (0.310) \\ 
  & & & & \\ 
 \hspace{1.6cm}Personalist & 0.319$^{***}$ & 0.128 & 0.134$^{*}$ & $-$0.145 \\ 
  & (0.075) & (0.170) & (0.080) & (0.205) \\ 
  & & & & \\ 
 Economic trend & $-$0.015$^{***}$ & $-$0.004 & $-$0.014$^{***}$ & 0.009 \\ 
  & (0.002) & (0.007) & (0.002) & (0.008) \\ 
  & & & & \\ 
 GDP per capita & $-$0.028$^{***}$ & $-$0.028$^{***}$ & $-$0.016$^{***}$ & $-$0.016 \\ 
  & (0.003) & (0.006) & (0.003) & (0.010) \\ 
  & & & & \\ 
 Political violence & 0.033$^{**}$ & 0.033$^{*}$ & 0.038$^{***}$ & 0.025 \\ 
  & (0.013) & (0.020) & (0.013) & (0.031) \\ 
  & & & & \\ 
 Previous coups (P) & 0.030$^{***}$ &  & 0.447$^{***}$ &  \\ 
  & (0.010) &  & (0.086) &  \\ 
  & & & & \\ 
 Yrs since coup (Y) &  &  & $-$0.018$^{***}$ &  \\ 
  &  &  & (0.004) &  \\ 
  & & & & \\ 
 Interaction term: P * Y &  &  & $-$0.013$^{***}$ &  \\ 
  &  &  & (0.005) &  \\ 
  & & & & \\ 
\hline \\[-1.8ex] 
Observations & 9,606 & 9,606 & 9,606 & 9,606 \\ 
Log Likelihood & $-$1,663.646 & $-$1,663.646 & $-$1,598.603 & $-$1,598.603 \\ 
$\rho$ & 0.898$^{***}$  (0.158) & 0.898$^{***}$  (0.158) & 0.386$^{*}$  (0.234) & 0.386$^{*}$  (0.234) \\ 
\hline 
\hline \\[-1.8ex] 
\textit{Note:}  & \multicolumn{4}{r}{$^{*}$p$<$0.1; $^{**}$p$<$0.05; $^{***}$p$<$0.01} \\ 
\end{tabular} 
\end{table}

The double \texttt{probit} model with sample selection, estimated using
the \texttt{sampleSelection} package
(\citeproc{ref-sampleSelection-2}{Toomet and Henningsen 2008}) in R,
provides valuable insights into the factors influencing coup attempts
and their outcomes across different regime types from 1950 to 2019
(Table \ref{results}). We present two models that differ slightly in
their treatment of previous coups: Model 1 incorporates the number of
previous coups, while Model 2 utilizes the time elapsed since the last
coup.

\newpage

\subsection{The Selection Model: Coup
Attempts}\label{the-selection-model-coup-attempts}

\begingroup
\setlength\LTleft{0.025\linewidth}
\setlength\LTright{0.025\linewidth}\fontsize{12.0pt}{14.4pt}\selectfont
\setlength{\LTpost}{0mm}

\begin{longtable}{@{\extracolsep{\fill}}llrr}

\caption{\label{tbl-mfx1}Average marginal effects of coup attempts
(Selection of Model 1)}

\tabularnewline

\toprule
Term & Contrast & AME\textsuperscript{\textit{1}} & Ratio Percent \\ 
\midrule\addlinespace[2.5pt]
Regime: Democracy & mean(democracy - dominant-party) & 0.003 & 13.030 \\ 
{\hspace{47.25pt}Military} & mean(military - dominant-party) & 0.070 & 277.730 \\ 
{\hspace{47.25pt}Monarchy} & mean(monarchy - dominant-party) & 0.020 & 80.620 \\ 
{\hspace{47.25pt}Personal} & mean(personal - dominant-party) & 0.024 & 93.980 \\ 
Economic trend & mean(+1) & -0.001 & -2.850 \\ 
GDP per capita & mean(+1) & -0.002 & -5.400 \\ 
Political violence & mean(+1) & 0.003 & 6.550 \\ 
Previous coups & mean(+1) & 0.002 & 5.930 \\ 
\bottomrule

\end{longtable}

\begin{minipage}{\linewidth}
\textsuperscript{\textit{1}}AME: Average Marginal Effect\\
\end{minipage}
\endgroup

In the selection model (Column 1), military and personalist regimes
exhibit significant positive coefficients at the 1\% level, indicating a
higher likelihood of experiencing coup attempts compared to
dominant-party regimes. Control variables also exhibit effects in
expected directions. Stronger economic performance, indicated by higher
economic growth trends and GDP per capita levels, correlates with a
lower risk of coup attempts. This suggests that better economic
conditions and relatively higher living standards reduce incentives for
coups. Political violence shows a positive and significant effect on
coup attempts, indicating that higher levels of instability increase the
likelihood of coups. The positive coefficient for the number of previous
coups suggests a ``copycat'' effect from earlier incidents.

While interpreting \texttt{probit} model results directly is not
intuitive, Table~\ref{tbl-mfx1}, generated using
\texttt{marginaleffects} package
(\citeproc{ref-marginaleffects}{Arel-Bundock, Greifer, and Heiss NaN}),
helps clarify the regime effects. The Average Marginal Effect (AME)
column shows changes in probability when a variable increases by one
unit (for numeric variables) or compared to a reference category (for
categorical variables). For instance, the military regime's marginal
effect of 0.07 indicates that the probability of coup attempts in
military regimes is 7 percentage points (pp) higher than in
dominant-party regimes, ceteris paribus. For GDP per capita, the
marginal effect of -0.002 suggests that a \$1000 increase reduces the
probability of coup attempts by 0.2 pp under average conditions.

The Ratio column displays the relative change in percent. While a 7 pp
increase might seem modest, it is substantial in context. Given that the
average probability of a coup attempt in dominant-party regimes is
approximately 2.53\%, the 7 pp increase means military regimes are about
277.7\% more likely to encounter coups than dominant-party regimes---a
significant change. Similarly, personalist regimes show a 2.4 pp higher
probability, about 94\% more likely compared to dominant-party regimes.
Monarchies display a positive effect similar to personalist regimes,
reflecting that monarchies are essentially a subset of personalist
regimes with royal titles.

Control variables, however, show weak effects in predicting coup
attempts. None of their marginal effects reach 1 pp in probability, and
all are less than 7\% in ratio. Given that the actual probability of a
coup attempt is only 2\% to 3\%, these small increases or decreases can
be considered negligible.

These results align with our theoretical expectations regarding internal
power struggles within military juntas and succession vulnerabilities in
personalist regimes, underscoring the importance of regime structure in
understanding coup likelihood.

Model 2 employs years since the last coup instead of the number of
previous coups. Due to most countries not experiencing coups, an
interaction term between previous coups (as a binary variable indicating
presence or absence of coups) and years since the last coup is used.
Generally, Model 2 shows results in the same direction as Model 1,
albeit with relatively lower coefficients (Table~\ref{tbl-mfx2}).

\begingroup
\setlength\LTleft{0.025\linewidth}
\setlength\LTright{0.025\linewidth}\fontsize{12.0pt}{14.4pt}\selectfont
\setlength{\LTpost}{0mm}

\begin{longtable}{@{\extracolsep{\fill}}llrr}

\caption{\label{tbl-mfx2}Average marginal effects of coup attempts
(Selection of Model 2)}

\tabularnewline

\toprule
Term & Contrast & AME\textsuperscript{\textit{1}} & Ratio Percent \\ 
\midrule\addlinespace[2.5pt]
Regime: Democracy & mean(democracy - dominant-party) & 0.003 & 8.930 \\ 
{\hspace{47.25pt}Military} & mean(military - dominant-party) & 0.028 & 91.650 \\ 
{\hspace{47.25pt}Monarchy} & mean(monarchy - dominant-party) & 0.018 & 57.040 \\ 
{\hspace{47.25pt}Personal} & mean(personal - dominant-party) & 0.009 & 30.090 \\ 
Economic trend & mean(+1) & -0.001 & -2.530 \\ 
GDP per capita & mean(+1) & -0.001 & -2.890 \\ 
Political violence & mean(+1) & 0.003 & 7.330 \\ 
Previous coups (P) & mean(1 - 0) & 0.023 & 91.990 \\ 
NA & mean(+1) & -0.002 & -5.050 \\ 
\bottomrule

\end{longtable}

\begin{minipage}{\linewidth}
\textsuperscript{\textit{1}}AME: Average Marginal Effect\\
\end{minipage}
\endgroup

\subsection{The Outcome Model: Coup
Success}\label{the-outcome-model-coup-success}

The outcome model (Columns 2 and 4 in Table \ref{results}) reveals
determinants of coup success. Military regimes demonstrate a higher
probability of coup success compared to dominant-party regimes, aligning
with expectations that military regimes face higher coup risks due to
their increased chances of success. Personalist and monarchical regimes
show slight positive effects on coup success, but these effects are not
statistically significant.

Control variables exhibit different patterns in the outcome model
compared to the selection model. Both GDP per capita and political
violence maintain a weak influence, similar to their effects in the
selection model. However, the economic trend shows a less significant
negative effect on coup success.

These results indicate that regime type remains a significant
determinant of both coup attempts and successes, even after controlling
for other factors, strongly supporting the proposed theoretical
framework.

\subsection{Discussion}\label{discussion}

The \(\rho\) values of 0.898 in Model 1 and 0.386 in Model 2,
significant at 1\% and 10\% levels respectively, are crucial parameters
in the sample selection model. These values represent the correlation
between the error terms of the selection equation (coup attempts) and
the outcome equation (coup outcomes). A high and significant \(\rho\)
suggests that unobserved factors influencing the likelihood of a coup
attempt are strongly correlated with those influencing the likelihood of
a successful coup. Practically, this indicates that the selection model
is appropriate and that accounting for selection bias (i.e., the fact
that only coups with high chances of success will be attempted) is
critical for obtaining unbiased estimates. The high \(\rho\) value
indicates that the same underlying conditions that lead to a coup
attempt also affect its success, underscoring the importance of
considering both stages in the analysis.

The results strongly support the choice of the sample selection model.
Significant coefficients with theoretically consistent directions
suggest the model effectively captures key aspects of coup dynamics.
Regimes with weaker institutional structures are more vulnerable to coup
attempts, while better economic conditions make coups less likely
overall. The model effectively addresses the non-random nature of coup
attempts by treating selection and outcome as separate processes.

The observed disparity between coup attempt rates and success rates
across regimes points towards selection bias, further validating the use
of the sample selection model. This model acknowledges that coups are
not random events, but rather strategic actions undertaken when the odds
appear favourable.

In summary, the double \texttt{probit} model with sample selection
proves to be a well-suited approach for this research. It provides
robust insights into the factors influencing both the likelihood of coup
attempts and their success rates across different regime types. The
findings highlight the crucial role of regime structure and the
selective nature of coup attempts, supporting the theoretical framework
and empirical strategy employed in this study.

\subsection{Implications}\label{implications}

The finding that regime type plays a crucial role in determining coup
attempts, while not surprising, has received relatively little attention
as a primary research focus. However, even if this result is
acknowledged more widely by academics or politicians, it is unlikely to
be implemented as a coup-proofing strategy.

Firstly, regime type is not a policy that can be easily altered. It is
largely established during the formation of the regime, resulting from
power struggles within the ruling group. As previously discussed, there
are specific reasons why one regime type is chosen over another. Regimes
established through irregular means are unlikely to transition to a more
stable regime type peacefully and smoothly.

Secondly, even if a particular regime type is more effective at
preventing coups, this may hold little significance for those in power.
Autocratic leaders often prioritize personal power and survival over
long-term regime stability, as more stable systems may not benefit them
personally. Democratic norms and constraints may be seen as obstacles to
their goals rather than as safeguards against instability.

While this conclusion might seem discouraging, it is not entirely
pessimistic. The power game involves not only the top leaders but also
the majority of other elites. While top leaders may prioritize personal
gains, other elites, including the military, economic elites, and civil
society, have a vested interest in a stable political environment.
Therefore, it is possible to reach a consensus for a political system
that is less susceptible to coups, potentially benefiting a broader
spectrum of society.

\section{Conclusion}\label{conclusion}

Motivated by the lack of consensus despite numerous empirical studies on
the determinants of coups, this study introduces a novel approach that
prioritizes determinants based on their impact on coup success. By
analysing coup success rates, the study hypothesizes that the expected
outcomes of coups are critical determinants of their occurrence.
Utilizing a double \texttt{probit} model with sample selection, I
investigate and confirm the relationship between regime types and coup
attempts.

The findings suggest that regime type plays a significant role in the
likelihood of coup attempts. Military and personalist regimes,
characterized by weaker institutional frameworks and higher
vulnerability during power transitions, are more susceptible to coups.
This underscores the importance of supporting initiatives that
strengthen constitutional institutions within these regimes.

The research also finds that stronger economic performance is associated
with a lower risk of coups, suggesting that policies promoting economic
development can be effective in reducing coup risk.

The study shows that the most efficient coup-proofing strategies involve
the establishment of strong institutions. In contrast, purges, random
shifting of military officers, or increased military expenditures are
less effective. However, few autocratic leaders, particularly dictators
or military juntas, are willing to institutionalize their regimes, as
such reforms may constrain their power or shorten their terms. While
institutions benefit the regime, they do not necessarily benefit the
leaders themselves.

Future research could explore specific institutional reforms that are
most effective in improving stability across different regimes.

\chapter{Autocoups: Conceptual Clarification and Analysis of Power
Extensions by Incumbent
Leaders}\label{autocoups-conceptual-clarification-and-analysis-of-power-extensions-by-incumbent-leaders}

\section*{Abstract}\label{abstract-1}
\addcontentsline{toc}{section}{Abstract}

This study aims to clarify the concept of autocoups, specifically
focusing on power extensions by incumbent leaders. By distinguishing
autocoups from the broader and more ambiguous concepts of self-coups or
executive takeover, which encompass both executive power aggrandizement
and power extension, this research redefines the concept of autocoups.
Based on this refined definition, I introduce a novel dataset of
autocoup events from 1945 to 2022. Using the newly compiled dataset, the
research includes three types of case studies that provide qualitative
insights into the dynamics of autocoups. Additionally, an empirical
analysis on the determinants of autocoup attempts and success is offered
to demonstrate how the autocoup dataset can be employed for more
quantitative research. This study contributes to the existing literature
by providing a clearer conceptual framework and a novel dataset of
autocoups. It enhances our understanding of the mechanisms and
motivations behind power extensions by incumbent leaders and examines
the implications for democratic backsliding, democratic breakdown,
personalization, and autocratic deterioration. The insights gained from
this study could draw more attention to the effects of autocoups on
power transitions, political stability, and democratic resilience.

keywords: \emph{Coups, Autocoups, Political Leadership}

\newpage

\section{Introduction}\label{introduction-2}

The study of irregular power transitions, encompassing both irregular
entries and exits from office, has long been a central topic in
political science. However, one form of irregular power transition - the
incumbent leader's refusal to relinquish power - remains relatively
understudied. This scenario, where leaders overstay their mandated term
limits for one or more terms, or even indefinitely, lacks a universally
accepted term in the existing literature. To address this gap, this
study adopts the term ``autocoup'' to denote this phenomenon.

Coups, being the primary form of irregular power transitions, have
received significant scholarly attention. Research by Goemans,
Gleditsch, and Chiozza (\citeproc{ref-goemans2009}{2009}) indicate that
coups account for a substantial portion of such transitions, with
roughly two-thirds (65.8\%) of 374 irregular exits occurring through
coups. Additionally, Frantz and Stein (\citeproc{ref-frantz2016}{2016})
demonstrate that coups are the most common form of exit in autocracies,
surpassing regular transitions by a third.

However, while coups were once frequent, recent decades have witnessed a
decline in their occurrence (\citeproc{ref-bermeo2016}{Bermeo 2016};
\citeproc{ref-thyne2019}{Thyne and Powell 2019}). Conversely, autocoups
appear to be on the rise, particularly since the Cold War's end
(\citeproc{ref-ginsburg2010evasion}{Ginsburg, Melton, and Elkins 2010};
\citeproc{ref-baturo2014}{Baturo 2014};
\citeproc{ref-versteeg2020law}{Versteeg et al. 2020}). Despite the
significant political impact and rising prevalence of autocoups, their
study lags behind that of classical coups, which are extensively
documented due to their historical frequency
(\citeproc{ref-thyne2019}{Thyne and Powell 2019}).

This research aims to bridge this gap by focusing on the understudied
phenomenon of autocoups as they are at least equally important and
warrant comprehensive analysis. Firstly, autocoups are typically
accompanied by a disregard for the rule of law, weakened institutions,
and potential democratic backsliding or authoritarian personalization.
Secondly, like traditional coups, successful autocoups increase the risk
of future irregular power transitions. Research indicates that since
1945, approximately 62 percent of leaders who overstayed their term
limits through autocoups in non-democratic countries were either ousted
or assassinated while in office (\citeproc{ref-baturo2019}{Baturo
2019}). Thirdly, failed autocoups often trigger instability that fuels
protests, violence, and even civil wars.

Despite their significance, autocoups are understudied compared to
traditional coups. Although several relevant concepts and terminologies
such as self-coups, autocoups, autogolpes, incumbent takeovers,
executive aggrandizement, overstay, and continuismo have been proposed
and discussed (\citeproc{ref-marsteintredet2019}{Marsteintredet and
Malamud 2019}; \citeproc{ref-baturo2022}{Baturo and Tolstrup 2022}),
there is no clear and widely accepted term to describe these events.
Different terms are often used interchangeably or inconsistently,
leading to confusion. Moreover, due to the lack of clear concepts, the
collection of autocoup datasets is still in its early stages compared to
the rich datasets of traditional coups. Hence, the limited existing
studies on autocoups have primarily relied on case studies
(\citeproc{ref-cameron1998}{Maxwell A. Cameron 1998b};
\citeproc{ref-antonio2021}{Antonio 2021};
\citeproc{ref-pion-berlin2022}{Pion-Berlin, Bruneau, and Goetze 2022}),
with few focusing on quantitative analyses.

To fully understand irregular political leadership transitions and
survival, we need to redefine and clarify the term ``autocoup'' first.
Based on a clear definition, we can introduce an autocoup dataset that
can be used for further empirical analysis. This study aims to address
these knowledge gap surrounding autocoups and contribute in three areas.
First, it clarifies the terminology by defining autocoups based mainly
on power extension. Second, it introduces a new dataset of autocoups
since 1945 based on this refined definition. Finally, the research
utilizes this dataset for a quantitative analysis of the factors that
influence leaders' decisions to attempt autocoups.

The subsequent section, based on a comprehensive review of the
definitions of power expansions and power extensions, proposes a precise
definition of autocoups. This will be followed by an introduction to the
new autocoup dataset in Section 3. In Section 4 and 5, I will explain
the determinants of autocoup attempts through two case studies and
demonstrate how the novel autocoup dataset collected in this study can
be used in empirical analysis. Finally, I will conclude this chapter in
Section, summarizing the key insights gained from the study and
suggesting avenues for future research.

\chapter{Power Acquisition and Leadership Survival: A Comparative
Analysis of Coup-Entry and Autocoup
Leaders}\label{power-acquisition-and-leadership-survival-a-comparative-analysis-of-coup-entry-and-autocoup-leaders}

\chapter{Conclusion}\label{conclusion-1}

\section{Main Findings}\label{main-findings}

This study delves into the dynamics and implications of irregular power
transitions, focusing on coups and autocoups. The findings illuminate
the complex interplay between incumbents and challengers fighting for
power.

Firstly, our analysis reveals that the expected success rate of a coup
attempt significantly influences its likelihood. This success rate is
heavily influenced by the balance of power between the incumbent regime
and challengers, which is largely determined by regime type. We find
that military regimes, although with more control over their own
military forces, face a higher risk of coups compared to dominant-party
regimes.

Secondly, the study introduces a redefined concept: the autocoup.
Defined as an incumbent leader's refusal to relinquish power as
mandated, this research distinguishes autocoups from broader terms like
self-coups. Based on this definition, we present the first publicly
available dataset of autocoup events from 1945 to 2022, encompassing 110
attempts and 87 successful autocoups. Case studies and empirical
analyses demonstrate the dataset's utility for quantitative research,
providing a robust foundation for further analysis on autocoups.

Thirdly, employing survival analysis techniques, the study finds clear
differences in leader longevity between those who come to power through
coups and those who extend their rule through autocoups. The results
indicate that coup-installed leaders face a significantly higher risk of
removal compared to autocoup leaders who manipulate the system to extend
their rule.

\section{Policy Implications}\label{policy-implications}

The findings of this study offer valuable insights for policy-makers
concerned with promoting and protecting global democracy, which has
faced increasing challenges despite a general post-WWII trend towards
democratization. Notably, the ``third wave'' of democratization
(\citeproc{ref-huntington1991democratization}{Huntington 1991})
witnessed a surge in democratic transitions in the late 20th century.
Since the Cold War's end, democratic nations have outnumbered
non-democratic ones (Figure~\ref{fig-democracy}) with the gap widening.

\begin{figure}

\centering{

\includegraphics{coups_and_autocoups_files/figure-pdf/fig-democracy-1.pdf}

}

\caption{\label{fig-democracy}Comparison of the number of democratic and
non-democratic countries (1945-2020)}

\end{figure}%

However, a ``democratic recession'' has emerged in recent years
(\citeproc{ref-diamond2008}{Diamond 2008}). Freedom House reports an
18th consecutive year of global freedom decline in 2023
(\citeproc{ref-freedomhouse2024freedom}{Freedom House 2024}). While few
countries have completely regressed to autocracy, the average global
democracy level has fallen back to pre-2000 levels. Notably, democratic
backsliding often occurs within regimes, with democracies becoming less
liberal and autocracies becoming less competitive
(\citeproc{ref-mechkova2017}{Mechkova, Lührmann, and Lindberg 2017}).

This research highlights irregular power transitions as a significant
factor in democratic backsliding within regimes. These transitions,
often coups or autocoups, violate democratic norms and disrupt the path
towards stable democracies. Leaders who gain power through irregular
means often resort to undemocratic tactics to maintain control, creating
a vicious cycle of eroding democratic institutions.

Our findings suggest that the shorter lifespans and potentially severe
consequences associated with coups may deter potential coup leaders.
Conversely, autocoups appear to be a more tempting option for
power-hungry leaders due to their higher success rates, seemingly
moderate consequences, and extended leader tenure after the autocoup.
This trend may explain the decline in classic coups since the 1990s
alongside the rise of autocoups (\citeproc{ref-bermeo2016}{Bermeo
2016}).

\section{Limitations and directions for future
research}\label{limitations-and-directions-for-future-research}

This study offers a novel framework for analysing irregular power
transitions, but some limitations require further exploration:

\begin{itemize}
\item
  \textbf{Data refinement:} Defining and classifying autocoups is a new
  approach. Future research should validate this classification system
  through additional studies and expert evaluations.
\item
  \textbf{Data harmonization:} The current analysis faces challenges due
  to mismatched units (country-year vs.~leader) between coup and
  autocoup datasets. Future efforts should explore data harmonization
  techniques for more robust comparisons.
\item
  \textbf{Democratic backsliding:} While this study establishes a
  connection between irregular power transitions and democratic
  backsliding, further empirical evidence is needed to solidify this
  link.
\end{itemize}

Several avenues exist for future research:

\begin{itemize}
\item
  \textbf{Terminology and data collection:} Refining the ``autocoup''
  concept and achieving wider recognition will facilitate more accurate
  and comprehensive data collection.
\item
  \textbf{Dataset expansion:} Expanding the autocoup dataset with more
  cases and integrating it with data on other irregular leadership
  transitions can provide a more holistic view of political survival
  after these events.
\item
  \textbf{Power dynamics and long-term impacts:} Utilizing this dataset,
  future studies can delve deeper into power dynamics at play and
  explore the long-term consequences of irregular transitions on
  political systems, particularly regarding democratic backsliding,
  breakdown, and personalization of power.
\end{itemize}

In conclusion, this study sheds light on the dynamics of irregular power
transitions, specifically focusing on coups and autocoups. By redefining
autocoups, classifying the dataset, analysing determinants, and
comparing leader longevity, we establish a framework for understanding
irregular transitions and leader survival. This work contributes to a
deeper understanding of democratic resilience and political stability.
Future research can build upon this foundation by conducting further
empirical analyses based on the novel autocoup dataset and continuing to
refine the framework.

\newpage

\chapter*{References}\label{references}
\addcontentsline{toc}{chapter}{References}

\phantomsection\label{refs}
\begin{CSLReferences}{1}{0}
\bibitem[\citeproctext]{ref-aidt2019}
Aidt, Toke, and Gabriel Leon. 2019. {``The Coup.''} Edited by Roger D.
Congleton, Bernard Grofman, and Stefan Voigt, February.
\url{https://doi.org/10.1093/oxfordhb/9780190469771.013.15}.

\bibitem[\citeproctext]{ref-antonio2021}
Antonio, Robert J. 2021. {``Democracy and Capitalism in the Interregnum:
Trump{'}s Failed Self-Coup and After.''} \emph{Critical Sociology} 48
(6): 937--65. \url{https://doi.org/10.1177/08969205211049499}.

\bibitem[\citeproctext]{ref-marginaleffects}
Arel-Bundock, Vincent, Noah Greifer, and Andrew Heiss. NaN. {``How to
Interpret Statistical Models Using
{\textbraceleft}Marginaleffects{\textbraceright} in
{\textbraceleft}r{\textbraceright} and
{\textbraceleft}Python{\textbraceright},''} NaN.

\bibitem[\citeproctext]{ref-baturo2014}
Baturo, Alexander. 2014. {``Democracy, Dictatorship, and Term Limits.''}
\url{https://doi.org/10.3998/mpub.4772634}.

\bibitem[\citeproctext]{ref-baturo2019}
---------. 2019. {``Continuismo in Comparison.''} In, 75--100. Oxford
University Press.
\url{https://doi.org/10.1093/oso/9780198837404.003.0005}.

\bibitem[\citeproctext]{ref-baturo2022}
Baturo, Alexander, and Jakob Tolstrup. 2022. {``Incumbent Takeovers.''}
\emph{Journal of Peace Research} 60 (2): 373--86.
\url{https://doi.org/10.1177/00223433221075183}.

\bibitem[\citeproctext]{ref-bermeo2016}
Bermeo, Nancy. 2016. {``On Democratic Backsliding.''} \emph{Journal of
Democracy} 27 (1): 5--19. \url{https://doi.org/10.1353/jod.2016.0012}.

\bibitem[\citeproctext]{ref-cameron1998a}
Cameron, Maxwell A. 1998a. {``Latin American Autogolpes : Dangerous
Undertows in the Third Wave of Democratisation.''} \emph{Third World
Quarterly} 19 (2): 219--39.
\url{https://doi.org/10.1080/01436599814433}.

\bibitem[\citeproctext]{ref-cameron1998}
Cameron, Maxwell A. 1998b. {``Self-Coups: Peru, Guatemala, and
Russia.''} \emph{Journal of Democracy} 9 (1): 125--39.
\url{https://doi.org/10.1353/jod.1998.0003}.

\bibitem[\citeproctext]{ref-chin2021}
Chin, John J, David B Carter, and Joseph G Wright. 2021. {``The
Varieties of Coups D{'}état: Introducing the Colpus Dataset.''}
\emph{International Studies Quarterly} 65 (4): 1040--51.
\url{https://doi.org/10.1093/isq/sqab058}.

\bibitem[\citeproctext]{ref-diamond2008}
Diamond, Larry. 2008. \emph{The Spirit of Democracy: The Struggle to
Build Free Societies Throughout the World}. Macmillan.

\bibitem[\citeproctext]{ref-fariss2022}
Fariss, Christopher J., Therese Anders, Jonathan N. Markowitz, and
Miriam Barnum. 2022. {``New Estimates of Over 500 Years of Historic GDP
and Population Data.''} \emph{Journal of Conflict Resolution} 66 (3):
553--91. \url{https://doi.org/10.1177/00220027211054432}.

\bibitem[\citeproctext]{ref-frantz2016}
Frantz, Erica, and Elizabeth A. Stein. 2016. {``Countering Coups:
Leadership Succession Rules in Dictatorships.''} \emph{Comparative
Political Studies} 50 (7): 935--62.
\url{https://doi.org/10.1177/0010414016655538}.

\bibitem[\citeproctext]{ref-freedomhouse2024freedom}
Freedom House. 2024. {``Freedom in the World 2024.''}
\url{https://freedomhouse.org/sites/default/files/2024-02/FIW_2024_DigitalBooklet.pdf}.

\bibitem[\citeproctext]{ref-gassebner2016}
Gassebner, Martin, Jerg Gutmann, and Stefan Voigt. 2016. {``When to
Expect a Coup d{'}état? An Extreme Bounds Analysis of Coup
Determinants.''} \emph{Public Choice} 169 (3-4): 293--313.
\url{https://doi.org/10.1007/s11127-016-0365-0}.

\bibitem[\citeproctext]{ref-geddes1999}
Geddes, Barbara. 1999. {``What Do We Know About Democratization After
Twenty Years?''} \emph{Annual Review of Political Science} 2 (1):
115--44. \url{https://doi.org/10.1146/annurev.polisci.2.1.115}.

\bibitem[\citeproctext]{ref-geddes2014}
Geddes, Barbara, Joseph Wright, and Erica Frantz. 2014. {``Autocratic
Breakdown and Regime Transitions: A New Data Set.''} \emph{Perspectives
on Politics} 12 (2): 313--31.
\url{https://doi.org/10.1017/s1537592714000851}.

\bibitem[\citeproctext]{ref-ginsburg2019}
Ginsburg, Tom, and Zachary Elkins. 2019. {``One Size Does Not Fit
All.''} In, 37--52. Oxford University Press.
\url{https://doi.org/10.1093/oso/9780198837404.003.0003}.

\bibitem[\citeproctext]{ref-ginsburg2010evasion}
Ginsburg, Tom, James Melton, and Zachary Elkins. 2010. {``On the Evasion
of Executive Term Limits.''} \emph{Wm. \& Mary L. Rev.} 52: 1807.

\bibitem[\citeproctext]{ref-goemans2009}
Goemans, Henk E., Kristian Skrede Gleditsch, and Giacomo Chiozza. 2009.
{``Introducing Archigos: A Dataset of Political Leaders.''}
\emph{Journal of Peace Research} 46 (2): 269--83.
\url{https://doi.org/10.1177/0022343308100719}.

\bibitem[\citeproctext]{ref-huntington1991democratization}
Huntington, Samuel P. 1991. {``The Third Wave: Democratization in the
Late Twentieth Century.''} \emph{Norman, OK: University of Oklahoma}.

\bibitem[\citeproctext]{ref-krishnarajan2019}
Krishnarajan, Suthan. 2019. {``Economic Crisis, Natural Resources, and
Irregular Leader Removal in Autocracies.''} \emph{International Studies
Quarterly} 63 (3): 726--41. \url{https://doi.org/10.1093/isq/sqz006}.

\bibitem[\citeproctext]{ref-leon2013a}
Leon, Gabriel. 2013. {``Loyalty for Sale? Military Spending and Coups
d{'}etat.''} \emph{Public Choice} 159 (3-4): 363--83.
\url{https://doi.org/10.1007/s11127-013-0124-4}.

\bibitem[\citeproctext]{ref-marshall2005current}
Marshall, Monty G. 2005. {``Current Status of the World's Major Episodes
of Political Violence.''} \emph{Report to Political Instability Task
Force.(3 February)}.

\bibitem[\citeproctext]{ref-marsteintredet2019}
Marsteintredet, Leiv, and Andrés Malamud. 2019. {``Coup with Adjectives:
Conceptual Stretching or Innovation in Comparative Research?''}
\emph{Political Studies} 68 (4): 1014--35.
\url{https://doi.org/10.1177/0032321719888857}.

\bibitem[\citeproctext]{ref-mechkova2017}
Mechkova, Valeriya, Anna Lührmann, and Staffan I. Lindberg. 2017. {``How
Much Democratic Backsliding?''} \emph{Journal of Democracy} 28 (4):
162--69. \url{https://doi.org/10.1353/jod.2017.0075}.

\bibitem[\citeproctext]{ref-peyton2024}
Peyton, Buddy, Joseph Bajjalieh, Dan Shalmon, Michael Martin, and Emilio
Soto. 2024. {``Cline Center Coup d{'}état Project Dataset.''} University
of Illinois at Urbana-Champaign.
\url{https://doi.org/10.13012/B2IDB-9651987_V7}.

\bibitem[\citeproctext]{ref-pion-berlin2022}
Pion-Berlin, David, Thomas Bruneau, and Richard B. Goetze. 2022. {``The
Trump Self-Coup Attempt: Comparisons and Civil{\textendash}Military
Relations.''} \emph{Government and Opposition} 58 (4): 789--806.
\url{https://doi.org/10.1017/gov.2022.13}.

\bibitem[\citeproctext]{ref-powell2012}
Powell, Jonathan. 2012. {``Determinants of the Attempting and Outcome of
Coups d{'}état.''} \emph{Journal of Conflict Resolution} 56 (6):
1017--40. \url{https://doi.org/10.1177/0022002712445732}.

\bibitem[\citeproctext]{ref-powell2011}
Powell, Jonathan M, and Clayton L Thyne. 2011. {``Global Instances of
Coups from 1950 to 2010: A New Dataset.''} \emph{Journal of Peace
Research} 48 (2): 249--59.
\url{https://doi.org/10.1177/0022343310397436}.

\bibitem[\citeproctext]{ref-powell2018}
Powell, Jonathan, Christopher Faulkner, William Dean, and Kyle Romano.
2018. {``Give Them Toys? Military Allocations and Regime Stability in
Transitional Democracies.''} \emph{Democratization} 25 (7): 1153--72.
\url{https://doi.org/10.1080/13510347.2018.1450389}.

\bibitem[\citeproctext]{ref-roessler2011}
Roessler, Philip. 2011. {``The Enemy Within: Personal Rule, Coups, and
Civil War in Africa.''} \emph{World Politics} 63 (2): 300--346.
\url{https://doi.org/10.1017/s0043887111000049}.

\bibitem[\citeproctext]{ref-singh2016}
Singh, Naunihal. 2016. \emph{Seizing Power}. Johns Hopkins University
Press. \url{https://doi.org/10.1353/book.31450}.

\bibitem[\citeproctext]{ref-sudduth2017}
Sudduth, Jun Koga. 2017. {``Strategic Logic of Elite Purges in
Dictatorships.''} \emph{Comparative Political Studies} 50 (13):
1768--1801. \url{https://doi.org/10.1177/0010414016688004}.

\bibitem[\citeproctext]{ref-thyne2019}
Thyne, Clayton L., and Jonathan Powell. 2019. {``Coup Research,''}
October. \url{https://doi.org/10.1093/acrefore/9780190846626.013.369}.

\bibitem[\citeproctext]{ref-sampleSelection-2}
Toomet, Ott, and Arne Henningsen. 2008. {``Sample Selection Models in
{\textbraceleft}r{\textbraceright}: Package
{\textbraceleft}sampleSelection{\textbraceright}''} 27.
\url{https://www.jstatsoft.org/v27/i07/}.

\bibitem[\citeproctext]{ref-versteeg2020law}
Versteeg, Mila, Timothy Horley, Anne Meng, Mauricio Guim, and Marilyn
Guirguis. 2020. {``The Law and Politics of Presidential Term Limit
Evasion.''} \emph{Colum. L. Rev.} 120: 173.

\end{CSLReferences}



\end{document}

% Options for packages loaded elsewhere
\PassOptionsToPackage{unicode}{hyperref}
\PassOptionsToPackage{hyphens}{url}
\PassOptionsToPackage{dvipsnames,svgnames,x11names}{xcolor}
%
\documentclass[
  12pt,
]{report}

\usepackage{amsmath,amssymb}
\usepackage{setspace}
\usepackage{iftex}
\ifPDFTeX
  \usepackage[T1]{fontenc}
  \usepackage[utf8]{inputenc}
  \usepackage{textcomp} % provide euro and other symbols
\else % if luatex or xetex
  \usepackage{unicode-math}
  \defaultfontfeatures{Scale=MatchLowercase}
  \defaultfontfeatures[\rmfamily]{Ligatures=TeX,Scale=1}
\fi
\usepackage{lmodern}
\ifPDFTeX\else  
    % xetex/luatex font selection
    \setmainfont[]{Times New Roman}
    \setsansfont[]{Arial}
    \setmonofont[]{Courier New}
\fi
% Use upquote if available, for straight quotes in verbatim environments
\IfFileExists{upquote.sty}{\usepackage{upquote}}{}
\IfFileExists{microtype.sty}{% use microtype if available
  \usepackage[]{microtype}
  \UseMicrotypeSet[protrusion]{basicmath} % disable protrusion for tt fonts
}{}
\usepackage{xcolor}
\usepackage[top = 3cm,bottom = 3cm,left = 3cm,right = 2.7cm]{geometry}
\setlength{\emergencystretch}{3em} % prevent overfull lines
\setcounter{secnumdepth}{5}
% Make \paragraph and \subparagraph free-standing
\makeatletter
\ifx\paragraph\undefined\else
  \let\oldparagraph\paragraph
  \renewcommand{\paragraph}{
    \@ifstar
      \xxxParagraphStar
      \xxxParagraphNoStar
  }
  \newcommand{\xxxParagraphStar}[1]{\oldparagraph*{#1}\mbox{}}
  \newcommand{\xxxParagraphNoStar}[1]{\oldparagraph{#1}\mbox{}}
\fi
\ifx\subparagraph\undefined\else
  \let\oldsubparagraph\subparagraph
  \renewcommand{\subparagraph}{
    \@ifstar
      \xxxSubParagraphStar
      \xxxSubParagraphNoStar
  }
  \newcommand{\xxxSubParagraphStar}[1]{\oldsubparagraph*{#1}\mbox{}}
  \newcommand{\xxxSubParagraphNoStar}[1]{\oldsubparagraph{#1}\mbox{}}
\fi
\makeatother


\providecommand{\tightlist}{%
  \setlength{\itemsep}{0pt}\setlength{\parskip}{0pt}}\usepackage{longtable,booktabs,array}
\usepackage{calc} % for calculating minipage widths
% Correct order of tables after \paragraph or \subparagraph
\usepackage{etoolbox}
\makeatletter
\patchcmd\longtable{\par}{\if@noskipsec\mbox{}\fi\par}{}{}
\makeatother
% Allow footnotes in longtable head/foot
\IfFileExists{footnotehyper.sty}{\usepackage{footnotehyper}}{\usepackage{footnote}}
\makesavenoteenv{longtable}
\usepackage{graphicx}
\makeatletter
\def\maxwidth{\ifdim\Gin@nat@width>\linewidth\linewidth\else\Gin@nat@width\fi}
\def\maxheight{\ifdim\Gin@nat@height>\textheight\textheight\else\Gin@nat@height\fi}
\makeatother
% Scale images if necessary, so that they will not overflow the page
% margins by default, and it is still possible to overwrite the defaults
% using explicit options in \includegraphics[width, height, ...]{}
\setkeys{Gin}{width=\maxwidth,height=\maxheight,keepaspectratio}
% Set default figure placement to htbp
\makeatletter
\def\fps@figure{htbp}
\makeatother
% definitions for citeproc citations
\NewDocumentCommand\citeproctext{}{}
\NewDocumentCommand\citeproc{mm}{%
  \begingroup\def\citeproctext{#2}\cite{#1}\endgroup}
\makeatletter
 % allow citations to break across lines
 \let\@cite@ofmt\@firstofone
 % avoid brackets around text for \cite:
 \def\@biblabel#1{}
 \def\@cite#1#2{{#1\if@tempswa , #2\fi}}
\makeatother
\newlength{\cslhangindent}
\setlength{\cslhangindent}{1.5em}
\newlength{\csllabelwidth}
\setlength{\csllabelwidth}{3em}
\newenvironment{CSLReferences}[2] % #1 hanging-indent, #2 entry-spacing
 {\begin{list}{}{%
  \setlength{\itemindent}{0pt}
  \setlength{\leftmargin}{0pt}
  \setlength{\parsep}{0pt}
  % turn on hanging indent if param 1 is 1
  \ifodd #1
   \setlength{\leftmargin}{\cslhangindent}
   \setlength{\itemindent}{-1\cslhangindent}
  \fi
  % set entry spacing
  \setlength{\itemsep}{#2\baselineskip}}}
 {\end{list}}
\usepackage{calc}
\newcommand{\CSLBlock}[1]{\hfill\break\parbox[t]{\linewidth}{\strut\ignorespaces#1\strut}}
\newcommand{\CSLLeftMargin}[1]{\parbox[t]{\csllabelwidth}{\strut#1\strut}}
\newcommand{\CSLRightInline}[1]{\parbox[t]{\linewidth - \csllabelwidth}{\strut#1\strut}}
\newcommand{\CSLIndent}[1]{\hspace{\cslhangindent}#1}

\usepackage{booktabs}
\usepackage{caption}
\usepackage{longtable}
\usepackage{colortbl}
\usepackage{array}
\usepackage{anyfontsize}
\usepackage{multirow}
\usepackage{sectsty}
\chapterfont{\centering}
\usepackage{lscape}
\newcommand{\blandscape}{\begin{landscape}}
\newcommand{\elandscape}{\end{landscape}}
\makeatletter
\@ifpackageloaded{caption}{}{\usepackage{caption}}
\AtBeginDocument{%
\ifdefined\contentsname
  \renewcommand*\contentsname{Table of contents}
\else
  \newcommand\contentsname{Table of contents}
\fi
\ifdefined\listfigurename
  \renewcommand*\listfigurename{Figures}
\else
  \newcommand\listfigurename{Figures}
\fi
\ifdefined\listtablename
  \renewcommand*\listtablename{Tables}
\else
  \newcommand\listtablename{Tables}
\fi
\ifdefined\figurename
  \renewcommand*\figurename{Figure}
\else
  \newcommand\figurename{Figure}
\fi
\ifdefined\tablename
  \renewcommand*\tablename{Table}
\else
  \newcommand\tablename{Table}
\fi
}
\@ifpackageloaded{float}{}{\usepackage{float}}
\floatstyle{ruled}
\@ifundefined{c@chapter}{\newfloat{codelisting}{h}{lop}}{\newfloat{codelisting}{h}{lop}[chapter]}
\floatname{codelisting}{Listing}
\newcommand*\listoflistings{\listof{codelisting}{List of Listings}}
\makeatother
\makeatletter
\makeatother
\makeatletter
\@ifpackageloaded{caption}{}{\usepackage{caption}}
\@ifpackageloaded{subcaption}{}{\usepackage{subcaption}}
\makeatother

\ifLuaTeX
  \usepackage{selnolig}  % disable illegal ligatures
\fi
\usepackage{bookmark}

\IfFileExists{xurl.sty}{\usepackage{xurl}}{} % add URL line breaks if available
\urlstyle{same} % disable monospaced font for URLs
\hypersetup{
  colorlinks=true,
  linkcolor={blue},
  filecolor={Maroon},
  citecolor={Blue},
  urlcolor={blue},
  pdfcreator={LaTeX via pandoc}}


\author{}
\date{}

\begin{document}

\begin{titlepage}
  \begin{center}
    \vspace*{2cm}
    
    \Huge{\textbf{Leadership Transitions and Survival: Coups, Autocoups, and Power Dynamics}}
    
    \vspace{1.5cm}
    
    \Large{Zhu Qi}
    
    \vspace{5cm}
    
    \large{A thesis submitted for the degree of \\ Doctor of Philosophy in Political Science}
    
    \vspace{0.8cm}
    
    \large{Department of Government}
    \vspace{0.5cm}
    
    \large{University of Essex}
    
    \vspace{1.5cm}
    
    \large{September 2024}
    \vspace{2cm}
    
    
  \end{center}
\end{titlepage}

\renewcommand*\contentsname{Contents}
{
\hypersetup{linkcolor=}
\setcounter{tocdepth}{2}
\tableofcontents
}
\listoffigures
\listoftables

\setstretch{1.618}
\chapter*{Acknowledgements}\label{acknowledgements}
\addcontentsline{toc}{chapter}{Acknowledgements}

The completion of this thesis marks the culmination of a remarkable
journey, filled with dedication, perseverance, and moments of profound
joy. I am deeply grateful to the numerous individuals who have supported
and encouraged me throughout this endeavour.

I would like to express my sincerest appreciation to my supervisor,
Professor Kristian Skrede Gleditsch, whose guidance, expertise, and
unwavering support have been instrumental in shaping my research. His
constructive feedback and encouragement have been invaluable, and I am
profoundly grateful for his mentorship.

I am also grateful to Professor Han Dorussen, the chair of my board
panel, for his continuous support and thoughtful input. His insightful
comments and suggestions have significantly enhanced the quality and
depth of my research.

I would like to acknowledge the important contributions of my initial
co-supervisors, Dr.~Saurabh Pant and Professor David Siroky, who laid a
strong foundation for this work during the early stages of my research.
Although they are no longer at the University of Essex, their
instruction and guidance were instrumental in shaping the direction of
this project.

I have been fortunate to receive feedback and guidance from several
esteemed scholars in the field, including Dr.~Brian J Phillips,
Dr.~Prabin Khadka, and Dr. Winnie Xia. Their expertise and insights have
enriched this research, and I am grateful for their contributions.

On a personal note, I would like to express my deepest gratitude to my
family, who have been a constant source of support and inspiration
throughout this journey. To my beloved wife, Ji Zhi, your patience,
love, and encouragement have been immeasurable. To my dear children,
Siyan and Sisheng, your joy and curiosity have motivated me to persevere
and strive for excellence.

I am also deeply grateful to my father for his enduring support and
belief in my abilities. To the cherished memory of my late mother, your
love, guidance, and values continue to shape my path and inspire my
endeavors. And to my three brothers, whose support enabled me to pursue
my PhD without worries, I am forever grateful.

While many individuals have contributed to the success of this work, I
take full responsibility for any errors or shortcomings that may remain.

\chapter*{Abstract}\label{abstract}
\addcontentsline{toc}{chapter}{Abstract}

This thesis undertakes a comprehensive examination of the dynamics of
irregular leadership transitions, focusing on coups and autocoups, to
understand their impact on leader survival and democratic resilience.
Utilizing innovative analytical techniques and a novel comparative
framework, this study investigates the factors influencing the
likelihood and frequency of these events, as well as their consequences
for democratic erosion and political stability.

The research first explores the key determinants of classic coups,
emphasizing the critical balance of power between incumbent leaders and
coup plotters. Employing a double probit model with sample selection,
the analysis reveals that the anticipated probability of coup success is
a significant driver of coup attempts. Notably, the findings show that
military regimes are approximately 277.7\% more likely, and personalist
regimes 94\% more likely, to experience coups compared to dominant-party
regimes, all else being equal. This heightened vulnerability of military
regimes can be attributed to the inherent power structures within these
types of political systems.

The study subsequently delves into the understudied phenomenon of
autocoups, where incumbents manipulate institutional structures to
extend their tenure beyond the originally mandated limits. I propose a
refined definition of autocoups and develop a novel dataset spanning
from 1945 to 2023. Through a combination of case studies and empirical
analysis, I demonstrate how this refined definition and dataset can
facilitate nuanced comparisons between traditional coups and autocoups.

Finally, using survival analysis, this study examines the relationship
between methods of power acquisition and leadership tenure. The findings
reveal that, on average, coup-installed leaders are 2.23 times more
likely to be ousted from power than autocoup leaders, all else being
equal. This disparity suggests that autocoups may serve as a catalyst
for both the personalization of power and the erosion of democratic
norms and institutions.

This thesis contributes to political science through:

\begin{itemize}
\item
  A precise definition of autocoups, addressing a critical literature
  gap.
\item
  A new longitudinal dataset on autocoups, enabling robust quantitative
  analyses.
\item
  A unified approach for studying coups and autocoups, offering
  comparative insights.
\end{itemize}

These findings have significant implications for both scholars and
policy-makers concerned with smooth power transitions and democratic
stability. Future research could explore strategies to bolster
democratic institutions against erosion and investigate the long-term
impacts of coups and autocoups on democratization and broader political
stability.

\emph{\textbf{Keywords:} Coups, Autocoups, Leadership transitions,
Leadership survival, Democratic resilience}

\chapter{Introduction}\label{introduction}

A seamless leadership transition is crucial in political life, as it
underpins the stability of the political system to a significant extent.
Conversely, social unrest and violence often erupt when conventional
leadership transitions are disrupted. Furthermore, unorthodox leadership
transfers can have profound and lasting consequences for political
systems, democratic resilience, and overall societal stability. As a
result, irregular leadership transitions have been a crucial research
focus within political science.

Among the various forms of irregular leadership transitions,
\textbf{coups and autocoups} stand out as the two most prevalent and
impactful types. This thesis seeks to provide a comprehensive analysis
of the dynamics of coups and autocoups, as well as their far-reaching
implications for leadership survival, regime stability, and democratic
resilience. By examining these phenomena through a unified theoretical
framework, this research offers a nuanced understanding of the intricate
interplay between power dynamics, institutional structures, and
political outcomes in contemporary governance.

\section{Motivation and research
questions}\label{motivation-and-research-questions}

This research is driven by a fundamental question at the heart of
contemporary political dynamics: \textbf{What factors determine whether
political leaders are prematurely ousted from power or manage to extend
their tenure beyond constitutionally prescribed limits?} This inquiry is
significant because irregular leadership transitions inevitably disrupt
the democratic process, often consolidating the power of autocoup or
coup-installed leaders through extra-legal means, and consequently
contributing to democratic erosion and the resurgence of
authoritarianism.

The relevance of this investigation is particularly acute in the current
global landscape. According to Freedom House's Freedom in the World 2024
report, global freedom has declined for the 18th consecutive year as of
2023, with political rights and civil liberties diminishing in 52
countries and improvements observed in only 21
(\citeproc{ref-freedomhouse2024freedom}{Freedom House 2024}). This
persistent global decline in freedom underscores the urgent need to
comprehend the factors determining irregular leadership transitions and
their far-reaching implications.

To address these complex issues, this study focuses on three primary
research questions, which form the foundation for the subsequent
chapters:

\begin{itemize}
\item
  \textbf{Coup Dynamics and Regime Types:} How do power dynamics and
  regime types influence the likelihood and success of coup attempts?
  This question seeks to unravel the intricate relationship between
  existing political structures and the propensity for forceful
  leadership changes.
\item
  \textbf{Comparative Analysis of Autocoups:} How can autocoups be
  effectively analysed within a comparative framework alongside classic
  coups? This inquiry seeks to expand the analysis of irregular power
  transitions by comparing autocoups to classic coups.
\item
  \textbf{Power Acquisition and Leadership Longevity:} Does the method
  of power acquisition, specifically whether through a coup or an
  autocoup, have a significant impact on leadership longevity? This
  question explores the long-term consequences of different modes of
  irregular power transitions on political stability and leadership
  tenure.
\end{itemize}

\section{Comparative framework: coups and
autocoups}\label{comparative-framework-coups-and-autocoups}

Coups have traditionally dominated the academic discourse on irregular
leadership transitions, owing to their frequency and visibility. The
well-established definition of coups---the forcible removal of incumbent
leaders by elites within the existing power structure---has facilitated
the development of comprehensive datasets and rigorous quantitative
analyses.

However, another form of irregular leadership transition has been
largely overlooked: instances where incumbent leaders refuse to
relinquish power and extend their mandated terms. This phenomenon, which
is termed ``autocoup'' in this thesis, has received comparatively less
attention, resulting in a lack of consensus on terminology, definition,
and data collection.

To address this gap, this study proposes a unified framework for
analysing both coups and autocoups. By conceptualizing these as related
yet distinct forms of irregular leadership transitions, we can develop a
more nuanced understanding of the factors that shape leadership survival
following unconstitutional changes in power. This comparative approach
is crucial for several reasons:

\begin{itemize}
\item
  \textbf{Significance and Frequency}: Both coups and autocoups are the
  most prevalent means of irregular leadership transitions and have a
  profound impact on democratic resilience. Understanding their dynamics
  is essential for assessing the stability and sustainability of
  democratic systems.
\item
  \textbf{Conceptual Similarities}: Notwithstanding their distinct
  characteristics, coups and autocoups share a crucial commonality: both
  involve the manipulation of power through illegitimate means to
  subvert established political processes. However, their objectives and
  targets differ significantly. A coup typically seeks to forcibly
  remove the incumbent leader, abruptly truncating their tenure. In
  contrast, an autocoup is launched by the incumbent leader against the
  legal successor or the transition process itself, thereby prolonging
  the leader's tenure.
\item
  \textbf{Comprehensive Analysis}: Integrating these phenomena within a
  unified analytical framework can contribute to a more comprehensive
  understanding of leadership transition, leadership longevity, and
  their impact on political systems across various regime types.
\end{itemize}

\section{Research objectives and
contributions}\label{research-objectives-and-contributions}

This study addresses a critical gap in the literature by presenting a
unified framework for analysing coups and autocoups. The key
contributions of this research are threefold, aiming to enhance our
understanding of irregular leadership transitions and their impact on
political systems.

\begin{itemize}
\item
  \textbf{Emphasis the impact of regime types on coup attempts:} This
  study examines the pivotal factor influencing the success or failure
  of coup attempts: the power dynamics between incumbent leaders and
  those seeking to usurp them. By focusing on this dynamic, the study
  highlights the critical role regime type plays in shaping coup risks.
\item
  \textbf{Refined definition and novel dataset for autocoups:} This
  study refines the definition of autocoups and develops a novel dataset
  encompassing events from 1945 to 2023. This addresses a significant
  gap in the existing literature and facilitates a comparative analysis
  between autocoups and classic coups.
\item
  \textbf{Survival analysis of leaders from different entry modes:}
  Employing survival analysis on existing coup data and the newly
  compiled autocoup dataset, this research demonstrates how different
  modes of power acquisition significantly impact leadership survival.
\end{itemize}

These contributions collectively advance the field by providing a more
holistic understanding of irregular leadership transitions, offering new
tools and data for quantitative analysis of autocoups, and demonstrating
the interconnectedness of power acquisition methods and leadership
survival.

\section{Policy implications}\label{policy-implications}

The findings of this study offer critical insights into the
interconnected phenomena of democratic backsliding, breakdown, and
autocratic intensification, providing a nuanced understanding of the
complex dynamics that underlie these trends.

\begin{itemize}
\item
  \textbf{Regression of global democracy levels}: The examination of
  irregular leadership transitions and leadership survival sheds light
  on the alarming decline in global freedom levels. This regression can
  be attributed, in part, to the proliferation of coups and autocoups,
  which inevitably violate democratic norms and disrupt the trajectory
  towards stable democracies.
\item
  \textbf{Within-regime democratic erosion}: This research elucidates
  the phenomenon of democratic backsliding within regimes
  (\citeproc{ref-mechkova2017}{Mechkova, Lührmann, and Lindberg 2017}),
  where democracies become increasingly illiberal and autocracies less
  competitive. The prevalence of autocoups since 2000
  (\citeproc{ref-bermeo2016}{Bermeo 2016}) has contributed significantly
  to this trend, as incumbents exploit their strategic advantages to
  extend their rule and undermine democratic institutions.
\item
  \textbf{Prevalence of autocoups since 2000}: The analyses in this
  study explain the rise of autocoups since 2000, highlighting the
  strategic advantages that incumbents possess, including a higher
  probability of success and relatively milder consequences in the event
  of failure. Furthermore, leaders who manage to extend their rule
  through an autocoup enjoy significantly longer tenures compared to
  those who come to power through a coup.
\item
  \textbf{Addressing the challenge}: Given the difficulties of mounting
  internal challenges against autocoups, external pressure from regional
  or international communities may play a crucial role in promoting
  democratic accountability and preventing the erosion of democratic
  norms.
\end{itemize}

\section{Limitations and future
research}\label{limitations-and-future-research}

While this study provides a novel framework for analysing irregular
leadership transitions and survival, several limitations underscore
opportunities for future research:

\begin{itemize}
\item
  \textbf{Refining the autocoup framework:} The attempt to define and
  classify autocoups represents a pioneering effort in this field.
  Nevertheless, both the definition and the dataset can be further
  refined through additional quantitative studies that critically
  evaluate and analyze the existing dataset. Future research should
  prioritize refining these definitions to enhance clarity,
  applicability, and generalizability, ultimately strengthening the
  framework's explanatory power.
\item
  \textbf{Addressing data harmonization challenges:} The current
  analysis is constrained by the mismatch between the units of analysis
  (country-year versus leader tenure) in the coup and autocoup datasets.
  To overcome this limitation, future research should explore data
  harmonization techniques, such as data fusion or calibration methods,
  to facilitate more robust comparisons and address these discrepancies.
  By doing so, researchers can enhance the accuracy and reliability of
  analyses concerning leadership longevity.
\item
  \textbf{Unpacking the link between irregular transitions and
  democratic backsliding:} Although this study suggests a connection
  between irregular leadership transitions and democratic backsliding,
  further empirical evidence is needed to solidify this link. Future
  research should prioritize longitudinal studies that can track the
  long-term impacts of coups and autocoups on democratic health across
  various regimes. Additionally, in-depth case studies of democratic
  erosion following irregular power transitions could provide valuable
  insights into the dynamics of this relationship, shedding light on the
  underlying mechanisms driving democratic decline.
\end{itemize}

\section{Overview of the thesis}\label{overview-of-the-thesis}

This thesis delves into the intricate dynamics of irregular leadership
transitions, examining their implications for leadership survival and
democratic processes. The study focuses on three key areas: the
determinants of classic coup attempts, the conceptualization and
analysis of autocoups, and the impact of power acquisition methods on
leadership longevity.

\subsection*{Chapter 2: Determinants of classic coup
attempts}\label{chapter-2-determinants-of-classic-coup-attempts}
\addcontentsline{toc}{subsection}{Chapter 2: Determinants of classic
coup attempts}

While coups have garnered significant scholarly attention, particularly
since the early 2000s (\citeproc{ref-thyne2019}{Thyne and Powell 2019}),
a consensus on the key factors driving coup attempts and their outcomes
remains elusive despite extensive research
(\citeproc{ref-gassebner2016}{Gassebner, Gutmann, and Voigt 2016}). This
chapter addresses this ongoing debate by shifting the focus from
pre-conditions to the strategic calculations of coup plotters.

Prior literature has emphasized the willingness and conditions necessary
for launching a coup, yet similar conditions in different contexts may
encounter varying levels of resistance due to differences in the
incumbent's authority and power. Therefore, coup plotters must consider
not only their own resources and conditions but also the strength of
their opponent, potential reactions from other political forces, and the
overall balance of power.

This chapter posits that coup plotters engage in a cost-benefit
analysis, evaluating the balance of power against the incumbent and
calculating their chances of success before staging a coup. Given the
severe consequences of failure, they are unlikely to act unless success
is highly probable. To operationalize this concept, the study utilizes
regime types as a proxy for the balance of power between coup
perpetrators and incumbent leaders. This approach stems from the
understanding that regime types are inherently defined by power
structures and dynamics (\citeproc{ref-geddes2014}{Geddes, Wright, and
Frantz 2014}).

Acknowledging that coup attempts are not random events but rather
strategically initiated by plotters, this chapter applies a double
probit model with sample selection, as used by J. Powell
(\citeproc{ref-powell2012}{2012}) and Böhmelt and Pilster
(\citeproc{ref-buxf6hmelt2014}{2014}), to address inherent selection
bias. This advanced approach allows for a more precise evaluation of the
factors that influence both the occurrence of coup attempts and their
likelihood of success. The analysis demonstrates that the anticipated
probability of success is a decisive factor in coup planning, providing
robust evidence for the role of power dynamics in shaping such
decisions. Importantly, the findings reveal that military regimes are
approximately 277.7\% more likely, and personalist regimes 94\% more
likely, to experience coup attempts compared to dominant-party regimes,
all other factors being equal.

\subsection*{Chapter 3: Conceptualising and analysing
autocoups}\label{chapter-3-conceptualising-and-analysing-autocoups}
\addcontentsline{toc}{subsection}{Chapter 3: Conceptualising and
analysing autocoups}

While autocoups have gained increased scholarly attention since the
2000s, previous research has been hindered by conceptual ambiguities and
a lack of systematic data. This chapter seeks to address the significant
gaps in the literature by refining the concept of autocoups and
developing a comprehensive dataset for quantitative analysis.

This chapter seeks to address two significant limitations in the
existing literature on autocoups. Firstly, the conflation of power
expansion and power extension has led to conceptual ambiguity, with
autocoups being used to describe both the consolidation of power by
incumbents and the extension of their tenure. Secondly, the definition
of autocoups has not been aligned with that of classic coups, despite
being a distinct subtype. Classic coups are uniformly defined as the
unconstitutional removal of an incumbent leader, with a clear focus on
the termination of their tenure. In contrast, autocoups often involve
the consolidation of power by incumbent leaders through the seizure of
control from other state institutions, rather than solely focusing on
the extension of their tenure. To bridge this analytical gap, it is
essential to integrate the study of autocoups with that of classic
coups, thereby establishing a more nuanced and comprehensive
understanding of the complex dynamics of power seizures.

This study proposes a more precise and theoretically consistent
definition of autocoups: \textbf{the illegitimate extension of an
incumbent leader's term in office beyond the originally mandated limits
through unconstitutional means.} By focusing on power extension, this
definition clearly distinguishes autocoups from other forms of power
consolidation while maintaining consistency with the concept of classic
coups, which inherently involve removing an incumbent from power.

Building on this refined conceptualization, the chapter introduces a
novel dataset encompassing autocoup events from 1945 to 2023, recording
110 attempts and 87 successes. This comprehensive dataset represents a
significant contribution to the field, enabling more robust quantitative
analysis of autocoups and their consequences. The chapter also
demonstrates the dataset's utility through case studies and empirical
analyses, highlighting the role of autocoups in driving democratic
backsliding and power personalization.

\subsection*{Chapter 4: Impact of power acquisition methods on
leadership
longevity}\label{chapter-4-impact-of-power-acquisition-methods-on-leadership-longevity}
\addcontentsline{toc}{subsection}{Chapter 4: Impact of power acquisition
methods on leadership longevity}

Building upon the analyses of classic coups and autocoups established in
previous chapters, this chapter presents a comparative analysis of these
two forms of irregular power transitions within a unified framework. The
focus is on examining the impact of power acquisition methods on
leadership survival, comparing the tenure of leaders installed by
successful coups against the extended tenures of autocoup leaders.

This chapter argues that leaders installed via autocoups and coups face
distinct challenges in consolidating their power, primarily due to
differences in the intensity of issues related to illegitimacy,
uncertainty, and instability. These disparities create an uneven playing
field in terms of power dynamics, with coup-installed leaders at a
significant disadvantage. Consequently, the chapter argues that autocoup
leaders are more likely to survive longer in their extended tenure
compared to coup-installed leaders.

To test this hypothesis, the study employs sophisticated statistical
techniques, including Cox proportional hazards model and time-dependent
Cox model. These methods provide a nuanced examination of how different
modes of power acquisition affect leadership longevity. The findings
support the central argument, revealing a significant impact of power
acquisition methods on leader tenure. Specifically, the results
demonstrate that, on average, coup-installed leaders are 2.23 times more
likely to be ousted from power than autocoup leaders, all else being
equal.

This insight contributes significantly to our understanding of the
strategic incentives for different forms of irregular power transitions
and their long-term consequences for political stability. By elucidating
the relationship between power acquisition methods and leadership
survival, this chapter offers valuable perspectives on the dynamics of
authoritarian persistence and the challenges of democratic consolidation
in regimes that have experienced irregular power transitions.

\subsection*{Chapter 5: Conclusion and future research
directions}\label{chapter-5-conclusion-and-future-research-directions}
\addcontentsline{toc}{subsection}{Chapter 5: Conclusion and future
research directions}

This concluding chapter synthesizes the key findings from each preceding
chapter, weaving together the insights gleaned from the analyses of
coups, autocoups, and leadership survival. It reflects on the broader
implications of these findings for understanding the interconnected
challenges of democratic backsliding, power personalization, and
political stability across diverse regime types.

The study's findings underscore the complex interplay between irregular
power transitions, leadership tenure, and regime stability. By
demonstrating the distinct dynamics and consequences of coups and
autocoups, this research provides valuable insights into the motivations
of political actors seeking to seize or maintain power through
extra-constitutional means. Moreover, by emphasizing the substantial
differences in leadership survival rates between these two forms of
power acquisition, this study sheds light on the factors that contribute
to the persistence or collapse of authoritarian regimes.

Recognizing the limitations inherent in any single study, this chapter
also outlines promising avenues for future research. It emphasizes the
need for continued exploration of irregular power transitions,
particularly in light of evolving global political dynamics
characterized by rising populism, democratic decline, and technological
advancements that empower both authoritarian actors and pro-democracy
movements.

By highlighting the significance of this research for understanding
contemporary challenges to democratic governance, the conclusion
underscores the broader relevance of the study's findings for both
scholars and policy-makers concerned with promoting political stability,
democratic resilience, and accountable governance.

\chapter{Power Dynamics and Coup Attempts: A Selection Mechanism
Analysis}\label{sec-chapter2}

\section*{Abstract}\label{abstract-1}
\addcontentsline{toc}{section}{Abstract}

This chapter examines coup attempts by focusing on the expected
probability of success as a key driver of coup decisions. Using a double
probit model with sample selection, the study investigates the
relationship between regime types and coup attempts, focusing on the
balance of power between coup plotters and incumbent leaders. The
findings reveal that regime type is a significant determinant of coup
attempts. Notably, compared to dominant-party regimes, military regimes
are approximately 277.7\% more likely, and personalist regimes are 94\%
more likely to experience coups, all else being equal. This heightened
vulnerability can be attributed to the weaker institutional frameworks
and internal power struggles characteristic of these regimes.

\emph{\textbf{Keywords:} Coups, Leadership transitions, Regime types,}
\emph{Sample selection}

\newpage

\section{Introduction}\label{introduction-1}

Coups d'état, defined as ``illegal and overt attempts by the military or
other elites within the state apparatus to unseat the sitting
executive'' (\citeproc{ref-powell2011}{Powell and Thyne 2011}),
represent a critical threat to constitutional leadership transitions and
political stability. The frequency and success of these attempts vary
significantly across countries and regions, prompting essential
questions about \textbf{why coups are more prevalent in certain
countries and why some coups succeed while others fail}.

For instance, according to the Global Instances of Coups dataset
(\citeproc{ref-powell2011}{Powell and Thyne 2011}) (GIC), Latin America
and Africa have seen notable variation in coup activity. Bolivia
experienced 23 coups between 1950 and 1984, and Argentina saw 20 coups
during a similar period. In Africa, Sudan endured 17 coups between 1955
and 2023. In stark contrast, countries such as Mexico and South Africa
have remained coup-free since 1950.

Despite decades of scholarly inquiry and the identification of over 100
potential determinants (\citeproc{ref-gassebner2016}{Gassebner, Gutmann,
and Voigt 2016}), a consensus on the key factors driving coup attempts
and their outcomes remains elusive. The proliferation of variables
presents a significant challenge: Is there a more effective analytical
framework that prioritizes the most relevant factors of coups, rather
than sifting through an ever-expanding list of possibilities?

This chapter introduces a novel perspective on understanding coup
dynamics by shifting the focus from pre-coup conditions to the strategic
calculations of potential coup plotters. I contend that the anticipated
probability of success is a crucial determinant in the decision to
initiate a coup. This argument is substantiated by several key
observations:

\begin{itemize}
\item
  \textbf{High stakes of coups}: Failed coup attempts often result in
  severe consequences for participants, including imprisonment, exile,
  or even death. This underscores the critical importance of success for
  potential plotters.
\item
  \textbf{Selectivity of Coup Attempts}: Despite 491 recorded attempts
  since 1950, these represent a small fraction (approximately 4\%) of
  over 12,000 country-years during the same period (GIC). This suggests
  that coups are not impulsive acts but rather carefully calculated
  gambles.
\item
  \textbf{Relatively high success rates}: Approximately half of all coup
  attempts succeed (GIC). This further suggests that plotters are
  selective, choosing to act only when they perceive a significant
  likelihood of success.
\end{itemize}

\begin{longtable}[]{@{}
  >{\raggedright\arraybackslash}p{(\columnwidth - 6\tabcolsep) * \real{0.2500}}
  >{\centering\arraybackslash}p{(\columnwidth - 6\tabcolsep) * \real{0.2500}}
  >{\centering\arraybackslash}p{(\columnwidth - 6\tabcolsep) * \real{0.2500}}
  >{\raggedleft\arraybackslash}p{(\columnwidth - 6\tabcolsep) * \real{0.2500}}@{}}

\caption{\label{tbl-coups}Top 10 countries with the most coup attempts}

\tabularnewline

\toprule\noalign{}
\begin{minipage}[b]{\linewidth}\raggedright
Country
\end{minipage} & \begin{minipage}[b]{\linewidth}\centering
Coup Attempted
\end{minipage} & \begin{minipage}[b]{\linewidth}\centering
Coup Succeeded
\end{minipage} & \begin{minipage}[b]{\linewidth}\raggedleft
Success Rate
\end{minipage} \\
\midrule\noalign{}
\endhead
\midrule\noalign{}
\multicolumn{4}{@{}>{\raggedright\arraybackslash}p{(\columnwidth - 6\tabcolsep) * \real{1.0000} + 6\tabcolsep}@{}}{%
\begin{minipage}[t]{\linewidth}\raggedright
\emph{Source: GIC dataset}
\end{minipage}} \\
\bottomrule\noalign{}
\endlastfoot
Bolivia & 23 & 11 & 47.8\% \\
Argentina & 20 & 7 & 35.0\% \\
Sudan & 17 & 6 & 35.3\% \\
Haiti & 13 & 9 & 69.2\% \\
Venezuela & 13 & 0 & 0.0\% \\
Iraq & 12 & 4 & 33.3\% \\
Syria & 12 & 8 & 66.7\% \\
Thailand & 12 & 8 & 66.7\% \\
Ecuador & 11 & 5 & 45.5\% \\
Burundi & 11 & 5 & 45.5\% \\
Guatemala & 10 & 5 & 50.0\% \\
Total & 491 & 245 & 49.9\% \\

\end{longtable}

Given the difficulty in directly observing the probability of coup
success, this study proposes using regime type as a crucial proxy for
predicting coup outcomes. The underlying premise is that the balance of
power within a regime, which is significantly influenced by its type,
ultimately shapes the outcomes of coups. By analysing power dynamics
across various regime types, we can gain more profound insights into the
structural factors that affect both the likelihood of coup attempts and
their success.

To address the inherent selection bias in studying coup attempts, this
research employs a double probit model with sample selection, building
on the methodological approaches used by J. Powell
(\citeproc{ref-powell2012}{2012}) and Böhmelt and Pilster
(\citeproc{ref-buxf6hmelt2014}{2014}). This method enables the
simultaneous analysis of factors influencing both the initiation and
success of coups.

This study aims to contribute to scholarly discourse and practical
efforts in two key ways:

\begin{itemize}
\item
  \textbf{Re-framing coup dynamics through the lens of expected
  success}: By emphasizing the expected chances of success as a driver
  of coup attempts, this research offers a more targeted approach to
  understanding coup dynamics. Specifically, it highlights the selective
  effect of anticipated outcomes on coup initiation, providing a more
  nuanced understanding of the strategic calculations underlying coup
  attempts.
\item
  \textbf{Illuminating the role of regime type in shaping coup
  outcomes}: In the absence of perfect knowledge of internal power
  balances, this study leverages regime types as a proxy for power
  structures. By examining the relationship between regime type and
  anticipated coup outcomes, this research sheds light on how these
  structures influence the likelihood of coup success, thereby informing
  strategies for coup prevention and mitigation.
\end{itemize}

The remainder of this chapter is structured as follows: Section 2 delves
into the complexities of coup attempts and their outcomes, providing a
nuanced examination of the underlying dynamics. Section 3 outlines the
research design, methodology, and variables employed in this study,
providing a transparent and detailed account of the analytical
framework. Section 4 presents and interprets the empirical findings,
highlighting the key patterns and trends that emerge from the data.
Section 5 concludes the chapter by distilling the main insights and
implications of the study, and exploring their potential applications in
understanding and mitigating coup risks.

\section{Dynamics of coup attempts and
outcomes}\label{dynamics-of-coup-attempts-and-outcomes}

Coup attempts represent a critical juncture in the political landscape
of regimes, driven by a complex interplay of factors that can be broadly
categorized into two main components: the \textbf{disposition} of
potential challengers (their motivations and willingness to act) and
their \textbf{capability} (the resources and opportunities available to
them). This analysis delves into the intricate dynamics of coup
attempts, examining the underlying motivations, factors influencing
their success, and the role of regime types in shaping a country's
susceptibility to coups.

\subsection{Motivations for coups}\label{motivations-for-coups}

Coup plotters are driven by a range of motivations, which can be
distilled into three primary categories: personal ambition, purported
national interest, and self-preservation.

\begin{itemize}
\item
  \textbf{Personal ambition:} The allure of absolute power is a potent
  motivator for many coup plotters. The prospect of seizing control
  offers the ability to shape national policies without constraint,
  command significant resources and wealth, make impactful decisions
  that affect millions, and bask in the prestige and recognition that
  comes with holding power. The opportunity to leave a lasting
  historical legacy is also a powerful draw.
\item
  \textbf{Purported national interest:} Coup leaders often justify their
  actions as necessary interventions for the greater good of the nation.
  While such claims should be scrutinized, some examples demonstrate
  genuine attempts to address critical issues, such as resolving
  constitutional crises, facilitating transitions to democracy, or
  addressing severe economic downturns or social unrest. The 2010 coup
  in Niger, which ousted President Tandja after he sought an
  unconstitutional third term, is a notable example
  (\citeproc{ref-ginsburg2019}{Ginsburg and Elkins 2019}).
\item
  \textbf{Self-preservation:} In some instances, coups serve as
  pre-emptive strikes against perceived existential threats. Motivations
  in this category include preventing elimination or political
  persecution by incumbent leaders, protecting the interests of specific
  military or political factions, and safeguarding ideological or ethnic
  groups from marginalization. The 1971 coup led by Idi Amin against
  Ugandan President Obote exemplifies this motivation, as Amin acted to
  prevent his removal from a key military command position
  (\citeproc{ref-sudduth2017}{Sudduth 2017}).
\end{itemize}

Despite these potential motivations, it is crucial to note that coups
remain relatively uncommon events. Since 1950, coups have occurred in
only about 4\% of country-years. This rarity underscores the importance
of the capability factor -- even the most motivated actors require
substantial resources and favourable circumstances to successfully
execute a coup.

\subsection{Capability for coups}\label{capability-for-coups}

While motivations may spark coup attempts, the capability to launch a
successful coup is often more decisive. This capability is pivotal in
determining whether potential coup plotters transition from mere
contemplation to concrete action. Several key factors influence this
capability, which can be broadly categorized into five main areas:

\begin{itemize}
\item
  \textbf{Military strength}: A significant advantage in military
  capabilities over the incumbent regime substantially increases the
  odds of a successful coup (\citeproc{ref-powell2018}{Powell et al.
  2018}; \citeproc{ref-choulis2022}{Choulis et al. 2022}). This factor
  encompasses control over elite military units or special forces,
  access to advanced weaponry and technology, loyalty of key military
  commanders, and strategic positioning of supportive military units.
  The balance of military power is often the most critical determinant
  of coup success, as it directly affects the ability to seize and hold
  key government installations.
\item
  \textbf{Internal divisions within the regime}: Exploiting existing
  fractures within the government's power structure can provide coup
  plotters with a critical advantage. These divisions may manifest as
  ideological disagreements among ruling elites, competing centers of
  power within the government, ethnic or regional tensions in the
  political leadership, or dissatisfaction among mid-level bureaucrats
  or military officers. Coup plotters can leverage these divisions to
  build a coalition of support, promising benefits or addressing
  grievances of marginalized factions within the regime.
\item
  \textbf{Public support}: Widespread discontent with the incumbent
  leaders, particularly within the military or key sectors of society,
  can create a fertile environment for a successful coup. Factors
  contributing to public support include economic hardship or
  inequality, perceived corruption or mismanagement by the government,
  human rights abuses or political repression, and failure to address
  critical national issues. While public support alone may not be
  sufficient for coup success, it can provide legitimacy to the coup
  plotters and reduce resistance from the general population.
\item
  \textbf{Foreign backing}: External support from powerful nations can
  provide resources, legitimacy, and even direct military intervention
  to tip the scales in favor of the coup plotters. This support may take
  various forms, including financial assistance, intelligence sharing,
  diplomatic recognition, covert military aid or training, and threats
  of intervention against the incumbent regime. The role of foreign
  backing in coups has been particularly significant during the Cold War
  era and continues to shape geopolitical dynamics in many regions.
\item
  \textbf{Timing and opportunity}: Identifying and exploiting moments of
  vulnerability in the incumbent leaders is crucial for coup success.
  Opportune moments may include national crises or emergencies, periods
  of political transition or uncertainty, major public events or
  celebrations, and times of internal conflict within the regime. Coup
  plotters must carefully assess these windows of opportunity and time
  their actions to maximize their chances of success.
\end{itemize}

\subsection{The selection mechanism in staging
coups}\label{the-selection-mechanism-in-staging-coups}

While historical data may suggest a high success rate for coups, it is
crucial to consider the inherent selection bias when interpreting this
information. Observers are limited to studying only attempted coups,
neglecting the numerous plots that never materialize. This limitation
presents a significant challenge in accurately assessing the true
likelihood of coup success.

Several factors contribute to this selection bias:

\begin{itemize}
\item
  \textbf{Unobserved deterrence}: Preemptive measures by incumbent
  regimes may effectively deter coup attempts before they materialize,
  rendering them invisible to statistical analysis.
\item
  \textbf{Self-selection of capable plotters}: Those who attempt coups
  are likely to do so only when they perceive a reasonable chance of
  success, leading to an overestimation of the overall success rate in
  observed data.
\item
  \textbf{Unreported failed attempts}: Many unsuccessful coup attempts,
  particularly those in the early planning stages, may go unreported or
  remain undetected, further skewing the available data.
\item
  \textbf{Varying definitions of ``coup attempt''}: Inconsistencies in
  how researchers define and classify coup attempts can introduce
  additional bias into the data, complicating cross-study comparisons.
\end{itemize}

To address these challenges and understand coup attempts more
comprehensively, a theoretical framework is needed to account for this
selection bias. A frequently cited framework
(\citeproc{ref-gassebner2016}{Gassebner, Gutmann, and Voigt 2016};
\citeproc{ref-aidt2019}{Aidt and Leon 2019}) offers a structured
approach to assess the disposition and capability of coup attempts by
evaluating the anticipated benefits for coup plotters.

The expected pay-off of a coup can be represented by the equation:

\begin{equation}\phantomsection\label{eq-eq1}{
\begin{aligned}
E(U) = p \times B + (1 - p) \times (-C)
\end{aligned}
}\end{equation}

Where:

\begin{itemize}
\tightlist
\item
  \(E(U)\)\textbf{:} Expected utility or pay-off of the coup attempt;
\item
  \(B\) represents the return of a successful coup;
\item
  \(C\) signifies the cost of a failed coup;
\item
  \(p\) represents the probability of coup success.
\end{itemize}

The condition for staging a coup is when the expected benefit is
positive (\(E(U) > 0\)). Rearranging the equation, we get:

\begin{equation}\phantomsection\label{eq-eq2}{
\begin{aligned}
p \times B > (1 - p) \times C
\end{aligned}
}\end{equation}

This implies that for a coup to be attempted, the expected benefits of
success must outweigh the expected costs of failure.

However, quantifying \(B\) and \(C\) is inherently difficult due to
several factors. The intangible nature of some costs and benefits (e.g.,
loss of life, personal freedom). The variability of outcomes across
different contexts. The subjective valuation of power and risk by
individual coup plotters.

Given the difficulty in precisely quantifying \(B\) and \(C\), we can
treat them as roughly equal for analytical purposes. This allows us to
shift our focus to the probability of success (\(p\)). The simplified
equation becomes:

\begin{equation}\phantomsection\label{eq-eq3}{
\begin{aligned}
p > (1-p)
\end{aligned}
}\end{equation}

This suggests that a success probability greater than 50\% is necessary
for a coup to be attempted. While empirical data shows a slightly lower
overall success rate for coups since 1950 (49.9\%, as shown in
Table~\ref{tbl-coups}), it is crucial to remember that this is an
average and does not reflect the specific probabilities assessed by coup
plotters beforehand.

Based on this discussion, we propose the following hypothesis:

\begin{quote}
\textbf{H2-1: \emph{The fundamental determinant of a coup attempt is the
perceived chance of success. Coup plotters likely require a success
threshold of at least 50\%.}}
\end{quote}

This hypothesis fundamentally shifts the focus of coup research from the
prerequisites for triggering a coup to the expected coup outcome as the
primary determinant. By doing so, it underscores the importance of
understanding the factors that influence coup plotters' perceptions of
success probability, thereby highlighting the need for a more nuanced
approach to coup analysis.

A natural follow-up question arises: What factors determine coup success
and, subsequently, influence the decision to attempt one? The answer
lies in understanding the complex balance of power inherent in the
diverse power dynamics across various regime types.

\subsection{Regime types and coup
susceptibility}\label{regime-types-and-coup-susceptibility}

Existing research on coup plotting often concentrates on the
dispositions and capabilities of potential perpetrators, overlooking a
crucial factor: the dynamic balance of power between those initiating a
coup and the incumbent regime. Even when dispositions and capabilities
are similar, coup outcomes can vary significantly depending on this
balance, underscoring the mediating role of power dynamics in
influencing the strategic calculus of potential coup plotters.

For instance, solely attributing coup attempts to military strength
oversimplifies the complex reality of these events
(\citeproc{ref-singh2016}{Singh 2016}). In a highly centralized
personalist regime, control over the military may be concentrated in the
hands of a small, loyalist faction, making it difficult for potential
rivals to garner the necessary support for a successful coup.
Conversely, in a more fragmented military regime, internal divisions and
competing power centres can create opportunities for ambitious officers
to mobilize support and challenge the ruling junta. As Geddes
(\citeproc{ref-geddes1999}{1999}) notes, a coup's success often hinges
on the reactions of these other factions, creating a complex
decision-making environment for potential plotters. Furthermore,
incumbents frequently employ strategic division within the military as a
coup-proofing measure (\citeproc{ref-buxf6hmelt2014}{Böhmelt and Pilster
2014}). This involves intentionally creating rival groups within the
armed forces, establishing an artificial balance and structural
obstacles to coup attempts. Additionally, factors beyond military force,
such as internal divisions among ruling elites, public support, and
foreign backing, significantly shape the balance of power.

Consequently, a more comprehensive approach involves analyzing the
overall balance of power within a political system. However, as noted in
the introduction, directly observing this balance presents significant
challenges for outside observers, including policy-makers, international
organizations, and even potential coup plotters themselves. An
alternative approach, and the one adopted in this study, is to analyze
factors that decisively shape power dynamics. This method aligns more
closely with academic research goals, which focus on understanding the
patterns and underlying factors of coup attempts rather than identifying
specific power challengers capable of launching successful coups.

Among the various factors influencing power dynamics, regime type
emerges as one of the most crucial and observable. Regime types are
fundamentally classifications based on power structures within a
political system, and different regime types allocate authority
differently for critical decisions such as the deployment of military
forces, appointment of key officials and military officers, and
formulation and implementation of main policies. This allocation of
authority, in turn, shapes the incentives and opportunities for
potential coup plotters, influencing their likelihood of success.

Geddes, Wright, and Frantz (\citeproc{ref-geddes2014}{2014}) provides a
classification of autocratic regime types based on their power
structures, offering valuable insights into the power dynamics within
different autocracies and their relative susceptibility to coups. The
three main types of autocratic regimes, their characteristics, and their
vulnerability to coup attempts are explored as follows (see
Table~\ref{tbl-regimes1}):

\begin{itemize}
\item
  \textbf{Military regimes:} These are characterized by a junta -- a
  group of military officers controlling leadership selection and policy
  formulation. Examples include regimes in Brazil (1964-1985), Argentina
  (1976-1983), and El Salvador (1948-1984)
  (\citeproc{ref-geddes1999}{Geddes 1999}). Despite their military
  nature, these regimes are surprisingly unstable due to internal power
  struggles within the junta. The absence of a clear final authority and
  the presence of multiple military factions increase the likelihood of
  resorting to force to resolve disputes, making these regimes the most
  vulnerable to coups.
\item
  \textbf{Personalist regimes:} In these regimes, power is concentrated
  in a single, charismatic leader who controls the military, policy, and
  succession. Examples include Rafael Trujillo's regime in the Dominican
  Republic (1930-1961), Idi Amin's regime in Uganda (1971-1979), and
  Jean-Bédel Bokassa's regime in the Central African Republic
  (1966-1979) (\citeproc{ref-geddes1999}{Geddes 1999}). Personalist
  regimes are relatively stable during the leader's tenure. However,
  they face a higher risk of coups due to unclear succession plans and
  vulnerabilities associated with the leader's personal weaknesses,
  health, and mortality.
\item
  \textbf{Dominant-rarty regimes:} In these systems, power resides
  within a well-organized ruling party, with leaders acting as its
  representatives. The party structure and ideology foster internal
  cohesion and a long-term vision. Examples include the Partido
  Revolucionario Institucional (PRI) in Mexico, the Revolutionary Party
  of Tanzania (CCM), and Leninist parties in various Eastern European
  countries (\citeproc{ref-geddes1999}{Geddes 1999}). Dominant-party
  regimes exhibit the greatest resilience against coups due to their
  institutionalized structures, unified leadership, clear ideology, and
  internal discipline.
\end{itemize}

Empirical data supports the theoretical framework that different regime
types exhibit varying levels of susceptibility to coups. Military
regimes, constituting only 5.6\% of country-years since 1950, experience
a disproportionate share of coups (over 22\%). In contrast,
dominant-party regimes, representing 22.6\% of country-years, account
for only 16.7\% of coups (see Table~\ref{tbl-regimes}). These statistics
demonstrate the disproportionate vulnerability of military and
personalist regimes to coups, while highlighting the relative stability
of dominant-party regimes.

The distinct power dynamics exhibited by these regime types
significantly influence their susceptibility to coups. This analysis
leads to the second hypothesis:

\begin{quote}
\textbf{H2-2: \emph{The susceptibility to coups varies significantly
among different types of autocratic regimes, with military regimes being
the most vulnerable, followed by personalist regimes, and dominant-party
regimes being the least susceptible.}}
\end{quote}

\blandscape

\begin{longtable}[]{@{}
  >{\raggedright\arraybackslash}p{(\columnwidth - 10\tabcolsep) * \real{0.1667}}
  >{\raggedright\arraybackslash}p{(\columnwidth - 10\tabcolsep) * \real{0.1667}}
  >{\raggedright\arraybackslash}p{(\columnwidth - 10\tabcolsep) * \real{0.1667}}
  >{\raggedright\arraybackslash}p{(\columnwidth - 10\tabcolsep) * \real{0.1667}}
  >{\raggedright\arraybackslash}p{(\columnwidth - 10\tabcolsep) * \real{0.1667}}
  >{\raggedright\arraybackslash}p{(\columnwidth - 10\tabcolsep) * \real{0.1667}}@{}}

\caption{\label{tbl-regimes1}Characteristics of Different Regime Types}

\tabularnewline

\toprule\noalign{}
Regime Type & Power Concentration & Succession & Military Alignment &
Stability & Examples \\
\midrule\noalign{}
\endhead
\midrule\noalign{}
\multicolumn{6}{@{}>{\raggedright\arraybackslash}p{(\columnwidth - 10\tabcolsep) * \real{1.0000} + 10\tabcolsep}@{}}{%
\emph{Source: GWF \& Author}} \\
\bottomrule\noalign{}
\endlastfoot
Military & Junta & Unclear & May have significant influence & Low &
Brazil (1964-1985), Argentina (1976-1983) \\
Personalist & Single Leader & Unclear or dependent on
leader\textquotesingle s will & Subordinated to leader & Moderate
(initially), Low (long-term) & Dominican Republic (Trujillo,
1930-1961) \\
Dominant-Party & Party Leadership & Institutionalized & Aligned with the
party & High & Mexico (PRI), China (CPC) \\

\end{longtable}

\clearpage

\begin{longtable}[]{@{}
  >{\raggedright\arraybackslash}p{(\columnwidth - 12\tabcolsep) * \real{0.1429}}
  >{\raggedleft\arraybackslash}p{(\columnwidth - 12\tabcolsep) * \real{0.1429}}
  >{\raggedleft\arraybackslash}p{(\columnwidth - 12\tabcolsep) * \real{0.1429}}
  >{\raggedleft\arraybackslash}p{(\columnwidth - 12\tabcolsep) * \real{0.1429}}
  >{\raggedleft\arraybackslash}p{(\columnwidth - 12\tabcolsep) * \real{0.1429}}
  >{\raggedleft\arraybackslash}p{(\columnwidth - 12\tabcolsep) * \real{0.1429}}
  >{\raggedleft\arraybackslash}p{(\columnwidth - 12\tabcolsep) * \real{0.1429}}@{}}

\caption{\label{tbl-regimes}Regime types and coups since 1950}

\tabularnewline

\toprule\noalign{}
\begin{minipage}[b]{\linewidth}\raggedright
Regime Type
\end{minipage} & \begin{minipage}[b]{\linewidth}\raggedleft
Country Year
\end{minipage} & \begin{minipage}[b]{\linewidth}\raggedleft
Share
\end{minipage} & \begin{minipage}[b]{\linewidth}\raggedleft
Coups
\end{minipage} & \begin{minipage}[b]{\linewidth}\raggedleft
Coups Percent
\end{minipage} & \begin{minipage}[b]{\linewidth}\raggedleft
Success Rate
\end{minipage} & \begin{minipage}[b]{\linewidth}\raggedleft
Likelihood
\end{minipage} \\
\midrule\noalign{}
\endhead
\midrule\noalign{}
\multicolumn{7}{@{}>{\raggedright\arraybackslash}p{(\columnwidth - 12\tabcolsep) * \real{1.0000} + 12\tabcolsep}@{}}{%
\begin{minipage}[t]{\linewidth}\raggedright
\emph{Source: REIGN and GIC Datasets}
\end{minipage}} \\
\bottomrule\noalign{}
\endlastfoot
Democracy & 5312 & 46.7\% & 122 & 24.8\% & 51.6\% & 2.3\% \\
Dominant-Party & 2569 & 22.6\% & 82 & 16.7\% & 53.7\% & 3.2\% \\
Personal & 1476 & 13.0\% & 113 & 23.0\% & 44.2\% & 7.7\% \\
Monarchy & 1056 & 9.3\% & 25 & 5.1\% & 56.0\% & 2.4\% \\
Military & 638 & 5.6\% & 110 & 22.4\% & 48.2\% & 17.2\% \\
Other & 322 & 2.8\% & 39 & 7.9\% & 53.8\% & 12.1\% \\
Total & 11373 & 100.0\% & 491 & 100.0\% & 49.9\% & 4.3\% \\

\end{longtable}

\elandscape

\section{Research design}\label{research-design}

\subsection{Double probit with sample selection
model}\label{sec-coup231}

This study employs a sophisticated statistical approach to account for
the selective nature of coup attempts. While coup attempt rates vary
across regimes, success rates tend to be surprisingly consistent,
hovering around 50\% (as shown in Table~\ref{tbl-regimes}). This
suggests that coup attempts are not random acts, but rather
strategically planned and undertaken only when the odds of success
appear favourable. A standard statistical model would not account for
this selectivity, potentially leading to biased results.

To address this issue, we utilize a double probit with sample selection
model. This model, known as a Heckman probit model or bivariate probit
model with sample selection, consists of two parts:

\begin{itemize}
\item
  \textbf{Selection equation (Stage 1)}: This stage analyses the factors
  influencing whether a coup attempt occurs in a particular
  country-year.
\item
  \textbf{Outcome equation (Stage 2)}: This stage focuses on the
  probability of success for those coup attempts that actually take
  place.
\end{itemize}

The selection equation (first stage) models the probability that a coup
attempt occurs:

\begin{equation}\phantomsection\label{eq-eq4}{
\begin{aligned}
y_{1i}^*&=\alpha_0 + \alpha_1 Regime_i + \mathbf{X}_i \boldsymbol{A} + \mu_{1i}
\\
\\
y_{1i} &= 
\begin{cases} 
1 &\text{if $y_{1i}^*>0$ (coup attempt occurs)} \\
\\
0 &\text{if $y_{1i}^*\le0$ (no coup attempt)}
\end{cases}
\end{aligned}
}\end{equation}

The outcome equation (second stage) models the probability of a coup
attempt succeeding, given it occurs.

\begin{equation}\phantomsection\label{eq-eq5}{
\begin{aligned}
y_{2i}^*&=\beta_0 + \beta_1 Regime_i + \mathbf{Z}_i \boldsymbol{B} + \mu_{2i}
\\
\\
y_{2i} &= 
\begin{cases} 
1 &\text{if $y_{2i}^*>0$ (coup succeeds)} \\
\\
0 &\text{if $y_{2i}^*\le0$ (coup fails)}
\end{cases}
\end{aligned}
}\end{equation}

Where:

\begin{itemize}
\item
  \(y_{1i}^*\) \emph{and} \(y_{2i}^*\) are latent variables
\item
  \(\text{Regime}_i\) is a categorical variable (military, personalist,
  or dominant-party)
\item
  \(\mathbf{X}_i\) and \(\mathbf{Z}_i\) are vectors of control variables
\item
  \(\mu_{1i}\) and \(\mu_{2i}\) are error terms, assumed to follow a
  bivariate normal distribution with correlation \(\rho\)
\end{itemize}

The model assumes:

\[
\binom{\mu_{1 i}}{\mu_{2 i}} \sim N\left(\binom{0}{0},\left(\begin{array}{ll}
1 & \rho \\
\rho & 1
\end{array}\right)\right)
\]

The probability equations are:

\begin{equation}\phantomsection\label{eq-eq7}{
\begin{aligned}
P\left(y_{1 i}=1\right) & =\Phi\left(\alpha_0+\alpha_1 \operatorname{Regime}_i+\mathbf{X}_i \boldsymbol{A}\right)
\end{aligned}
}\end{equation}

\begin{equation}\phantomsection\label{eq-eq8}{
\begin{aligned}
P\left(y_{2 i}=1 \mid y_{1 i}=1\right) & =\Phi\left(\beta_0+\beta_1 \operatorname{Regime}_i+\mathbf{Z}_i \boldsymbol{B}\right)
\end{aligned}
}\end{equation}

Where \(\Phi(\cdot)\) is the cumulative distribution function of the
standard normal distribution.

\subsection{Variables}\label{variables}

\subsubsection{Dependent variable}\label{dependent-variable}

The analysis utilizes data on coup attempts and outcomes from Powell and
Thyne (\citeproc{ref-powell2011}{2011}). A successful coup is defined as
one where the incumbent leader is removed from power for more than seven
days. The dataset covers the period from 1950 to 2023 and includes
information on 491 coup attempts, with roughly half (245) being
successful. Descriptive statistics for these coup attempts and regime
types can be found in Table~\ref{tbl-coups} and Table~\ref{tbl-regimes}.

\begin{itemize}
\item
  \textbf{Coup attempt}: Binary variable indicating whether a coup
  attempt occurred (1) or not (0) in a given country-year.
\item
  \textbf{Coup success}: Binary variable indicating whether a coup
  attempt was successful (1) or failed (0), conditional on a coup
  attempt occurring.
\end{itemize}

\subsubsection{Key independent variable: regime
type}\label{sec-chapter2322}

I categorize regime types following Geddes, Wright, and Frantz
(\citeproc{ref-geddes2014}{2014}) (GWF), focusing on military,
personalist, and dominant-party regimes, with democracies and monarchies
included for comparison. Descriptive statistics for regime types are
presented in Table~\ref{tbl-regimes}.

\subsubsection{Control variables}\label{control-variables}

The control variables are chosen based on the research of Gassebner,
Gutmann, and Voigt (\citeproc{ref-gassebner2016}{2016}). They analysed
66 factors potentially influencing coups and found that slow economic
growth, prior coup attempts, and other forms of political violence are
particularly significant factors. Therefore, we include economic
performance, political violence, and the number of previous coups as our
main control variables.

\begin{itemize}
\tightlist
\item
  \textbf{Economic Level:} Represented by GDP per capita. This measure
  provides an indication of the overall economic health and standard of
  living in a country. We use GDP per capita data (in constant 2017
  international 1000 dollars, PPP) from the V-Dem dataset by Fariss et
  al. (\citeproc{ref-fariss2022}{2022}).
\item
  \textbf{Economic Performance:} Measured using the current-trend
  (\(CT\)) ratio developed by Krishnarajan
  (\citeproc{ref-krishnarajan2019}{2019}). This ratio compares a
  country's current GDP per capita to the average GDP per capita over
  the previous five years. A higher \(CT\) ratio indicates stronger
  economic performance. For a country \(i\) at year \(t\), the \(CT\)
  ratio is calculated as follows:
\end{itemize}

\begin{equation}\phantomsection\label{eq-eq6}{
    \begin{aligned}
    CT_{i,t} = {GDP/cap_{i,t} \over {1 \over 5} {\sum_{k=1}^5GDP/cap_{i,t-k}}}
    \end{aligned}
}\end{equation}

\begin{itemize}
\tightlist
\item
  \textbf{Political stability:} This variable captures overall regime
  stability by including a violence index that encompasses all types of
  internal and interstate wars and violence. The data for this index is
  sourced from the variable ``actotal'' in the Major Episodes of
  Political Violence dataset
  (\citeproc{ref-marshall2005current}{Marshall 2005}), with 0
  representing the most stable conditions (no violence at all) and 18
  representing the most unstable.
\item
  \textbf{Previous coups:} Included in the selection equation as either:
  a) The number of previous coups in a country (Model 1), or b) The time
  since the last coup attempt (Model 2 for robustness check).
\end{itemize}

\section{Results and discussion}\label{results-and-discussion}

\begin{table}

\caption{\label{tbl-coupmodel}Sample Selection Model of Regime Type and
Coup Success, 1950-2019}

\centering{

\begingroup 
\small 
\begin{tabular}{@{\extracolsep{7pt}}lcccc} 
\\[-1.8ex]\hline 
\hline \\[-1.8ex] 
\\[-1.8ex] & \multicolumn{2}{c}{Model 1} & \multicolumn{2}{c}{Model 2} \\ 
 & Coup Attempts & Coup Outcome & Coup Attempts & Coup Outcome \\ 
\\[-1.8ex] & (1) & (2) & (3) & (4)\\ 
\hline \\[-1.8ex] 
 Constant & $-$1.774$^{***}$ & $-$1.803$^{***}$ & $-$1.663$^{***}$ & $-$0.654 \\ 
  & (0.058) & (0.360) & (0.088) & (0.518) \\ 
  & & & & \\ 
 Regime: Democracy & 0.056 & 0.068 & 0.043 & 0.042 \\ 
  & (0.072) & (0.121) & (0.075) & (0.192) \\ 
  & & & & \\ 
 \hspace{1.6cm}Military & 0.687$^{***}$ & 0.596$^{***}$ & 0.345$^{***}$ & 0.247 \\ 
  & (0.084) & (0.170) & (0.091) & (0.229) \\ 
  & & & & \\ 
 \hspace{1.6cm}Monarchy & 0.282$^{**}$ & 0.178 & 0.233$^{*}$ & 0.088 \\ 
  & (0.118) & (0.201) & (0.123) & (0.310) \\ 
  & & & & \\ 
 \hspace{1.6cm}Personalist & 0.319$^{***}$ & 0.128 & 0.134$^{*}$ & $-$0.145 \\ 
  & (0.075) & (0.170) & (0.080) & (0.205) \\ 
  & & & & \\ 
 Economic trend & $-$0.015$^{***}$ & $-$0.004 & $-$0.014$^{***}$ & 0.009 \\ 
  & (0.002) & (0.007) & (0.002) & (0.008) \\ 
  & & & & \\ 
 GDP per capita & $-$0.028$^{***}$ & $-$0.028$^{***}$ & $-$0.016$^{***}$ & $-$0.016 \\ 
  & (0.003) & (0.006) & (0.003) & (0.010) \\ 
  & & & & \\ 
 Political violence & 0.033$^{**}$ & 0.033$^{*}$ & 0.038$^{***}$ & 0.025 \\ 
  & (0.013) & (0.020) & (0.013) & (0.031) \\ 
  & & & & \\ 
 Previous coups (P) & 0.030$^{***}$ &  & 0.448$^{***}$ &  \\ 
  & (0.010) &  & (0.086) &  \\ 
  & & & & \\ 
 Yrs since coup (Y) &  &  & $-$0.018$^{***}$ &  \\ 
  &  &  & (0.004) &  \\ 
  & & & & \\ 
 Interaction term: P * Y &  &  & $-$0.013$^{***}$ &  \\ 
  &  &  & (0.005) &  \\ 
  & & & & \\ 
\hline \\[-1.8ex] 
Observations & 9,606 & 9,606 & 9,606 & 9,606 \\ 
Log Likelihood & $-$1,663.683 & $-$1,663.683 & $-$1,598.656 & $-$1,598.656 \\ 
$\rho$ & 0.898$^{***}$  (0.158) & 0.898$^{***}$  (0.158) & 0.386$^{*}$  (0.234) & 0.386$^{*}$  (0.234) \\ 
\hline 
\hline \\[-1.8ex] 
\textit{Note:}  & \multicolumn{4}{r}{$^{*}$p$<$0.1; $^{**}$p$<$0.05; $^{***}$p$<$0.01} \\ 
\end{tabular} 
\endgroup

}

\end{table}%

The double probit model with sample selection, estimated using the
\textbf{\emph{sampleSelection}} package
(\citeproc{ref-sampleSelection-2}{Toomet and Henningsen 2008}) in R,
provides valuable insights into the factors influencing coup attempts
and their outcomes across different regime types from 1950 to 2019
(Table~\ref{tbl-coupmodel}). I present two models that differ slightly
in their treatment of previous coups: Model 1 incorporates the number of
previous coups, while Model 2 utilizes the time elapsed since the last
coup.

\begin{longtable}[]{@{}llrr@{}}

\caption{\label{tbl-mfx1}Average marginal effects of coup attempts
(Selection of Model 1)}

\tabularnewline

\toprule\noalign{}
Term & Contrast & AME{\textsuperscript{1}} & Ratio Percent \\
\midrule\noalign{}
\endhead
\midrule\noalign{}
\multicolumn{4}{@{}l@{}}{%
{\textsuperscript{1}} AME: Average Marginal Effect} \\
\bottomrule\noalign{}
\endlastfoot
Democracy & mean(democracy - dominant-party) & 0.003 & 13.040 \\
Military & mean(military - dominant-party) & 0.070 & 277.730 \\
Monarchy & mean(monarchy - dominant-party) & 0.020 & 80.280 \\
Personal & mean(personal - dominant-party) & 0.024 & 93.980 \\
Economic trend & mean(+1) & −0.001 & −2.850 \\
GDP per capita & mean(+1) & −0.002 & −5.400 \\
Political violence & mean(+1) & 0.003 & 6.550 \\
Previous coups & mean(+1) & 0.002 & 5.930 \\

\end{longtable}

\subsection{Selection model: coup
attempts}\label{selection-model-coup-attempts}

The selection model (Column 1) reveals that military and personalist
regimes exhibit significant positive coefficients at the 1\% level,
indicating a higher likelihood of experiencing coup attempts compared to
dominant-party regimes. This aligns with the theoretical expectations
regarding internal power struggles within military juntas and succession
vulnerabilities in personalist regimes.

Table~\ref{tbl-mfx1} presents the Average Marginal Effects (AME) and
ratio percentages to clarify the regime effects. The military regime's
marginal effect of 0.07 indicates that the probability of coup attempts
in military regimes is 7 percentage points (pp) higher than in
dominant-party regimes, ceteris paribus. This translates to military
regimes being about 277.7\% more likely to encounter coups than
dominant-party regimes. Similarly, personalist regimes show a 2.4 pp
higher probability, about 94\% more likely compared to dominant-party
regimes.

The control variables show effects in expected directions but with
weaker magnitudes. Stronger economic performance, indicated by higher
economic growth trends and GDP per capita levels, correlates with a
lower risk of coup attempts. Political violence shows a positive effect,
indicating that higher levels of instability increase the likelihood of
coups. The positive coefficient for the number of previous coups
suggests a ``copycat'' effect from earlier incidents.

\subsection{Outcome model: coup
success}\label{outcome-model-coup-success}

Turning to the outcome model (Columns 2 and 4 in
Table~\ref{tbl-coupmodel}), the results reveal determinants of coup
success. Military regimes demonstrate a higher probability of coup
success compared to dominant-party regimes, aligning with expectations
that military regimes face higher coup risks due to their increased
chances of success. Personalist and monarchical regimes show slight
positive effects on coup success, but these effects are not
statistically significant.

The control variables exhibit different patterns in the outcome model
compared to the selection model. Both GDP per capita and political
violence maintain a weak influence, similar to their effects in the
selection model. However, the economic trend shows a less significant
negative effect on coup success.

\subsection{Model comparison (model 1 versus model
2)}\label{model-comparison-model-1-versus-model-2}

Comparing the two models, Model 2 employs years since the last coup
instead of the number of previous coups, with an interaction term
between previous coups (as a binary variable) and years since the last
coup. Generally, Model 2 shows results in the same direction as Model 1,
albeit with relatively lower coefficients (see Table~\ref{tbl-mfx2}).

The differences between Model 1 and Model 2 suggest that while the
recency of coups matters, the overall history of coups in a country may
have a stronger influence on future coup attempts.

\begin{longtable}[]{@{}llrr@{}}

\caption{\label{tbl-mfx2}Average marginal effects of coup attempts
(Selection of Model 2)}

\tabularnewline

\toprule\noalign{}
Term & Contrast & AME{\textsuperscript{1}} & Ratio Percent \\
\midrule\noalign{}
\endhead
\midrule\noalign{}
\multicolumn{4}{@{}l@{}}{%
{\textsuperscript{1}} AME: Average Marginal Effect} \\
\bottomrule\noalign{}
\endlastfoot
Democracy & mean(democracy - dominant-party) & 0.003 & 8.920 \\
Military & mean(military - dominant-party) & 0.028 & 91.630 \\
Monarchy & mean(monarchy - dominant-party) & 0.018 & 56.730 \\
Personal & mean(personal - dominant-party) & 0.009 & 30.080 \\
Economic trend & mean(+1) & −0.001 & −2.530 \\
GDP per capita & mean(+1) & −0.001 & −2.890 \\
Political violence & mean(+1) & 0.003 & 7.330 \\
Previous coups (P) & mean(1 - 0) & 0.023 & 92.090 \\
Yrs since coup (Y) & mean(+1) & −0.002 & −5.050 \\

\end{longtable}

\subsection{Discussion of key
findings}\label{discussion-of-key-findings}

The \(ρ\) values of 0.898 in Model 1 and 0.386 in Model 2, significant
at 1\% and 10\% levels respectively, indicate strong correlation between
unobserved factors influencing coup attempts and coup success. This
supports the appropriateness of the sample selection model and
underscores the importance of considering both stages in the analysis.

The significant coefficients with theoretically consistent directions
suggest the model effectively captures key aspects of coup dynamics. The
observed disparity between coup attempt rates and success rates across
regimes points towards selection bias, further validating the use of the
sample selection model.

\subsection{Implications}\label{implications}

The results of this study strongly corroborate the theoretical framework
delineated in this chapter, highlighting the critical role that regime
structure and the anticipated probability of coup success play in
determining a regime's susceptibility to overthrow. These findings
underscore that coups are strategic actions undertaken when
circumstances appear favourable, rather than random occurrences.

The implications of these results for real-world politics are profound.
Given that the perceived likelihood of successful coup execution is
pivotal in coup attempts, incumbent leaders---particularly those in
autocratic regimes---will be strongly motivated to concentrate their
efforts on diminishing the capabilities of potential coup plotters,
rather than merely addressing the underlying dispositions toward staging
a coup.

In practice, this implies that autocratic leaders are likely to
prioritize ``coup-proofing'' strategies. These may include appointing
loyal generals and officials over competent ones, fostering factional
divisions within the elite, and removing capable or popular military
officers and government officials---even those who remain loyal.
However, this approach can precipitate a vicious cycle.

The adage, ``New levels, new devils'', aptly captures the cyclical
nature of coup risk mitigation. Incumbents, often lulled into a false
sense of security after neutralizing immediate threats, may find
themselves facing new challenges as a direct consequence of their
actions. The very act of suppressing potential rivals, while seemingly
strengthening their grip on power, inevitably fosters resentment and
grievances among those disenfranchised. This, in turn, makes the
prospect of a coup increasingly enticing, shifting the calculus of risk
for potential plotters. While capability remains paramount in
determining the success of any coup attempt, the disposition for such an
undertaking, fuelled by discontent and perceptions of vulnerability,
cannot be disregarded. Paradoxically, heavy-handed crackdowns on
opposition, intended to deter challenges, can inadvertently lower the
perceived costs of staging a coup, creating a fertile ground for future
instability.

Moreover, if incumbent leaders devote excessive attention to
coup-proofing at the expense of economic growth and social development,
they are likely to resort to increasingly autocratic forms of
governance. This, in turn, can lead to democratic backsliding or even
outright breakdown, as well as the entrenchment of authoritarian and
personalistic rule.

In conclusion, the findings of this study offer critical insights into
the complex dynamics underlying coup attempts and the strategic calculus
of incumbent regimes. Navigating these challenges will require a
delicate balance between addressing the capabilities and dispositions of
potential coup plotters while maintaining a focus on broader governance,
economic, and social priorities. Future research should explore the
long-term consequences of coup-proofing strategies on regime stability
and societal well-being, as well as potential alternative approaches to
mitigating coup risks without compromising democratic values and
institutions.

\section{Summary}\label{summary}

This chapter explores the most frequent irregular leadership transition,
coups, that destabilize political systems and disrupt democratic
transitions worldwide, a phenomenon that has garnered extensive
scholarly attention yet remains a challenging puzzle. To shed new light
on this issue, this study introduces a novel perspective by emphasizing
the determinants that impact coup success, moving beyond the mere
capability to launch a coup and focusing on the expected probability of
success. This approach highlights the critical role played by the
perceived balance of power between coup plotters and the incumbents in
determining outcomes.

The study employs a double probit model with sample selection, which
enables a nuanced analysis of the relationship between regime types and
coup likelihood. The results affirm that regime type significantly
influences coup vulnerability, with military and personalist regimes
exhibiting higher susceptibility to coups compared to dominant-party
regimes. This finding is corroborated by historical examples, such as
the frequent instability and coup attempts experienced by military
regimes in Latin America during the 20th century and various African
personalist regimes. The study underscores the importance of building
regimes with strong constitutional institutions, as those reliant on
military or personal authority prove to be more prone to coups.

A key contribution of this study is its exploration of the paradox faced
by autocratic rulers, such as dictators and military juntas. While
institutional strengthening could promote regime longevity and reduce
the risk of coups, many autocrats resist such reforms due to the
potential limitations they might impose on their power.
Institutionalization, while beneficial for the regime's long-term
stability, may not align with the short-term interests of individual
leaders who prioritize their personal control and authority over broader
regime security.

The study also identifies policy implications for enhancing political
stability, including promoting economic development, strengthening
institutional frameworks, and encouraging political inclusivity. This
shift towards institutionalization could help reduce the volatility that
often accompanies military and personalist regimes.

Furthermore, this research opens several avenues for further inquiry,
including:

\begin{itemize}
\tightlist
\item
  Examining the long-term effects of regime types on political
  stability, including trends toward democratization or authoritarian
  personalization.
\item
  Investigating the role of international factors in influencing regime
  types and coup dynamics, especially when internal mechanisms are
  insufficient to prevent coups.
\end{itemize}

In conclusion, this chapter provides empirical evidence confirming that
regime type plays a decisive role in determining coup risk. It also
acknowledges the complex interplay of power dynamics within regimes,
offering insights for both scholars and policy-makers focused on
promoting democratic transitions and political stability. By shedding
light on the strategic decisions behind coup attempts, this study
contributes to a deeper understanding of political instability and its
potential solutions, ultimately informing strategies for promoting more
stable and resilient political systems.

\chapter{Autocoups: Conceptual Clarification and Analysis of Power
Extensions by Incumbent Leaders}\label{sec-chapter3}

\section*{Abstract}\label{abstract-2}
\addcontentsline{toc}{section}{Abstract}

This chapter clarifies the concept of autocoups, focusing on their
distinct characteristics, particularly the extension of power by
incumbent leaders. By delineating autocoups from the more encompassing
and ambiguous concepts of self-coups or executive takeovers, which
conflate executive power aggrandizement and power extension, this
research offers a refined definition of autocoups. Building on this
conceptual framework, a novel dataset of autocoup events spanning 1945
to 2023 is introduced. The study employs a mixed-methods approach,
combining three types of qualitative case studies that provide in-depth
insights into the dynamics of autocoups with a quantitative analysis of
the determinants of autocoup attempts and success. This empirical
examination demonstrates the utility of the autocoup dataset for future
research. This study contributes to the existing literature by providing
a clearer conceptual framework, a novel dataset, and a deeper
understanding of the mechanisms and motivations underlying power
extensions by incumbent leaders. Furthermore, it explores the
implications of autocoups for democratic resilience, shedding light on
the effects of autocoups on leadership transitions and political
stability.

\emph{\textbf{keywords}: Autocoups, Coups, Dataset}

\newpage

\section{Introduction}\label{introduction-2}

While the study of irregular leadership transitions has predominantly
focused on coups due to their frequency and significant impact, another
form---the incumbent leader's refusal to relinquish power---has received
comparatively less attention despite its importance. Recent decades,
particularly since the end of the Cold War, have witnessed a decline in
classic coups and a concomitant rise in this incumbent retention or
overstay type of irregular leadership transition
(\citeproc{ref-ginsburg2010evasion}{Ginsburg, Melton, and Elkins 2010};
\citeproc{ref-baturo2014}{Baturo 2014};
\citeproc{ref-versteeg2020law}{Versteeg et al. 2020}).

This chapter aims to redefine and clarify this type of irregular
leadership transition, where leaders overstay their mandated term
limits, as an \textbf{\emph{autocoup}}. Although analyses related to
autocoups are not uncommon, the existing literature exhibits several
notable shortcomings:

\begin{itemize}
\item
  \textbf{Terminological ambiguity:} The use of terms like
  ``self-coups'', ``autocoups'', ``autogolpes'', ``incumbent
  takeovers'', ``executive aggrandizement'', ``overstay'', and
  ``continuismo'' in different literature lacks clear, universally
  accepted definitions, leading to confusion and inconsistent
  application (\citeproc{ref-marsteintredet2019}{Marsteintredet and
  Malamud 2019}; \citeproc{ref-baturo2022}{Baturo and Tolstrup 2022}).
  This terminological ambiguity hinders accurate analysis and comparison
  across studies
\item
  \textbf{Limited dataset:} Due to the conceptual ambiguity surrounding
  autocoups, data collection remains in its nascent stages compared to
  the rich datasets available for classic coups.
\item
  \textbf{Methodological gaps:} The study of autocoups has been hindered
  by a limited dataset, resulting in a reliance on in-depth case studies
  (\citeproc{ref-cameron1998}{Maxwell A. Cameron 1998b};
  \citeproc{ref-antonio2021}{Antonio 2021};
  \citeproc{ref-pion-berlin2022}{Pion-Berlin, Bruneau, and Goetze 2022})
  to explore this phenomenon. Notably, quantitative analysis has been
  underutilized in this field, with few studies employing statistical
  methods to examine autocoups.
\end{itemize}

More importantly, analyses of autocoups are often not integrated with
those of classic coups, despite their interconnected nature. As a
distinct category of coup, autocoups lack a clear and differentiated
definition in relation to traditional coups. Classic coups are typically
defined as the complete removal of incumbent leaders, with a focus on
the termination of their tenure. In contrast, autocoups often focus more
on incumbent leaders consolidating power by seizing control from other
state institutions, rather than on extending their tenure. As a result,
coups and autocoups are frequently analysed in isolation. This
separation has led to a dearth of comparative analyses, hindering a more
comprehensive understanding of the complex dynamics between these two
types of irregular leadership transitions.

Examining autocoups, particularly in conjunction with classic coups, is
essential for several compelling reasons. Firstly, both coups and
autocoups represent significant and frequent means of irregular
leadership transitions, underscoring the need for a comprehensive
understanding of these phenomena. Secondly, autocoups, much like coups,
have a profoundly detrimental impact on governance, as they undermine
the rule of law, erode institutional capacity, and contribute to
democratic backsliding or the personalization of authoritarian power.
Thirdly, successful autocoups, akin to successful coups, create a
precedent that increases the likelihood of future irregular power
transitions, thereby perpetuating a cycle of instability. For instance,
since 1945, a striking 62\% of leaders who extended their terms through
autocoups in non-democratic countries ultimately met a tumultuous end,
either ousted or assassinated while in office
(\citeproc{ref-baturo2019}{Baturo 2019}). Lastly, failed autocoups,
similar to failed coups, often precipitate instability, inciting
widespread protests, violence, and even civil wars, which can have
far-reaching and devastating consequences for the affected country and
its citizens.

This chapter addresses these gaps by focusing on autocoups, aiming to
clarify terminology, refine concepts and definitions, enhance data
collection, and explore determinants through empirical analysis,
contributing in three key areas:

\begin{itemize}
\item
  \textbf{Conceptual clarification:} The term autocoup will be redefined
  and clarified, with a focus on power extension.
\item
  \textbf{Data collection:} A new dataset of autocoups since 1945 will
  be introduced based on this refined definition.
\item
  \textbf{Empirical analysis:} Utilizing this dataset, a quantitative
  analysis of the factors influencing leaders' decisions to attempt
  autocoups will be conducted.
\end{itemize}

The structure of this chapter is as follows: Section 2 will review
definitions related to power expansions and extensions, leading to a
precise definition of autocoups. Section 3 will present the new autocoup
dataset. Sections 4 and 5 will explore the determinants of autocoup
attempts through case studies and demonstrate the application of the
dataset in empirical analysis. The conclusion will summarize key
findings and suggest directions for future research.

\section{Autocoups: A literature review and clarification of
definitions}\label{autocoups-a-literature-review-and-clarification-of-definitions}

A significant limitation in the study of irregular leadership transition
is the conspicuous lack of integration between research on autocoups and
classic coups. Despite both coups and autocoups being crucial mechanisms
of irregular leadership change, the existing literature has largely
treated these two phenomena in isolation, neglecting to explore their
interconnectedness and the nuanced dynamics that govern their
occurrence.

This separation is attributable to two primary factors. Firstly,
previous research has often overlooked autocoups as a distinct form of
irregular leadership transition. Secondly, a persistent conceptual
ambiguity has hindered the development of a clear and consistent
definition and categorization of autocoups.

Classic coups are typically characterized by the abrupt and
comprehensive removal of incumbent leaders, focusing on the swift
termination of their tenure. In contrast, autocoups are often defined as
events wherein incumbent leaders consolidate their authority by
systematically usurping power from other state institutions, rather than
merely extending their own tenure.

The absence of a clear, differentiated definition of autocoups in
relation to their classic counterparts has exacerbated this divide.
While classic coups have been extensively studied and well-defined,
autocoups remain a distinct yet under-explored category. This
definitional ambiguity has resulted in a dearth of comparative analyses
that could illuminate the complex interplay between these two forms of
power subversion.

Integrating the study of autocoups with that of classic coups is crucial
for bridging this analytical gap. Such an approach would offer a more
nuanced and comprehensive perspective on the various mechanisms by which
political power can be subverted or consolidated. By examining these
phenomena in tandem, researchers can better understand the full spectrum
of irregular power transitions, from outright removal to internal power
grabs, providing valuable insights into the nature of political
instability and regime change.

To strengthen the analysis of autocoups, a crucial first step is to
establish a clear and consistent terminology, followed by a refinement
of the definition of autocoup to mitigate ambiguity and clarify its
distinct characteristics.

\subsection{Terminology}\label{terminology}

The most prevalent term in autocoup literature is ``self-coup,'' or
``autogolpe'' in Spanish (\citeproc{ref-przeworski2000}{Przeworski et
al. 2000}; \citeproc{ref-cameron1998a}{Maxwell A. Cameron 1998a};
\citeproc{ref-bermeo2016}{Bermeo 2016}; \citeproc{ref-helmke2017}{Helmke
2017}; \citeproc{ref-marsteintredet2019}{Marsteintredet and Malamud
2019}). This term gained academic prominence following Peruvian
President Alberto Fujimori's actions in 1992, when he dissolved
Congress, temporarily suspended the constitution, and ruled by decree
(\citeproc{ref-mauceri1995}{Mauceri 1995};
\citeproc{ref-cameron1998}{Maxwell A. Cameron 1998b}). However, as
Marsteintredet and Malamud (\citeproc{ref-marsteintredet2019}{2019})
astutely points out, the term ``self-coup'' can be misleading, as it
implies a coup against oneself, which is inaccurate since the action
typically targets other state institutions or apparatus.

Another approach to describing coups staged by incumbents involves using
terms with adjectives or modifiers, such as ``presidential coup,''
``executive coup,'' ``constitutional coup,'' ``electoral coup,''
``judicial coup,'' ``slow-motion coup,'' ``soft coup,'' and
``parliamentary coup'' (\citeproc{ref-marsteintredet2019}{Marsteintredet
and Malamud 2019}). While these terms can be useful in specific
instances, their proliferation often creates more confusion than
clarity. Most of these terms focus on the specific methods employed by
coup perpetrators but fail to clearly identify the perpetrator,
necessitating further explanation. Moreover, many of these methods could
be employed either by or against executive leaders, further muddying the
waters.

A third alternative involves terms like ``incumbent takeover,''
``executive takeover,'' or ``overstay.'' Incumbent takeover refers to
``an event perpetuated by a ruling executive that significantly reduces
the formal and/or informal constraints on his/her power''
(\citeproc{ref-baturo2022}{Baturo and Tolstrup 2022, 374}), building on
earlier research by (\citeproc{ref-svolik2014}{Svolik 2014}). Meanwhile,
overstay is defined as ``staying longer than the maximum term as it
stood when the candidate originally came into office''
(\citeproc{ref-ginsburg2011evasion}{Ginsburg, Melton, and Elkins 2011,
1844}). These terms effectively identify the perpetrator (the incumbent)
and/or the nature of the event (overstaying/extending power). However,
they fall short in highlighting the illegality or illegitimacy of these
actions. Consequently, they cannot serve as a direct counterpart to
``coup,'' which clearly denotes the illegality of leadership ousters,
while ``takeover'' or ``overstay'' diminish the severity of the act.

Given that these terms often lack precision, focusing on specific
methods rather than the core act of power usurpation, this study
proposes ``autocoup'' as the most suitable term for this phenomenon.
Unlike other terms, `autocoup' clearly identifies the perpetrator and
the illegitimate nature of the power grab, distinguishing it from
classic coups while maintaining a parallel structure in terminology.

\subsection{Definition}\label{definition}

While precise terminology is undoubtedly crucial, another issue arises
with previous definitions of autocoups: what is the primary
emphasis---power expansion, power extension, or a combination of the
two?

Definitions of power expansion and power extension within the field of
political science can often be ambiguous or overlapping, presenting a
potential source of confusion. To ensure greater clarity in the study of
autocoups, it is necessary to distinguish these two distinct conceptual
frameworks more clearly:

\begin{itemize}
\item
  \textbf{Power expansion:} This refers to the process by which an
  incumbent leader acquires additional authority or control over state
  apparatuses beyond their original mandate. This may involve
  centralizing power, reducing checks and balances, or encroaching on
  the authority of other branches such as the legislature or judiciary.
\item
  \textbf{Power extension:} This describes situations where a leader
  prolongs their tenure beyond the originally mandated term in office,
  often through constitutional amendments, cancellation of elections, or
  other means of circumventing term limits.
\end{itemize}

Existing definitions of autocoups or related concepts often suffer from
ambiguity between power expansion and extension, or they focus more on
power expansion, which has several drawbacks.

Firstly, defining autocoups primarily in terms of power expansion does
not align well with the traditional definition of a coup. A classical
coup is clearly focused on the ouster of the current leader, not merely
a limitation or restriction on their power. Using the same logic, a more
appropriate definition of an autocoup should prioritize the tenure
extension of executive leadership. Power restrictions on incumbents
would not be coded as a coup as long as they remain in office.
Similarly, an executive leader acquiring more power from other branches
could be coded as power aggrandizement but not an autocoup, as long as
they step down when their term expires.

Secondly, emphasizing power expansion in autocoups often neglects the
ultimate purpose of incumbents. It is irrational for an incumbent to
expand executive power only to pass the powerful role to future leaders.
Although the term ``self-coup'' gained prominence from the 1992 Fujimori
case in Peru, which initially involved seizing power from other
institutions, it is important to note that Fujimori ultimately extended
his term limits through constitutional amendments. The 1993 Constitution
allowed Fujimori to run for a second term, which he won in April 1995.
Shortly after Fujimori began his second term, his supporters in Congress
passed a law of ``authentic interpretation'' that effectively allowed
him to run for another term in 2000, which he won amid suspicions and
rumors. However, he did not survive the third term; in 2000, facing
charges of corruption and human rights abuses, Fujimori fled Peru and
took refuge in Japan (\citeproc{ref-ezrow2019}{Ezrow 2019}).

Thirdly, measuring the extent of power expansion to qualify as an
autocoup can be challenging. As Maxwell A. Cameron
(\citeproc{ref-cameron1998a}{1998a}) defined, a self-coup is ``a
temporary suspension of the constitution and dissolution of congress by
the executive, who rules by decree until new legislative elections and a
referendum can be held to ratify a political system with broader
executive power'' (p.~220). However, defining ``broader executive
power'' is inherently problematic and disputable.

Therefore, this study argues that a more accurate definition of
autocoups should prioritize power extension as the core characteristic.
This approach is straightforward and easy to identify in practice. In
most cases, autocoups involving power extension also involve power
expansions as their prerequisite and foreshadowing.

Based on these criteria, I define \textbf{an autocoup as the
illegitimate extension of an incumbent leader's term in office beyond
the originally mandated limits through unconstitutional means}. This
definition emphasizes the core characteristic of power extension while
acknowledging the potential for power expansion as a related phenomenon:

\begin{itemize}
\item
  \textbf{Leadership Focus}: This definition refers to the actual
  leaders of the country, regardless of their official titles.
  Typically, this would be the president; however, in some cases, such
  as in Germany, the primary leader is the premier, as the president is
  a nominal head of state.
\item
  \textbf{Primary Characteristic}: While the primary characteristic of
  an autocoup is extending the term in office, this definition does not
  exclude instances of power expansion. Both aspects can coexist, but
  the extension of the term is the central element.
\item
  \textbf{Illegitimacy}: Autocoups, by their nature, subvert legal norms
  and established leadership transfer mechanisms. No matter how
  legitimate they claim to be, their illegitimacy is not beyond a
  reasonable doubt as long as the incumbents are the direct
  beneficiaries. This critical aspect will be explored further in
  Section 3.
\end{itemize}

By clarifying these definitions, this study aims to provide a more
precise and consistent framework for understanding and analysing
autocoups, thereby enhancing the clarity and rigor of research in this
field.

\section{Introduction to the autocoup
dataset}\label{introduction-to-the-autocoup-dataset}

\subsection{Defining the scope}\label{defining-the-scope}

Categorizing political events as autocoups inevitably involves
challenging borderline cases. To maintain consistency and avoid
ambiguity, this study adopts a broad coding approach: All instances of
incumbents extending their original mandated term in office are coded as
autocoups, regardless of the apparent legality of the extension.

This approach is justified because truly legitimate amendments to power
transition institutions should apply only to subsequent leaders, not the
incumbent. Even when extension procedures appear legal, the legitimacy
is questionable when the incumbent is the direct beneficiary.

\subsection{Classifying autocoups}\label{sec-classify}

Autocoups manifest in various forms. I categorize them based on several
key factors:

\begin{itemize}
\item
  \textbf{Methods employed}: Specific strategies used by incumbents
  (e.g., constitutional amendments, election cancellation).
\item
  \textbf{Degree of legality}: Extent of deviation from established
  legal norms.
\item
  \textbf{Duration of extension}: Length of time the incumbent remains
  in office beyond designated term limits.
\item
  \textbf{Outcomes}: Whether the autocoup attempt succeeds or fails.
\end{itemize}

This study primarily focuses on the methods employed, while coding for
other aspects when information is available.

\subsubsection*{Evasion of term limits}\label{evasion-of-term-limits}
\addcontentsline{toc}{subsubsection}{Evasion of term limits}

Evasion of term limits is a common tactic employed in autocoups.
Incumbents often resort to seemingly legal manoeuvres to extend their
hold on power. These manoeuvres primarily involve manipulating
constitutional provisions through various means. The incumbents may
pressure legislative bodies (congress) or judicial institutions (Supreme
Court) to reinterpret existing term limits, amend the constitution to
extend terms, or even replace the constitution altogether. This might
also involve popular vote through referendums, or a combination of these
approaches. The extension can range from a single term to indefinite
rule.

These manoeuvres primarily involve manipulating constitutional
provisions through various means.

\begin{itemize}
\item
  \textbf{Changing term length:} Incumbents might lengthen the official
  term duration (e.g., from 4 to 6 years) to stay in office longer, even
  if the number of allowed terms remains unchanged. Examples, in the
  dataset, include Presidents Dacko (CAR, 1962), Kayibanda (Rwanda,
  1973), and Pinochet (Chile, 1988).
\item
  \textbf{Enabling re-election:} This approach involves incumbents
  modifying legal or constitutional frameworks to permit themselves to
  run for leadership again, despite initial restrictions. These
  restrictions might include prohibitions on re-election, bans on
  immediate re-election, or term limits that the incumbents have already
  reached. An illustrative example is President Menem of Argentina in
  1993, who leveraged this tactic to extend his tenure.
\item
  \textbf{Removing term limits altogether:} This approach was
  implemented by President Paul Biya of Cameroon in 2008. Biya, who had
  been in power since 1982, successfully pushed for a constitutional
  amendment that abolished presidential term limits. This change allowed
  him to run for re-election indefinitely, effectively opening the
  possibility for him to rule for life.
\item
  \textbf{Declaring} \textbf{leader for life:} This differs from
  removing term limits as the leader still faces elections (although
  potentially rigged or uncontested). An example is Indonesia's
  President Sukarno, who attempted to declare himself president for life
  in 1963 (ultimately unsuccessful).
\end{itemize}

These methods are often used in combination. Initially, the duration of
a term is extended, followed by amendments to allow re-election, then
the removal of term limits, and finally, the declaration of the leader
for life. For example, Haitian President François Duvalier amended the
constitution in 1961 to permit immediate re-election and then declared
himself president for life in 1964.

\subsubsection*{Election manipulation or
rigging}\label{election-manipulation-or-rigging}
\addcontentsline{toc}{subsubsection}{Election manipulation or rigging}

Election manipulation or rigging is the second most commonly used tactic
to extend an incumbent's tenure.

\begin{itemize}
\item
  \textbf{Delaying or removing elections:} Delaying or removing
  scheduled elections without legitimate justification is a frequent
  method used by incumbents to maintain power. For instance, Chadian
  President François Tombalbaye delayed general elections until 1969
  after assuming power in 1960. Similarly, Angolan President José
  Eduardo dos Santos suspended elections throughout his rule from 1979
  to 2017.
\item
  \textbf{Refusing unfavourable election results:} Incumbents may refuse
  to accept unfavourable election results and attempt to overturn them
  through illegitimate means. For example, President Donald Trump of the
  United States refused to accept the results of the 2020 election and
  tried to overturn them.
\item
  \textbf{Rigging elections:} Winning elections with an extraordinarily
  high percentage of votes is highly questionable. This study will code
  elections where the incumbent wins more than 90\% of the vote as
  autocoups. For instance, President Teodoro Obiang of Equatorial Guinea
  has consistently won elections with over 95\% of the vote in
  multi-party elections since 1996, indicating election rigging.
\item
  \textbf{Excluding opposition in elections:} Manipulating the electoral
  process by excluding opposition parties or candidates from
  participation, effectively creating a one-candidate race, clearly
  signifies an autocoup.
\end{itemize}

\subsubsection*{Figurehead Installation}\label{figurehead-installation}
\addcontentsline{toc}{subsubsection}{Figurehead Installation}

One strategy employed by incumbents to evade term limits is to install a
trusted associate as a figurehead, allowing the incumbent to maintain de
facto control while formally relinquishing office. This can be achieved
through the creation of seemingly subordinate positions, which in
reality serve as conduits for the incumbent's continued influence.

A notable example of this tactic is the 2008 Russian presidential
transition. Confronted with constitutional term limits, President
Vladimir Putin hand-picked Dmitry Medvedev to succeed him as president.
Following Medvedev's election, he appointed Putin as Prime Minister,
ostensibly reversing their roles. However, most observers and analysts
concur that Putin continued to wield significant behind-the-scenes
influence, effectively rendering Medvedev a proxy leader.

\subsubsection*{Reassigning supreme authority to a new
role}\label{reassigning-supreme-authority-to-a-new-role}
\addcontentsline{toc}{subsubsection}{Reassigning supreme authority to a
new role}

This tactic involves an incumbent leader manipulating the constitution
or legal framework to create a new position of power, or elevate an
existing one, before stepping down from their current role. They then
strategically take on this new position, effectively retaining
significant control despite appearing to relinquish power. For example,
in 2017, Recep Tayyip Erdoğan, the Prime Minister of Turkey, spearheaded
a constitutional referendum that transitioned the country from a
parliamentary system to a presidential one. This new system concentrated
significant executive power in the presidency. Following the
referendum's approval, Erdoğan successfully ran for the newly
established presidency, effectively retaining control under a different
title.

\subsubsection*{One-time arrangement for current
leaders}\label{one-time-arrangement-for-current-leaders}
\addcontentsline{toc}{subsubsection}{One-time arrangement for current
leaders}

This strategy involves special arrangements that extend the term or
tenure of current leaders without altering the underlying institutions.
For example, Lebanon extended President Émile Lahoud's term by three
years in 2004 through a one-time arrangement.

\subsection{Data coding}\label{data-coding}

The autocoup dataset is built upon existing studies and datasets,
ensuring a comprehensive and reliable foundation. Table~\ref{tbl-source}
outlines the main sources used for coding the autocoup dataset.

The Archigos dataset (\citeproc{ref-goemans2009}{Goemans, Gleditsch, and
Chiozza 2009}) and the Political Leaders' Affiliation Database (PLAD)
(\citeproc{ref-bomprezzi2024wedded}{Bomprezzi et al. 2024}) provide
comprehensive data on all leaders from 1875 to 2023, although our coding
only includes autocoups since 1945. These datasets are invaluable for
identifying actual rulers, distinguishing them from nominal heads of
state.

The Incumbent Takeover dataset (\citeproc{ref-baturo2022}{Baturo and
Tolstrup 2022}) integrates data from 11 related datasets, offering a
broad spectrum of cases where leaders significantly reduced constraints
on their power. This dataset includes both power expansions and
extensions, necessitating cross-referencing with Archigos to verify
qualifications for autocoups.

\begin{longtable}[]{@{}llrr@{}}

\caption{\label{tbl-source}Main Data Sources for Coding the Autocoup
Dataset}

\tabularnewline

\toprule\noalign{}
Dataset & Authors & Coverage & Obervations \\
\midrule\noalign{}
\endhead
\bottomrule\noalign{}
\endlastfoot
Archigos & Goemans et al (2009) & 1875-2015 & 3409 \\
PLAD & Bomprezzi et al. (2024) & 1989-2023 & 1334 \\
Incumbent Takeover & Baturo and Tolstrup (2022) & 1913-2019 & 279 \\

\end{longtable}

In total, 110 observations were coded, with 95 overlapping with the
candidate data from Incumbent Takeover. The remaining 15 events were
newly coded by the author through verification with other sources such
as Archigos, PLAD and news reports.

The main deviation from the Incumbent Takeover dataset arises from
excluding power expansions that do not involve attempts to extend
tenure.

The dataset encompasses a total of 14 variables along with the
\emph{notes} field.

\begin{itemize}
\item
  \textbf{Country identification:} Country code (\emph{ccode}) and
  country name (\emph{country}) from Correlates of War project
  (\citeproc{ref-stinnett2002}{Stinnett et al. 2002}).
\item
  \textbf{Leader information:} Name of the de facto leader
  (\emph{leader\_name},coded following Archigos and PLAD datasets).
\item
  \textbf{Timeline variables:} Date the leader assumed power
  (\emph{entry\_date}), date the leader left office(\emph{exit\_date}),
  date of the significant event marking the autocoup
  (\emph{autocoup\_date}), and Start date of the leader's additional
  term acquired through the autocoup (\emph{extending\_date}).
\item
  \textbf{Power transition methods:} Categorical variable for how the
  leader entered power (\emph{entry\_method}), categorical variable for
  how the leader exited power (\emph{exit\_method}), dummy variable
  indicating regular (1) or irregular (0) entry (\emph{entry\_regular}),
  and dummy variable indicating regular (1) or irregular (0) exit
  (\emph{exit\_regular}).
\item
  \textbf{Autocoup details:} Key variable capturing methods used to
  extend power (\emph{autocoup\_method}) and outcome of the autocoup
  attempt (\emph{autocoup\_outcome}, ``fail and lose power'', ``fail but
  complete original tenure'', or ``successful''). For successful coups,
  the additional term length can be calculated from the difference
  between \emph{exit\_date} and \emph{extending\_date}.
\item
  \textbf{Data source:} Identifies the dataset source used for coding
  (\emph{source}).
\item
  \textbf{Additional notes:} Provides context for exceptional cases
  (\emph{notes}).
\end{itemize}

There are a few coding challenges and decisions worth mention. For cases
where extensions happen incrementally, the \emph{autocoup\_date}
reflects a significant event marking the extension, such as a
legislative vote or successful referendum. In cases where a leader
undertook multiple autocoup attempts, details are recorded in the notes
field. Care was taken to differentiate between cases of power expansion
and actual attempts to extend tenure, which required cross-referencing
multiple sources. Determining the success or failure of an autocoup
attempt often required in-depth research, especially for less documented
cases.

\subsection{Data descriptions}\label{data-descriptions}

The primary coding has identified 110 autocoup cases from 1945 to 2023,
involving 73 countries. This comprehensive dataset provides a rich
source of information for analysing trends and patterns in autocoup
attempts across different political contexts.

Table~\ref{tbl-autocoup_method} presents a breakdown of the autocoup
methods employed by leaders:

\begin{table}

\caption{\label{tbl-autocoup_method}Autocoup methods and success rates
(1945-2021)}

\centering{

\fontsize{12.0pt}{14.4pt}\selectfont
\begin{tabular*}{0.99\linewidth}{@{\extracolsep{\fill}}lccr}
\toprule
Autocoup Method & Attempted & Succeeded & Success Rate \\ 
\midrule\addlinespace[2.5pt]
Enabling re-election & 46 & 33 & 71.7\% \\ 
Removing term limits & 14 & 14 & 100.0\% \\ 
Delaying elections & 9 & 9 & 100.0\% \\ 
Leader for life & 9 & 9 & 100.0\% \\ 
Changing term length & 7 & 5 & 71.4\% \\ 
Figurehead & 6 & 5 & 83.3\% \\ 
One-time arrangement & 5 & 4 & 80.0\% \\ 
Refusing election results & 4 & 1 & 25.0\% \\ 
Reassigning power role & 4 & 2 & 50.0\% \\ 
Rigging elections & 3 & 2 & 66.7\% \\ 
Cancelling elections & 3 & 3 & 100.0\% \\ 
Total & 110 & 87 & 79.1\% \\ 
\bottomrule
\end{tabular*}
\begin{minipage}{\linewidth}
\emph{Source: Autocoup dataset}\\
\end{minipage}

}

\end{table}%

The most common autocoup method is ``enabling re-election'', accounting
for 46 events. This is followed by ``removing term limits'' (14 cases),
and then ``delaying elections'' and ``declaring the leader for life''
(each with 9 cases).

The overall success rate of autocoups is 79\%, which is significantly
higher than the approximately 50\% success rate of classical coups. This
high success rate can be attributed to several factors:

\begin{itemize}
\item
  Incumbent Advantage: Leaders already in power have access to resources
  and institutional mechanisms that can be leveraged to their advantage.
\item
  Gradual Implementation: Unlike sudden coups, autocoups can be
  implemented gradually, allowing leaders to build support and
  legitimacy over time.
\item
  Legal Facade: Many autocoup methods operate within a veneer of
  legality, making them harder to oppose openly.
\item
  Control of State Apparatus: Incumbents often have significant control
  over state institutions, which can be used to facilitate their
  autocoup attempts.
\end{itemize}

However, success rates vary significantly across different methods.

\begin{itemize}
\item
  \textbf{100\% success rate}: Removing term limits, delaying elections,
  declaring the leader for life, and cancelling elections all have
  perfect success rates. This suggests that once these processes are set
  in motion, they are difficult to reverse.
\item
  \textbf{Lower success rates:} Refusing to accept election results has
  the lowest success rate, with only 1 out of 4 attempts succeeding.
  Although the sample size is limited (only 4 cases in total), this
  trend might suggest several factors at play. These include greater
  democratic resilience in systems where general elections are regularly
  held, heightened international scrutiny and pressure in response to
  blatant manipulation of election results, and stronger domestic
  opposition to such overt power grabs.
\end{itemize}

\section{Determinants of autocoup attempts: Case
studies}\label{determinants-of-autocoup-attempts-case-studies}

\subsection{High frequency and success rate of autocoups in
post-communist
regimes}\label{high-frequency-and-success-rate-of-autocoups-in-post-communist-regimes}

Analysis of our dataset reveals a notably high frequency and success
rate of autocoups in post-communist countries. These nations, formerly
communist regimes prior to the collapse of the Soviet Union, have
largely evolved into `hybrid regimes'
(\citeproc{ref-nurumov2019}{Nurumov and Vashchanka 2019}), with only a
few retaining their communist status. The data documents 12 cases of
autocoups aimed at prolonging incumbency in these countries, with only
two attempts failing. Examination of these cases highlights several
distinctive characteristics:

\begin{itemize}
\item
  \textbf{Inherited authoritarian systems}: Despite most of these 12
  countries transitioning from communist to non-communist governments
  (with the exception of China), they retained many authoritarian
  systems from their communist past.
\item
  \textbf{Continuity of former elites}: The transitions did not result
  in the removal or overthrow of previous ruling groups. Instead, former
  communist elites often maintained their positions of power.
\item
  \textbf{Subverted democratic processes}: While general elections and
  term limits were introduced in most of these countries, the legacy of
  former communist regimes frequently led to the circumvention of term
  limits and manipulation of elections
  (\citeproc{ref-nurumov2019}{Nurumov and Vashchanka 2019}).
\end{itemize}

\subsubsection*{Case 1: Lifelong ruler--Alexander Lukashenko in
Belarus}\label{case-1-lifelong-ruleralexander-lukashenko-in-belarus}
\addcontentsline{toc}{subsubsection}{Case 1: Lifelong ruler--Alexander
Lukashenko in Belarus}

Alexander Lukashenko, a former member of the Supreme Soviet of the
Byelorussian Soviet Socialist Republic, became the head of the interim
anti-corruption committee of the Supreme Council of Belarus following
the dissolution of the Soviet Union. Elected as Belarus's first
president in 1994, he has maintained this position ever since.
Initially, the 1994 constitution limited presidents to two successive
terms. However, Lukashenko removed this restriction in 2004.
International monitors have not regarded Belarusian elections as free
and fair since his initial victory. Despite significant protests,
Lukashenko has consistently claimed to win with a high vote share, often
exceeding 80\% in each election. This pattern is evident across all five
Central Asian countries of the former Soviet Union, where
post-dissolution leaders were typically high officials or heads of the
former Soviet republics who continued their leadership in the
presidency.

\subsubsection*{Case 2: Transferring power to a handpicked
successor--Nursultan Nazarbayev in
Kazakhstan}\label{case-2-transferring-power-to-a-handpicked-successornursultan-nazarbayev-in-kazakhstan}
\addcontentsline{toc}{subsubsection}{Case 2: Transferring power to a
handpicked successor--Nursultan Nazarbayev in Kazakhstan}

Nursultan Nazarbayev served as the first president of Kazakhstan from
1991 until 2019. Prior to the dissolution of the Soviet Union, he held
de facto leadership as the First Secretary of the Communist Party of
Kazakhstan. Following independence, he was elected as the first
president and retained office until 2019 through various means,
including resetting term limits due to the implementation of new
constitutions. Notably, Nazarbayev did not officially eliminate term
limits but instead created an exemption for the ``First President''
(\citeproc{ref-nurumov2019}{Nurumov and Vashchanka 2019}). Unlike
Lukashenko, who remains the incumbent of Belarus, Nazarbayev transferred
the presidency to a designated successor, Kassym-Jomart Tokayev, in
2019. However, he retained significant influence as the Chairman of the
Security Council of Kazakhstan until 2022.

\subsection{Autocoups for immediate re-election: Cases of Latin
America}\label{autocoups-for-immediate-re-election-cases-of-latin-america}

Latin America has a long-standing tradition of maintaining term limit
conventions. Simón Bolívar, the founding father of Bolivia, was
initially a strong advocate for term limits, stating in 1819, ``Nothing
is as dangerous as allowing the same citizen to remain in power for a
long time\ldots{} That's the origin of usurpation and tyranny''
(\citeproc{ref-ginsburg2019}{Ginsburg and Elkins 2019, 38}). Although
Bolívar eventually modified his stance, arguing in his 1826 Constitution
Assembly speech that ``a president for life with the right to choose the
successor is the most sublime inspiration for the republican order,''
term limits became a convention in Latin America. Approximately 81\% of
Latin American constitutions between independence and 1985 imposed some
form of term limits on the presidency
(\citeproc{ref-marsteintredet2019a}{Marsteintredet 2019}).

An analysis of cases in Latin American countries reveals two notable
patterns.

\subsubsection*{Often successful at breaking non-re-election or
non-immediate re-election
restrictions}\label{often-successful-at-breaking-non-re-election-or-non-immediate-re-election-restrictions}
\addcontentsline{toc}{subsubsection}{Often successful at breaking
non-re-election or non-immediate re-election restrictions}

Unlike other presidential systems where two terms are more common,
non-re-election or non-immediate re-election used to be prevalent in
Latin America. According to Marsteintredet
(\citeproc{ref-marsteintredet2019a}{2019}), non-consecutive re-election
was mandated in about 64.9\% of all constitutions between independence
and 1985, while 5.9\% banned re-election entirely.

However, adherence to these conventions has varied across the region.
Since Mexico introduced non-re-election institutions in 1911 at the
start of the Mexican Revolution, they have remained inviolate
(\citeproc{ref-klesner2019}{Klesner 2019}). Similarly, Panama and
Uruguay have never altered their re-election rules, and Costa Rica has
only experienced a brief period (1897-1913) permitting immediate
presidential re-election since prohibiting it in 1859
(\citeproc{ref-marsteintredet2019a}{Marsteintredet 2019}). In many other
countries, however, constitutions have been frequently amended or
violated.

The pursuit of re-election or consecutive re-election, therefore, has
been a significant trigger for autocoups aimed at power extension in
this region. Our research documents 32 autocoup cases, with over 50\%
(17 cases) attempting to enable re-election or immediate re-election,
and about 59\% (10 cases out of 17) being successful.

Unlike those who attempt to overstay in office indefinitely, many Latin
American leaders exit after their second term expires. Examples include
President Fernando Henrique Cardoso of Brazil (1995-2003), President
Danilo Medina of the Dominican Republic (2012-2020), and President Juan
Orlando Hernández of Honduras (2014-2022)
(\citeproc{ref-ginsburg2019}{Ginsburg and Elkins 2019};
\citeproc{ref-marsteintredet2019a}{Marsteintredet 2019};
\citeproc{ref-landau2019}{Landau, Roznai, and Dixon 2019};
\citeproc{ref-baturo2019}{Baturo 2019}; \citeproc{ref-neto2019}{Neto and
Acácio 2019}).

\subsubsection*{Failing to further extend
tenure}\label{failing-to-further-extend-tenure}
\addcontentsline{toc}{subsubsection}{Failing to further extend tenure}

This trend does not imply that none of these leaders attempted further
extensions, but rather that most accepted their unsuccessful outcomes
without abusing their power to manipulate the process. While autocoups
aimed at securing one additional term are often successful, attempts to
overstay beyond this are frequently unsuccessful.

In contrast to the previous examples, two contrasting cases illustrate
the varied outcomes of term limit challenges:

\begin{itemize}
\item
  \textbf{Unsuccessful extension -- Carlos Menem (Argentina)}: President
  Menem successfully extended his tenure by one term through a 1994
  constitutional amendment allowing one executive re-election. He was
  subsequently re-elected in 1995. However, his attempt to reset his
  term count, arguing that his first term (1988-1995) should not count
  as it was under previous constitutions, was unanimously rejected by
  the Supreme Court in March 1999 (\citeproc{ref-llanos2019}{Llanos
  2019}). A similar scenario unfolded with President Álvaro Uribe of
  Colombia (2002-2010) (\citeproc{ref-baturo2019}{Baturo 2019}).
\item
  \textbf{Successful extension -- Daniel Ortega (Nicaragua)}: In
  contrast, Daniel Ortega, the incumbent president of Nicaragua,
  successfully extended his presidency. In 2009, the Supreme Court of
  Justice of Nicaragua permitted his re-election in 2011. Subsequently,
  in 2014, the National Assembly of Nicaragua approved constitutional
  amendments abolishing presidential term limits, allowing Ortega to run
  for an unlimited number of five-year terms. As a result, he has held
  the presidency since 2007 (\citeproc{ref-close2019}{Close 2019}).
\end{itemize}

\subsection{As common as classical coups: Cases of African
countries}\label{as-common-as-classical-coups-cases-of-african-countries}

Classical coups have been prevalent in Africa, accounting for
approximately 45\% of all global coups (219 out of 491 cases) since
1950, involving 45 out of 54 African countries (GIC dataset). While
autocoups are less frequent compared to traditional coups, they maintain
a significant presence in Africa. Among 110 documented autocoup cases
globally, 46\% (51 cases) occurred in Africa, involving 36 countries.
Notably, the success rate of autocoups in Africa is over 84\% (43 out of
51 attempts), which surpasses both the success rate of classical coups
in the region (roughly 50\%) and the global average success rate of
autocoups (79\%).

Identifying a clear pattern of autocoups in Africa is challenging,
mirroring the complexity observed with classical coups. Various factors
have been proposed to explain this phenomenon:

\begin{itemize}
\item
  \textbf{Natural Resources}: Countries rich in natural resources,
  particularly oil or diamonds, may see leaders more likely to attempt
  and succeed in extending their terms (\citeproc{ref-posner}{Posner and
  Young, n.d.}; \citeproc{ref-cheeseman2015}{Cheeseman 2015};
  \citeproc{ref-cheeseman2019a}{Cheeseman and Klaas 2019}).
\item
  \textbf{Quality of democracy}: The quality of democracy is a critical
  factor influencing respect for term limits
  (\citeproc{ref-reyntjens2016}{Reyntjens 2016}).
\item
  \textbf{International influence}: International aid or donor influence
  can play a significant role in discouraging attempts at power
  extension (\citeproc{ref-brown2001}{Brown 2001};
  \citeproc{ref-tangri2010}{Tangri and Mwenda 2010}).
\item
  \textbf{Organized opposition and party unity}: The extent of organized
  opposition and the president's ability to enforce unity within the
  ruling party are crucial factors
  (\citeproc{ref-cheeseman2019}{Cheeseman 2019}).
\end{itemize}

Utilizing the Africa Executive Term Limits (AETL) dataset, Cassani
(\citeproc{ref-cassani2020}{2020}) highlights human rights abuses and
the desire for impunity as main drivers for incumbents to cling to
power. The more authoritarian a leader, the more likely they are to
attempt to break term limits and overstay in office. A leader's ability
to secure the loyalty of the armed forces through public investment
increases the chances of success in overstaying.

Despite both coups and autocoups being prevalent, there has been a
noticeable shift since the end of the Cold War in 1991: Traditional
coups have decreased in frequency while autocoups have become more
prevalent.

This trend can be partially attributed to the introduction of
multi-party elections in Africa in the 1990s, which also brought in term
limits for executives (\citeproc{ref-cassani2020}{Cassani 2020};
\citeproc{ref-cheeseman2019}{Cheeseman 2019}). Before 1991, personal or
military rule was more common, and term limits were less frequent.
Post-1991, with more term limits introduced, challenges to these limits
have increased. However, it is crucial to note that this increase in
challenges does not necessarily imply that violations are more common
than adherence to term limits, because total power transitions have
increased compared to the past.

\section{Empirical analysis: An example of utilizing the autocoup
dataset}\label{empirical-analysis-an-example-of-utilizing-the-autocoup-dataset}

The availability of the autocoup dataset has made it feasible to conduct
quantitative analyses that extend beyond traditional case studies. This
section provides a straightforward example of how to utilize this
dataset effectively. To analyze the determinants of autocoup attempts, I
employ a probit regression model. This approach differs from the double
probit model with sample selection used in Chapter~\ref{sec-chapter2}
for coup attempts and success analyses. Instead, I use two separate
probit models. Due to the high probability of success in autocoups, they
do not exhibit the typical sample selection characteristics that
necessitated the use of a sample selection model in our earlier analysis
of traditional coups.

\subsection{Dependent variables}\label{dependent-variables}

\begin{itemize}
\item
  \textbf{Autocoup attempt}: Binary variable indicating whether an
  autocoup attempt occurred (1) or not (0) during the tenure of an
  incumbent leader.
\item
  \textbf{Autocoup success}: Binary variable indicating whether an
  autocoup attempt was successful (1) or failed (0), conditional on an
  autocoup attempt occurring.
\end{itemize}

\subsection{Independent variables}\label{independent-variables}

The selection of independent variables are consistent with the coup
analysis in Chapter~\ref{sec-chapter2}, plus the population size and the
leader's age.

\begin{itemize}
\item
  \textbf{Population size:} To account for its potential impact on
  leaders' tenures, we consider the log of the population size. This
  transformation helps in managing the wide range of population sizes
  across different countries. The data is sourced from the V-Dem dataset
  and is evaluated to understand its influence on power transitions.
  Larger populations may present more governance challenges and
  potential sources of opposition, thereby affecting the stability and
  longevity of a leader's tenure.
\item
  \textbf{Leader's age:} The age of the leader is included as an
  additional variable in the analysis, offering insights into potential
  correlations with leadership strength. Older leaders may have
  different experiences, networks, and health considerations that could
  influence their ability to maintain power. This data is sourced from
  Archigos and PLAD datasets.
\end{itemize}

Unlike the analysis of coup determinants, which could theoretically
occur in any given year, I assume that an autocoup happens only once
during an incumbent leader's tenure, as a successful autocoup negates
the need for another attempt. However, this assumption does not always
reflect reality, as leaders might attempt further extensions or try
again after a failed attempt. For simplicity, I overlook these
possibilities in our analysis.

Therefore, in our probit model, the unit of analysis for autocoups is
the entire tenure of a leader, rather than a country-year. I establish a
base year for the variables: for leaders who staged an autocoup, we use
the year of their first attempt as the base year; for leaders who did
not attempt to overstay, I use the middle year of their tenure as the
base year.

\subsection{Results and discussions}\label{results-and-discussions}

\begin{table}

\caption{\label{tbl-autocoupmodel}Determinants of autocoup attempts and
success (1945-2018)}

\centering{

\begin{tabular}{@{\extracolsep{50pt}}lcc} 
\\[-1.8ex]\hline 
\hline \\[-1.8ex] 
 & Autocoup Attempts & Autocoup Outcome \\ 
\\[-1.8ex] & (1) & (2)\\ 
\hline \\[-1.8ex] 
 Constant & $-$1.674$^{***}$ & $-$0.888 \\ 
  & (0.624) & (1.935) \\ 
  & & \\ 
 Regime: Dominant-party & 0.070 & 0.672$^{*}$ \\ 
  & (0.145) & (0.402) \\ 
  & & \\ 
 \hspace{1.6cm}Military & $-$0.255 & 0.615 \\ 
  & (0.189) & (0.541) \\ 
  & & \\ 
 \hspace{1.6cm}Personalist & 0.737$^{***}$ & 1.609$^{***}$ \\ 
  & (0.157) & (0.448) \\ 
  & & \\ 
 GDP per capita & $-$0.009 & 0.064 \\ 
  & (0.011) & (0.045) \\ 
  & & \\ 
 Economic trend & 0.653 & 0.197 \\ 
  & (0.533) & (1.772) \\ 
  & & \\ 
 Political stability & $-$0.044 & 0.126 \\ 
  & (0.036) & (0.130) \\ 
  & & \\ 
 Age & $-$0.001 & 0.004 \\ 
  & (0.001) & (0.017) \\ 
  & & \\ 
 Population(log) & $-$0.048 & 0.029 \\ 
  & (0.042) & (0.144) \\ 
  & & \\ 
\hline \\[-1.8ex] 
Observations & 1,028 & 102 \\ 
Log Likelihood & $-$308.494 & $-$43.651 \\ 
Akaike Inf. Crit. & 634.988 & 105.302 \\ 
\hline 
\hline \\[-1.8ex] 
\textit{Note:}  & \multicolumn{2}{r}{$^{*}$p$<$0.1; $^{**}$p$<$0.05; $^{***}$p$<$0.01} \\ 
\end{tabular}

}

\end{table}%

Table~\ref{tbl-autocoupmodel} summarizes the findings from the probit
regression models based on our analysis of the determinants of autocoup
attempts and their success.

Model 1, which examines autocoup attempts, reveals only one significant
predictor besides the constant term. Among the regime types, personalist
regimes significantly increase the likelihood of autocoup attempts, all
else being equal. This suggests that leaders in personalist regimes are
more prone to attempt to extend their power through autocoups compared
to leaders in democratic regimes (reference regime). Leaders in
dominant-party and military regimes, however, show no significant
difference in the likelihood of attempting an autocoup compared to
democratic leaders.

The model for autocoup success (Model 2) shows similar dynamics.
Personalist regimes again have a strong positive and significant effect
on the success of autocoups compared to democratic leaders.
Dominant-party regimes also show a positive and marginally significant
effect. However, a detailed examination reveals that about half of the
successful autocoups in dominant-party regimes (9 out of 20) exhibit a
personalist style, such as ``party-personal-military'' regimes.

This outcome is logical since personalist leaders are typically much
more powerful than other types of leaders, making them more inclined and
capable of overstaying in power.

Other factors play an insignificant role in determining the attempts and
outcomes of autocoups. This aligns with our conclusions on the
determinants of classic coups. Both coups and autocoups are
significantly affected by power dynamics. As power transitions involve
the struggle between seizing and maintaining power, the balance of power
status quo inevitably matters in both coups and autocoups. This also
explains the high success rate of autocoups. Compared to power
challengers, incumbents are in an obviously advantageous position.
Incumbent leaders can use state power to their benefit, which is
difficult to counteract. Even the abuse of power is often unchecked
under a powerful leader's rule.

The empirical analysis of autocoups yields significant implications for
real-world politics. In particular, the high overall success rate of
autocoups highlights the vulnerability of democratic institutions to
gradual erosion by incumbent leaders. The threshold for ousting or
impeaching an incumbent leader through constitutional means is
exceptionally high, with success often requiring more than a simple
majority and substantial support across various sectors. Resorting to
illegal means, such as a coup, presents even greater challenges due to
high costs, severe consequences, and a low likelihood of success.

Conversely, political dynamics, whether in democracies or autocracies,
tend to favour incumbents even when they act unconstitutionally.
Incumbents can leverage state resources to achieve their political
ambitions, benefiting from a high probability of success and minimal
consequences in case of failure. This asymmetry in power and risk
creates a concerning scenario: for incumbents who do not respect
constitutional institutions, the opportunity to launch an autocoup
appears sufficiently low-risk to warrant an attempt.

\section{Summary}\label{summary-1}

This chapter conducts a thorough and comprehensive analysis of
autocoups, with a specific focus on political events where incumbent
leaders illegitimately extend their tenure in power. By refining the
existing definition and distinguishing autocoups from related concepts
such as ``self-coups,'' ``autogolpes,'' and ``executive takeovers,''
this research introduces a novel dataset that catalogues autocoups from
1945 to 2023. This refined definition and the accompanying dataset
enable the study to broaden its analysis of irregular leadership
transitions. While traditional analyses often concentrate on the abrupt
termination of tenure through coups, this research expands the scope to
include the irregular extension of tenure through autocoups. This
approach provides a more comprehensive and nuanced understanding of the
phenomenon, highlighting the various mechanisms by which incumbent
leaders can subvert democratic processes to maintain their power.

The findings reveal that personalist regimes are significantly more
likely to experience autocoup attempts and succeed in these attempts
compared to democracies. Dominant-party systems, often exhibiting
personalist characteristics, also show an association with successful
autocoups. While regime type significantly influences autocoups, other
factors appear less impactful, mirroring classic coups where the balance
of power is a more essential determinant. The high success rate of
autocoups can be attributed to the inherent advantages incumbents
possess, such as control over or abuse of state power and the difficulty
of removing or impeaching them through legal or illegal means.

However, several limitations warrant consideration for future research.
Firstly, the definition of an autocoup requires further commentary and
discussion to gain wider acceptance in the academic community. Despite
efforts to maintain objectivity, some coding decisions may involve
subjective judgments, particularly in borderline cases. Secondly, due to
the nature of autocoups, which are less frequent than classic coups (491
coups versus 110 autocoups during the same period), the quantitative
analysis cannot be conducted as a country-year variable as in coup
studies. This raises the issue of choosing an appropriate base year for
the analysis, which requires further discussion and potentially
sensitivity analyses.

Despite these limitations, this research significantly enhances our
understanding of the mechanisms and motivations behind autocoups,
contributing to the literature on political stability and democratic
resilience. The findings highlight the vulnerability of political
systems, particularly democracies, to erosion from within by incumbent
leaders.

Future studies could build on this work by employing the dataset to
explore more nuanced power dynamics or examine the long-term impacts of
these events on political systems. Particularly fruitful areas for
investigation include the relationship between autocoups and democratic
backsliding, democratic breakdown, and the personalization of power.
Additionally, comparative analyses between autocoups and traditional
coups could yield insights into the evolving nature of power
consolidation strategies in different political contexts.

In conclusion, this study not only provides a valuable resource for
future research but also contributes to our understanding of the complex
interplay between leadership, institutional structures, and political
stability. As autocratic tendencies continue to challenge democratic
norms globally, the insights gained from this analysis of autocoups
become increasingly relevant for both scholars and policy-makers
concerned with preserving and strengthening democratic institutions.

\chapter{Power Acquisition and Leadership Survival: A Comparative
Analysis of coup-installed and Autocoup
Leaders}\label{power-acquisition-and-leadership-survival-a-comparative-analysis-of-coup-installed-and-autocoup-leaders}

\section*{Abstract}\label{abstract-3}
\addcontentsline{toc}{section}{Abstract}

This chapter investigates the intricate relationship between the methods
of power acquisition and the tenure of leaders who ascend to power
through unconventional means, focusing on coup-installed and autocoup
leaders. The central hypothesis posits that the mode of accession has a
profound impact on leadership longevity. Utilizing Cox proportional
hazards and time-dependent Cox models, this study provides robust
empirical evidence of disparate survival times between these two leader
types. The findings reveal that, on average, coup-installed leaders are
2.23 times more likely to be ousted from power than autocoup leaders,
all else being equal. These results have far-reaching implications for
political stability and democratic processes, suggesting that the
perceived low costs and high rewards associated with autocoups may
incentivize incumbents to prolong their tenure through this means,
potentially contributing to democratic erosion. This research makes a
notable contribution to the academic literature by offering nuanced
insights into the dynamics of irregular leadership transitions and
enhances our understanding of the complex interplay between power
acquisition methods and leadership longevity.

\emph{\textbf{keywords}: Coups, Autocoups, Leadership Survival, Cox
Model}

\newpage

\section{Introduction}\label{introduction-3}

The enduring fascination with the longevity of political leaders has
sparked extensive research in political science, with scholars seeking
to understand why some leaders maintain power for decades while others
are ousted in a matter of months or even days. Despite this broad
interest, a specific subset of leaders---those who ascend to power
through coups or extend their tenure through autocoups---has received
relatively limited attention. Examining the tenures of these leaders is
crucial, as it sheds light on the dynamics of irregular leadership
transitions and their implications for political stability and
democratic processes.

In contrast to leaders who attain power through conventional means,
those who rise through irregular channels, such as coups or autocoups,
present more complex and intriguing cases for study. The Archigos
dataset highlights the prevalence of irregular power transitions.
Between 1945 and 2015, over half of leaders who assumed power
irregularly also exited irregularly, a rate significantly higher than
that of leaders who accessed office through regular channels.

Coup-installed and autocoup leaders constitute a substantial portion of
these irregular cases. The Archigos dataset notes that of 374 leaders
who exited irregularly, 246 (65.8\%) were ousted through coups.
Furthermore, research by Frantz and Stein
(\citeproc{ref-frantz2016}{2016}) demonstrates that coup-related exits
account for approximately one-third of all exits in autocracies,
surpassing any other transition type. Additionally, the autocoup
dataset, introduced in \hyperref[sec-chapter3]{Chapter 3}, documents 110
autocoup attempts between 1945 and 2023, of which 87 were successful.

Measuring the survival tenure of coup-installed and autocoup leaders
poses challenges due to the inherent irregularity and uncertainty of
their positions. Nevertheless, a comparative analysis reveals that
leaders who extend power through autocoups tend to have longer average
post-autocoup tenures (approximately 11 years) compared to
coup-installed leaders (approximately 5.7 years), suggesting a potential
tenure gap of over five years.

\begin{figure}

\centering{

\includegraphics{_coups_and_autocoups_files/figure-pdf/fig-logrank-1.pdf}

}

\caption{\label{fig-logrank}Survival curves of autocoup and
coup-installed leaders}

\end{figure}%

A preliminary log-rank test in survival analysis, as illustrated in
Figure~\ref{fig-logrank}, demonstrates a statistically significant
difference between the tenures of autocoup and coup-installed leaders.
The survival curve for autocoup leaders consistently exceeds that of
coup-installed leaders, indicating longer survival times and a reduced
risk of ouster for autocoup leaders.

This study posits that the method of accession significantly influences
leadership longevity. Coup-installed leaders likely confront greater
challenges to their rule, resulting in shorter average tenures compared
to autocoup leaders. The analysis, employing Cox proportional hazards
and time-dependent Cox models, supports this hypothesis, demonstrating
that autocoup leaders generally experience longer tenures than
coup-installed leaders.

This research offers two primary contributions to the field. First, it
highlights an understudied factor in leadership survival analysis: the
impact of the method of accession to power. The findings suggest that
leader survival is influenced not only by ruling strategies but also by
the initial method of acquiring power. Second, by employing survival
models, this study provides empirical evidence of the significant
difference in tenure duration between autocoup and coup-installed
leaders. This insight may explain the increasing prevalence of tenure
extensions through autocoups since 2000, as more incumbents observe and
potentially emulate successful precedents.

The remainder of this chapter is structured as follows: Section 2
provides a comprehensive literature review on political survival,
establishing the context for this research. Section 3 explores the
factors influencing the survival of coup and autocoup leaders. Section 4
outlines the methodology and data used, including the application of
survival models to analyze the determinants of leadership longevity.
Section 5 presents the analysis findings and a detailed discussion of
the results. Finally, Section 6 concludes by synthesizing key takeaways
and exploring their broader implications for political stability and
democratic processes.

\section{Literature review}\label{literature-review}

The longevity of political leaders, marked by wide-ranging variations
across regimes, countries, and historical periods, has captivated
political science research for decades. This field encompasses two
interconnected aspects: regime survival and individual leader survival.
Regime survival focuses on the endurance of political systems, such as
monarchies, political parties, or specific ideological structures, while
leader survival concerns the duration of individual leaders' time in
office.

Political survival patterns vary widely across different systems.
Parliamentary democracies (e.g., Japan, United Kingdom) often experience
prolonged periods of party dominance coupled with frequent leadership
changes. Similarly, communist regimes (e.g., China) typically
demonstrate enduring party rule with more frequent leadership
transitions. Presidential systems (e.g., United States) and many
military regimes tend to exhibit more frequent changes in both ruling
party or junta and leader.

The existing literature on leader survival is extensive and
multifaceted. Some studies explore specific mechanisms influencing
leadership longevity within particular regimes, such as democracies
(\citeproc{ref-svolik2014}{Svolik 2014}) or autocracies
(\citeproc{ref-davenport2021}{Davenport, RezaeeDaryakenari, and Wood
2021}). Others aim to develop more generalizable theoretical frameworks
explaining leader survival across different political systems
(\citeproc{ref-buenodemesquita2003}{Bueno de Mesquita et al. 2003}).
While a universal theory remains an aspirational goal, the complexities
of leadership survival across diverse regime types present significant
challenges.

Power transition mechanisms vary substantially across different types of
regimes, particularly between democracies and autocracies. Autocratic
systems often feature closed leadership selection processes, restricted
to a narrow pool of individuals. While some autocracies may hold
elections, significant barriers to entry for legitimate challengers
typically persist. The opacity of selection processes in autocracies
makes it difficult to assess genuine levels of public support compared
to democracies. Conceptualizing selectorates or winning coalitions, as
proposed by Bueno de Mesquita et al.
(\citeproc{ref-buenodemesquita2003}{2003}), becomes problematic in many
autocratic contexts.

Given these complexities, focusing research on specific regimes or
leader types may be more fruitful. The study of irregular leaders, such
as those who ascend to power through coups or extend their tenures
through autocoups, offers a compelling avenue for research due to the
inherent complexities and uncertainties surrounding their leadership
trajectories.

Two primary perspectives have emerged to explain the dynamics of leader
survival. The first emphasizes objective factors and resources, such as
personal competence (\citeproc{ref-yu2016}{Yu and Jong-A-Pin 2016}),
societal stability (\citeproc{ref-arriola2009}{Arriola 2009}), economic
development (\citeproc{ref-palmer1999}{Palmer and Whitten 1999};
\citeproc{ref-williams2011}{Williams 2011}), natural resource endowments
(\citeproc{ref-smith2004}{Smith 2004};
\citeproc{ref-quirozflores2012}{Quiroz Flores and Smith 2012};
\citeproc{ref-wright2013}{Wright, Frantz, and Geddes 2013}), and
external support (\citeproc{ref-licht2009}{Licht 2009};
\citeproc{ref-wright2008}{Wright 2008}; \citeproc{ref-thyne2017}{C.
Thyne et al. 2017}). The second focuses on subjective factors and
strategies, including political policies, responses to opposition, and
tactics for consolidating power (\citeproc{ref-gandhi2007}{Gandhi and
Przeworski 2007}; \citeproc{ref-morrison2009}{Morrison 2009};
\citeproc{ref-escribuxe0-folch2013}{Escribà-Folch 2013};
\citeproc{ref-davenport2021}{Davenport, RezaeeDaryakenari, and Wood
2021}).

Coups, a significant aspect of irregular leadership transitions, have
received considerable scholarly attention. Research has examined coup
prevention strategies (\citeproc{ref-powell2017}{J. Powell 2017};
\citeproc{ref-sudduth2017}{Sudduth 2017}; \citeproc{ref-debruin2020}{De
Bruin 2020}). Studies have explored the impact of coups on leadership
and the subsequent actions of coup leaders Easton and Siverson
(\citeproc{ref-easton2018}{2018}).

However, a significant gap remains in the literature regarding the
comparison of leadership survival between coup-installed and autocoup
leaders. This study aims to address this gap by investigating and
comparing the duration of leadership survival for these two leader
types.

By focusing on the comparison between coup-installed and autocoup
leaders, this study seeks to contribute to a more nuanced understanding
of political survival in irregular leadership transitions. This approach
may offer valuable insights into the complex dynamics of leader
longevity across different political contexts.

\section{Survival dynamics of autocoup and coup-installed
leaders}\label{survival-dynamics-of-autocoup-and-coup-installed-leaders}

The study of leadership survival in political systems presents inherent
challenges due to the opacity and diverse mechanisms of power
transitions. These challenges, however, underscore the significance of
this research, as it illuminates understudied dynamics in political
leadership. While the survival of political leaders exhibits complexity
and variation, it is not entirely devoid of patterns. Leaders of similar
types often display significant comparability, providing a foundation
for meaningful analysis.

\subsection{Key definitions and scope}\label{key-definitions-and-scope}

Before delving into the comparison, it is essential to clarify several
key terminologies:

\begin{itemize}
\item
  \textbf{Coup and autocoup}: These terms adhere to the definitions
  established in previous chapters, maintaining consistency throughout
  the study.
\item
  \textbf{Tenure length threshold}: To ensure meaningful analysis, this
  study focuses on leaders with substantial periods in power, applying a
  six-month threshold to both autocoup and coup-installed leaders. This
  criterion filters out ephemeral leadership episodes, allowing for a
  more robust examination of survival dynamics.
\item
  \textbf{Autocoup leader}: An incumbent leader who successfully employs
  illegitimate or unconstitutional means to extend their tenure in
  power. This definition encompasses various methods of power
  consolidation that circumvent established democratic processes or
  constitutional limits.
\item
  \textbf{Coup-installed leader}: The individual who assumes power after
  a successful coup, regardless of their role in the coup itself. This
  broad definition allows for the inclusion of both coup instigators and
  those selected to lead post-coup, providing a comprehensive view of
  leadership dynamics following forceful regime change.
\end{itemize}

This study focuses on comparing the post-autocoup tenure of autocoup
leaders with the post-coup tenure of coup-installed leaders. This
comparative approach is motivated by the relevance and similarity of
these leader types in terms of illegitimacy, uncertainty, and
instability. By examining these parallel yet distinct paths to power, we
can gain insights into the factors that influence leadership longevity
in irregular leadership transitions.

\subsection{Challenges in power
consolidation}\label{challenges-in-power-consolidation}

Both autocoup and coup-installed leaders confront distinct challenges in
consolidating their power, primarily stemming from the varying intensity
of issues related to illegitimacy, uncertainty, and instability. This
disparity creates an uneven playing field in terms of power dynamics,
placing coup-installed leaders at a significant disadvantage.
Table~\ref{tbl-leaders} provides a comparative overview of the main
features of autocoup and coup-installed leaders, highlighting these key
differences.

\blandscape

\begin{table}

\caption{\label{tbl-leaders}Main features of autocoup and coup-installed
leaders}

\centering{

\fontsize{12.0pt}{14.4pt}\selectfont
\begin{tabular*}{1\linewidth}{@{\extracolsep{\fill}}>{\raggedright\arraybackslash}p{\dimexpr 112.50pt -2\tabcolsep-1.5\arrayrulewidth}>{\raggedright\arraybackslash}p{\dimexpr 225.00pt -2\tabcolsep-1.5\arrayrulewidth}>{\raggedright\arraybackslash}p{\dimexpr 225.00pt -2\tabcolsep-1.5\arrayrulewidth}}
\toprule
Feature & Autocoup Leader & Coup Entry Leader \\ 
\midrule\addlinespace[2.5pt]
Illegitimacy & Normally attained through
lawful procedures, but
lacking consensus
legitimacy & Blatantly illegal \\ 
Uncertainty & Initially with some certainty, but decreases as the leader's age grows or health worsens & Significant uncertainty initially \\ 
Instability & Relatively stable & Unstable except when a strongman emerges or constitutional institutions are established \\ 
Balance of Power & Generally in a better position of power & Initially unclear and challenging to establish a balance \\ 
\bottomrule
\end{tabular*}

}

\end{table}%

\elandscape

\subsubsection*{Illegitimacy}\label{illegitimacy}
\addcontentsline{toc}{subsubsection}{Illegitimacy}

While both types of leaders suffer from a legitimacy deficit, the nature
and perception of this deficit differ significantly:

\begin{itemize}
\item
  \textbf{Coup-installed leaders}: Their illegitimacy is blatant and
  unambiguous, stemming from the overt and often violent seizure of
  power. This brazen act undermines pre-existing norms and institutions,
  generating immediate domestic and international condemnation.
\item
  \textbf{Autocoup leaders}: In contrast, autocoup leaders employ a more
  subtle and deceptive strategy, manipulating legal processes and
  institutions to create a façade of democratic legitimacy. This veneer
  of legality, while often thin, can provide a degree of cover and buy
  time for consolidating power.
\end{itemize}

\subsubsection*{Uncertainty}\label{uncertainty}
\addcontentsline{toc}{subsubsection}{Uncertainty}

The irregular paths to power inherent to both leadership types
inevitably generate uncertainty regarding the longevity of their reigns
and the mechanisms of their eventual departures. However, the levels and
sources of uncertainty differ significantly:

\begin{itemize}
\item
  \textbf{Coup-installed leaders:} These leaders face a trifecta of
  uncertainties. First, the immediate aftermath of a coup often involves
  a struggle for power within the junta or ruling coalition, creating
  ambiguity about who will ultimately consolidate control. Second, the
  tenure of coup-installed leaders is inherently precarious, subject to
  internal rivalries, popular uprisings, or counter-coups. Third, the
  lack of established succession mechanisms further amplifies
  uncertainty, making it difficult to predict the transfer of power and
  potentially triggering future instability.
\item
  \textbf{Autocoup leaders:} While not immune to uncertainty, autocoup
  leaders generally present a clearer picture. The question of who will
  rule post-autocoup is largely settled, as the incumbent retains power.
  Furthermore, many autocoup leaders openly aspire to extend their rule
  indefinitely or incrementally, attempting to establish a sense of
  permanence. This perceived stability, whether real or manufactured,
  can contribute to a more predictable political environment, at least
  in the short term.
\end{itemize}

\subsubsection*{Instability}\label{instability}
\addcontentsline{toc}{subsubsection}{Instability}

The awareness of shaky legitimacy and persistent uncertainty inevitably
breeds insecurity and a sense of crisis, forcing both autocoup and
coup-installed leaders to prioritize stabilization measures. However,
the nature and intensity of these challenges differ:

\begin{itemize}
\item
  \textbf{Coup-installed leaders:} These leaders face the daunting task
  of rapidly reshaping power dynamics, often resorting to purges and
  crackdowns to eliminate potential adversaries and consolidate control.
  This process of dismantling existing structures and building new ones
  generates significant instability, potentially alienating former
  allies and triggering resistance from various segments of society. The
  need to appease powerful actors both domestically and internationally
  further limits their options, forcing them into compromises that can
  undermine their authority and long-term stability.
\item
  \textbf{Autocoup leaders:} In contrast, autocoup leaders often benefit
  from a degree of continuity in regime personnel and institutions. This
  relative stability allows them to implement changes gradually,
  minimizing disruptions and mitigating potential backlash. While they
  may still face opposition, they are less likely to confront immediate
  and existential threats to their rule, providing them with more time
  and leverage to consolidate power.
\end{itemize}

By understanding these contrasting challenges, we can better appreciate
the relative advantages and disadvantages faced by autocoup and
coup-installed leaders. This comparative perspective provides a nuanced
framework for analyzing the strategies these leaders employ to
consolidate power and navigate the perilous terrain of irregular
leadership transitions.

\subsection{Empirical evidence and
hypothesis}\label{empirical-evidence-and-hypothesis}

Empirical evidence substantiates the disadvantage faced by
coup-installed leaders, revealing a complex interplay between historical
precedent, power consolidation challenges, and leadership longevity.
This section presents key data points and introduces the central
hypothesis guiding this study.

Data analysis reveals a striking correlation between the frequency of
coup attempts in a country and the likelihood of future coups. Notably,
over a third of coups occur in the top ten countries with the most
attempts since 1950 (Table~\ref{tbl-coups}). This pattern suggests a
self-reinforcing cycle of political instability, where each successful
coup increases the probability of subsequent attempts, creating an
environment of persistent uncertainty for coup-installed leaders.

The disparity in leadership longevity between autocoup and
coup-installed leaders is starkly illustrated by survival data. As
depicted in Figure~\ref{fig-logrank}, the average survival period
following an autocoup is approximately five years longer than that of
coup-installed leaders. This substantial difference in tenure length
underscores the divergent challenges faced by these two types of leaders
in maintaining their grip on power.

The distinct challenges faced by autocoup leaders and coup-installed
leaders in consolidating power create a self-perpetuating cycle that
significantly influences their tenure length:

\begin{itemize}
\item
  Coup-installed leaders: Face greater legitimacy challenges and
  internal instability; Struggle to attract and retain strong support;
  More vulnerable to internal and external challenges; Shorter average
  tenures reinforce perception of instability.
\item
  Autocoup leaders: Often benefit from a veneer of legitimacy and a
  stronger initial position; Better able to consolidate power and
  attract supporters; Face less immediate threat of overthrow; Longer
  average tenures contribute to perception of stability.
\end{itemize}

This cycle suggests that the initial method of power acquisition or
extension has far-reaching consequences for a leader's ability to
maintain their position over time.

Based on these observations and the theoretical framework outlined
earlier, I propose the following hypothesis:

\textbf{\emph{H4-1: Political leaders who successfully extend their
tenure through autocoups are more likely to survive longer extended
tenure compared to coup-installed leaders.}}

This hypothesis encapsulates the expected outcome of the divergent
challenges and advantages faced by autocoup and coup-installed leaders.
By testing this hypothesis, I aim to quantify the impact of the method
of power acquisition or extension on leadership longevity, contributing
to a more nuanced understanding of political survival in contexts of
irregular transitions.

\section{Research design}\label{research-design-1}

This section employs survival analysis to test the hypothesis that
autocoup leaders have longer survival times in office compared to
coup-installed leaders. The research design utilizes Cox models to
analyze the survival tenures of these two types of leaders, accounting
for multiple factors that may influence their time in power.

\subsection{Methodology: Survival
analysis}\label{methodology-survival-analysis}

Two Cox models will be employed to analyze the survival tenures of
coup-installed and autocoup leaders:

\begin{itemize}
\item
  \textbf{Cox proportional hazards (PH) model}: This model uses only the
  variables present at the entry year, without considering changes over
  time.
\item
  \textbf{Time-dependent Cox model}: This model accounts for variations
  in time-dependent control variables such as economic performance and
  political stability.
\end{itemize}

The Cox model is preferred over the Kaplan-Meier model as it allows for
the estimation of multiple factors' impacts. While it does not directly
estimate the duration of tenure in office, it evaluates the hazard rate
associated with being ousted from power. This approach captures
different facets of the same phenomenon: as a leader's cumulative hazard
of being ousted increases, their probability of survival in office
decreases.

\subsection{Data and variables}\label{data-and-variables}

The dependent variables include survival time and end point status:

\begin{itemize}
\item
  \textbf{Survival time:} Survival time refers to the duration of a
  leader's tenure, measured in days. For coup-installed leaders, the
  survival time begins on the day they assume power through a coup. For
  autocoup leaders, the survival time starts on the expiration date of
  their original legitimate term. For example, Russia's president
  Vladimir Putin assumed power in 2000 and, after serving two terms,
  stepped down in 2008. However, he remained in a powerful position as
  the prime minister and hand-picked Dmitry Medvedev to succeed him as
  president, while continuing to control the power behind the scenes. In
  this case, Putin's survival time begins in 2008, marking the start of
  his post-autocoup tenure. The survival time concludes on the day the
  leader finally exits office, applicable to both coup-installed and
  autocoup leaders.
\item
  \textbf{End point status:} This variable indicates the manner in which
  the leader's tenure concluded, categorized as follows:

  \textbf{0 = Censored:} This status is assigned to leaders who leave
  office through regular means other than being ousted. This includes
  leaders transferring power to their designated successors, leaving
  office as their terms expire, losing in general elections, voluntarily
  leaving office due to health issues, or dying of natural causes.

  \textbf{1 = Ousted:} This status is assigned to leaders who are forced
  to leave office. This includes leaders resigning under pressure, being
  ousted by coups or other forces, or being assassinated.
\end{itemize}

The key independent variable is the leader type, which categorizes
leaders into two distinct groups:

\begin{itemize}
\tightlist
\item
  \textbf{Group A = Autocoup leader}: Leaders who extend their tenure
  through autocoups.
\item
  \textbf{Group B = Coup-installed leader}: Leaders who assume power
  through coups.
\end{itemize}

This variable is the primary independent variable of interest, serving
as the basis for comparing the survival time between these two types of
leaders.

The data for both dependent and independent variables are sourced from
the autocoup dataset introduced in this study, Archigos, and PLAD.

Control variables include economic performance, political stability,
population size, and the leader's age, which are consistent with the
autocoup analysis in Chapter~\ref{sec-chapter3}.

\section{Results and discussion}\label{results-and-discussion-1}

\subsection{Model results}\label{model-results}

Using the \textbf{\emph{survial}} package in R
(\citeproc{ref-survival}{Therneau 2024}), I present the regression
results for both the Cox Proportional Hazards model (Cox PH) and the
time-dependent Cox model in Table~\ref{tbl-cox}.

\begin{table}

\caption{\label{tbl-cox}Cox models for survival time of different types
of leaders}

\centering{

\fontsize{12.0pt}{14.4pt}\selectfont
\begin{tabular*}{\linewidth}{@{\extracolsep{\fill}}lcccccccc}
\toprule
 & \multicolumn{4}{c}{\textbf{Cox PH Model}} & \multicolumn{4}{c}{\textbf{Time-dependent Cox Model}} \\ 
\cmidrule(lr){2-5} \cmidrule(lr){6-9}
\textbf{Characteristic} & \textbf{N} & \textbf{Event N} & \textbf{HR}\textsuperscript{\textit{1,2}} & \textbf{SE}\textsuperscript{\textit{2}} & \textbf{N} & \textbf{Event N} & \textbf{HR}\textsuperscript{\textit{1,2}} & \textbf{SE}\textsuperscript{\textit{2}} \\ 
\midrule\addlinespace[2.5pt]
{\bfseries Leader Type} &  &  &  &  &  &  &  &  \\ 
    Autocoup leaders & 76 & 31 & 1.00 & — & 737 & 29 & 1.00 & — \\ 
    Coup-installed leaders & 148 & 73 & 2.71*** & 0.252 & 853 & 73 & 2.23*** & 0.246 \\ 
{\bfseries GDP Growth Trend} & 224 & 104 & 1.94 & 1.08 & 1,590 & 102 & 0.20* & 0.981 \\ 
{\bfseries GDP per capita} & 224 & 104 & 0.97* & 0.020 & 1,590 & 102 & 0.95** & 0.023 \\ 
{\bfseries Population: log} & 224 & 104 & 0.98 & 0.083 & 1,590 & 102 & 0.90 & 0.079 \\ 
{\bfseries Polity 5} & 224 & 104 & 0.99 & 0.025 & 1,590 & 102 & 1.01 & 0.023 \\ 
{\bfseries Political stability} & 224 & 104 & 1.00 & 0.053 & 1,590 & 102 & 1.11* & 0.049 \\ 
{\bfseries Age} & 224 & 104 & 1.01 & 0.010 & 1,590 & 102 & 1.00 & 0.011 \\ 
\bottomrule
\end{tabular*}
\begin{minipage}{\linewidth}
\textsuperscript{\textit{1}}*p\textless{}0.1; **p\textless{}0.05; ***p\textless{}0.01\\
\textsuperscript{\textit{2}}HR = Hazard Ratio, SE = Standard Error\\
\end{minipage}

}

\end{table}%

Both the Cox PH model and the time-dependent Cox model analyses revealed
a statistically significant association between leadership type and the
hazard of removal from power. Since time-dependent Cox model use the
control variables which change over time, I interpret the main findings
based on time-dependent model.

Coup-installed leaders were found to have a hazard ratio of 2.23 in the
time-dependent model compared to autocoup leaders (reference group),
assuming all other variables in the model are held constant. This
suggests that coup-installed leaders face a significantly greater risk
of removal from power compared to autocoup leaders. At any given time
during their tenure, coup-installed leaders are 2.23 times more likely
to be ousted from power compared to autocoup leaders, all else being
equal in the model.

The control variables perform differently in the two models. Economic
level (GDP per capita) exhibits statistically significant effects in
both models. In the time-dependent model, the hazard ratio of 0.95
indicates that for each unit increase in GDP per capita (measured in
units of \$10,000), the hazard (or risk) of being ousted at any given
time is reduced by 5\%, assuming all other variables in the model are
held constant.

GDP growth trend demonstrates a more substantial effect in reducing the
risk of coups. Specifically, a 1 percentage point higher economic growth
trend is associated with an 80\% reduction in the risk of being ousted,
although this effect is only statistically significant at the 10\%
level. This suggests a possible trend where positive economic
performance might mitigate the risk of removal from power, but the
evidence is not robust enough to confirm this conclusively.

Political stability, as measured by the violence index, shows that a
1-point increase in the index correlates with an 11\% higher risk of
being ousted. However, this effect is also only statistically
significant at the 10\% level, indicating a weaker but potentially
important relationship between increased violence and the risk of
removal from office.

\subsection{Discussion}\label{discussion}

\begin{figure}

\begin{minipage}{0.50\linewidth}

\centering{

\includegraphics{_coups_and_autocoups_files/figure-pdf/fig-coxSurv-1.pdf}

}

\subcaption{\label{fig-coxSurv-1}Cox PH Model}

\end{minipage}%
%
\begin{minipage}{0.50\linewidth}

\centering{

\includegraphics{_coups_and_autocoups_files/figure-pdf/fig-coxSurv-2.pdf}

}

\subcaption{\label{fig-coxSurv-2}Time-dependent Cox Model}

\end{minipage}%

\caption{\label{fig-coxSurv}Survival curves for Cox Model}

\end{figure}%

The survival curves depicted in Figure~\ref{fig-coxSurv} illustrate the
survival rates for leaders of both types. Both the Cox PH model and the
time-dependent Cox model produce similar plots. Notably, the survival
curve for coup-installed leaders exhibits a significantly lower
trajectory compared to that of autocoup leaders. The steeper drop at the
early stage for coup-installed leaders indicates they are more likely to
be ousted shortly after assuming power. Additionally, the survival curve
for coup-installed leaders crosses the median survival line much earlier
(about 3,000 days) than that of autocoup leaders (about 8,500 days).
This disparity suggests that autocoup leaders tend to remain in power
for longer durations than their coup-installed counterparts.

\begin{figure}

\begin{minipage}{0.50\linewidth}

\centering{

\includegraphics{_coups_and_autocoups_files/figure-pdf/fig-coxHR-1.pdf}

}

\subcaption{\label{fig-coxHR-1}Cox PH Model}

\end{minipage}%
%
\begin{minipage}{0.50\linewidth}

\centering{

\includegraphics{_coups_and_autocoups_files/figure-pdf/fig-coxHR-2.pdf}

}

\subcaption{\label{fig-coxHR-2}Time-dependent Cox Model}

\end{minipage}%

\caption{\label{fig-coxHR}Hazard ratios and 95\% CIs for Leader Ousting}

\end{figure}%

Figure~\ref{fig-coxHR} displays the hazard ratios and corresponding 95\%
confidence intervals for the variables incorporated in the Cox model.
Both the Cox Proportional Hazards (PH) model and the time-dependent
model produce similar plots, reinforcing the robustness of the findings.
Key points to note include:

\begin{itemize}
\item
  The closer the hazard ratio (represented by the dots) is to 1, the
  less impact the variable has on the risk of being ousted. A hazard
  ratio of 1 indicates no effect.
\item
  The whiskers extending from the dots represent the 95\% confidence
  intervals. If these whiskers cross the vertical blue line at 1, it
  indicates that the variable is not statistically significant at the
  5\% level.
\item
  The hazard ratio for coup-installed leaders is significantly greater
  than 1 and statistically significant at the 5\% level. This indicates
  that coup-installed leaders face a substantially higher risk of being
  ousted compared to autocoup leaders.
\item
  Most other variables have hazard ratios close to 1, suggesting that a
  one-unit increase in these variables does not significantly affect the
  risk of being ousted.
\item
  Although the hazard ratio for GDP growth trend is considerably less
  than 1 in the time-dependent model, indicating a potential protective
  effect, it is not statistically significant at the 5\% level. However,
  it is statistically significant at the 10\% level, suggesting that
  better economic performance may help to consolidate the rule of the
  incumbents to some extent, albeit the evidence is not as strong.
\end{itemize}

\subsection{Assessing the proportional hazards
assumption}\label{assessing-the-proportional-hazards-assumption}

Assessing the proportional hazards assumption is crucial for the
validity of the Cox model results. To evaluate this, we used the
chi-square test based on Schoenfeld residuals to determine whether the
covariate effects remain constant (proportional) over time. Although the
Cox PH model violates the proportional hazards assumption, our primary
analysis relies on the time-dependent Cox model, which does not show
strong evidence of violating the proportional hazards assumption for any
covariate. The global p-value of 0.416 is much greater than the 5\%
significance level, indicating that the proportional hazards assumption
is reasonably met for the time-dependent Cox model.

\section{Summary}\label{summary-2}

This chapter explores the survival durations of political leaders who
come to power through unconventional means, specifically focusing on
coups and autocoups. Building on the hypothesis that the mode of
accession plays a critical role in determining leader tenure, I employ
survival analysis techniques, including the Cox proportional hazards
model and a time-dependent Cox model, to investigate this phenomenon.
The results provide compelling evidence that leaders who consolidate
power through autocoups generally enjoy longer tenures than those
installed by coups.

Empirical analysis reveals a significant disparity in tenure length:
leaders who assume power via autocoup remain in office for an average of
11 years, compared to just 5.6 years for those installed by coups.
Moreover, the time-dependent Cox model indicates that coup-installed
leaders are 2.23 times more likely to be ousted from power at any given
time compared to their autocoup counterparts, all other factors being
equal. These findings underscore the importance of understanding the
autocoup as a mechanism through which leaders extend their rule by
manipulating legal frameworks and weakening institutional constraints.

The implications of these findings are profound. The relative ease and
potential rewards of an autocoup could incentivize more leaders to
resort to this method of power retention, particularly in fragile
democracies or transitioning regimes. As a result, democratic
backsliding may become more common, as autocoups erode democratic
institutions and undermine constitutional norms.

This study makes a meaningful contribution to the literature on
political leadership survival by demonstrating that the mode of
accession significantly impacts leader tenure---an aspect that has been
underexplored in prior research. Methodologically, this work advances
the field by applying robust survival analysis techniques, including
both Cox models, to provide a nuanced understanding of the dynamics that
influence leadership stability.

However, the study is not without limitations. The analysis relies on an
autocoup dataset that was collected and coded by the author, a
relatively novel concept in the academic sphere. As the understanding
and recognition of ``autocoup'' as a term continue to evolve, future
research should refine and expand the dataset. Incorporating additional
cases and cross-referencing with other forms of irregular leadership
transitions would contribute to a more comprehensive view of political
survival under such conditions.

In conclusion, this chapter highlights the need for more refined
approaches to studying political tenure and irregular power retention.
By offering valuable insights into the dynamics of political stability
and the risks associated with non-democratic leadership transitions,
this research emphasizes the importance of continued investigation into
the complex relationships between power, legitimacy, and survival in
political leadership.

\chapter{Conclusion}\label{conclusion}

\section{Main findings}\label{main-findings}

This study provides critical insights into the dynamics and implications
of irregular power transitions, with a specific focus on coups and
autocoups. The research illuminates the complex interplay between
incumbents and challengers vying for power, yielding three key findings.

\begin{itemize}
\item
  \textbf{Coup attempt determinant}: The expected success rate
  significantly influences the likelihood of a coup attempt. This
  success rate is largely determined by the balance of power between
  incumbent leaders and challengers, which varies by regime type.
  Notably, the findings show that military regimes are approximately
  277.7\% more likely, and personalist regimes 94\% more likely, to
  experience coups compared to dominant-party regimes, all else being
  equal.
\item
  \textbf{Autocoup concept and dataset}: I introduce a refined concept
  of ``autocoup'', defined as an incumbent leader's refusal to
  relinquish power as mandated. We present the first publicly available
  dataset of autocoup events from 1945 to 2023, encompassing 110
  attempts and 87 successful autocoups. Case studies and empirical
  analyses demonstrate the dataset's utility for quantitative research.
\item
  \textbf{Leader longevity}: Survival analysis techniques reveal clear
  differences in leader longevity between coup-installed leaders and
  autocoup leaders. The findings reveal that, on average, coup-installed
  leaders are 2.23 times more likely to be ousted from power than
  autocoup leaders, all else being equal.
\end{itemize}

\section{Policy implications}\label{policy-implications-1}

The examination of irregular power transitions and leadership survival
offers crucial insights into democratic backsliding, breakdown, and
autocratic intensification. The findings provide logical explanations
for several political trends:

The examination of irregular power transitions and leadership survival
offers a crucial perspective on the interrelated phenomena of democratic
backsliding, breakdown, and autocratic intensification. The findings of
this study provide logical explanations for several political trends:

\begin{itemize}
\item
  \textbf{Global democracy regression:} This study elucidates why global
  freedom has declined for the 18th consecutive year. Irregular power
  transitions, whether through coups or autocoups, inherently violate
  democratic norms and disrupt the trajectory toward stable democracies.
\item
  \textbf{Within-regime democratic erosion:} The research explains why
  democratic backsliding often occurs within regimes
  (\citeproc{ref-mechkova2017}{Mechkova, Lührmann, and Lindberg 2017}),
  rather than through regime change. Democracies are becoming less
  liberal and autocracies less competitive, particularly due to the
  prevalence of autocoups since 2000 (\citeproc{ref-bermeo2016}{Bermeo
  2016}). As discussed in Chapter 3, autocoups extend the tenure of
  incumbent leaders without overturning the regime itself.
\item
  \textbf{Rise of autocoups since 2000:} The analysis also clarifies why
  autocoups have been on the rise since 2000. Incumbent leaders possess
  several strategic advantages: firstly, they have a significantly
  higher probability of success due to their incumbent vantages compared
  to coup plotters. Secondly, the consequences of failed autocoups are
  relatively milder than those for failed coup plotters, resulting in
  lower costs even if they fail. Lastly, leaders who manage to extend
  their rule through an autocoup often enjoy considerably longer tenures
  compared to coup-installed leaders, thus benefiting more
  substantially.
\item
  \textbf{Role of external pressure:} Due to the challenges of internal
  opposition to autocoups, where power is concentrated in the hands of
  incumbent leaders, external pressure from regional or international
  communities may play a vital role in encouraging adherence to
  constitutional processes of power transition. For instance, after the
  general election in Venezuela on July 29, 2024, at least nine Latin
  American countries rejected the election results and called for
  dialogue\footnote{While the world waited for the outcome, nine Latin
    American countries released a joint statement urging transparency
    and recognition of the voters' will. The nine countries are
    Argentina, Costa Rica, the Dominican Republic, Ecuador, Guatemala,
    Panama, Paraguay, Peru, and Uruguay. On the morning after the
    election, the same group released a second statement demanding a
    complete review of the results in the presence of independent
    electoral observers
    (\href{https://www.as-coa.org/articles/how-have-international-leaders-responded-venezuelas-2024-election}{AS/COA},
    accessed on September 9, 2024).}. Although this pressure might not
  be effective in every case, it showcases the potential influence of
  the international community in discouraging future autocoup attempts.
\end{itemize}

\section{Limitations and directions for future
research}\label{limitations-and-directions-for-future-research}

While the study offers a novel framework for analysing irregular
leadership transitions, several limitations require further exploration:

\begin{itemize}
\item
  \textbf{Data refinement:} Defining and classifying autocoups is a new
  approach. Future research should validate this classification system
  through additional studies and expert evaluations.
\item
  \textbf{Data harmonization:} The current analysis faces challenges due
  to mismatched units (country-year vs.~leader) between coup and
  autocoup datasets. Future efforts should explore data harmonization
  techniques for more robust comparisons.
\item
  \textbf{Democratic backsliding:} While this study establishes a
  connection between irregular power transitions and democratic
  backsliding, further empirical evidence is needed to solidify this
  link.
\end{itemize}

Future research avenues include:

\begin{itemize}
\item
  \textbf{Terminology and data collection:} Refining the ``autocoup''
  concept and achieving wider recognition will facilitate more accurate
  and comprehensive data collection.
\item
  \textbf{Dataset expansion:} Expanding the autocoup dataset with more
  cases and integrating it with data on other irregular leadership
  transitions can provide a more holistic view of political survival
  after these events.
\item
  \textbf{Power dynamics and long-term impacts:} Utilizing this dataset,
  future studies can delve deeper into power dynamics at play and
  explore the long-term consequences of irregular transitions on
  political systems, particularly regarding democratic backsliding,
  breakdown, and personalization of power.
\end{itemize}

In conclusion, this study significantly contributes to our understanding
of irregular leadership transitions, focusing on coups and autocoups. By
redefining autocoups, classifying the dataset, analysing determinants,
and comparing leader longevity, I establish a robust framework for
understanding irregular power transitions and leadership survival. This
work deepens our comprehension of democratic resilience and political
stability, providing a foundation for future research to conduct further
empirical analyses based on the novel autocoup dataset and continue
refining the framework.

\chapter*{References}\label{references}
\addcontentsline{toc}{chapter}{References}

\phantomsection\label{refs}
\begin{CSLReferences}{1}{0}
\bibitem[\citeproctext]{ref-aidt2019}
Aidt, Toke, and Gabriel Leon. 2019. {``The Coup.''} Edited by Roger D.
Congleton, Bernard Grofman, and Stefan Voigt, February.
\url{https://doi.org/10.1093/oxfordhb/9780190469771.013.15}.

\bibitem[\citeproctext]{ref-antonio2021}
Antonio, Robert J. 2021. {``Democracy and Capitalism in the Interregnum:
Trump{'}s Failed Self-Coup and After.''} \emph{Critical Sociology} 48
(6): 937--65. \url{https://doi.org/10.1177/08969205211049499}.

\bibitem[\citeproctext]{ref-arriola2009}
Arriola, Leonardo R. 2009. {``Patronage and Political Stability in
Africa.''} \emph{Comparative Political Studies} 42 (10): 1339--62.
\url{https://doi.org/10.1177/0010414009332126}.

\bibitem[\citeproctext]{ref-baturo2014}
Baturo, Alexander. 2014. {``Democracy, Dictatorship, and Term Limits.''}
\url{https://doi.org/10.3998/mpub.4772634}.

\bibitem[\citeproctext]{ref-baturo2019}
---------. 2019. {``Continuismo in Comparison.''} In, 75--100. Oxford
University Press.
\url{https://doi.org/10.1093/oso/9780198837404.003.0005}.

\bibitem[\citeproctext]{ref-baturo2022}
Baturo, Alexander, and Jakob Tolstrup. 2022. {``Incumbent Takeovers.''}
\emph{Journal of Peace Research} 60 (2): 373--86.
\url{https://doi.org/10.1177/00223433221075183}.

\bibitem[\citeproctext]{ref-bermeo2016}
Bermeo, Nancy. 2016. {``On Democratic Backsliding.''} \emph{Journal of
Democracy} 27 (1): 5--19. \url{https://doi.org/10.1353/jod.2016.0012}.

\bibitem[\citeproctext]{ref-buxf6hmelt2014}
Böhmelt, Tobias, and Ulrich Pilster. 2014. {``The Impact of
Institutional Coup-Proofing on Coup Attempts and Coup Outcomes.''}
\emph{International Interactions} 41 (1): 158--82.
\url{https://doi.org/10.1080/03050629.2014.906411}.

\bibitem[\citeproctext]{ref-bomprezzi2024wedded}
Bomprezzi, Pietro, Axel Dreher, Andreas Fuchs, Teresa Hailer, Andreas
Kammerlander, Lennart Kaplan, Silvia Marchesi, Tania Masi, Charlotte
Robert, and Kerstin Unfried. 2024. {``Wedded to Prosperity? Informal
Influence and Regional Favoritism.''} Discussion Paper. CEPR.

\bibitem[\citeproctext]{ref-brown2001}
Brown, Stephen. 2001. {``Authoritarian Leaders and Multiparty Elections
in Africa: How Foreign Donors Help to Keep Kenya's Daniel Arap Moi in
Power.''} \emph{Third World Quarterly} 22 (5): 725--39.
\url{https://doi.org/10.1080/01436590120084575}.

\bibitem[\citeproctext]{ref-buenodemesquita2003}
Bueno de Mesquita, Bruce, Alastair Smith, Randolph M. Siverson, and
James D. Morrow. 2003. \emph{The Logic of Political Survival}. The MIT
Press. \url{https://doi.org/10.7551/mitpress/4292.001.0001}.

\bibitem[\citeproctext]{ref-cameron1998a}
Cameron, Maxwell A. 1998a. {``Latin American Autogolpes : Dangerous
Undertows in the Third Wave of Democratisation.''} \emph{Third World
Quarterly} 19 (2): 219--39.
\url{https://doi.org/10.1080/01436599814433}.

\bibitem[\citeproctext]{ref-cameron1998}
Cameron, Maxwell A. 1998b. {``Self-Coups: Peru, Guatemala, and
Russia.''} \emph{Journal of Democracy} 9 (1): 125--39.
\url{https://doi.org/10.1353/jod.1998.0003}.

\bibitem[\citeproctext]{ref-cassani2020}
Cassani, Andrea. 2020. {``Autocratisation by Term Limits Manipulation in
Sub-Saharan Africa.''} \emph{Africa Spectrum} 55 (3): 228--50.
\url{https://doi.org/10.1177/0002039720964218}.

\bibitem[\citeproctext]{ref-cheeseman2015}
Cheeseman, Nic. 2015. {``Democracy in Africa,''} March.
\url{https://doi.org/10.1017/cbo9781139030892}.

\bibitem[\citeproctext]{ref-cheeseman2019}
---------. 2019. {``Should I Stay or Should I Go? Term Limits,
Elections, and Political Change in Kenya, Uganda, and Zambia.''} In,
311--38. Oxford University PressOxford.
\url{https://doi.org/10.1093/oso/9780198837404.003.0016}.

\bibitem[\citeproctext]{ref-cheeseman2019a}
Cheeseman, Nic, and Brian Klaas. 2019. \emph{How to Rig an Election}.
Yale University Press. \url{https://doi.org/10.12987/9780300235210}.

\bibitem[\citeproctext]{ref-choulis2022}
Choulis, Ioannis, Marius Mehrl, Abel Escribà-Folch, and Tobias Böhmelt.
2022. {``How Mechanization Shapes Coups.''} \emph{Comparative Political
Studies} 56 (2): 267--96.
\url{https://doi.org/10.1177/00104140221100194}.

\bibitem[\citeproctext]{ref-close2019}
Close, David. 2019. {``Presidential Term Limits in Nicaragua.''} In,
159--78. Oxford University PressOxford.
\url{https://doi.org/10.1093/oso/9780198837404.003.0009}.

\bibitem[\citeproctext]{ref-davenport2021}
Davenport, Christian, Babak RezaeeDaryakenari, and Reed M Wood. 2021.
{``Tenure Through Tyranny? Repression, Dissent, and Leader Removal in
Africa and Latin America, 1990{\textendash}2006.''} \emph{Journal of
Global Security Studies} 7 (1).
\url{https://doi.org/10.1093/jogss/ogab023}.

\bibitem[\citeproctext]{ref-debruin2020}
De Bruin, Erica. 2020. {``Preventing Coups d{'}état.''} In, 1--12.
Cornell University Press.
\url{https://doi.org/10.7591/cornell/9781501751912.003.0001}.

\bibitem[\citeproctext]{ref-easton2018}
Easton, Malcolm R, and Randolph M Siverson. 2018. {``Leader Survival and
Purges After a Failed Coup d{'}état.''} \emph{Journal of Peace Research}
55 (5): 596--608. \url{https://doi.org/10.1177/0022343318763713}.

\bibitem[\citeproctext]{ref-escribuxe0-folch2013}
Escribà-Folch, Abel. 2013. {``Repression, Political Threats, and
Survival Under Autocracy.''} \emph{International Political Science
Review} 34 (5): 543--60. \url{https://doi.org/10.1177/0192512113488259}.

\bibitem[\citeproctext]{ref-ezrow2019}
Ezrow, Natasha. 2019. {``Term Limits and Succession in Dictatorships.''}
In, 269--88. Oxford University PressOxford.
\url{https://doi.org/10.1093/oso/9780198837404.003.0014}.

\bibitem[\citeproctext]{ref-fariss2022}
Fariss, Christopher J., Therese Anders, Jonathan N. Markowitz, and
Miriam Barnum. 2022. {``New Estimates of Over 500 Years of Historic GDP
and Population Data.''} \emph{Journal of Conflict Resolution} 66 (3):
553--91. \url{https://doi.org/10.1177/00220027211054432}.

\bibitem[\citeproctext]{ref-frantz2016}
Frantz, Erica, and Elizabeth A. Stein. 2016. {``Countering Coups:
Leadership Succession Rules in Dictatorships.''} \emph{Comparative
Political Studies} 50 (7): 935--62.
\url{https://doi.org/10.1177/0010414016655538}.

\bibitem[\citeproctext]{ref-freedomhouse2024freedom}
Freedom House. 2024. {``Freedom in the World 2024.''}
\url{https://freedomhouse.org/sites/default/files/2024-02/FIW_2024_DigitalBooklet.pdf}.

\bibitem[\citeproctext]{ref-gandhi2007}
Gandhi, Jennifer, and Adam Przeworski. 2007. {``Authoritarian
Institutions and the Survival of Autocrats.''} \emph{Comparative
Political Studies} 40 (11): 1279--1301.
\url{https://doi.org/10.1177/0010414007305817}.

\bibitem[\citeproctext]{ref-gassebner2016}
Gassebner, Martin, Jerg Gutmann, and Stefan Voigt. 2016. {``When to
Expect a Coup d{'}état? An Extreme Bounds Analysis of Coup
Determinants.''} \emph{Public Choice} 169 (3-4): 293--313.
\url{https://doi.org/10.1007/s11127-016-0365-0}.

\bibitem[\citeproctext]{ref-geddes1999}
Geddes, Barbara. 1999. {``What Do We Know About Democratization After
Twenty Years?''} \emph{Annual Review of Political Science} 2 (1):
115--44. \url{https://doi.org/10.1146/annurev.polisci.2.1.115}.

\bibitem[\citeproctext]{ref-geddes2014}
Geddes, Barbara, Joseph Wright, and Erica Frantz. 2014. {``Autocratic
Breakdown and Regime Transitions: A New Data Set.''} \emph{Perspectives
on Politics} 12 (2): 313--31.
\url{https://doi.org/10.1017/s1537592714000851}.

\bibitem[\citeproctext]{ref-ginsburg2019}
Ginsburg, Tom, and Zachary Elkins. 2019. {``One Size Does Not Fit
All.''} In, 37--52. Oxford University Press.
\url{https://doi.org/10.1093/oso/9780198837404.003.0003}.

\bibitem[\citeproctext]{ref-ginsburg2010evasion}
Ginsburg, Tom, James Melton, and Zachary Elkins. 2010. {``On the Evasion
of Executive Term Limits.''} \emph{Wm. \& Mary L. Rev.} 52: 1807.

\bibitem[\citeproctext]{ref-ginsburg2011evasion}
---------. 2011. {``On the Evasion of Executive Term Limits.''}
\emph{William and Mary Law Review} 52: 1807.

\bibitem[\citeproctext]{ref-goemans2009}
Goemans, Henk E., Kristian Skrede Gleditsch, and Giacomo Chiozza. 2009.
{``Introducing Archigos: A Dataset of Political Leaders.''}
\emph{Journal of Peace Research} 46 (2): 269--83.
\url{https://doi.org/10.1177/0022343308100719}.

\bibitem[\citeproctext]{ref-helmke2017}
Helmke, Gretchen. 2017. {``Institutions on the Edge,''} January.
\url{https://doi.org/10.1017/9781139031738}.

\bibitem[\citeproctext]{ref-klesner2019}
Klesner, Joseph L. 2019. {``The Politics of Presidential Term Limits in
Mexico.''} In, 141--58. Oxford University Press.
\url{https://doi.org/10.1093/oso/9780198837404.003.0008}.

\bibitem[\citeproctext]{ref-krishnarajan2019}
Krishnarajan, Suthan. 2019. {``Economic Crisis, Natural Resources, and
Irregular Leader Removal in Autocracies.''} \emph{International Studies
Quarterly} 63 (3): 726--41. \url{https://doi.org/10.1093/isq/sqz006}.

\bibitem[\citeproctext]{ref-landau2019}
Landau, David, Yaniv Roznai, and Rosalind Dixon. 2019. {``Term Limits
and the Unconstitutional Constitutional Amendment Doctrine.''} In,
53--74. Oxford University PressOxford.
\url{https://doi.org/10.1093/oso/9780198837404.003.0004}.

\bibitem[\citeproctext]{ref-licht2009}
Licht, Amanda A. 2009. {``Coming into Money: The Impact of Foreign Aid
on Leader Survival.''} \emph{Journal of Conflict Resolution} 54 (1):
58--87. \url{https://doi.org/10.1177/0022002709351104}.

\bibitem[\citeproctext]{ref-llanos2019}
Llanos, Mariana. 2019. {``The Politics of Presidential Term Limits in
Argentina.''} In, 473--94. Oxford University Press.
\url{https://doi.org/10.1093/oso/9780198837404.003.0023}.

\bibitem[\citeproctext]{ref-marshall2005current}
Marshall, Monty G. 2005. {``Current Status of the World's Major Episodes
of Political Violence.''} \emph{Report to Political Instability Task
Force.(3 February)}.

\bibitem[\citeproctext]{ref-marsteintredet2019a}
Marsteintredet, Leiv. 2019. {``Presidential Term Limits in Latin
America: {\emph{C}}.1820{\textendash}1985.''} In, 103--22. Oxford
University PressOxford.
\url{https://doi.org/10.1093/oso/9780198837404.003.0006}.

\bibitem[\citeproctext]{ref-marsteintredet2019}
Marsteintredet, Leiv, and Andrés Malamud. 2019. {``Coup with Adjectives:
Conceptual Stretching or Innovation in Comparative Research?''}
\emph{Political Studies} 68 (4): 1014--35.
\url{https://doi.org/10.1177/0032321719888857}.

\bibitem[\citeproctext]{ref-mauceri1995}
Mauceri, Philip. 1995. {``State Reform, Coalitions, and The Neoliberal
{\emph{Autogolpe}} in Peru.''} \emph{Latin American Research Review} 30
(1): 7--37. \url{https://doi.org/10.1017/s0023879100017155}.

\bibitem[\citeproctext]{ref-mechkova2017}
Mechkova, Valeriya, Anna Lührmann, and Staffan I. Lindberg. 2017. {``How
Much Democratic Backsliding?''} \emph{Journal of Democracy} 28 (4):
162--69. \url{https://doi.org/10.1353/jod.2017.0075}.

\bibitem[\citeproctext]{ref-morrison2009}
Morrison, Kevin M. 2009. {``Oil, Nontax Revenue, and the
Redistributional Foundations of Regime Stability.''} \emph{International
Organization} 63 (1): 107--38.
\url{https://doi.org/10.1017/s0020818309090043}.

\bibitem[\citeproctext]{ref-neto2019}
Neto, Octavio Amorim, and Igor P. Acácio. 2019. {``Presidential Term
Limits as a Credible-Commitment Mechanism.''} In, 123--40. Oxford
University PressOxford.
\url{https://doi.org/10.1093/oso/9780198837404.003.0007}.

\bibitem[\citeproctext]{ref-nurumov2019}
Nurumov, Dmitry, and Vasil Vashchanka. 2019. {``Presidential Terms in
Kazakhstan.''} In, 221--46. Oxford University PressOxford.
\url{https://doi.org/10.1093/oso/9780198837404.003.0012}.

\bibitem[\citeproctext]{ref-palmer1999}
Palmer, Harvey D., and Guy D. Whitten. 1999. {``The Electoral Impact of
Unexpected Inflation and Economic Growth.''} \emph{British Journal of
Political Science} 29 (4): 623--39.
\url{https://doi.org/10.1017/s0007123499000307}.

\bibitem[\citeproctext]{ref-pion-berlin2022}
Pion-Berlin, David, Thomas Bruneau, and Richard B. Goetze. 2022. {``The
Trump Self-Coup Attempt: Comparisons and Civil{\textendash}Military
Relations.''} \emph{Government and Opposition} 58 (4): 789--806.
\url{https://doi.org/10.1017/gov.2022.13}.

\bibitem[\citeproctext]{ref-posner}
Posner, Daniel N., and Daniel J. Young. n.d. {``Term Limits: Leadership,
Political Competition and the Transfer of Power.''} In, 260--78.
Cambridge University Press.
\url{https://doi.org/10.1017/9781316562888.011}.

\bibitem[\citeproctext]{ref-powell2012}
Powell, Jonathan. 2012. {``Determinants of the Attempting and Outcome of
Coups d{'}état.''} \emph{Journal of Conflict Resolution} 56 (6):
1017--40. \url{https://doi.org/10.1177/0022002712445732}.

\bibitem[\citeproctext]{ref-powell2017}
---------. 2017. {``Leader Survival Strategies and the Onset of Civil
Conflict: A Coup-Proofing Paradox.''} \emph{Armed Forces \& Society} 45
(1): 27--44. \url{https://doi.org/10.1177/0095327x17728493}.

\bibitem[\citeproctext]{ref-powell2018}
Powell, Christopher Faulkner, William Dean, and Kyle Romano. 2018.
{``Give Them Toys? Military Allocations and Regime Stability in
Transitional Democracies.''} \emph{Democratization} 25 (7): 1153--72.
\url{https://doi.org/10.1080/13510347.2018.1450389}.

\bibitem[\citeproctext]{ref-powell2011}
Powell, and Thyne. 2011. {``Global Instances of Coups from 1950 to 2010:
A New Dataset.''} \emph{Journal of Peace Research} 48 (2): 249--59.
\url{https://doi.org/10.1177/0022343310397436}.

\bibitem[\citeproctext]{ref-przeworski2000}
Przeworski, Adam, Michael E. Alvarez, Jose Antonio Cheibub, and Fernando
Limongi. 2000. {``Democracy and Development,''} August.
\url{https://doi.org/10.1017/cbo9780511804946}.

\bibitem[\citeproctext]{ref-quirozflores2012}
Quiroz Flores, Alejandro, and Alastair Smith. 2012. {``Leader Survival
and Natural Disasters.''} \emph{British Journal of Political Science} 43
(4): 821--43. \url{https://doi.org/10.1017/s0007123412000609}.

\bibitem[\citeproctext]{ref-reyntjens2016}
Reyntjens, Filip. 2016. {``A New Look at the Evidence.''} \emph{Journal
of Democracy} 27 (3): 61--68.
\url{https://doi.org/10.1353/jod.2016.0044}.

\bibitem[\citeproctext]{ref-singh2016}
Singh, Naunihal. 2016. \emph{Seizing Power}. Johns Hopkins University
Press. \url{https://doi.org/10.1353/book.31450}.

\bibitem[\citeproctext]{ref-smith2004}
Smith, Benjamin. 2004. {``Oil Wealth and Regime Survival in the
Developing World, 1960{\textendash}1999.''} \emph{American Journal of
Political Science} 48 (2): 232--46.
\url{https://doi.org/10.1111/j.0092-5853.2004.00067.x}.

\bibitem[\citeproctext]{ref-stinnett2002}
Stinnett, Douglas M., Jaroslav Tir, Paul F. Diehl, Philip Schafer, and
Charles Gochman. 2002. {``The Correlates of War (Cow) Project Direct
Contiguity Data, Version 3.0.''} \emph{Conflict Management and Peace
Science} 19 (2): 59--67.
\url{https://doi.org/10.1177/073889420201900203}.

\bibitem[\citeproctext]{ref-sudduth2017}
Sudduth, Jun Koga. 2017. {``Strategic Logic of Elite Purges in
Dictatorships.''} \emph{Comparative Political Studies} 50 (13):
1768--1801. \url{https://doi.org/10.1177/0010414016688004}.

\bibitem[\citeproctext]{ref-sudduth2018}
Sudduth, Jun Koga, and Curtis Bell. 2018. {``The Rise Predicts the Fall:
How the Method of Leader Entry Affects the Method of Leader Removal in
Dictatorships.''} \emph{International Studies Quarterly} 62 (1):
145--59. \url{https://doi.org/10.1093/isq/sqx075}.

\bibitem[\citeproctext]{ref-svolik2014}
Svolik, Milan W. 2014. {``Which Democracies Will Last? Coups, Incumbent
Takeovers, and the Dynamic of Democratic Consolidation.''} \emph{British
Journal of Political Science} 45 (4): 715--38.
\url{https://doi.org/10.1017/s0007123413000550}.

\bibitem[\citeproctext]{ref-tangri2010}
Tangri, Roger, and Andrew M. Mwenda. 2010. {``President Museveni and the
Politics of Presidential Tenure in Uganda.''} \emph{Journal of
Contemporary African Studies} 28 (1): 31--49.
\url{https://doi.org/10.1080/02589000903542574}.

\bibitem[\citeproctext]{ref-survival}
Therneau, Terry M. 2024. {``A Package for Survival Analysis in r.''}
\url{https://CRAN.R-project.org/package=survival}.

\bibitem[\citeproctext]{ref-thyne2017}
Thyne, Clayton, Powell, Sarah Parrott, and Emily VanMeter. 2017. {``Even
Generals Need Friends.''} \emph{Journal of Conflict Resolution} 62 (7):
1406--32. \url{https://doi.org/10.1177/0022002716685611}.

\bibitem[\citeproctext]{ref-thyne2019}
Thyne, and Powell. 2019. {``Coup Research,''} October.
\url{https://doi.org/10.1093/acrefore/9780190846626.013.369}.

\bibitem[\citeproctext]{ref-sampleSelection-2}
Toomet, Ott, and Arne Henningsen. 2008. {``Sample Selection Models in
{\textbraceleft}r{\textbraceright}: Package
{\textbraceleft}sampleSelection{\textbraceright}''} 27.
\url{https://www.jstatsoft.org/v27/i07/}.

\bibitem[\citeproctext]{ref-versteeg2020law}
Versteeg, Mila, Timothy Horley, Anne Meng, Mauricio Guim, and Marilyn
Guirguis. 2020. {``The Law and Politics of Presidential Term Limit
Evasion.''} \emph{Colum. L. Rev.} 120: 173.

\bibitem[\citeproctext]{ref-williams2011}
Williams, Laron K. 2011. {``Pick Your Poison: Economic Crises,
International Monetary Fund Loans and Leader Survival.''}
\emph{International Political Science Review} 33 (2): 131--49.
\url{https://doi.org/10.1177/0192512111399006}.

\bibitem[\citeproctext]{ref-wright2008}
Wright, Joseph. 2008. {``To Invest or Insure?''} \emph{Comparative
Political Studies} 41 (7): 971--1000.
\url{https://doi.org/10.1177/0010414007308538}.

\bibitem[\citeproctext]{ref-wright2013}
Wright, Joseph, Erica Frantz, and Barbara Geddes. 2013. {``Oil and
Autocratic Regime Survival.''} \emph{British Journal of Political
Science} 45 (2): 287--306.
\url{https://doi.org/10.1017/s0007123413000252}.

\bibitem[\citeproctext]{ref-yu2016}
Yu, Shu, and Richard Jong-A-Pin. 2016. {``Political Leader Survival:
Does Competence Matter?''} \emph{Public Choice} 166 (1-2): 113--42.
\url{https://doi.org/10.1007/s11127-016-0317-8}.

\end{CSLReferences}

\chapter*{\texorpdfstring{Appendix\textbf{:
Datasets}}{Appendix: Datasets}}\label{appendix-datasets}
\addcontentsline{toc}{chapter}{Appendix\textbf{: Datasets}}

\begin{itemize}
\item
  \textbf{Coup Model Dataset}

  \begin{itemize}
  \item
    \textbf{Dataset Name:} \textbf{\texttt{coup\_model.csv}}
  \item
    \textbf{Description:} This dataset is specifically cleaned for the
    coup model and contains the relevant data points necessary for
    analysis.
  \end{itemize}
\item
  \textbf{Autocoup Dataset}

  \begin{itemize}
  \item
    \textbf{Dataset Name:} \textbf{\texttt{autocoup.csv}}
  \item
    \textbf{Description:} This dataset is an original contribution of
    this thesis, compiled and curated by the author to support the
    research objectives.
  \end{itemize}
\item
  \textbf{Autocoup Model Dataset}

  \begin{itemize}
  \item
    \textbf{Dataset Name:} \textbf{\texttt{autocoup\_model.csv}}
  \item
    \textbf{Description:} This dataset is cleaned for the autocoup model
    and includes the data required for the modelling process.
  \end{itemize}
\item
  \textbf{Cox Proportional Hazards (Cox PH) Model Dataset}

  \begin{itemize}
  \item
    \textbf{Dataset Name:}
    \textbf{\texttt{survival\_cox\_ph\_model.csv}}
  \item
    \textbf{Description:} This dataset is used for the Cox Proportional
    Hazards model and contains the data necessary for analysing survival
    rates and hazard ratios.
  \end{itemize}
\item
  \textbf{Time-Dependent Cox Model Dataset}

  \begin{itemize}
  \item
    \textbf{Dataset Name:}
    \textbf{\texttt{survival\_cox\_td\_model.csv}}
  \item
    \textbf{Description:} This dataset is cleaned for the time-dependent
    Cox model, incorporating variables that account for time-dependent
    effects in survival analysis.
  \end{itemize}
\end{itemize}




\end{document}

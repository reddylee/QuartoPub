% Options for packages loaded elsewhere
% Options for packages loaded elsewhere
\PassOptionsToPackage{unicode}{hyperref}
\PassOptionsToPackage{hyphens}{url}
\PassOptionsToPackage{dvipsnames,svgnames,x11names}{xcolor}
%
\documentclass[
  12pt,
]{article}
\usepackage{xcolor}
\usepackage[top = 3cm,bottom = 3cm,left = 3cm,right = 2.7cm]{geometry}
\usepackage{amsmath,amssymb}
\setcounter{secnumdepth}{5}
\usepackage{iftex}
\ifPDFTeX
  \usepackage[T1]{fontenc}
  \usepackage[utf8]{inputenc}
  \usepackage{textcomp} % provide euro and other symbols
\else % if luatex or xetex
  \usepackage{unicode-math} % this also loads fontspec
  \defaultfontfeatures{Scale=MatchLowercase}
  \defaultfontfeatures[\rmfamily]{Ligatures=TeX,Scale=1}
\fi
\usepackage{lmodern}
\ifPDFTeX\else
  % xetex/luatex font selection
  \setmainfont[]{Times New Roman}
  \setsansfont[]{Arial}
  \setmonofont[]{Courier New}
\fi
% Use upquote if available, for straight quotes in verbatim environments
\IfFileExists{upquote.sty}{\usepackage{upquote}}{}
\IfFileExists{microtype.sty}{% use microtype if available
  \usepackage[]{microtype}
  \UseMicrotypeSet[protrusion]{basicmath} % disable protrusion for tt fonts
}{}
\usepackage{setspace}
% Make \paragraph and \subparagraph free-standing
\makeatletter
\ifx\paragraph\undefined\else
  \let\oldparagraph\paragraph
  \renewcommand{\paragraph}{
    \@ifstar
      \xxxParagraphStar
      \xxxParagraphNoStar
  }
  \newcommand{\xxxParagraphStar}[1]{\oldparagraph*{#1}\mbox{}}
  \newcommand{\xxxParagraphNoStar}[1]{\oldparagraph{#1}\mbox{}}
\fi
\ifx\subparagraph\undefined\else
  \let\oldsubparagraph\subparagraph
  \renewcommand{\subparagraph}{
    \@ifstar
      \xxxSubParagraphStar
      \xxxSubParagraphNoStar
  }
  \newcommand{\xxxSubParagraphStar}[1]{\oldsubparagraph*{#1}\mbox{}}
  \newcommand{\xxxSubParagraphNoStar}[1]{\oldsubparagraph{#1}\mbox{}}
\fi
\makeatother


\usepackage{longtable,booktabs,array}
\usepackage{calc} % for calculating minipage widths
% Correct order of tables after \paragraph or \subparagraph
\usepackage{etoolbox}
\makeatletter
\patchcmd\longtable{\par}{\if@noskipsec\mbox{}\fi\par}{}{}
\makeatother
% Allow footnotes in longtable head/foot
\IfFileExists{footnotehyper.sty}{\usepackage{footnotehyper}}{\usepackage{footnote}}
\makesavenoteenv{longtable}
\usepackage{graphicx}
\makeatletter
\newsavebox\pandoc@box
\newcommand*\pandocbounded[1]{% scales image to fit in text height/width
  \sbox\pandoc@box{#1}%
  \Gscale@div\@tempa{\textheight}{\dimexpr\ht\pandoc@box+\dp\pandoc@box\relax}%
  \Gscale@div\@tempb{\linewidth}{\wd\pandoc@box}%
  \ifdim\@tempb\p@<\@tempa\p@\let\@tempa\@tempb\fi% select the smaller of both
  \ifdim\@tempa\p@<\p@\scalebox{\@tempa}{\usebox\pandoc@box}%
  \else\usebox{\pandoc@box}%
  \fi%
}
% Set default figure placement to htbp
\def\fps@figure{htbp}
\makeatother


% definitions for citeproc citations
\NewDocumentCommand\citeproctext{}{}
\NewDocumentCommand\citeproc{mm}{%
  \begingroup\def\citeproctext{#2}\cite{#1}\endgroup}
\makeatletter
 % allow citations to break across lines
 \let\@cite@ofmt\@firstofone
 % avoid brackets around text for \cite:
 \def\@biblabel#1{}
 \def\@cite#1#2{{#1\if@tempswa , #2\fi}}
\makeatother
\newlength{\cslhangindent}
\setlength{\cslhangindent}{1.5em}
\newlength{\csllabelwidth}
\setlength{\csllabelwidth}{3em}
\newenvironment{CSLReferences}[2] % #1 hanging-indent, #2 entry-spacing
 {\begin{list}{}{%
  \setlength{\itemindent}{0pt}
  \setlength{\leftmargin}{0pt}
  \setlength{\parsep}{0pt}
  % turn on hanging indent if param 1 is 1
  \ifodd #1
   \setlength{\leftmargin}{\cslhangindent}
   \setlength{\itemindent}{-1\cslhangindent}
  \fi
  % set entry spacing
  \setlength{\itemsep}{#2\baselineskip}}}
 {\end{list}}
\usepackage{calc}
\newcommand{\CSLBlock}[1]{\hfill\break\parbox[t]{\linewidth}{\strut\ignorespaces#1\strut}}
\newcommand{\CSLLeftMargin}[1]{\parbox[t]{\csllabelwidth}{\strut#1\strut}}
\newcommand{\CSLRightInline}[1]{\parbox[t]{\linewidth - \csllabelwidth}{\strut#1\strut}}
\newcommand{\CSLIndent}[1]{\hspace{\cslhangindent}#1}



\setlength{\emergencystretch}{3em} % prevent overfull lines

\providecommand{\tightlist}{%
  \setlength{\itemsep}{0pt}\setlength{\parskip}{0pt}}



 


\usepackage{sectsty}
\chapterfont{\centering}
\usepackage{lscape}
\newcommand{\blandscape}{\begin{landscape}}
\newcommand{\elandscape}{\end{landscape}}
\makeatletter
\@ifpackageloaded{caption}{}{\usepackage{caption}}
\AtBeginDocument{%
\ifdefined\contentsname
  \renewcommand*\contentsname{Table of contents}
\else
  \newcommand\contentsname{Table of contents}
\fi
\ifdefined\listfigurename
  \renewcommand*\listfigurename{List of Figures}
\else
  \newcommand\listfigurename{List of Figures}
\fi
\ifdefined\listtablename
  \renewcommand*\listtablename{List of Tables}
\else
  \newcommand\listtablename{List of Tables}
\fi
\ifdefined\figurename
  \renewcommand*\figurename{Figure}
\else
  \newcommand\figurename{Figure}
\fi
\ifdefined\tablename
  \renewcommand*\tablename{Table}
\else
  \newcommand\tablename{Table}
\fi
}
\@ifpackageloaded{float}{}{\usepackage{float}}
\floatstyle{ruled}
\@ifundefined{c@chapter}{\newfloat{codelisting}{h}{lop}}{\newfloat{codelisting}{h}{lop}[chapter]}
\floatname{codelisting}{Listing}
\newcommand*\listoflistings{\listof{codelisting}{List of Listings}}
\makeatother
\makeatletter
\makeatother
\makeatletter
\@ifpackageloaded{caption}{}{\usepackage{caption}}
\@ifpackageloaded{subcaption}{}{\usepackage{subcaption}}
\makeatother
\usepackage{bookmark}
\IfFileExists{xurl.sty}{\usepackage{xurl}}{} % add URL line breaks if available
\urlstyle{same}
\hypersetup{
  pdftitle={Autocoups and Democracy},
  pdfauthor={Zhu Qi},
  colorlinks=true,
  linkcolor={blue},
  filecolor={Maroon},
  citecolor={Blue},
  urlcolor={blue},
  pdfcreator={LaTeX via pandoc}}


\title{Autocoups and Democracy}
\author{Zhu Qi}
\date{2025-11-19}
\begin{document}
\maketitle


\setstretch{1.618}
\section*{Abstract}\label{abstract}
\addcontentsline{toc}{section}{Abstract}

This chapter investigates the impact of autocoups on political
institutions, comparing them with traditional coups through an analysis
of variations in Polity V scores. It advances two primary hypotheses:
first, that incumbent leaders frequently consolidate power by
systematically undermining institutional constraints in the period
leading up to an autocoup, resulting in a decline in Polity V scores
attributable to the autocoup. Second, unlike traditional coups, which
exhibit a ``U-shaped'' trajectory in Polity V scores, autocoups
precipitate a persistent decline in these scores without subsequent
recovery. This is attributed to autocoup leaders' deliberate intent to
suppress opposition and dismantle institutional checks and balances to
secure prolonged tenure. Employing a country-fixed effects model, this
study demonstrates that Polity V scores typically decline following
autocoups, mirroring the magnitude of decline observed after traditional
coups. However, while traditional coups often lead to an immediate
reduction in Polity V scores followed by conditions conducive to
recovery over time, autocoups result in sustained democratic erosion.
These findings highlight the divergent political trajectories induced by
coups and autocoups. This research addresses a critical gap in the
empirical analysis of autocoups and contributes to academic and policy
discussion by elucidating their detrimental effects, particularly in
terms of democratic backsliding and the entrenchment of authoritarian
governance.

\textbf{Keywords:} \emph{Coups, Autocoups, Democratization}

\newpage

\section{Introduction}\label{introduction}

The decline in global political rights and civil liberties, as
documented by reports such as Freedom House's Freedom in the World 2024,
marks the eighteenth consecutive year of democratic backsliding
worldwide (\citeproc{ref-freedomhouse2024freedom}{Freedom House 2024}).
This sustained erosion naturally raises the question: What political
mechanisms primarily drive the decline in liberties and democratic
quality?

One of the primary suspects historically linked to episodes of
democratic recession and the decline in global liberties since 2000 has
been the coup d'état---the violent, non-constitutional seizure of power.
However, an emerging paradox challenges the view of coups as the main
contemporary perpetrator of backsliding. Data from the most cited coup
datasets suggest a secular decline in the frequency of traditional
coups; specifically, the number of coup attempts between 2008 and 2017
represented the lowest ten-year total since at least 1960
(\citeproc{ref-powell2011a}{J. M. Powell and Thyne 2011};
\citeproc{ref-thyne2019}{Thyne and Powell 2019}). Furthermore, despite
ongoing scholarly debates, a significant body of empirical literature
contends that coups may exert a positive or complex long-term effect on
democratization by removing entrenched dictatorships and breaking
political logjams (\citeproc{ref-powell2014a}{J. Powell 2014};
\citeproc{ref-thyne2014}{C. Thyne and Powell 2014};
\citeproc{ref-dahl2023}{Dahl and Gleditsch 2023}). If traditional coups
are decreasing and their long-term democratic effect is ambiguous or
even restorative, they may not be the principal driver of the current,
steady decline in global liberties. This critical gap compels us to
search for an alternative, more insidious mechanism of regime erosion.

In sharp contrast to the trend in traditional coups, a similar but
distinct political event---the autocoup---has increased notably since
2000 (\citeproc{ref-bermeo2016}{Bermeo 2016};
\citeproc{ref-baturo2022}{Baturo and Tolstrup 2022};
\citeproc{ref-zhu2024}{Zhu 2024}). An autocoup is defined as the
extension of an incumbent leader's tenure in office beyond the
originally mandated limit via extra-constitutional manipulation in this
study (\citeproc{ref-zhu2024}{Zhu 2024}). While both coups and autocoups
disrupt established political orders, autocoups involve the insidious
erosion of democratic norms from within by the very leader sworn to
uphold them. Despite the growing prevalence and conceptual significance
of this phenomenon, its specific impact on democracy and regime
transitions remains under-examined.

This article undertakes the first empirical investigation into the
democratic consequences of autocoups. Its primary objective is to
determine whether autocoups, in the current global context, entrench
authoritarian rule, facilitate democratization, or have no substantive
impact on regime trajectories. Given the conceptual and empirical
parallels to coups, a secondary aim is to conduct a focused comparative
analysis of their respective effects on democratization to clarify their
broader political ramifications.

To address these questions, this study employs a fixed-effects model to
evaluate the respective impacts of coups and autocoups on democratic
quality, operationalized through Polity V scores. The findings
demonstrate a critical divergence:

While both coups and autocoups are associated with an immediate decline
in democratic quality, the impact differs significantly over time.
Polity V scores affected by coups typically exhibit a notable recovery
within two years. Conversely, democratic quality impacted by autocoups
shows no such improvement over the same period, indicating a longer and
deeper impact on the erosion of democracy.

This study makes two principal contributions to political science.
Firstly, it provides the first systematic empirical analysis of the
impact of autocoups on democratization, establishing them as a distinct
political phenomenon and addressing a critical gap in the literature on
democratic backsliding. Secondly, by comparing the effects of coups and
autocoups, this research demonstrates the more severe and sustained
damage to democratic institutions caused by the latter. This underscores
the urgent need to treat autocoups not merely as a variant of executive
aggrandizement but as a distinct political pathology warranting focused
scholarly and policy attention.

The remainder of this article is structured as follows. Section 2
examines the mechanisms through which autocoups impact democratic
institutions, with particular emphasis on their comparison with
traditional coups. Section 3 outlines the research design,
methodological approach, and variables employed. Section 4 presents the
empirical findings and discusses their broader implications. Section 5
concludes by summarising the key findings and reflecting on their
significance for understanding and addressing autocoup dynamics.

\section{Theoretical Framework: Disaggregating Irregular
Transitions}\label{theoretical-framework-disaggregating-irregular-transitions}

The scholarly literature on irregular transitions has traditionally
focused on coups d'état, evaluating political outcomes through binary
regime classifications---whether countries democratize, autocratize, or
remain stable (\citeproc{ref-thyne2014}{C. Thyne and Powell 2014};
\citeproc{ref-derpanopoulos2016}{Derpanopoulos et al. 2016}). This
framework is appropriate for traditional coups, which trigger abrupt
leadership replacement and elite rearrangement, making the political
disruption easily captured through dichotomous shifts.

However, this binary framework is inadequate for the autocoup, an
increasingly prevalent form of irregular power consolidation. An
autocoup occurs when an incumbent leader extends their tenure beyond
constitutionally mandated limits through extra-constitutional means.
Unlike a classic coup d'état, an autocoup retains the existing leader
and governing coalition, meaning it rarely triggers immediate changes in
regime labels. The absence of nominal transition thus obscures the true
consequence: the subversion of institutional constraints that regulate
executive power.

For this reason, to evaluate the political impact of autocoups, a more
sensitive approach is required. This study employs the Polity V score
(from the Polity5 dataset), which ranges from \(\text{–10}\) to
\(+\text{10}\). This continuous measure enables the detection of
incremental degradation in executive constraints and political
participation, aligning with research on subtle democratic backsliding
(\citeproc{ref-dahl2023}{Dahl and Gleditsch 2023}).

\subsection{Mechanisms of democratic damage in
autocoups}\label{mechanisms-of-democratic-damage-in-autocoups}

Autocoups pose a distinct and acute threat to democratic institutions
because they require the incumbent to systematically dismantle the
political rules \emph{before} and \emph{after} the decisive act. The
democratic damage unfolds through three reinforcing mechanisms:

The decline associated with autocoups begins before the formal extension
of tenure. To ensure a successful and uncontested breach of
constitutional limits, incumbents must first neutralize potential veto
players and dismantle oversight mechanisms. This preparation involves
purging rival elites, restricting media freedoms, weakening judicial
independence, and harassing opposition parties (e.g., Peru's 1992
autogolpe, where the Polity score collapsed from 8 to \(\text{–4}\)
following the dissolution of Congress
(\citeproc{ref-cameron1998}{Cameron 1998})). These actions constitute
the first wave of democratic erosion.

Autocoups are, once they are executed, fundamentally, an unlawful
violation of existing constitutional constraints. Term limits are
foundational safeguards; when incumbents circumvent them, they destroy
one of the most important institutional protections against
authoritarianism. The act itself permanently weakens the credibility of
legal norms, establishing a precedent that facilitates future violations
by subsequent leaders.

After the Event, following the illegal extension of power, leaders face
a powerful structural incentive for long-term control. The committed
constitutional crime exposes the leader to the threat of future
prosecution and retribution should they be removed. This structural
necessity differentiates autocoup leaders from coup-installed leaders;
the former cannot credibly commit to liberalization without
significantly increasing their personal risk. The result is a
self-reinforcing authoritarian drift, characterized by sustained
repression and the indefinite postponement of democratic restoration.

The immediate impact of the autocoup event, despite the differences in
preparation (gradual vs.~abrupt), is thus hypothesized to be comparable
to that of a traditional coup:

\textbf{\emph{H1: Autocoups will result in a significant decline in
Polity V scores immediately following their occurrence, in a manner
comparable to the effects observed in traditional coups.}}

\subsection{Comparative long-term outcomes: autocoups
vs.~coups}\label{comparative-long-term-outcomes-autocoups-vs.-coups}

Despite the shared feature of immediate institutional disruption,
autocoups and traditional coups produce markedly different long-term
political outcomes.

The literature shows that coups produce highly variable consequences.
Several studies argue that coups can act as catalysts for political
renewal, especially when they remove entrenched authoritarian leaders or
aim to restore constitutional order (\citeproc{ref-thyne2014}{C. Thyne
and Powell 2014}; \citeproc{ref-miller2016}{Miller 2016}). In Niger, for
example, President Mamadou Tandja's attempt in 2009 to amend the
constitution to permit a third term precipitated a military coup in 2010
(\citeproc{ref-miller2016}{Miller 2016}). Similarly, in Honduras the
same year, President Manuel Zelaya was removed from office by the
military after seeking to alter the constitution to allow immediate
re-election (\citeproc{ref-muuxf1oz-portillo2019}{Muñoz-Portillo and
Treminio 2019}).

Coup leaders, taking power from outside the existing executive, may feel
incentives to promise reforms to build legitimacy and mitigate elite
resistance. This dynamic supports the cited U-shaped trajectory: an
initial sharp decline in democratic quality followed by a gradual
rebound toward the mean.

Autocoups, however, follow a fundamentally different logic. Because the
act is inherently anti-democratic and self-serving, the leader cannot
later pursue liberalization without endangering their own political
survival. The structural incentives created by the successful autocoup
are thus to centralize power, maintain long-term coercive control, and
prohibit political openings. The result is a unidirectional, durable
authoritarian trajectory. Democratic decline associated with autocoups
tends not only to be sharp but also sustained, with little evidence of
recovery even in the medium term.

This difference leads to the second hypothesis:

\textbf{\emph{H2: Autocoups produce significant declines in Polity V
scores that typically do not rebound, whereas traditional coups often
exhibit a U-shaped trajectory, with initial deterioration followed by
gradual democratic improvement.}}

\section{Methodology and variables}\label{methodology-and-variables}

\subsection{Methodology}\label{methodology}

As outlined above, autocoups are less likely to result in full regime
transitions---whether from democracy to autocracy or vice versa.
Consequently, evaluating their effects solely in terms of regime change
or shifts across democratic thresholds is analytically inappropriate.
Instead, this study assesses political change by examining variations in
Polity V scores, which capture more subtle shifts in institutional
quality and democratic performance.

To differentiate between immediate and medium-term effects, the analysis
considers both event-year and two-year impacts of autocoups. The
event-year effect is measured as the change in Polity V score in the
year of the autocoup relative to the preceding year:

\[
Polity_{t} - Polity_{t-1}
\]

The three-year effect captures the change in Polity V score two years
after the event, relative to the year of the autocoup:

\[
Polity_{t+3} - Polity_t
\]

This three-year specification is intended to capture medium-term
political developments, as autocoups typically entrench existing power
structures rather than inducing immediate systemic change. Short-term
fluctuations may not fully reflect the institutional consequences of
such events.

To empirically test the hypotheses, the study employs a linear
fixed-effects model at the country level. To distinguish between
attempted and successful autocoups, separate models are estimated using
binary variables that code for autocoup attempts and successes,
respectively.

\subsection{Variables}\label{variables}

The analysis draws upon a global panel of country-year observations
spanning from 1950 to 2020, resulting in approximately 9,100
observations. The primary dependent variable is the change in Polity V
score, calculated either as a one-year or three-year difference,
depending on the model specification. Polity V scores range from −10
(full autocracy) to +10 (full democracy). To address missing data caused
by transitional codes (−66, −77, −88), these values are replaced with
the nearest valid Polity score to preserve temporal continuity and
reduce bias associated with listwise deletion.

The primary independent variable is the occurrence of an autocoup, as
defined in Chapter 2. The dataset includes 83 attempted and 64
successful autocoups. For models analysing attempted autocoups, the
variable is coded as 1 in the year of the attempt and 0 otherwise. In
the three-year specification, a decay function is applied to measure the
persistence of effects, following the approach of Dahl and Gleditsch
(\citeproc{ref-dahl2023}{2023}). To account for temporal diffusion, a
half-life of five years is specified, allowing the model to capture both
immediate and delayed consequences from the year of the autocoup (
\(y_t\) ) through to four years post-event ( \(y_{t+4}\) ).

In addition, traditional coups are included as a secondary independent
variable for two reasons. First, they enable a comparative evaluation of
the political consequences of coups versus autocoups. Second, coups and
autocoups may occur in close proximity or in causal sequence,
necessitating analytical disaggregation. The coup data are drawn from
Powell and Thyne (\citeproc{ref-powell2011}{2011}), and are coded in a
manner consistent with the autocoup variables---using a binary indicator
for one-year effects and a decay function for three-year impacts.

A set of control variables is included to account for alternative
explanations. These comprise: economic performance, proxied by GDP
growth and GDP per capita; political violence, to capture variations in
political stability; and the logarithm of population size, which serves
as a proxy for state capacity and scale effects. To mitigate concerns
regarding reverse causality, all control variables are lagged by one
year, ensuring that their values precede the outcome being measured.

Two additional dummy variables are incorporated:

\textbf{Non-democracy:} This variable captures regime type by
distinguishing cases with Polity V scores below −6 (already autocratic
and less prone to further decline) and above +6 (institutionally
resilient to democratic erosion).

\textbf{Cold War:} A temporal dummy variable to account for the
geopolitical context, in line with previous studies on the relationship
between coups and democratisation (\citeproc{ref-thyne2014}{C. Thyne and
Powell 2014}; \citeproc{ref-derpanopoulos2016}{Derpanopoulos et al.
2016}; \citeproc{ref-dahl2023}{Dahl and Gleditsch 2023}). It captures
broad international trends, such as the stagnation or decline in
democratic scores during the Cold War (1960s--1990) and the more
pronounced democratising trend after 1990.

\section{Results and discussion}\label{results-and-discussion}

This section examines the democratic implications of autocoups by
analysing their effects on Polity V scores, both in the immediate
aftermath and in the medium term. Table~\ref{tbl-demomodel} presents
four models: Models 1 and 2 report results for attempted autocoups,
while Models 3 and 4 pertain to successful autocoups. Within each group,
Models 1 and 3 assess immediate effects (in the event year), whereas
Models 2 and 4 evaluate medium-term effects (three years after the
event).

\subsection{Immediate democratic
impact}\label{immediate-democratic-impact}

Consistent with the first hypothesis, autocoups and coups are associated
with significant immediate declines in Polity V scores. In both Models 1
and 3, autocoups---whether attempted or successful---lead to a
statistically significant reduction of approximately 1.3 points in
Polity V scores in the event year, all else equal. These effects are
comparable in magnitude across both attempted and successful autocoups,
suggesting that the democratic damage materialises irrespective of
whether the attempt fully succeeds.

Traditional coups are associated with larger immediate declines. Model 1
shows that attempted coups reduce Polity V scores by 1.31 points, while
successful coups, in Model 3, lead to a drop of 2.12 points, both
significant at the \(1\%\) level. These findings confirm that both types
of irregular power grabs deliver immediate shocks to democratic
institutions, though coups---especially successful ones---inflict
greater disruption.

\begin{table}

\caption{\label{tbl-demomodel}The Impacts on
Democratization(1950--2018): Autocoups vs Coups}

\centering{

\begin{tabular}{@{\extracolsep{30pt}}lcccc} 
\\[-1.8ex]\hline 
\hline \\[-1.8ex] 
 & \multicolumn{4}{c}{Dependent variable: Differences of Polity V scores} \\ 
\cline{2-5} 
\\[-1.8ex] & \multicolumn{2}{c}{Attempted} & \multicolumn{2}{c}{Succeeded} \\ 
 & (1) & (2) & (3) & (4) \\ 
\hline \\[-1.8ex] 
 Autocoup & $-$1.276$^{***}$ & $-$0.338 & $-$1.290$^{***}$ & $-$0.130 \\ 
  & (0.201) & (0.322) & (0.226) & (0.360) \\ 
  & & & & \\ 
 Coup & $-$1.312$^{***}$ & 1.203$^{***}$ & $-$2.120$^{***}$ & 1.868$^{***}$ \\ 
  & (0.091) & (0.127) & (0.124) & (0.183) \\ 
  & & & & \\ 
 GDP per Capita & $-$0.003$^{**}$ & $-$0.009$^{***}$ & $-$0.003$^{**}$ & $-$0.010$^{***}$ \\ 
  & (0.001) & (0.002) & (0.001) & (0.002) \\ 
  & & & & \\ 
 Economic Trend & $-$0.428 & $-$0.563 & $-$0.329 & $-$0.635 \\ 
  & (0.277) & (0.480) & (0.275) & (0.480) \\ 
  & & & & \\ 
 Log Population & 0.178$^{**}$ & 0.755$^{***}$ & 0.188$^{***}$ & 0.734$^{***}$ \\ 
  & (0.070) & (0.122) & (0.070) & (0.122) \\ 
  & & & & \\ 
 Political Violence & 0.015 & 0.033 & 0.012 & 0.033 \\ 
  & (0.014) & (0.024) & (0.014) & (0.024) \\ 
  & & & & \\ 
 Non-Democracy & 0.809$^{***}$ & $-$0.776$^{***}$ & 0.797$^{***}$ & $-$0.775$^{***}$ \\ 
  & (0.062) & (0.109) & (0.062) & (0.109) \\ 
  & & & & \\ 
 Cold War & $-$0.235$^{***}$ & $-$0.092 & $-$0.224$^{***}$ & $-$0.116 \\ 
  & (0.063) & (0.109) & (0.063) & (0.109) \\ 
  & & & & \\ 
\hline \\[-1.8ex] 
Observations & 9,104 & 9,104 & 9,104 & 9,104 \\ 
R$^{2}$ & 0.047 & 0.028 & 0.055 & 0.030 \\ 
Adjusted R$^{2}$ & 0.029 & 0.009 & 0.036 & 0.011 \\ 
F Statistic & 55.436$^{***}$ & 32.690$^{***}$ & 64.970$^{***}$ & 34.462$^{***}$ \\ 
\hline 
\hline \\[-1.8ex] 
\textit{Note:}  & \multicolumn{4}{r}{$^{*}$p$<$0.1; $^{**}$p$<$0.05; $^{***}$p$<$0.01} \\ 
\end{tabular}

}

\end{table}%

\subsection{Medium-term divergence: coups
vs.~autocoups}\label{medium-term-divergence-coups-vs.-autocoups}

In the medium term, however, the political trajectories begin to
diverge: while coups are followed by significant improvements in Polity
V scores, autocoups continue to exert a negative effect, albeit one that
does not reach statistical significance.

Models 2 and 4 evaluate changes in Polity V scores three years after the
event. The results indicate that autocoups have no statistically
significant effect in the medium term---whether attempted or
successful---implying that the initial democratic decline is not
followed by subsequent institutional reform or recovery. In contrast,
attempted coups are associated with a significant increase of 1.2
points, and successful coups show a particularly strong rebound of 1.87
points, both at the \(1\%\) significance level.

These findings provide clear support for the second hypothesis. Whereas
coups tend to exhibit a ``U-shaped'' pattern---with democratic erosion
followed by recovery---autocoups demonstrate a consistent,
unidirectional decline in democratic quality, with no evidence of
rebound.

The results suggest that autocoups exert their impact primarily in the
short term, as reflected in the immediate drop in Polity V scores, while
offering no potential for democratic revitalisation in the medium term.
This contrasts with coups, which, although initially disruptive,
sometimes serve as catalysts for institutional renewal, particularly in
cases where they are followed by electoral processes or popular
mobilisation.

These findings reinforce the notion that autocoups function to entrench
incumbents, undermining constitutional safeguards and consolidating
executive power. By contrast, coups---particularly those that displace
entrenched regimes---may open space for institutional realignment or
liberalisation, depending on the post-coup political context.

The models incorporate a range of control variables to isolate the
effects of coups and autocoups:

GDP per capita is negatively and significantly associated with changes
in Polity V scores across all models. This counterintuitive negative
association may reflect the limited potential for democratic gains in
already high-income democracies, where Polity V scores are near their
ceiling.

Log of population size is positively and significantly associated with
Polity score changes, suggesting that larger states may possess greater
institutional adaptability or reform potential.

The results for non-democratic regimes (defined as those with Polity V
scores below −6) reveal a temporal asymmetry in their effects on
democratic outcomes. In the event-year models (Models 1 and 3),
non-democratic regimes are associated with significant positive changes
in Polity V scores. This likely reflects cases where short-term
liberalisation or reform efforts follow leadership crises or
institutional ruptures, producing modest democratic gains even within
authoritarian contexts. By contrast, in the three-year models (Models 2
and 4), the effect reverses direction: non-democratic regimes are
associated with significant declines in Polity V scores over the medium
term. This pattern suggests that early signs of liberalisation often
fail to consolidate and may be followed by renewed authoritarian
entrenchment. In essence, while non-democratic regimes may exhibit
initial democratic openings---whether symbolic or procedural---these
gains are frequently short-lived, with longer-term trajectories
reverting to autocratic norms. This dynamic underscores the fragility of
democratic progress in authoritarian contexts, where reforms introduced
in the aftermath of institutional disruption are often superficial or
strategically instrumental, lacking the structural support required for
sustained democratisation.

Cold War context is statistically significant only in the event-year
models, where it correlates with a decline in Polity V scores,
reflecting the broader global pattern of democratic suppression during
the Cold War period.

Political violence and economic growth do not show consistent or
significant effects, indicating that immediate democratic outcomes are
more sensitive to regime characteristics and structural factors than to
short-term economic or security conditions.

Overall, the empirical results offer robust support for both hypotheses.
Autocoups and coups both lead to significant immediate declines in
democratic quality, with coups inflicting greater short-term damage. In
the medium term, coups are often followed by democratic recovery,
whereas autocoups result in persistent democratic erosion with no
evidence of rebound.

These findings suggest that autocoups represent a particularly insidious
form of democratic backsliding, less dramatic than coups but ultimately
more damaging in their long-term effects. They reinforce the need for
greater scholarly and policy attention to constitutional manipulations
by incumbents, which, although often gradual and legally framed, can
produce lasting democratic decay.

\subsection{Robustness tests}\label{robustness-tests}

To assess the robustness of the main findings, a series of alternative
model specifications were estimated. The results confirm that the core
conclusions remain stable under these variations.

First, the operationalisation of the autocoup variable was modified: the
decay function used in the baseline analysis was replaced with a binary
indicator distinguishing between attempted and successful autocoups.
Additionally, the broad `non-democracy' category was disaggregated into
more specific regime types---military, personalist, presidential,
parliamentary, and `other'---with dominant-party regimes serving as the
reference category. This classification mirrors the approach used in the
determinants analysis of autocoups presented in earlier chapters. The
results of these robustness models are presented in Models 5 to 8 in
Table~\ref{tbl-demomodel2}.

\begin{table}

\caption{\label{tbl-demomodel2}The Impact of Autocoups on
Democratization: Binary Autocoups}

\centering{

\begin{tabular}{@{\extracolsep{20pt}}lcccc} 
\\[-1.8ex]\hline 
\hline \\[-1.8ex] 
 & \multicolumn{4}{c}{Dependent variable: Differences of Polity V scores} \\ 
\cline{2-5} 
\\[-1.8ex] & \multicolumn{2}{c}{Attempted} & \multicolumn{2}{c}{Succeeded} \\ 
 & (5) & (6) & (7) & (8) \\ 
\hline \\[-1.8ex] 
 Autocoup & $-$1.236$^{***}$ & $-$0.148 & $-$1.234$^{***}$ & $-$0.057 \\ 
  & (0.200) & (0.359) & (0.226) & (0.402) \\ 
  & & & & \\ 
 Coup & $-$1.366$^{***}$ & 1.240$^{***}$ & $-$2.190$^{***}$ & 1.712$^{***}$ \\ 
  & (0.091) & (0.157) & (0.123) & (0.215) \\ 
  & & & & \\ 
 GDP per Capita & $-$0.003$^{**}$ & $-$0.010$^{***}$ & $-$0.003$^{**}$ & $-$0.010$^{***}$ \\ 
  & (0.001) & (0.002) & (0.001) & (0.002) \\ 
  & & & & \\ 
 Economic Trend & $-$0.387 & $-$0.569 & $-$0.282 & $-$0.629 \\ 
  & (0.277) & (0.482) & (0.276) & (0.482) \\ 
  & & & & \\ 
 Log Population & 0.247$^{***}$ & 0.890$^{***}$ & 0.262$^{***}$ & 0.879$^{***}$ \\ 
  & (0.072) & (0.126) & (0.072) & (0.126) \\ 
  & & & & \\ 
 Political Violence & 0.015 & 0.044$^{*}$ & 0.012 & 0.046$^{*}$ \\ 
  & (0.014) & (0.024) & (0.014) & (0.024) \\ 
  & & & & \\ 
 Regime: Military & 0.602$^{***}$ & $-$0.545$^{***}$ & 0.574$^{***}$ & $-$0.584$^{***}$ \\ 
  & (0.101) & (0.177) & (0.101) & (0.178) \\ 
  & & & & \\ 
 \hspace{1.5cm} Personal & $-$0.042 & $-$0.532$^{***}$ & $-$0.065 & $-$0.526$^{***}$ \\ 
  & (0.094) & (0.164) & (0.094) & (0.164) \\ 
  & & & & \\ 
 \hspace{1.5cm} Presidential & $-$0.576$^{***}$ & 0.399$^{**}$ & $-$0.578$^{***}$ & 0.381$^{**}$ \\ 
  & (0.091) & (0.158) & (0.090) & (0.158) \\ 
  & & & & \\ 
 \hspace{1.5cm} Parliamentary & $-$0.475$^{***}$ & 0.965$^{***}$ & $-$0.468$^{***}$ & 0.966$^{***}$ \\ 
  & (0.105) & (0.182) & (0.104) & (0.182) \\ 
  & & & & \\ 
 \hspace{1.5cm} Other & 0.999$^{***}$ & 1.094$^{***}$ & 1.013$^{***}$ & 1.115$^{***}$ \\ 
  & (0.114) & (0.199) & (0.114) & (0.199) \\ 
  & & & & \\ 
 Cold War & $-$0.168$^{***}$ & $-$0.002 & $-$0.156$^{**}$ & $-$0.011 \\ 
  & (0.064) & (0.111) & (0.063) & (0.111) \\ 
  & & & & \\ 
\hline \\[-1.8ex] 
Observations & 9,036 & 9,036 & 9,036 & 9,036 \\ 
R$^{2}$ & 0.060 & 0.033 & 0.068 & 0.033 \\ 
Adjusted R$^{2}$ & 0.041 & 0.014 & 0.049 & 0.014 \\ 
F Statistic & 47.043$^{***}$ & 25.244$^{***}$ & 53.742$^{***}$ & 25.364$^{***}$ \\ 
\hline 
\hline \\[-1.8ex] 
\textit{Note:}  & \multicolumn{4}{r}{$^{*}$p$<$0.1; $^{**}$p$<$0.05; $^{***}$p$<$0.01} \\ 
\end{tabular}

}

\end{table}%

Consistent with the main models, autocoups remain significantly
associated with negative changes in Polity V scores in the short term
(Models 5 and 7), with coefficients of −1.236 and −1.234, respectively
(both significant at the \(1\%\) level). However, in the three-year
models (Models 6 and 8), the effect becomes statistically insignificant,
indicating that the negative effect of autocoups is immediate but not
sustained over time.

By contrast, coups continue to show a distinct ``U-shaped'' effect. In
the event-year models (Models 5 and 7), coups are associated with
significant declines in Polity V scores (−1.366 and −2.190), both at the
\(1\%\) level. Yet in the three-year models (Models 6 and 8), the effect
reverses direction: coups are now associated with large positive changes
in Polity V scores (+1.240 and +1.712, also significant at the \(1\%\)
level). This confirms the earlier interpretation that while coups may
cause immediate democratic disruption, they are often followed by
democratic recovery in the medium term.

The disaggregated regime type variables provide additional insights.
Military regimes show significant positive effects in the event-year
models (Models 5 and 7), with coefficients of +0.602 and +0.574, but
become negative and significant in the three-year models (−0.545 and
−0.584 in Models 6 and 8). This reversal suggests that initial
post-event liberalisation in military regimes is not sustained, and may
even regress.

Personalist regimes are consistently associated with negative and
significant effects in the three-year models (Models 6 and 8: −0.532 and
−0.526), but not in the two-year models, suggesting that their
democratic erosion becomes more evident over time.

Presidential and parliamentary democracies follow a similar pattern:
both show significant negative effects in the short term (Models 5 and
7), and positive, statistically significant effects in the medium term
(Models 6 and 8). For example, parliamentary democracies are associated
with a drop of −0.475/−0.468 in the short term but show a gain of
+0.965/0.966 over three years. This pattern supports the idea that
democratic institutions may initially be shaken by political disruption
but recover when institutional mechanisms are strong.

``Other'' regimes (likely transitional or provisional systems) show
consistently large and positive effects across all models, ranging from
+0.999 to +1.115, all significant at the \(1\%\) level. This implies
that these regimes tend to transition toward more democratic forms over
both short and medium time frames.

Several control variables also behave consistently with the baseline
models. GDP per capita is negatively and significantly associated with
changes in Polity V scores across all models, again likely reflecting
ceiling effects in advanced democracies with limited room for
improvement. Log of population is positively and significantly related
to Polity changes, reinforcing earlier interpretations that larger
states may possess greater reform potential or be more likely to
register changes in democratic performance. Political violence becomes
statistically significant only in the three-year models (Models 6 and
8), where it has a small positive effect (+0.044, +0.046), suggesting
that prolonged unrest may precede some form of institutional response or
democratic opening. The Cold War variable is significant only in the
event-year models (Models 5 and 7), where it is associated with small
negative effects (−0.168 and −0.156), consistent with broader patterns
of democratic suppression during the Cold War period.

These robustness models confirm the main findings while offering
additional nuance. These results underscore the importance of both
regime context and temporal scope in evaluating the consequences of
irregular power grabs. Autocoups, unlike coups, represent a consistently
negative force for democratic institutions---one that undermines without
paving the way for recovery.

\section{Conclusion}\label{conclusion}

This chapter has examined the impact of autocoups on democratic
institutions by analysing changes in Polity V scores, with a comparative
focus on traditional coups. Two key hypotheses guided the analysis:
first, that autocoups are associated with consistent declines in
democratic quality, particularly in the short term; and second, that
while coups often generate initial disruptions, they tend to produce a
``U-shaped'' effect, marked by subsequent democratic recovery or even
advancement in the medium term.

The empirical results offer strong support for these hypotheses. Across
multiple model specifications, autocoups---whether attempted or
successful---exhibit significant negative effects on Polity V scores in
the event year, but these effects do not persist into the medium term.
In contrast, coups are associated with significant democratic
improvement three years after the event, despite an initial decline.
This pattern is robust across models incorporating disaggregated regime
types, alternative lag structures, and extended time horizons.

The analysis further reveals important variation across regime types.
Military and personalist regimes, while sometimes exhibiting modest
democratic gains in the immediate aftermath, tend to experience declines
in Polity V scores over time, suggesting a return to entrenched
authoritarianism. Presidential and parliamentary democracies, by
contrast, initially register democratic decline but tend to recover
within three years---consistent with institutional resilience. Notably,
transitional or provisional regimes (``other'' types) display
consistently strong democratic gains, underscoring their potential for
reform during periods of flux.

The findings carry several theoretical and policy-relevant implications.
While coups are widely recognised as pivotal events in the study of
regime change, autocoups deserve greater scholarly attention. Unlike
coups, which may at times catalyse democratic transitions, autocoups
represent a systematically anti-democratic mechanism, typically employed
to erode checks on executive power and extend incumbents' rule.
Moreover, as shown in Chapter 4, autocoup leaders tend to retain power
for longer periods---nearly a decade on average---compared to less than
seven years for coup-installed leaders, implying more durable
institutional consequences.

This chapter also advances the methodological literature by emphasising
the importance of temporal framing. Many political shocks---particularly
autocoups---are preceded by elite purges, electoral manipulation, or
institutional weakening. Consequently, focusing solely on post-event
changes risks overlooking the cumulative nature of democratic decline.
The findings thus support a more longitudinal and process-oriented
approach to studying regime erosion.

Nevertheless, limitations remain. Notably, coups and autocoups
occasionally occur in close temporal proximity, making it difficult to
disentangle their respective contributions to changes in Polity V
scores. Future research should seek to better isolate these overlapping
effects, perhaps through finer-grained event sequencing or qualitative
case tracing.

In sum, this chapter reinforces the view that autocoups are a critical
yet underexplored driver of democratic backsliding. Their often-subtle
execution belies their long-term consequences. As such, they warrant
continued empirical scrutiny and deeper integration into both the
comparative democratisation literature and policy frameworks concerned
with defending constitutional governance.

\phantomsection\label{refs}
\begin{CSLReferences}{1}{0}
\bibitem[\citeproctext]{ref-baturo2022}
Baturo, Alexander, and Jakob Tolstrup. 2022. {``Incumbent Takeovers.''}
\emph{Journal of Peace Research} 60 (2): 373--86.
\url{https://doi.org/10.1177/00223433221075183}.

\bibitem[\citeproctext]{ref-bermeo2016}
Bermeo, Nancy. 2016. {``On Democratic Backsliding.''} \emph{Journal of
Democracy} 27 (1): 5--19. \url{https://doi.org/10.1353/jod.2016.0012}.

\bibitem[\citeproctext]{ref-cameron1998}
Cameron, Maxwell A. 1998. {``Self-Coups: Peru, Guatemala, and Russia.''}
\emph{Journal of Democracy} 9 (1): 125--39.
\url{https://doi.org/10.1353/jod.1998.0003}.

\bibitem[\citeproctext]{ref-dahl2023}
Dahl, Marianne, and Kristian Skrede Gleditsch. 2023. {``Clouds with
Silver Linings: How Mobilization Shapes the Impact of Coups on
Democratization.''} \emph{European Journal of International Relations},
January, 135406612211432.
\url{https://doi.org/10.1177/13540661221143213}.

\bibitem[\citeproctext]{ref-derpanopoulos2016}
Derpanopoulos, George, Erica Frantz, Barbara Geddes, and Joseph Wright.
2016. {``Are Coups Good for Democracy?''} \emph{Research \& Politics} 3
(1): 205316801663083. \url{https://doi.org/10.1177/2053168016630837}.

\bibitem[\citeproctext]{ref-freedomhouse2024freedom}
Freedom House. 2024. {``Freedom in the World 2024.''}
\url{https://freedomhouse.org/sites/default/files/2024-02/FIW_2024_DigitalBooklet.pdf}.

\bibitem[\citeproctext]{ref-miller2016}
Miller, Michael K. 2016. {``Reanalysis: Are Coups Good for Democracy?''}
\emph{Research \& Politics} 3 (4): 205316801668190.
\url{https://doi.org/10.1177/2053168016681908}.

\bibitem[\citeproctext]{ref-muuxf1oz-portillo2019}
Muñoz-Portillo, Juan, and Ilka Treminio. 2019. {``The Politics of
Presidential Term Limits in Central America.''} In, 495--516. Oxford
University PressOxford.
\url{https://doi.org/10.1093/oso/9780198837404.003.0024}.

\bibitem[\citeproctext]{ref-powell2014a}
Powell, Jonathan. 2014. {``An Assessment of the {`}Democratic{'} Coup
Theory.''} \emph{African Security Review} 23 (3): 213--24.
\url{https://doi.org/10.1080/10246029.2014.926949}.

\bibitem[\citeproctext]{ref-powell2011a}
Powell, Jonathan M, and Clayton L Thyne. 2011. {``Global Instances of
Coups from 1950 to 2010: A New Dataset.''} \emph{Journal of Peace
Research} 48 (2): 249--59. \url{http://www.jstor.org/stable/29777507}.

\bibitem[\citeproctext]{ref-powell2011}
Powell, and Thyne. 2011. {``Global Instances of Coups from 1950 to 2010:
A New Dataset.''} \emph{Journal of Peace Research} 48 (2): 249--59.
\url{https://doi.org/10.1177/0022343310397436}.

\bibitem[\citeproctext]{ref-thyne2014}
Thyne, Clayton, and Jonathan Powell. 2014. {``Coup d{'}état or Coup
d'Autocracy? How Coups Impact Democratization, 1950-2008.''}
\emph{Foreign Policy Analysis}, April, n/a--.
\url{https://doi.org/10.1111/fpa.12046}.

\bibitem[\citeproctext]{ref-thyne2019}
Thyne, and Powell. 2019. {``Coup Research,''} October.
\url{https://doi.org/10.1093/acrefore/9780190846626.013.369}.

\bibitem[\citeproctext]{ref-zhu2024}
Zhu, Qi. 2024. {``Leadership Transitions and Survival: Coups, Autocoups,
and Power Dynamics.''} PhD thesis, University of Essex.

\end{CSLReferences}




\end{document}
